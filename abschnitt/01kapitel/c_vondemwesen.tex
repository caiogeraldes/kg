% TeX root=../../main.tex

\renewcommand{\chaptitlefont}{\centering\large\bfseries\scshape}
\chap{Von dem Wesen der Sprachlaute und von dein Verhältnisse
	derselben zu einander in den Mundarten}
\renewcommand{\chaptitlefont}{\centering\LARGE\bfseries\scshape}

Um eine klare und sichere Einsicht in das Wesen der griechischen Laute und in
das Verhältnis derselben zu einander in den Mundarten zu gewinnen, ist es
notwendig, einen Blick zu thun auf die Laute der mit dem Griechischen
urverwandten Sprachen. Die Arbeiten der vergleichenden Grammatik\footnote{}
haben zum teil mit vollkommen genügender Sicherheit aufgewiesen, was von den
Lauten einer jeden der indogermanischen Sprachen so zu sagen ursprünglich ist
und was nicht, und auch wo ein solcher Nachweis nicht zu erbringen wäre, ist es
doch lehrreich und wichtig, das in den verschiedenen Sprachen Entsprechende zu
kennen.

\section{Von dem Wesen der Vokale}

\paragraph{} Der A-Laut, im Sanskrit ungeteilt, hat sich im Griechischen in die
drei Laute a e o (α ε ο, ᾶ η ω ει ου) gespalten; dasselbe ist im Lateinischen
der Fall, nur dass hier ĕ und ŏ grossenteils zu ĭ und ŭ weiter entwickelt sind.
Doch zeigt sich die Zusammengehörigkeit dieser A-Vokale, wie man sie mit L.
Meyer nennen kann, auch im Griechischen fort und fort, man vergl. καλ (dor.),
καλή (att.), Masc. καλός, Voc. καλέ, N. Plur. καλα, dazu καλα-ῖς, καλοῦ (aus
ό-ο, strengdor. καλῶ), καλῷ.54) Beispiele des Entsprechens in den verwandten
Sprachen, zunächst für die kurzen Laute: a) gr. α, sk. a, lat. u. s. w. a,
καλ-ός, sk. kalj-as (gesund), δάκρυ, l. lacruma, goth. tagr, δάκ-νω, sk.
da[cnull ]-āmi, goth. tah-ja (zerreisse), καν-αχέω, rausche, sk. ka[ndot ]-kani,
Glocke, l. can-o; — b) gr. o, sk. a (ā), lat. o (e), deutsch meist a: γόνυ, sk.
ǵānu, l. gĕnu, ὄψ (F όψ), l. vōx, sk. vā[kacute], δόμος, l. domus, sk. damas,
ὀκτώ, sk. ash[tnull ]āu, goth. ahtau, d. acht, ὄϊς (ὄϝις), sk. avis, l. ovis,
althochd. auwi Schäfchen; c) gr. ε, sk. a, lat. u. s. w. e (i): ἕρπ-ω. l.
serp-o, sk. sarp-āmi, ἑπτά, l. septem, sk. saptan, γένος, l. genus, sk. ǵanus,
μένος, sk. mánas, Gen. μένους (st. μένες-ος), sk. mánas-as, ἔδ-ω, l. edo, sk.
ad-mi, ἐς-τί, l. est, sk. ás-ti, φέρ-ω, l. fer-o, sk. bhár-āmi, ἔ-φερ-ον, sk.
á-bharam. Über den Wechsel von α ο ε in den Dialekten s. § 24, 1.

\paragraph{} Aus dieser Spaltung der A-Laute erwuchsen der griechischen Sprache
grosse Vorteile.55) Zuerst wurde dadurch eine grössere Lautabwechslung bewirkt;
vgl. sk. á-bhar-am u. ἔ-φερ-ον, sk. á-labh-am u. ἔ-λαβ-ον, a-bhar-āmahi u.
ἐ-φερ-όμεθα, ǵa-ǵan-a u. γέ-γον-α, da-dar<*>-a u. δέ-δορ-κ-α; ein Wort wie
<*>atapathabrâhmaṇa würde dem griechischen Ohre unerträglich gewesen sein.
Sodann treten die verschiedenen Flexionsformen eines Wortes deutlicher hervor;
vgl. sk. Nom. u. Akk. Pl. pádas, Gen. S. padás u. πόδες, πόδας, ποδός, avahata =
εἴχετε u. εἴχετο. Ferner hat die Sprache diesen Wechsel auf das Sinnreichste für
die Flexionsbildung der Verben und für die Wortbildung verwendet; man umfasst
ihn unter dem Namen der Ablautung; z. B. τρέφω, τέτροφα, ἐτραφην; τροφή,
τροφεύς, τραφερός; κλέπτω, κέκλοφα, ἐκλάπην; κλοπή, vgl. stehle, stahl,
gestohlen. Auch ist dadurch die Anzahl der Wurzeln in der griechischen Sprache
grösser, vgl. μαν u. μεν in μαίνομαι u. μένω, δαμ u. δεμ in δαμάζω u. δέμω u. s.
w. Das lange a hat sich in gleicher Weise wie das kurze im Griechischen in drei
Laute: ᾶ, η, ω gespalten. Z. B. δᾶήρ, sk. dēvṛ Nom. dēvā (d. i. daivā), l.
lēvir, στα- στη-, sk. sthâ, l. stā-re, ἡμι-, sk. sāmi, l. sēmi, althochd. sāmi-,
halb, δῶρον, sk. dānam, l. dōnum, γι-γνώ-σκω, sk. ǵānā-mi, l. (g)nōsco, ὠκύς,
sk. ā[cnull ]us, lat. Komp. ōcior. Über die Dialekte s. § 26. Auch diesen
Wechsel der langen Laute hat die Sprache zu Flexions- und Wortbildungen vielfach
benutzt. S. §§ 36 und 37. Bezüglich der innerhalb des Griechischen entstandenen
langen Laute ist zu bemerken, dass die üblichsten Dialekte ein geschlossenes
langes e und o, welches sich zu ει bezw. ου entwickelte, neben den offenen η und
ω gewonnen haben; auch dies kann zu Unterscheidungen dienen, als τὼ λόγω u. τοῦ
λόγου. 

\paragraph{} Die Schwächung eines ursprünglichen A-Vokales, nämlich des ε, in ι
ist im ganzen selten und tritt fast nur vor zwei Konsonanten ein,56) z. B. sk.
ē-dhi st. as-dhi [root ] as, ἴς-θι [root ] ἐς (esse); sk. hjas, χθές, χθιζός;
sk. a<*>va, l. equus, gr. ἵππος. Die Schwächung eines ursprünglichen A-Vokales,
nämlich des ο, in υ ist gleichfalls selten, z. B. νύξ, sk. naktam, l. nox, ξύν,
l. com, cum, ὄ-νυξ, sk. nakhas; hie und da ist auch das υ aus einem
ursprünglichen ϝ entstanden, vgl. das Suffix σύνη m. tvana, θύρα m. dvāra-m,
σῦριγξ m. √ svar, d. swirran, γυνή aus γFανή (böot. βανά aus γFανά) m. goth.
quinô.57) (Ausserdem im äol. Dial., s. § 24, 2.)

\paragraph{}Die beiden anderen Grundvokale ι und υ haben ihren Laut fast
durchweg ohne Vermischung erhalten. Dadurch, dass das υ seinen ursprünglichen
vollen Laut u in den dünneren ü verwandelte (§ 5, 3), wurde es dem ι näher
gerückt, und so geschah es, dass es zuweilen in ι überging, so durch
Dissimilation in dem bei Homer noch nicht vorkommenden φῖτυ, φῖτύω [root ] φυ;
dagegen σίαλος, δρίον (schon Hom. Hes.) kommen zwar von σῦς, δρῦς her, aber die
Art der Ableitung ist nicht klar.58) Über ι st. υ im äol. Dial. s. § 24, 2; in
der gew. Sprache zeigt sich sporadischer Übergang von ι zu υ etwa seit dem 4.
Jahrh. v. Chr., als in Ἀμφικτύονεις (seit 410 nachzuw.) für Ἀμφικτίονες, ἥμυσυ
(schon 378 v. Chr.) st. ἥμισυ, Μουνυχίων (s. 306) st. Μουνιχιών; auf e.
delischen Inschr. (um 180 v. Chr.) κυλύχνιον zu κυλίχνη, Κυνθυκῶι für — ικῶι,
Χοιρύλος für Χοιρίλος; es ist hier Assimilation wie in ἥμυσυ, neben welchem
keineswegs ἡμύσεος ἡμύσεια vorkommt. Begreiflich ist ein solches Schwanken zumal
in Fremdwörtern: βίβλος βιβλίον att. Inschr. der guten Zeit, Plato (Schanz
Praef. Euthyd. VI) u. s. w.; βυβλίον βυβλιοθήκη Inschr. seit dem 1. Jahrh. v.
Chr., aber βύβλος auch schon in Herodots Hdschr. überwiegend und in anderen
Bedeutungen als “Buch” ausschliessliche Form, so auch βύβλου Aesch. Suppl. 761,
βύβλους Hermipp. 63, 13 Kock. Ferner ist μόλυβδος attisch (Inschr.); aber
βόλιβος βόλιμος dorisch; μόλιβος hat Homer (Sophokl.) neben μολύβδαινα, wonach
Herodian (II, 551) μόλιβος und μόλυβδος will; nach Moeris ist μόλυβος hellenist.
für att. μόλυβδος. — Schwanken in Dialekten: αἰσιμνάτας megar. für αἰσυμνήτης,
Τινδαρίδαι lakon. Inschr., Ἐλευὕνια (Ἐλευσίνια) desgl., Ἄρταμις Ἱακυνθοτρόφος
auf Knidos nb. Ὑάκινθος in Lakonien.59) — Dagegen in der Zeit, wo der U-Laut des
υ noch feststand, war eher ein Schwanken nach ο hinüber; darum in der
Reduplikation zur Vermeidung des vollen Gleichlauts μορ-μύρ-ω, murmuro, πορφύρω,
κόκκυξ cucūlus. So auch Ὄλομπος auf einer (freilich auch sonst Fehler zeigenden)
Vase wohl chalkidischen Ursprungs C. J. G. 8412; bei den Chalkidiern und
überhaupt auf Euböa ist nach allem Anschein das υ wie bei den benachbarten
Böotern lange noch u gewesen.\footnote{}

\section*{II. Konsonanten\protect\footnote{}}
\addcontentsline{toc}{section}{II. Konsonanten}

\section*{A. Mutae}
\addcontentsline{toc}{section}{A. Mutae}

\section{(a) Die harten Mutae κ, π, τ}


\paragraph{} Dem k der verwandten Sprachen, als des Lateinischen, bezw. dem k
[kacute] <*> des Sanskrit entspricht im Griechischen a) meistenteils κ, als:
κύ-ων, sk. [cnull ]van, N. çvā, l. can-is, δείκ-νυμι, sk. di[cnull ]-āmi, l.
in-dĭc-o; b) wo im Sanskrit [kacute] (k, <*>), im Lateinischen qu = kv, pflegt
im Griechischen dem letzteren entsprechend (mit rückwirkender Assimilation) der
Lippenlaut π zu stehen, als: ἕπομαι (st. σέπομαι), sk: si-sa[kacute]-mi, l.
sequor, secutus, secundus, εἶπον (d. i. ἔϝειπον), ἔπος (ϝέπος), ὄψ, ὀπ-ός (ϝόψ),
sk. va[kacute]-mi, rede, va[kacute]-as, Wort, l. vōx, vōc-is, vŏc-are; s. indes
über den Wechsel des π u. κ in den Dialekten § 28, a); — c) im gleichen Falle
vor ε ι der Zahnlaut τ, als: τέ, sk. [kacute]a, l. que, τίς, τί u. τὶς, τὶ, sk.
na-kis, Niemand, l. quis, quid, πέντε, pañ[kacute]an, l. quinque, τέτταρες, sk.
[kacute]átvāras, l. quattuor; doch ist hier vollends fast überall in den
Dialekten Schwanken, als: πέμπε, πέτταρες, s. das.; — d) zuweilen ist im Inlaute
k in γ erweicht, als: μείγνυμι, μίσγω, A. P. ἐμιγην, sk. mi[cnull ]rajāmi, l.
misc-eo, πήγ-νυμι, sk. pā[cnull ]-ajāmi, binde, l. pac-iscor, pāx, pāc-is,
τήγ-ανον, Schmelztiegel, v. τήκ-ω; desgl. im Anlaute vor ν: γνόφος neben κνέφας,
γναφεύς neben κναφεύς. Vgl. über die Dialekte § 30.

\paragraph{} Das p der verwandten Sprachen erscheint im Griechischen fast
durchweg als π, als: ἕρπω (st. σέρπω), sk. sarp-āmi, l. serpo, πόσις (st.
πότις), sk. patis, l. pot-is, pot-ens, πατήρ sk. pitā (St. pitar), l. pater.
Über die Dialekte s. § 32.

\paragraph{}Desgleichen entspricht dem t des Sanskr., Latein. u. s. w. fast
durchweg τ, als: τείνω ([root ] τεν) sk. tan-ō-mi, l. ten-do, στρών-νυμι, sk.
stṛ-ṇōmi, ἵ-στη-μι (st. σί-στη-μι), stelle, sk. ti-sh[tnull ]hā-mi, stehe, l.
sto, si-sto; ganz vereinzelt sind Erweichungen des τ zu δ, als in dem Fremdworte
δάπις neben dem älteren τάπις, Teppich, beides b. Xenoph., δάπιδας auch Hermipp.
com. Kock fr. 63, 23 (I, p. 243), dagegen τάπης Hom.; ἕβδομος u. ὄγδοος neben
ἑπτά, ὀκτώ, l. septimus, octavus; die Erweichung ist hier dem urspr. unmittelbar
auf die Mutae folgenden μ, ϝ zuzuschreiben, s. § 181, 3. Umfangreicher aber ist
die Assimilierung des τ zu ς vor ι (υ), besonders im Ionismus und Atticismus,
als: πόσις, sk. pátis, δίδωσι, dor. δίδωτι. S. § 31.

\section{(b) Die weichen Mutae γ, β, δ}


\paragraph{} Bei g ist das Verhältnis der Sprachen ähnlich wie bei k, d. h. es
entspricht dem g ǵ des Sanskrit a) gr. γ, als γένος, sk. gánus, l. genus, γεύω,
sk. ǵush-āmi, l. gusto; — b) es ist aus gv vielfach β hervorgegangen, als:
βαίνω, ἔβην, sk. ǵi-gā-mi, A. a-gām, lat. venio für gvenio; βαρύς, sk. gurus, l.
grav-is; βοῦς, sk. gāus, hier auch lat. bos; βρέφος (τό), sk. garbh-as (masc.);
βίος, βίοτος, sk. ǵîvas, ǵîvathas, Leben, lat. vîvus, lebendig, für gvivus; über
die Dialekte s. § 28 b); c) vereinzelt auch δ vor hellem Laute, als: δελφύς
(uterus), sk. garbhas (Mutterleib); dazu vor ν in δνόφος neben γνόφος. Vgl. über
die Dialekte § 28 b. 
\paragraph{} B als allgemein indogermanischer Laut ist selten; im Latein. indes
entspricht griechischem β oft b, als in βραχύς, l. brevis, βληχάομαι, l.
blactero, d. blöke, ὄμβρος, l. imber. 

\paragraph{} Sanskr. lat. d = gr. δ, als: δί-δω-μι, sk. da-dā-mi, l. do,
δάμνημι, sk. dām-jāmi, l. domo, ἔδω (ἐσθίω), sk. admi, l. edo.

\section{(c) Die gehauchten Mutae χ, φ, θ}


\paragraph{} Den weichen Aspiraten des Sanskrit: gh, bh, dh entsprechen im
Griechischen die harten: χ, φ, θ, d. h. kh, ph, th, wie man deutlich aus der
Reduplikation sieht, als: κέ-χυκα, πέ-φυκα, τέ-θυκα, ferner aus θρέψω neben
τρέφω, τροφή, aus dem Ionischen ἐνθαῦτα, κιθών neben ἐνταῦθα, χιτών, aus ἀφʼ οὗ
st. ἀπʼ οὗ, aus d. Lat. Poenus, Pilemo, purpura u. Φοῖνιξ, Φιλήμων, πορφυρᾶ
(vgl. Curtius, Et.\textsuperscript{5} 415 f.); vgl. sk. bhû = φῦναι, Perf.
ba-bhû-va (nicht pa-bhû-va wie im Gr. πέ-φῦκα). Beispiele: ἐ-λαχ-ύς, klein,
ἐ-λάχ-ιστος, sk. lagh-us (leicht), lagh-iš[tnull ]has, χοῖρος, Ferkel, sk.
ghṛshvis, Schwein; νέφος Wolke, sk. nabhas, Luft; φόβος, sk. bhaj-am, φρᾶτήρ,
sk. bhrātṛ Nom. bhrātā, l. frater; ἄνθος, Keim, Blume, sk. an-dhas, Kraut, Grün,
τί-θη-μι, sk. da-dhā-mi. Dem griechischen χ entspricht im Sanskrit ausser gh
auch h, im Lateinischen im Anlaut und zuweilen im Inlaut desgl. h, im Inlaut
meistens g, als: χθές, sk. hjas, l. heri, χείρ, sk. har-aṇam (Hand), altlat. hir
b. Lucil., χαίρω, sk. harj-âmi (amo, desidero), ὀχέω (Fοχέω), sk. vah-āmi, l.
veho; ἔγχελυς, lat. anguilla, λείχω, sk. [root ] lih, l. lingo. Für φ hat das
Lateinische im Anlaut f, als φεύγω, fugio, im Inlaut b, als ἄμφω, ambo; für θ
anlautend ebenfalls f, als θυμός, sk. dhūmas (Rauch), lat. fūmus, inlautend
wieder die Media d, als μέσσος st. μέθjος, sk. madhjas, lat. medius, oder b, als
ἐλεύθερος, liber, οὖθαρ, uber. 

\paragraph{} Übergang von χ (χϝ) in φ findet sich in dem Akk. νίφ-α, Schnee,
νιφάς, Schneeflockè, νιφετός, Schneegestöber, νείφει, schneit, l. ninguit, nix
(St. niv st. nigv); vor hellem Vokale in θ: θέρομαι, θέρος, θερμός, θέρμη, sk.
ġi-ghar-mi, leuchte, ghar-mas, Glut. Über den Wechsel der Aspiraten in den
Dialekten § 28 c.

\section{B. Liquidae ρ und λ}

\paragraph{} Die beiden Liquidae ρ und λ, welche offenbar nah mit einander
verwandt sind, schwanken häufig unter einander. In den bei weitem meisten Fällen
entspricht indes dem r des Sanskrit das griechische ρ, dem l jener Sprache das
griechische λ; doch gibt es auch nicht wenige Beispiele, wo dem sanskr. r
griech. (lat.) λ (l) gegenübersteht; dazu auch solche, wo das Griechische selber
zwischen ρ und λ schwankt. S. Giese, Aeol. Dial. 276 ff.; Curtius,
Et.\textsuperscript{5}, 554. Z. B. ἐρέσσειν, sk. ar-i-tras (Ruder), l. rēmus,
φέρῳ, sk. bharâmi, l. fero, goth. baira; — λύω, sk. lu-nāmi (seca, disseco), l.
re-luo, löse wieder ein, so-lvo (st. se-luo, solū-tus), goth. lau-sja, löse,
μέλᾶς, sk. malas, schmutzig, schwarz, l. mălus; — aber sk. r = griech. l z. B.
λευκός, weiss, λύχνος, Leuchte, l. luceo, sk. rō[kacute]-ē, leuchte, ru[kacute],
Glanz, πολύς, πλεῖον, l. plus, sk. purus, viel, κλύω, lat. cluo, sk. [root ]
[cnull ]ru, πίμπλημι, l. pleo, sk. piparmi; — ρ und λ schwanken im Griechischen
in einigen Wörtern, als: ῥάκος u. λάκος, Fetzen, κόρυμβος u. κολοφών, Gipfel,
αἱρέω (dial. auch mit λ) u. εἶλον, ἐρέβινθοι, Kichererbsen, u. λεβίνθιοι
(Hesych.), βρύω u. βλύω, ἀρκέω, l. arceo, u. ἀλ-αλκεῖν, ἀλκή; κρίβανος u.
κλίβανος (Lobeck ad Phryn. p. 179, Rutherford, Phryn, p. 267 f.); dazu
γλώσσαργος u. γλώσσαλγος, κεφαλαργία u. κεφαλαλγία u. s. w., § 67, 4. — Über den
Wechsel des ρ mit λ, des λ mit ν in den Dialekten s. § 29 a.

\section{C Nasale ν und μ}


{\noindent\small Vorbemerk. Der Nasal ν geht vor einem Kehllaute in γ über, also γκ = nk, γγ =
ng, γχ = nch, vor einem Lippenlaute in μ.}

\paragraph{} Das n der verwandten Sprachen findet sich im Griechischen im
allgemeinen als ν wieder, so: ναυς, sk. nāus, l. navis, ἀνήρ, sk. naras (Mann,
Mensch), ἐννέα (ἐννέFα), sk. navan, l. novem, ὄνυξ, G. -χος, sk. nakhas, l.
unguis, d. Nagel; in λ ist ν anscheinend übergegangen in att. πλεύμων (auch
Hippokr. VI, 374 nach cod. θ; dor. Inschr. Epidauros), wofür πνεύμων in der
κοινή u. b. Hom. Il. δ, 528 (υ, 486), wo indes nach Photius πλεύμονι;62) da auch
im Lat. (pulmo) und im Slavischen l erscheint, so kann auch πνεύμων aus πλ.
durch Anlehnung an πνέω entstanden sein (L. Meyer I\textsuperscript{2}, 129);
über λίτρον st. νίτρον s. die Dialekte § 29 a; über den Übergang des ν in α s. §
68, 4. 

\paragraph{} Ebenso hat sich im Griechischen grösstenteils das m der verwandten
Sprachen als μ erhalten, als: μή, sk. mā, μέσσος, μέσος, sk. madhjas, l. medius,
μήτηρ, dor. μάτηρ, sk. mātā (St. matar), l. mater, ἅμα, sk. samā, l. simul, ἐμέω
(Fεμέω), sk. vam-āmi, l. vomo. Im Inlaut ist es in wenigen Wörtern (vor j) in ν
übergegangen, als: βαίνω (st. βαν-jω), sk. gam; s. Curtius,
Et.\textsuperscript{5} 534 ff., G. Meyer, 185\textsuperscript{2}; aber als
Auslaut immer, als: τόν st. τόμ, sk. tam, ἔ-φερον, sk. á-bharam, l. ferebam,
ἀγρόν, sk. aǵram, l. agrum. Über das dor. νίν st. μίν s. § 29 a, über den
Wechsel von μ, π, β im Aeol. u. Dor. § 32.

\section*{D. Spirans ς, Halbvokale ϝ, j}
\addcontentsline{toc}{section}{D. Spirans ς, Halbvokale ϝ, j}


\section[(a) Spirans ς]{(a) Spirans ς\protect\footnote{}}


\paragraph{} Wesentlich unterscheidend ist für das Griechische im Verhältnis zu den verwandten Sprachen die Behandlung des Spirans ς und der Halbvokale v und j. Jene ist in starkem Masse beseitigt, und zwar schon in vorhistorischer Zeit; das j völlig in der gleichen Zeit; das v ϝ sehen wir auch aus den Dialekten, die es länger als das Attische und Ionische hatten, sichtlich mehr und mehr verschwinden. 

Das ς hat sich hauptsächlich in zwei Fällen regelmässig verwandelt oder verflüchtigt: im Anlaut vor Vokal und im Inlaut zwischen Vokalen. 

a) Anlautendes ς vor Vokal hat sich fast nie erhalten: σῦς neben ὗς, Σάλμων u.
Ἄλμων, Σάλμος u. Ἄλμος (St. in Böotien), σοφός σαφής, l. sapiens; in der Regel
ist es in den Spiritus asper übergegangen, als: ἅμα, sk. samas (ähnlich), goth.
sama (derselbe), d. samt; ἥμι-συς, sk. sâmi-, l. sēmi-, ahd. sâmi-(halb); ὁδός
[root ] ἑδ, σεδ, sk. sād-ajâmi gehe hinzu; ἑζόμην ἵζω [root ] ἑδ, σεδ, sk.
sîdâmi, l. sedeo, d. sitzen; ὕπνος, sk. svapnas (also gr. entspr, mit sva
zunächst συ), l. somnus; ἅλλομαι, l. salio; ἅλς, sk. saras, l. sal, d. Salz;
ὕλη, l. silva; ὕραξ (υ), lat. sōrex (auch hier sva = συ); ὗς neben σῦς, l. sus,
d. Sau; ἕρπω, sk. sarpâmi, l. serpo; ἑπτά, sk. saptan, l. septem; ἕβδομος, sk.
saptamas, l. septimus; ἕπομαι, sk. sisa[kacute]mi, l. sequor; ἵστημι, l. sisto;
desgleichen zum Teil ς mit folgendem ϝ: ἱδρώς, sk. svidâmi, schwitze, l. sūdor
(aus svoidōs od. sveidōs, L. Meyer), d. Schweiss; ἁνδάνω, ἥδομαι, sk. svad-âmi,
koste, gefalle, Med. svâdê, gefalle, ἡδύς, sk. svâdus, l. suāvis a. svadv-is;
οὗ, οἷ, ἕ (σϝοῦ, σϝοῖ, σϝέ), ὅς, ἑός (σϝός, σεϝός), sk. sva- (selbst), svas
(eigen), l. sui, sibi, se, suus (aus sevos sovos), ἑκυρός, sk. çvaçuras, l.
sŏcer a. svocer, goth. svaihra; ἕξ dor. ϝέξ, sk. [sbreve]a[sbreve], sex, sechs;
ἕκτος, sk. šaš[tnull ]has, l. sextus. In anderen Fällen ist indes von σϝ das ς
geblieben: σιγᾶν, ahd. swîgên; man sucht die meisten mit ς und Vokal anlautenden
Wörter auf den Anlaut σϝ zurückzuführen, als: σάττω, σήπω, σίδηρος, σίνομαι (G.
Meyer, Gr.\textsuperscript{2} 220 f.). Das Kyprische ging nach dem Zeugnis der
Glossen in der Verwandlung des anlaut. ς in h noch über die Gemeinsprache
hinaus.64) — Verflüchtigung auch des Spir. asper ist in einer Reihe von Wörtern,
die z. T. dialektisch sind, eingetreten: ὀπός, sucus, ahd. saf, nhd. Saft,
οὖλος, e p. st. ὅλος, sk. sarvas, altl. sollus, ganz; ὀρός, ὁ, die Molken, l.
serum, ἐτεός kypr. ἐτεϝός, vgl. ἔτυμος, ἐτήτυμος, sk. satjas, wahr; εἴρω,
knüpfe, neben εἱρμός, ὅρμος, σειρά ([root ] σϝερ? Curtius,
Et.\textsuperscript{5} 353 f.) über das kopul. ἀ oder ὀ st ἁ oder ὁ (sk. sa,
sam) s. § 44. 

b) Inlautendes ς zwischen Vokalen ist meist innerhalb des Griechischen aus τ neu
entwickelt, als πόσις (§ 10, 3), τίθησι dor. τίθητι, oder aus σς vereinfacht,
als γένες (ς) ι ἴς (ς) ος μές (ς) ος τός (ς) ος; unklarer Herkunft sind νόσος,
νῆσος, μισεῖν. Doch behauptet sich ς in der Deklination (Dat. Plur. τῇσι τοῖσι)
und besonders in der Konjugation: 2 sg. Med. σαι bei den Verba auf μι und in den
Perfecta; desgl. 2 sg. act. σι im Dorischen bei den Verba auf μι: τίθησι 2.
Person; ferner σο im Impf. Plusqu. analog dem σαι; σα im Aorist auch nach Vokal,
als ἐνίκησα ἐμίσθωσα; desgl. σω im Futurum: νικήσω, μισθώσω. (Eine einheimische
Nebenform Ὑσάμπολις für Ὑάμπολις wird Hdn. II, 35 angeführt.) In der grossen
Masse der Fälle aber hat es sich verflüchtigt, worauf vielfach Kontraktion der
nun zusammenstossenden Vokale eingetreten ist: Konjugation λέγῃ aus λέγε (ς) αι,
ἐλέγου aus ἐλέγε (ς) ο, so in der Masse der (barytonen und perispomenierten)
Verba; auch im Futur ohne ς νεμῶ, μενῶ aus νεμέ (ς) ω, μενέ (ς) ω, κομιῶ, τελῶ,
σκεδῶ § 228; Deklin. Gen. ἀγροῖο st. ἀγρόςjο, sk. aǵrasja; μῦς μυός. l. mus,
muris st. musis; γένος, G. γένεος, sk. manas, G. manasas, l. generis, st.
genesis; ἀληθής, ἀληθέος, dazu ἀλήθεια aus ἀληθεςjα; ferner (ς) ἕρπω, Impf.
ἔἑρπον, εἷρπον, (ς) ἕπομαι, Impf. ἐἑπόμην, εἱπὀμην; ἦα, ἦ, sk. âsam, l. eram,
εἴην st. ἔςjην; ἠώς, ἕως, äol. αὔως, sk. ušas, l. aurora; ἰός Gift, l. virus,
ἔαρ Frühling, sk. vasantas; νυός Schwiegertochter, l. nurus, d. Schnur u. s. w.
Die Massenhaftigkeit dieser Verflüchtigung des inlautenden ς beweist, wie sehr
die Scheu vor dem σὰν κίβδαλον (Pindar fr. 79 A Byk.) den Griechen im Gefühle
lag; es haben also auch die Musiker nicht aus blosser Willkür das ς gescholten
und gemieden, in dem Grade, dass einige Dichter (Lasos) lyrische Gedichte ohne
ein einziges ς verfertigten (Athen. X. 455,b—d; Eustath. Il. 1335, 52; Dionys.
Comp. verb. p. 80 sq. R.), und dass Dionysios nach musischen Quellen über den
euphonischen Charakter des ς so urteilt: ἄχαρι δὲ καὶ ἀηδὲς τὸ ς, καὶ εἰ
πλεονάσειε, σφόδρα λυπεῖ. θηριώδους γὰρ καὶ ἀλόγου μᾶλλον ἢ λογικῆς ἐφάπτεσθαι
δοκεῖ φωνῆς ὁ συριγμός. So sind denn auch, wie wir § 23, 2 sehen werden,
einzelne Dialekte noch weiter als die Gemeinsprache in der Tilgung des
intervokalischen ς gegangen. 

\paragraph{} Auch anlautendes ς vor Konsonant hat wenigsens starke Einbusse
erlitten. Stets fällt es ab vor ρ und ν,65) als: ῥέω (ῥέϝω), sk. sravâmi; ῥεῦμα
vgl. sk. srôtas, d. Strom (ahd. stroum); ῥοφέω, vgl. sorbeo; νέω, νήχομαι,
schwimme, sk. snaûmi fliesse, νίφα (Akkus.) νείφει Schneegestöber, es schneit,
goth. snaiv-s, a h d. sneo sniwit; νυός, sk. snušâ, ahd. snŭr, jetzt Schnur;
νευρά [root ] σνυρ, vgl. ahd. snuor, Schnur. Auch σλ kommt als Anlaut nirgends
vor. Aus der homerischen Prosodie, welche anlautendem ρ λ ν (auch μ) vielfach
Positionskraft verleiht, haben Viele wohl nicht mit Recht gefolgert, dass die
vor der Liquida verschwundenen Konsonanten wie ς hier noch eine Wirkung
ausübten, vgl. § 75, 12. — Der Anlaut σμ (gespr. zm, mit französ. z) kann
bleiben, schwankt aber sehr. Die Form mit und die ohne ς bestehen nebeneinander
in: σμικρός u. μικρός (σμικρ. Hom. Il. ρ, 757, dazu h. Ven. 115, sonst ep.
μικρός, neuionisch gew. σμικρός,66) oft auch bei den älteren Attikern, als den
Tragg. u. Plato;67) auf att. Inschriften erst einmal gefunden68); σμήρινθος Pl.
Leg. 1, 644, e (ubi v. Stallb.) u. μήρινθος; σμῖλαξ u. μῖλαξ (s. Schneider ad
Pl. Civ. 2, 372, b); σμάραγδος die gewöhnl. Form, auch bei Herodot, seltener
μάραγδος; ἐπισμυγερῶς Hom., σμυγερός Ap. Rh., σμογ. Gramm., gew. μογερός;
σμύραινα u. μύραινα; σμῦς Hesych. = μῦς; ἀπομύσσω μυκτήρ u. Hesych. σμυκτήρ
σμύσσεται, so auch σμύξων (Fischart) nb. μύξων; μύρον u. ἐσμυρισμένας Archil.
(ἐσμυριγμέναι Hesych.), σμύρνα u. μύρρα.69) Dauernd geblieben ist σμ in σμῆν,
σμῆνος, σμίλη σμινύη u. s. w. — Vor den Tenues und Aspiraten kann ς bleiben, ist
aber wiederum nicht selten abgefallen. Vgl. bei Homer σκίδνασθαι u. κίδνασθαι
Il. π, 375 u. ψ 226, σκέδασεν ρ, 749 u. κεδασθέντες β, 398; Σκάμανδρος u.
Κάμανδρος (davon äol. Namen wie Κάμων u. Καμμῦς zu Σκαμανδρώνυμος); ferner
nebeneinander σκάπτω u. σκάπετος κάπετος; σκερβόλλω Aristoph., σκερβολέω
κερβολέω Hesych.; σκαφώρη u. καφώρη (Fuchs), σκάρῖφος u. κάρφος; σκίμπτειν,
stützen, u. κίμψαντες = ἐρείσαντες, Hesych.; σκνίψ (σκίψ) u. κνίψ (eine
Ameisenart); σκαρδαμύσσειν u. καρδαμύσσειν b. Hesych.; — σχ u. χ, σχελυνάζειν u.
χελυνάζειν Hesych. (spotten); — σπ u. π, σπάνις u. πένομαι, πένης, πενία;
σπέλεθος u. πέλεθος, Kot; σπύραθος u. πύραθος, Mist; Πολυπέρχων äol. Inschr.
D.-I. 304 A = Πολυσπέρχων; σπυρός syrakusan. (u. Inschr. Kos Bullet. de corresp.
hell. V, 217) = πυρός, Weizen; Hesych. σπυρρούς = πυρρούς; — σφ und φ, Σφίγξ u.
böol. Φίξ (auch Hes. th. 326, s. Göttling ad h. l.; daher Φίκιον ὄρος, vgl.
Lobeck Paralip. p. 104); σφίν, lakon. φίν, ebenf. wohl lakon. φαιρίδδειν =
σφαιρίζειν u. φαιρωτήρ (Hes.); — στ und τ, στέγω (decke). στέγος, στέγη (Dach),
στεγανός (bedeckt), στεγνός (dicht), sk. sthagâmi (decke), u. τέγος, τέγη (tego,
d. decke); στυρβάζειν u. τυρβάζειν (turbare); στρύχνος u. τρύχνος; στρύζειν u.
τρύζειν; στριγμός u. τρίζειν.70) — Über das mit σπ π wechselnde ψ πτ und das
analog mit σκ κ wechselnde ξ κτ s. § 33. — Mit δ verschmilzt ς zu ζ = σδ, wofür
dialektisch vielfach δ δδ (§ 33); σβ findet sich nur in σβέννυμι anlautend, σγ
lautet überhaupt nicht an. 

\paragraph{} Über die Schicksale von ς mit Konsonant im Inlaut s. §§ 64, 5. 66,
3. Auslautendes ς ist im allgemeinen geblieben; Ausnahmen s. § 29.

\section{b) Halbvokal ϝ (§ 7)}

\paragraph{} Der aus der vorhistorischen Ursprache überkommene Halbvokal ϝ (§ 7, 2), nach seiner Gestalt später Digamma (Doppelgamma), dagegen von Haus aus gemäss seiner Aussprache (wie engl. w, lat. v § 3, 14, S. 59) Vau (geschr. nachmals Βαῦ) genannt, im alten Alphabete die sechste Stelle einnehmend (§ 2, 1), war ursprünglich ohne Zweifel bei allen griechischen Stämmen im Gebrauche. Da aber sein Laut dem griechischen Ohre und Munde unangenehm war, so wurde er von einigen Stämmen früher, von anderen später beseitigt, indem er entweder durch andere Konsonanten oder durch Vokale ersetzt oder ganz verdrängt wurde. Es wird diese Beseitigung des Vau mit dem Übergange von u in ü in einem gewissen Zusammenhange stehen; denn wenn das silbenbildende u zu ü wurde, so konnte das halbvokalische weder diesen Übergang mitmachen, ohne silbenbildend zu werden, noch als einzig vorhandenes U sich auf die Dauer behaupten. Wir sehen somit auch mehrfach, wie sich das Digamma da am zähesten hält, wo das υ seinen alten Laut bewahrt hatte, und umgekeh<*> da früh ausgegangen ist, wo auch das υ frühzeitig getrübt wurde. Letzteres ist bei den östlichen Ioniern (auch den Attikern) der Fall, und so ist bei ihnen auch das Vau frühzeitig beseitigt worden. Während es nämlich in den Homerischen Gesängen, wie wir § 17 sehen werden, noch im Gebrauche, freilich schon vielfach durch υ und im Anlaute durch ε ersetzt, in vielen Fällen auch aufgegeben war; ist es in der neuionischen Mundart des Herodot spurlos verschwunden, und hat auch bei den Lyrikern und Iambographen nur schwache Spuren hinterlassen.71) Dagegen kennen es, nach dem Zeugnisse ihrer Vasen, noch die chalkidischen Ionier Italiens, und selbst auf Naxos finden wir noch ein sicheres und ein mehr unsicheres Beispiel (ΑFΥΤΟ d. i. αὐτοῦ, C. I. Gr. 10, vgl. Kirchhoff, Gr. Alph.\textsuperscript{4}, 86; Fιφικαρτίδης? Bull. de corr. hell. 1888, 464). Bei den anderen Stämmen hält sich ϝ zwar länger, und bei den Böotern, sowie bei Italioten bis in die hellenistische Zeit, ja bei den Lakoniern ist der Laut v nie ausgegangen, wenn auch das Zeichen F aufgegeben wurde (s. unten 3 a, α); aber im allgemeinen sehen wir fast überall, dass das Digamma schon in alter Zeit nicht mehr in seiner vollen und unversehrten Kraft bestand, sondern teilweise bereits durch andere Laute ersetzt oder ganz weggelassen war. Den unversehrtesten Gebrauch des F zeigen uns die älteren kyprischen, sowie die altkorinthischen Inschriften. Bei den Lesbiern dagegen, nach welchen doch das Digamma das äolische heisst,72) wurde sein Gebrauch schon frühzeitig sehr schwankend; denn bei ihren Dichtern geht es häufig als Inlaut zwischen Vokalen in υ und vor ρ in β über, noch öfter verschwindet es im Inlaut, und auch im Anlaut wird es nach Bedarf des Verses bald gebraucht, bald weggelassen. Auf lesbischen Inschriften findet sich von ϝ keine Spur mehr. Bei den Böotern und einigen dorischen Stämmen behauptete sich das ϝ zwar länger und gleichmässiger; zuerst ging es als Inlaut, später als Anlaut verloren; aber schon bei Alkman finden wir es bisweilen vernachlässigt, mehr noch bei Epicharmus73); auch auf den Tafeln von Herakleia, die es noch kennen, ist es doch in sehr vielen Wörtern weggelassen. 

\paragraph{} Beispiele:\footnote{} a) Dor. bei Alkman ϝ überliefert oder nach der Überlieferung hergestellt fr. 99 Bgk. ϝά (= ἑά), 79 δάϝιον (= δήϊον), ϝάναξ; auf dem ägypt. Papyrusfragment (23 B.) col. I, 6 ϝάνακτα; an anderen Stellen ist das ϝ zwar nicht geschrieben, aber ausgesprochen worden, wie man teils aus dem Hiatus, teils aus dem Metrum sieht: Papyr. II, 24 τὸ εἶδος, III, 8 τε Ἰανθεμίς, frg. 51 ἐγώνγα ἄνασσα, 76 τὸ ἦρ, 31 ἔειξε wahrscheinlich ἔϝειξε v. ϝείκω, 69 ὅς ἕθεν (-υ-), 36 Κύπριδο̄ς ἕκατι; aber hie und da zeigt sich das Digamma erloschen: τοῦθ ἁδεᾶν fr. 37 (τοῦτο ϝαδ. ändert Bergk), τίς ποκα ῥᾴ 42, σιειδής d. i. θεο (ϝ) ειδής Papyr. III, 3. Inlautend als υ das. II, 29 αὐειρομέναι (˘ ¯ ˘ ˘ ¯). In den lakonischen Stellen von Aristophanes' Lysistrate kann man Digamma ziemlich durchführen, wiewohl es nie geschrieben ist (V. 1096 τὸ ἔσθος). — Bei Epicharmus fr. 19 Ahrens ἥκω οἴκαδις, 29 τῷ ἦρι, 60 χορδαί τε ἁδύ, 98 σάφα ἴσαμι u. dgl., 113 ἄγροθε̄ν ἔοικε. Zahlreiche Beispiele des Digamma bewahren die dorischen Inschriften, namentlich die älteren, während auf den jüngeren der Buchstabe entweder ganz fehlt, oder nur in wenigen Wörtern enthalten ist. So auf den herakleischen Tafeln in ϝέτος (aber inlautend das. πενταἑτηρίς), ϝίκατι od. ϝείκατι, ϝίδιος, 1, 109 ἐγϝηληθίωντι (= ἐξειληθῶσι von ϝηλίω = εἰλέω), ϝέξ u. Ableitungen (aus σϝέξ); dagegen ohne ϝ ἐργάζομαι, οἰκία ἐποίκια (doch ἐπιοικοδομά), ἕργω, ἀφέργω, ἐφέργω, συνἕργω, ἵσος od. ἴσος, ῥήτρα, ἄρρηκτος. S. Ahrens II, 42 f. Von altdorischen Inschriften haben die des korinthischen Dialekts das ϝ, auch das inlautende nach Konsonant und Vokal, am treuesten bewahrt, als Fεκάβα, Fίφιτος, Fιόλαϝος (Fιόλας), Λαϝοπτόλεμος, Πυρϝός, ἀμοιϝά (= ἀμοιβή), Αἴϝας, Δϝεινίας u. a. m. Korinth, πρόξενϝος, ὄρϝος (ὅρος), Ξενϝάρεος, ῥοϝαῖσι, ἀϝυτάν (missverständlich Τλασίαϝο in der epischen Genetivform) Korkyra. In Argos alt Διϝί, ἐποίϝηἑ; in Lakonien ναϝῶν noch Ende 5. Jahrh., Γαιαϝόχω Stele des Damonon; ΗΙΛΕFΟΙ ἱλήϝωι lakonisches Epigr. Olympia; Kreta (Gortyn. Tafeln) ϝήμα (von ἕννυμι), διαϝεῖπαι u. a., aber ausser ϝίσϝος (ἴσος) im eigentlichen Inlaute verschwunden; Mittelgriechenland αἰϝεί und κλέϝος, Altar von Krisa, ϝεϝαδηκότα (von ἁνδάνω), ϝασστός, ϝέκαστος, ϝότι Lokr. (im eigentlichen Inlaute auch hier nicht mehr, auch nicht in ιστία = Fεστία ἑστία). 

b) Böot. auf Inschr. ϝάστιος = ἄστεος, ϝέτος, ϝίκατι, ϝισοτέλιαν v. ϝίσος =
ἴσος, ϝοικία, ϝεϝυκονομειόντων = ᾠκονομηκότων u. a., als Inlaut noch ΠτωιΕϝι (=
Πτωϊεῖ Dativ) alt Bullet. de corr. hell. X, 191, nachmals im Kompos.
ϝικατιϝέτιες, auffällig auf jungen Inschr. ῥαψαϝυδός, αὐλαϝυδός = ῥαψῳδός,
αὐλῳδός. S. Ahrens I, 169 sq., Meister, Dial. I, 253 ff. Bei der Dichterin
Korinna, die den böot. Dialekt anwandte, scheint fr. 19 πῆδʼ ἑϝόν mit Beermann
aus πηδεγον herzustellen. Dagegen bei Pindar, der zwar ein Böoter war, aber
einen gemischten Dialekt gebrauchte, zeigt sich das Digamma geschrieben nie,
latent beständig nur bei dem Pron. οὗ, als O. 1, 23. 65; 6, 20 u. so an sehr
vielen Stellen; im Übrigen ist er im Gebrauche desselben unbeständig,75) wovon
der Grund in seiner gemischten Sprache liegt; denn keineswegs hat er vor den
digammierten Wörtern die Kürze einer konsonantisch auslautenden Silbe, noch den
Apostroph vermieden; dagegen verleiht er dem Digamma kaum irgend mehr
positionsbildende Kraft.\footnote{}

c) Lesbisch (mit leichter Corruptel in den Hdschr.) bei Aleäus fr. 39 ϝάδεα =
ἡδεῖα, bei A. 55 und S. 28 ϝείπην = εἰπεῖν; Apollon. bezeugt ϝέθεν, ϝοῖ, ϝός,
letzteres geschrieben in e. Frg. (Alkaios 50 Bgk.\textsuperscript{4}) in den
Vol. Herc. Ox. I, 122; in Balbillas äol. Gedichten steht für FΟΙ, FΕ ΓΟΙ, ΓΕ auf
dem Steine; vor e. Kons. ϝρῆξις einmal b. Alc. nach Tryphon πάθ. λέξ. § 11; an
einzelnen Dichterstellen sieht man aus dem unerlaubten Hiatus, dass ϝ im Anlaute
gestanden hat, als: A. 15, 7 ὑπὸ ἔργον, S. 2, 9 γλῶσσα ἔαγε. S. Ahrens I, p. 32,
Meister I, 103 ff. 

d) Thessalisch (Meister I, 300) wenige Beispiele: Fασίδαμος (St. ἡδ-), Δάϝων,
Fεκέδαμος, Κόρϝαι (thessal. nur nach Vermutung, s. Dial.-lnschr. 373). —
Arkadisch (ders. II, 103) ebenfalls nicht oft: Fασστυόχω, Fανακισίας u. a., im
Inlaut κάταρϝος Bull. de corr. hell. 1889, 281 von ἀρϝά, att. ἀρά. — Kyprisch
(ders. 242 ff.) in den Inschr. epichorischer Schrift noch sehr reichlich:
Νικοκλέϝης, Νεϝαγόρας, βασιλῆϝος, Διϝείθεμις, ΣαϝοκλέFης (= att. Σωκλῆς),
wiewohl auch hier in manchen Beispielen inlautendes Digamma fehlt; anlautendes
fehlt fast nirgends.

\paragraph{} In betreff der Änderungen, welche das ϝ erfahren, hat, sind folgende Fälle zu unterscheiden:

a) anlautendes ϝ.

α) es ist dafür der verwandte Lippenlaut β gesetzt, so bei den Lakedämoniern und
anderen Doriern77), z. B. Βορθαγόρας Argos, Röhl, J. Gr. ant. 30, βάννας
(italiot.) = ἄναξ, βάδομαι = ἥδομαι, βείκατι = εἴκοσι, βεκάς = ἑκάς, βέργον =
ἔργον, βεστόν od. βεττόν, vestis, βέτος = ἔτος, βιδεῖν = ἰδεῖν, βίως = ἴσως,
βοῖνος = οἶνος u. a. (Glossen), Βαστίας, Βιόλας u. a. (lakon. Inschr.), auch im
Inlaute Glossen ἀβείδω = ἀείδω, ἀβέλιος = ἀέλιος (ἥλιος), ἀβηδών = ἀηδών, ἀβώρ =
ἠώς, ἀκροβᾶσθαι = ἀκροᾶσθαι, θαβακός = θᾶκος, φάβος = φάος, ὤβεα = ᾠά, ova,
λαίβα, Schild, Kret. (v. d. linken Hand, vgl. l. laeva) u. a., Inschr.
Βολοεντίοι nb. Βολοντίοι, Ὀλοντίοι Kret., Φάβεννος Lakedämonier auf e. delph.
Inschr., Dittenberger, Syll. 189, in der Komposition lak. Inschr. Εὐρυβάνασσα
und (mit aus ευ entwickeltem ϝ) Εὐβάλκης; διαβειπάμενος kret. Inschr.; vor e.
Konson. nur in dem kret. Ortsnamen Βλισσήν = Λισσήν, Ὀλισσήν; (einige Glossen
haben anlautendes β, obwohl ihnen ϝ fremd ist, als: βαγός = ἀγός v. ἄγω,
βαλικιώτης Kret. = ἡλικιώτης). — Eleisch: Βηλεύς (richtiger Βαλ.) = Ἠλεύς, Βαδύ
= Ἡδύ, βοικία (Damokratesinschr.) u. a., Meister II, 47; — Lesb. vor ρ: bei
Sapph. βρόδον, βράκεα, βράδινος; b. Theokrit. βραϊδίως, b. d. Gramm. βρίζα, vgl.
Wurzel, βρύτις = ῥυτίς, βρύτηρ (cf. ϝερύω Hom.), βρᾴ = ῥᾴ (Alkm.) ῥέα,
Βραδάμανθυς, βρήτωρ; aber b. Alc. [s. 2, d)] ϝρῆξις. Es ist bei diesen
Schreibungen sehr schwer zu unterscheiden, was wirklicher Lautübergang und was
nur notdürftiger Schriftausdruck in Ermangelung des verlorenen Digammazeichens
ist; in letztere Klasse gehört sicher der arkad. Name Βασας (= Fασίας) Xen.
Anab. 4, 1, 18 (Meister II, 103). In manchen Dialekten scheint auch β
spirantischen Laut angenommen zu haben, so dass es von ϝ nicht weit abstand. 

β) ϝ wird μ. Der Übergang des ϝ in den Lippennasal erstreckt sich jedenfalls nur
auf eine kleine Anzahl von Wörtern, als: μάλευρον, Mehl, nach Curtius =
ϝάλευρον, ἄλευρον, [root ] ϝαλ, ἀλέω, mahle (doch zeigt ἀλέω keine Spur
anlautenden Digammas, dem auch schon die att. Reduplikation bei diesem Verbum
widerspricht); μαλλός, Zotte, l. villus?, μολπίς, Hesych. = ϝελπίς, ἐλπίς. In
anderen Beispielen, die man hierherzieht, ist dieser Übergang vollends schwierig
nachzuweisen, wie Curtius, Et\textsuperscript{5}. 591 ff. selbst gezeigt hat. 

γ) ϝ wird γ. Dies scheint aber eher ein Übergang in der Schrift zu sein: Gamma
statt Digamma, und aus blosser Unkunde hervorgegangen, gleichwie in den
Steininschriften der Balbilla (oben 2, d) durch Unkunde des Steinmetzen ΓΟΙ, ΓΕ
steht. So führt denn Hesychius eine nicht geringe Anzahl von Glossen, denen ϝ
zukommt, unter Γ und mit γ an, als: γάδεσθαι = ἥδεσθαι, γανδάνειν = ἁνδ., γακτός
v. Fάγνυμι, γάλι = ἅλις, γέαρ = ἔαρ, ver, γέμματα = ϝέμματα, εἵματα, γέτος =
ἔτος, γήθεα = ἤθη, γία = ἴα, γίο, γοί = οὗ, οἷ, γίς = ἴς, vis, γιστία = ἱστία,
ἑστία, γοῖδα = οἶδα, γοῖνος = οἶνος vinum, u. a., mit γ als Inlaut ἀγατᾶσθαι =
ἀϝατᾶσθαι = βλάπτεσθαι (vgl. αὐάτα d. i. ἀFάτα b. Pind. = ἄτη).78) Über das Hom.
γέντο s. § 19, Anm. 1. — Doch ist im Inlaute aus νϝ γγ in φέγγος geworden,
welches sich zu φάϝος verhält wie πένθος zu πάθος, βένθος zu βάθος u. s. w., und
auch im Anlaut vor ρ scheint γρῖνος (ϝρῖνος, ῥῖνος) Haut durch Herodian
bezeugt.

δ) ϝ wird Spiritus asper, doch nur selten, so tab. Heracl. I, 57. II, 35
πενταἑτηρίς neben Fέτος, desgl. ἕτος oft in der κοινή, s. § 22, Anm.; ferner
tab. Her. ἕργω (ἀφέργω, ἐφέργω, συνἕργω) I, 83. 85; ἵσος nb. ἴσος (ebenso in der
κοινή oft, s. § 22, 10); in der gewöhnlichen Sprache ἕσπερος, l. vesper, ἕν-νυμι
(aus ϝές-νυμι), sk. vas-man (Kleid), l. ves-tio, ἑκών, ἕκηλος, sk. va[cnull ]-mi
(will), ἑστία, l. Vesta, ἕρση (Tau), sk. varš-as (Regen), ἵστωρ, ἱστορία,
ἱστορεῖν v. [root ] ϝιδ, vid-ere, neben ἰδεῖν (weshalb auch einige alte
Grammatiker die Schreibung ἴστωρ vorzogen, s. Spitzner ad Il. ς, 591); aber σϝ
werden gewöhnlich (wenn nicht ς bleibt) Sp. asper (durch hw hindurch), als:
ἑκυρός, sk. [cnull ]va[cnull ]uras Kühners ausführl. Griech. Grammatik. I. T.
(st. svakuras), goth. svaihra, ἡδύς, sk. svâdus, l. suavis (schon Alkman 37
τοῦθʼ ἁδεᾶν), οὗ, οἷ, ἕ, St. sve-, ἱδρώς, St. svid (doch ἰδίω).

ε) ϝ verhärtet sich zu π in dem Namen Πάξος, den Skylax p. 19 für das kret.
Fάξος (s. unter ζ) bietet (Vossius korrigiert Ἀξός). Schreibfehler sind bei
Hesychius τηράνθεμον für ϝηρ., λαῖτα Schild für λαιϝά (oben α) u. s. w., Ahrens,
p. 56. (Auch φ steht anlautend für ϝ in λαῖφα ἀσπίς, ebenfalls Hesych.)

ζ) ϝ vokalisiert sich, was indes, ähnlich wie die Ersetzung durch β γ,
grossenteils Sache der Schreibung ist. So steht ο für ϝ in dem Namen Ὀαλίδιος zu
Eretria, d. i. Fαλίδιος Ἠλεῖος, in den kret. Ortsnamen Ὄαξος (b. Apoll. Rh. I,
1126 Οἰαξίς st. Ὀαξίς) aus Fάξος (von Fάγνυμι, St. Byz. s. v., Cobet, Misc. 355)
und Ὀλισσήν (nb. Βλισσήν, oben α), aus Fλισσήν; vgl. Fαξίων, Brief der Vaxier,
Ὀαξίοι, ätolisches Dekret, Bull. de corr. hell. VI, 460; vor ρ in Ζεὺς Ὀράτριος
auf Kreta, was doch = Fράτριος sein wird; υ schreibt eine kret. Inschrift in
ὔεργον, so Ὑέλη Velia, Hartel, Hom. Stud. III, 36; Bechtel, Inschr. des ion.
Dial. S. 106, nach welchem daraus hervorgeht, dass υ damals bei den Gründern der
Stadt, den Phokäern, damals noch u war. — Vgl. die spätere Wiedergabe des
lateinischen v durch (ο) ου. Eigentümlich der lakon. Ortsname Οἴτυλος (Il. β,
585) oder Βείτυλος (besser doch Βίτυλος), Ahrens, p. 46.

η) Dem anlautenden Digamma wird der (prothetische) Vokal ε (α) vorgeschlagen, hinter dem es selbst verschwindet. Vgl. § 19, 1. So bei Homer ἐέλδομαι, ἐέλπομαι, ἐέλδωρ, ἐέργω, ἐέλσαι, ἔεδνα, ἐείκοσι, ἐση, also vor ε, ει, ι (so auch Pind. einzeln: ἐέρσαν ἔειπε ἐειδόμενος); auch vor ρ in ἐρύω für ϝρύω; das ϝ konnte sich hier als υ halten, daher εὔληρα (αὔλ.) Zügel, vgl. lat. lorum, L. Meyer I, 1\textsuperscript{2}, S. 146, Εὔρυτος neben Ἔρυτος, Εὐρυσίλαος lesb. Inschr. nb. Ἐρύλαος, ἐρυσίπτολις u. s. w. (s. z. Dial.-Inschr. 3129). (Auch εὐρύς weit ist aus ἐϝρύς entwickelt, sk. uru aus vru, G. Meyer, Gr. S. 114\textsuperscript{2}.) Aus einem wirklichen Dialekte sind jene Homer. Formen mit εε noch nicht nachgewiesen (vgl. § 19, 1); dagegen steht α in kret. ἄερσα Hesych., in ἀείρω (Alkman αὐείρω), ἀέξω u. s. w.; s. über die prothetischen Vokale unten § 44. In der Regel ist anlautendes ϝ vor Vokalen wie vor Konsonanten spurlos verschwunden. 

b) inlautendes ϝ zwischen Vokalen. 

Der Prozess des Verschwindens ist hier allgemein viel rascher und gründlicher
vor sich gegangen als im Anlaut. Das verschwundene Digamma hat freilich gerade
im Attischen insofern eine Wirkung hinterlassen, als die beiden nun
zusammenstossenden Vokale sich schwerer durch Kontraktion vereinigen, vgl. ῥέϝω
πλέϝω = ῥέω πλέω, aber δέω “binde” wird δῶ, ὄγδοος octavus, ἐννέα novem, νέος
novus u. s. w. In Mundarten, die das ϝ länger gebrauchten, findet es sich
zwischen Vokalen in υ verwandelt, welches sich mit dem vorhergehenden Vokale zum
Diphthonge verbindet, so einzeln in der böotischen, als: βούων, bovum, βούεσσι,
bovibus, Ἀρχεναυΐδας v. ναῦς, navis, Ἄρευα Cor. 11 von Ἄρευς = Ἄρης; häufiger in
der lesbischen nach α, als: αὔηρ, ναῦος Tempel (dies auch Inschr.), φαῦος,
φαυοφόροι, αὔελλα, αὐΐδετος (alles dies Gramm.; die Fragm. von Sappho und Alkm.
liefern keine Belege als ναύω A. 9); ferner Ἄρευος Alc. wie böot., ἐνδεύη
δευομένοις Inschr., χεύω (ἔγχευε Alc. 41) θεύω ἐρεύω νεύω Gr., ἐπιπνεύοισα Alc.
66, εὐάλωκε Gr., εὐέθωκε = εἴωθε (Hesych.), εὔιδε (Balbilla). In αὔως aurora
(also urspr. αὔσως), viell. auch παραύα Wange (παρ-αύα, von παραυσια, eig. das
neben dem Ohre?), ist der Diphthong von Haus aus da; in πλεύω u. s. w. nehmen
dies Manche ebenfalls an, ich möchte indes meinen, dass urgr. wie sk. αυ vor
Vokalen αϝ gewesen wäre, also πλυ πλευ πλέϝω, indem sich das gew. πλέω πλόος aus
πλεύω πλοῦος schwer oder gar nicht erklärt. Aber der lesbische Dialekt ist weit
entfernt, dies αυ ευ durchzuführen: wir finden in den Fragm. der Dichter
ἐάνασσε, αείσω, φαος u. s. w. (Meister, Dial. I, 112) und die Variante ἔγχευε
neben ἔγχεε Alc. 41 zeigt durch das daktylische Mass, dass υ hier nur
graphischer Ausdruck für ϝ ist, wie in αυειρομέναι Alkm. 23, II, 29, αυάτα
Pind., der wohl selbst ἀ άτα schrieb. Jenes äolische αυ, ευ finden wir auch bei
Homer einzeln: αὐέρυσαν aus ἀ (ν) ϝέρ., αὐίαχοι ἀϝίαχοι, δεύω δεύομαι, ebenso
Hesiod καυάξαις f. κα (τ) ϝάξαις, καταχεύεται Op. 583. Vor ρ haben die Lesbier
den Diphthong in αὔρηκτος = ἄρρηκτος, εὐράγη ἐρράγη, Εὐρυσίλαος Inschr. f.
Ἐϝρυσίλαος, vgl. Homer ταλαύρινος aus ταλά-ϝρινος, ἀπούρας aus ἀπο-ϝρας, Εὔρυτος
nb. Ἔρυτος u. s. w. (oben a, η), und auf kypr. Inschr. ἐϝρητάσατυ neben
εὐϝρητάσατυ d. i. ὡμολόγησε von ϝρήτασθαι, zu ϝρήτα (ῥήτρα) ὁμολογία. Im
allgemeinen aber wiegt bei Homer die andere Weise vor, den Vokal vorher arbiträr
zu dehnen, was das Attische wenigstens bei α in den meisten Fällen regelmässig
thut: (Ἀϝίδης) αδης aber auch ϊδος, att. ιδης, αείδω und ἆείδω, att. δω, ἆΐσσω
(spätere Dichter auch αΐσσω), att. ᾄττω; man ist hiernach nicht berechtigt, mit
Hartel (Hom. Stud. III, 27 f.) Αὔιδος, αὐείδῃ u. dgl. als Homerisch anzusetzen.
In einzelnen Fällen ist die Dehnung auch durch zugesetztes ι ausgedrückt: οἴιες
Od. ι, 425 = ὄϝιες; κοίιλος? Alc. fr. 15, 5, wo Ahrens κώιλαι, Hdschr. κοῖλαι;
Mimnerm. 12, 6, Hdschr. ebenf. κοίλη; οἰέτεας Il. β, 765 (Hartel a. a. O. 31
f.), vgl. Οἰαξίς b. Apollon. oben a, ζ); nach ε in λείουσι Il. ε, 782. η, 256.
ο, 592 (das. 33 f.), sowie in πνείω (ἀποπνείοντ Tyrtae. 10, 24) θείω u. s. w.
(vgl. § 38, Anm. 4; § 231, Anm. 1). 

c) Inlautendes ϝ nach Konsonanten kann sich diesem assimilieren, doch wird die
Verdoppelung des Konsonanten nachmals selbst meistens beseitigt, mit oder ohne
Dehnung des vorhergehenden Vokals, als: ϝίσϝος gleich, äol. ἴσσος, Hom. ἶσος (G.
Meyer will ἴσσος), att. ἴσος; ξένϝος (korinth.), äol. ξέννος, ion. ξεῖνος, att.
dor. ξένος; ὅρϝος Grenze Korkyra (ὅρρος nicht mehr nachzuw.), ion. οὖρος, att.
ὅρος, dor. ὄρος; aber πυρϝός (korinth.) allgemein πυρρός, vgl. § 29, Anm. Ferner
γοῦνα δοῦρα für γόνϝα, δόρϝα, πολλοῦ doch für πολϝοῦ, u. s. w.; nach Muta
tarent. ἴκκος gew. ἵππος für ἴκϝος equus; vgl. den Namen eines päonischen
Fürsten Λύκκειος und Λύππειος (aus Λύκϝ.) Meisterhans, Gr. d. att. Inschr.
59\textsuperscript{2}; sonst steht für altes κϝ ππ, π, als ὅππως ὅπως, ion.
ὅκως, bei anlautendem κϝ natürlich einfacher Konson.: πῶς, ion. κῶς; vor hellem
Vokale ττ, τ: ὅττι, ὅτι, thessal. πόκκι; anlaut. τίς, thess. κίς. Die Verbindung
τϝ wird ττ att. böot., τ dor., σς (ς) gew.: att. τέτταρες, böot. πέτταρες, dor.
τέτορες, äol. πέσσυρες, hom. πίσυρες, ion. τέσσερες, vgl. sk. [kacute]atvâras,
lat. quattuor. Bei δϝ ist Assimilation in der hom. Schreibung ἀδδεές Il. θ 423
u. s. w., doch Aristarch ἀδεές, La Roche, Hom. Textkr. 178; Dehnung des
vorhergehenden Vokals in δείδοικα u. s. w. Hom., spurloser Ausfall gew.: δέδια
δεινός δίς für δϝίς (lat. bis) δισσός u. s. w.; auch δήν δηρόν s. § 19, Anm. 2;
Alkman sagte für δϝήν δοάν mit Vokalisierung, Bk. An. II, 949. Ein
eigentümlicher Lautübergang von τϝ, δϝ in τρ, δρ zeigt sich in Glossen bei
Hesych.: kret. τρέ für σέ (τϝέ), ohne nähere Angabe δεδροικώς f. δεδοικώς,
Ahrens II, p. 51. 

d) Inlautendes ϝ vor Konsonant (selten, und nur in der Komposition und
Ableitung) wird ziemlich analog dem anlautenden behandelt, vgl. oben b) εὐράγη,
gew. ἐρράγη, wo das ρρ als Ersatz für ϝρ gefasst werden kann, wiewohl richtiger
diese Verdoppelung des inlautend werdenden ῥ auf die Aussprache dieses
Konsonanten im Anlaut zurückgeführt wird. Wenigstens steht für inlautendes ρ mit
Kons. attisch nicht ρρ wie lesbisch, sondern ρ, z. B. φθείρω.

\paragraph{} Das Schwinden des Digamma auch in den Dialekten, die den Gebrauch
des ϝ bewahrt hatten (oben 1. 2), genügt es mit wenigen Beispielen zu belegen.
So lassen die Böotier das ϝ als Inlaut gewöhnlich weg, als: ἀΐδων Cor. 18, auf
Inschr. εὐεργέτας, Διΐ (eleisch Δί, kontrah. aus Διϝΐ, Δι Corp. Inscr. I, 29),
Δαμοκλεῖος, Τιμόλαος; dasselbe geschieht bei den lesbischen Dichtern, als: Δίος,
ὤϊον öfter Sapph., und sogar im Anlaut ἐπεμμένα (od. ἐπαμμένα? wie Alkm. 18)
Sapph. 70 b. Maxim. Tyr. XXIV, 9, ὦ ναξ Alc. 1, τὸ δ ἔργον 14, προσίδοισαν S.
69, οὐκ οἶδα 36, φάεννον εἶδος 3 u. s. w.; im Inlaut auch so, dass Kontraktion
eintritt, als: Ἄλιε Adesp. 61 Bgk. (aus Ἀέλιε), ἀλίω S. 69 (aus ἀελίω); sonst
st. ᾶϝ regelmässig blosses ᾱ als: νᾶος, νᾶϊ A. 19. 18; ἐξεκλάϊσε ἐλαΐζετο
Inschr., λᾶον (λαόν) Alc. 92 (Meister, Dial. I, 111 f.), ebenso α, ε oft st.
lesb. αυ, ευ (oben 2), als αείσω S. 11, αοίδαν Alc. 39, αήδων S. 39, αέρρει A.
78, ἐάνασσε A. 64. Auf lesbischen Inschriften findet sich keine Spur des ϝ.79) —
Bei dem dor. Dichter Alkman wird bisweilen ein Wort vor einem sonst digammierten
Worte apostrophiert, also das ϝ weggelassen, als: 117 οἶνον δ Οἰνουντιάδαν v.
ϝοῖνος, vinum, vor ρ 42 in ῥᾴ (ä o l. βρᾴ), vgl. oben 2a; auch bei Epicharmus
oft, als: ἔσθοντ ἴδῃς 18, πλατίον οἰκεῖ 72 u. s. w. — In der gewöhnlichen
Sprache z. B. ἡδεῖα st. σϝᾶδεϝ-ια, sk. svâdvî, ἐσθής, vestis, ἰδεῖν, videre,
οἶκος, sk. vê[kacute]as (Haus), l. vicus, goth. veihs (Dorf), εἴκω, weiche, ὄψ,
vox, ἔργον, Werk, ἐργάζομαι, wirke, ἔτος, sk. vatsas (Jahr), l. vetus, ῖτέα,
richtig εἰτέα, Weide, l. vitex, ἄστυ, sk. vâstu (Haus), l. Vesta, ἦρ, ἔαρ, vēr,
ἴον, viola, ἰός (Gift), sk. vish-as, l. virus, ἴς, vis, ἐμέω, sk. vam-âmi, l.
vomo, u. v. a.; vor ρ, als: ῥόδον st. ϝρόδον, ῥήγνυμι st. ϝρ.; hinter δ und ς
(δϝ, σϝ), als: δώδεκα neben δυώδεκα, δίς st. δϝίς, sk. dvis, l. bis, ἦδος, ἡδύς,
sk. svâdus, l. suavis, ἔθος, ἦθος, ἐθίζω [root ] σϝεθ, σάλος, Schwanken, ahd.
swellan, jetzt schwellen, σέλας, σελήνη, σείριος, sk. svar (Himmel), l. sôl,
goth. sauil, σιγή, σιγᾶν, mhd. swîgen, jetzt schweigen, σομφός, goth. svamms;80)
im Inlaute, als: οἶς, ovis, sk. avis, ᾠόν, ovum, κληΐς, dor. κλᾶἰς, clavis,
νέος, novus, sk. navas, σκαιός, scaevus, sk. savjas, δῖος, divus, sk. divjas,
αἰών, aevum, βοῦς, l. bôs, sk. gâus, βοός, bovis, sk. gav-as, δαήρ, sk. dêvâ (A.
dêvaram), l. lēvir, πνέω (st. πνέϝω, äol. πνεύω, F. πνεύσομαι), ῥέω (st. σρέϝω),
sk. srav-âmi, λεῖος, lēvis; nach einem Konsonanten, als: ξένος, dor. noch
ξένϝος, ὅρος, dor. noch ὅρϝος, ἴσος aus ϝίσϝος (kret.).

\section{ϝ in den Homerischen Gedichten}


\paragraph{} In den Homerischen Gedichten, wie sie uns überliefert sind, findet
sich keine Spur von dem Zeichen des Digamma. Auch erwähnen die alten Grammatiker
Nichts von dem Gebrauche des Vau bei Homer. Beide Umstände dürfen uns jedoch
nicht befremden. Denn diese ionisch verfassten Gedichte verloren das
geschriebene und das gesprochene Digamma in demselben Masse, wie der Dialekt es
verlor, d. i. sehr früh; die alexandrinischen Grammatiker hatten daher
selbstverständlich nur Exemplare ohne ϝ im Gebrauche und konnten somit auch
Nichts von dem Digamma bei Homer wissen.81) Dass aber Homer den Laut des Digamma
gekannt und angewendet hat, lässt sich jetzt schon von vorn herein daraus
vermuten, dass die Schwestersprachen diesen Laut besitzen, und daher derselbe
ohne Zweifel der Ursprache angehört hat, aus der die griechische Sprache
hervorgegangen ist. Hierzu treten aber noch thatsächliche Erscheinungen in den
Homerischen Gesängen, welche uns zu der Annahme dieses Lautes in derselben
nötigen. Wir bemerken vorweg, dass in der Massenhaftigkeit der betr.
Erscheinungen vor gewissen Wörtern das Nötigende liegt, indem vereinzelt
derartiges sich auch da findet, wo ein Konsonant nicht gestanden haben kann,
weswegen eben bei manchen, namentlich selteneren Wörtern für Zweifel genug Raum
bleibt. 

\paragraph{}Erstens: die Wörtchen καί, ἐπεί und alle Encliticae, die auf einen
Diphthongen ausgehen, als: οἱ, τοι, τευ, μοι u. s. w., in welchen Homer überall,
mit Ausnahme sehr weniger Stellen,82) vor einem Vokale den Diphthongen kurz
gebraucht, haben denselben vor einem digammierten Worte sowohl in der Hebung als
in der Senkung lang.83) Il. κ, 328 καί ϝοι ὄμοσσεν. μ, 407 χάζετʼ ἐπεί ϝοι θυμὸς
ἐϝέλπετο κῦδος ἀρέσθαι. δ, 17 πᾶσι φίλον καὶ ϝηδὺ γένοιτο. χ, 510 γυμνόν· ἀτάρ
τοι ϝείματʼ ἐνὶ μεγάροισι κέονται. α, 124 οὐδέ τί που ϝίδμεν. ς, 192 ἄλλου δ οὔ
τευ ϝοῖδα. β, 215 ἀλλʼ ὅτι ϝοι ϝείσαιτο. Ebenso andere Wörter auf αι, οι u. s.
f., gerade auch in der Senkung des Verses: Il. ω, 479 δεινὰς ἀνδροφόνους, αἵ ϝοι
πολέας κτάνον υἷας. Od. ε, 106 τῶν ἀνδρῶν, οἳ ϝάστυ πέρι Πριάμοιο μάχοντο.

\paragraph{}Zweitens: während die Verlängerung einer kurzen konsonantisch
auslautenden Endsilbe ausser vor der männlichen Cäsur des III. Fusses nur sehr
selten stattfindet, tritt sie öfter vor einem digammierten Worte ein. Il. ι, 284
γαμβρός κέν ϝοι ἔοις. ε, 836 χειρὶ πάλιν ϝερύσασ. ι, 56 οὐδὲ πάλιν ϝερέει. — ψ,
298 ἀλλʼ αὐτοῦ τέρποιτο μένων· μέγα γάρ ϝοι ἔδωκεν. ω, 583 νόσφιν ἀειράσας, ὡς
μὴ Πρίαμος ϝίδοι υἱόν. γ, 372 ὃς ϝοι ὑπʼ ἀνθερεῶνος. ζ, 351 ὃς ϝῄδη (besser
ϝείδη). ι, 147 πρὸς ϝοῖκον. In der Senkung findet die Verlängerung nur vor dem
Pron. ϝέο und vor Formen der Wurzel ϝιδ statt, s. § 19. Ferner: die Verlängerung
einer kurzen vokalisch auslautenden Endsilbe vor einem folgenden Vokale findet
in der Senkung nie und in der Hebung höchst selten, vor den digammierten Wörtern
ἕο, ἕθεν, οἷ, ὅς, ἑκυρός, ἰκέλη, ἰακή, ἰάχων hingegen an einigen Stellen statt,
als: ἀπ ϝέο Il. ε, 343, ἀπ ϝέθεν ζ, 62, προτ ϝοῖ φ, 507, τ ϝοι χ, 307, πόσεῗ ϝῷ
ε, 71, θυγατέρᾶ ϝήν ε, 371, ἐπίσταιτο̄ ϝῇσι φρεσίν ξ, 92, οὐδ ϝοὺς παῖδας β, 832,
φίλε̄ ϝεκυρέ γ, 172, ἀνδρ ϝικέλη δ, 86, γένετο̄ ἰαχή δ, 456.84)

\paragraph{}Drittens: die unerlaubten Hiatus85) werden durch die digammierten
Wörter aufgehoben. Il. ζ, 203 Fίσανδρον δέ ϝοι υἱὸν Ἄρης ἆτος πολέμοιο. ω, 778
ἄξετε νῦν, Τρῶες, ξύλα ϝάστυδε, μηδέ τι θυμῷ. β, 803 πολλοὶ γὰρ κατὰ ϝάστυ. ζ,
505 ἀνὰ ϝάστυ. β, 261 εἰ μὴ ἐγώ σε λαβὼν ἀπὸ μὲν φίλα ϝείματα δύσω. α, 85
θαρσήσας μάλα ϝειπέ. β, 38 νήπιος, οὐδὲ τὰ ϝῄδη (ϝείδη), ἅ ῥα Ζεὺς μήδετο ϝέργα.
υ, 122 δευέσθω, ἵνα ϝείδῃ. 
\paragraph{}Viertens: οὐ vor einem Vokale statt οὐκ
(οὐχ), nur bei dem Pron. der 3. Person. Il. β, 392 οὔ ϝοι. α, 114 οὔ ϝεθεν. ω,
214 οὔ ϝε. 

\paragraph{} Fünftens: das syllabische Augment vor einem Vokale zeigt das ϝ an,
als: ἔαξε, d. i. ἔϝαξε, κατέϝαξε v. ϝάγνυμι, ἔειπας d. i. ἔϝειπας, ἑέσσατο d. i.
ἐϝέσσατο v. ϝέν-νυμι; die Reduplikation im Pf. u. Plusq., als: ϝέϝολπα, ϝέϝοικε,
ϝέϝοργα. Das ν ἐφελκυστικὸν fällt weg, als: δαῖέ ϝοι Il. ε, 4, οἵ κέ ϝε ι, 155,
ἐνὼ ϝιδέειν Il. ε, 475 (nicht ἐγών); die elisionsfähigen Vokale in Kompositionen
und bei Präpositionen erleiden keine Veränderung, als: κακοϝεργός, θεοϝειδής,
μενο εικής, κατὰ ϝάστυ, ἀνὰ ϝάστυ (nie κατʼ ἄστυ, ἀνʼ ἄστυ), μετὰ ϝέθνος Il. η,
115, ἀπὸ ϝῆς Il. β, 292, ὑπόϝειξιν, ἐπιϝάνδανε u. s. w.; statt ἀν- (α privat.)
tritt ἀ vor, als ἀεικής, ἀελπτέοντες, ἀαγής. 

\paragraph{}Dass auch mehrere Wörter, welche in unserem jetzigen Homerischen
Texte mit einem einfachen Konsonanten anlauten, zu Homers Zeit noch Digamma
hinter demselben hatten, werden wir § 19 sehen. 

{\small\noindent\subparagraph{} Über das Digamma bei Hesiod s. Rzach, hes.
Untersuch. (Prag 1875); Fl. Jahrb., Suppl. VIII, 377; Flach, D. dial. Dig. b.
Hesiod, Berl. 1876; über das Dig. in den späteren epischen Dichtungen (Homer.
Hymnen u. s. w.) Flach in Bezzenb. Btr. II, 1 ff.}

\section[Alphabetisches Verzeichnis der digammierten Wörter bei
  Homer]{Alphabetisches Verzeichnis der digammierten Wörter bei
  Homer\protect\footnote{}}\label{sec:digammahomer}


{\small\noindent Vorbemerk. Die abgeleiteten Wörter sind unter die Stammwörter
gestellt.}

ἄγ-νυμι, Hiatus Il. ε, 161, θ, 403, ψ, 341, 467. π, 769. Α. ἔαξα (ἔ-ϝαξα), Pf.
Hes. op. 534 οὔτʼ ἐπὶ νῶτα ἔαγε (ϝέϝαγε), Sapph. 2, 9 γλῶσσα ἔαγε (aber Il. ψ,
392 ἵππειον δέ ϝοι ἦξε δέ ϝ ἔαξε van Leeuwen Mnem. N. S. XIII, 197] θεὰ ζυγόν,
Od. τ, 539 πᾶσι κατʼ αὐχένας ἦξε, wofür Herodian II, 921 αὐχένʼ ἔηξε, daher Bk.
αὐχένʼ ἔϝαξε); d. Kompos. ἀϝαγής. Aber ἀκτή zeigt kein ϝ, während im ion.
κυματωγή b. Hdt. aus κυματοαγή noch eine Nachwirkung desselben zu erkennen
scheint; dazu κατξω aus κατα (ϝ) άξω u. s. w., § 343. 

[αἴνυμαι, nehme: nur die Form ἀποαίνυμαι, neben der jedoch auch ἀπαίνυμαι
vorkommt, scheint auf Dig. zu weisen; doch s. § 42, 2, b.] 

[αἱρέω, ebenso: ἀποαιρεῖσθαι Il. α, 230, ἀποαίρεο 275, aber ἀφαιρεῖται 182. Auch
die Dialekte ohne ϝ.] 

ἅλις, reichlich, gedrängt, zu [root ] ϝελ (unten εἴλω), Hiat. Od. ν, 136 u. s.
w.; εἰνάτερε̄ς ἅλις V. F. Il. χ, 473; Il. φ, 236 ist v. l. κατʼ αὐτὸν ἅλις ἔσαν
und κατʼ αὐτὸν ἔσαν ϝάλις; nur ersteres ist überl. in der Wiederholung des V.
344; ρ, 54 ὅθʼ ἅλις will Bentl. ὃ ϝάλις lesen. 

ἁλίσκομαι, mit ϝ Inschr. Stymphalos ϝαλόντοις = ἁλοῦσι, Meister, Dial. II, 103:
Hiat. Il. μ, 172. ξ, 81. φ, 281; Α. ἐϝάλων (über Il. ε, 487 λίνου ἁλόντε mit ᾱ
s. § 343 unter ἁλίσκομαι); vgl. das Lesb. εὐάλωκεν st. ἑάλωκεν, oben § 16, 3, b,
u. Hdt. 9.120 νεοάλωτοι (v. l. νεάλ.); ἀνᾶλίσκω aus ἀνα (ϝ) αλίσκω, Wackernagel
K. Z. XXV, 269. 

ἄναξ, ἄνασσα, ἀνάσσω, vgl. Eigenn. Fανάξανδρος Fαναξίων u. s. w., böot. Meister,
Dial. I, 253, argiv. ϝανάκοι Röhl J. G. A. 43 a: Hiat. in jedem Versfusse, vgl.
ἐ-άνασσε Alc. 64 (Bergk).

ἁνδάνω (d. i. σϝανδάνω, sk. svad-âmi, gusto, wie ἡδύς = suavis, sk. svâd-us,
dulcis, lokr. Inschr. ϝεϝαδηκότα, böot. Eigenn. Fαδιούλογος = Ἡδύλογος, Fάδων,
Fάσανδρος, Fασίας, Meister, Dial. I, 253), καί lang in der IV. Senkung Od. β,
114; über ἑήνδανον ἑάνδ. s. Anm. S. 97, aber εὔαδον d. i. ἔϝαδον Il. ξ, 340 u.
sonst; Il. ι, 173 u. Od. ς, 422: τοῖσι δὲ πᾶσι ϝεϝαδότα μῦθον ἔϝειπεν; ἡδύς, καί
lang in d. IV. Senk. Il. δ, 17, η, 387. Od. ν, 69, ῳ u. ου lang in d. IV. Senk.
Il. ψ, 784. Od. υ, 358. φ, 376. β, 340, in d. I. Od. γ, 391; Hiat. Il. λ, 378.

ἁραιός (so Aristarch st. ἀρ., La Roche, Hom. Textkr. 201), Hiat. im V. Fusse Il.
ε, 425. ς, 411 = υ, 37. 

ἀρήν, G. ἀρνός, Lamm, böot. Eigenn. Fάρνων, Meister I, 253, Hiat. im V. Fusse
Il. δ, 158. 435; θ, 131; δίφρο̄ν ἄρνας i. d. Hebung des III. F. γ, 103; λύκοῖ
ἄρνεσσι Hebung des II. F. π, 352. Aber ἀρνειός, Widder, zeigt von Dig. keine
Spur. Vgl. πολύρρην aus πολύϝρην, ὑπόρρηνος, § 122, Anm. 13.

ἄστυ, sk. [root ] vas, wohnen, vastu, Haus, böot. Eigenn. Fαστίας u. s. w.,
ϝαστός arkad., ϝασστός lokr. Inschr., lang καί in der IV. Senk. Il. ρ, 144,
μετάλλᾶ Od. τ, 190 in d. IV. Senk., οἵ Od. ε, 106 in d. II. Senk.; Hiat. oft
κατὰ ϝάστυ, ἀνὰ ϝάστυ. 

ἔαρ, ver, sk. vas-antas, Frühling, Od. τ, 519 καλὸν ἀείδῃσῖν ϝέαρος, vgl. Alkm.
fr. 76 τὸ ἦρ; dazu εἰαρινός Hiat. V. F. Il. θ, 307; böot. Eigenn. Fειαρινώ; in
ὥρῃ ἐν εἰαρινῇ Il. β, 271 u. s. tilgt Bentley ἐν nach Od. ε, 485. 

ἕδνα, Nbf. ἔεδνα, Hiat. im V. F. Il. π, 178. 190. χ, 472; καί lang in d. IV.
Senk. Od. ν, 378. 

ἔθειραι Hiat. im V. F. Il. π, 795. χ, 315; aber Elision τ, 382.

ἔθνος Hiat. Il. μ, 330. η, 115. ρ, 581. 680, u. s. w.

ἔθοντες v. ἔθω (σϝέθω, vgl. suesco). Il. π, 260 ἐριδμαίνωσι ϝέθοντες Bekk., vulgo ἐριδμαίνωσιν ἔθ., vgl. ι, 540; ἦθος unten; εὐέθωκα ob. § 16, 3, b.

εἶδον, εἶδος, εἴδωλον s. ἰδεῖν.

εἴκελος, s. ἔοικα.

εἴκοσι, böot. ϝίκατι, lak. βείκατι Hesych., ϝείκατι, ϝίκατι tab. Heracl., sk. v<*>[cnull ]ati, l. viginti, καί in Compositis lang Il. β, 510. 748 u. s. w.; häufig ἐείκοσι (Cobet, Misc. cr. 379).

εἴκω, vgl. weiche, καί lang in d. IV. Senk. Il. ω, 718, οἷ (οἱ) in d. II. Heb. Il. ν, 807. Od. χ, 91, αἰδοῖ in d. V. Senk. Il. κ, 238 (doch besser αἰδόϊ, also Hiat.); ὑποείκω sehr oft, ὑπείκω nur Il. α, 294, Od. μ, 117 (letztere St. leicht zu ändern); vgl. Alkman fr. 31 ἔειξε, d. i. ἔϝειξε.

εἰλύω, wickle ein, vgl. lat. volvo: σάκεσῖν εἰλυμένοι ω<>μους Od. ξ, 479, vgl. ε, 403. Il. α, 186. π, 640. ρ, 492, Od. ξ, 136; εἴλυμα Od. ζ, 179 u. εἰλυφάω Il. λ, 156 ohne Dig.; aber εἰλυφάζει mit Hiat. υ, 492.

εἴλω, dränge, vgl. καταϝΕλμένων f. καταϝεϝελμ. Gortyn. Tafeln, ἐγϝηληθίωντι ἐξειληθῶσι tab. Heracl., Il. υ, 278 ἐϝαλη, Il. ς, 287 κεκόρησθε ϝεϝελμένοι, ω, 662 ὡς κατὰ ϝάστυ ϝεϝέλμεθα; Hiat. Il. φ, 607. χ, 308 u. s. w. Od. ω, 538; aber Elision Il. ς, 294 θαλάσσῃ τʼ ἔλσαι Ἀχαιούς; so οὐλαμός, Gedränge, ἀνὰ ϝουλαμὸν ἀνδρῶν Il. δ, 273 u. sonst; ἅλις s. o.

εἶπον, [root ] ϝεπ, sk. va[kacute]-mi = sage, l. voc-are, Augm. ἔϝειπον; lang οἷ (οἱ) in d. II. Senk. Il. ω, 113, in d. II. Heb. Il. ν, 821. ω, 75. Od. ο, 525, μοι Od. δ, 379. 468. τ, 162, καί in d. I. Heb. Od. χ, 133, in d. II. Heb. Il. ο, 57; Hiat. sehr oft im II. F. Il. α, 85 u. sonst, im V. F. Il. α, 90 u. sonst, am Ende des III. F. Od. χ, 288; ἔπος, Hiat. im II. F. Il. π, 686, im V. ο, 234; vgl. lesb. ϝείπην = εἰπεῖν, ϝέπος eleische Inschr., ἀποϝειπάθθω u. s. Gortyn. Taf. — Für ὄψ vox sind die Belege nicht so entscheidend (s. Od. λ, 421. Il. α, 604. γ, 152. δ, 435 u. s. w.); widerstrebend δ ὄπ Il. λ, 137, φ, 98, ἀοιδιάουσʼ ὀπὶ Od. ε, 61. Für ὄσσα ist aus Homer nichts zu ersehen.

εἴρω, l. ver-bum, goth. vaur-d, Wort, Curt.\textsuperscript{5} 343. Präs. sage,
Hiat. im V.F. Od. β, 162. λ, 137. ν, 7; Fut. ἐρέω, τοι lang in d. II. Heb. Il.
α, 204 u. sonst oft. Med. εἴρομαι, lasse mir sagen, frage, περὶ ξείνοιο
ϝερέσθαι? Od. α, 405, vgl. 135. γ, 77. δ, 61. ι, 503. λ, 542. τ, 46. 95 (doch
ohne ϝ Il. α, 513. 553, η, 127. ο, 247. ω, 390, Od. α, 188. 284. γ, 69. 243. θ,
549. ι, 402. ξ, 378. ο, 263. 362. π, 137. 465. ρ, 368. 509. 571. υ, 137. ψ, 106.
ω, 114. 474, dazu ἀνείρομαι, διείρομαι).\footnote{}

ἑκάς (vgl. βεκάς, lakon.), ἐπεί lang Od. ε, 358; Hiat. in jedem Fusse; sehr
selten nach kurzen mit einem Konsonanten auslautenden Silben; sehr selten vor
ἑκάς lange Vokale kurz. Dazu ἕκατος, ἑκάεργος, ἑκατηβελέτης, ἑκηβόλος, ἑκηβολία,
z. B. Ἀπόλλωνο̄ς ϝεκάτοιο Il. η, 83. υ, 295. Böot. Inschr. Fἑκάδαμος (FΗΕΚ.),
Meister, Dial. I, 254; bei Hom. d. Name Ἑκαμήδη Il. α, 624. ξ, 6 ἐϋπλόκαμο̄ς Ἑκ.;
Ἑκάβη (Fεκάβα korinth. Vase Dial.-Inschr. 3130) hat b. Hom. keine sehr sicheren
Anzeichen des ϝ (Il. ω, 193. π, 718. ζ, 293), u. drei widerstrebende Stellen
finden sich: χ, 430. ω, 283. 747.

ἕκαστος: ϝέκαστος kret., lokr., arkad. Inschr. (vgl. Allen, C. Stud. III, 248; L. Meyer, K. Ztschr., XXI (1873), 350 ff.); dazu ἑκάτερθε: Hiat. an zahlreichen Stellen, zuw. auch Verlängerung auslautender kurzer Silben bezw. Länge auslautender langer Vokale und Diphthongen; andererseits auch nicht wenige (nach Hartel 56) Belege vernachlässigten Digammas, so in der häufigen Phrase μένος καὶ θυμὸν ἑκάστου.

ἕκηλος, sk. [root ] va<*>, va[cnull ]-mi = volo, s. Curt., Et.\textsuperscript{5} 136: Hiat. im V. F. Il. ε, 759. ι, 376 u. s. w. (für ἕκ. auch εὔκηλος d. i. ἐϝ (έ) κηλος); ἕκητι, Hiat. Od. ο, 319. τ, 86. υ, 42 u. s. w., vgl. ἀέκητι; ἑκών, Hiat. ebenso im Kompos. ἀέκων, b. ἑκ. Od. π, 95. Il. ζ, 523 u. s. w. Vgl. ϝεκών lokr. Inschr. Dial.-Inschr. 1478, 12.

ἑκυρός § 16, 3, a, δ. Il. γ, 172 φίλε̄ ϝεκυρέ.

ἔλδομαι, wünsche, wahrsch. digammiert, vgl. ἐέλδομαι, ἐέλδωρ.

[ἑλεῖν, ἕλωρ, ἑλώριον zeigen anscheinende Spuren eines anlautenden Konsonanten:88) Il. ε, 576 ἔνθα Πυλαιμένεᾶ ἑλέτην (Hiat. mit Verlängerung, aber in d. Heb. des III. F., Knoes vergleicht θ, 556. Od. κ, 322), bloss Hiat. β, 332. ε, 118; an anderen Stellen (ο, 71 χ, 253. ε, 210. 37. ρ, 276. χ, 142) ist er entschuldigt; zahlreiche St. verschmähen den konson. Anlaut; ἕλωρ, ἑλώριον Il. α, 4. ε, 684. ρ, 667 Hiat. nach der I. Kürze des III. F.; aber Od. ν, 208 πώς μοι ἕλωρ, Il. ς, 93 Πατρόκλοιο δ ἕλωρα rein vokalisch. Dass der Stamm ἑλ ursprünglich konsonantisch anlautete, geht auch aus dem Augmente εἶλον hervor; indes ein Digamma zeigt sich auch in den Dialekten nirgends.]

ἑλίσσω, vgl. böot. Eigenn. Fελιξίων, Meister, Dial. I, 254: der Hiatus in der weiblichen Cäsur des III. F., z. B. Il. θ, 340 u. sonst, beweist Nichts; ἕλιξ, καί lang im V. F. Il. ο, 524 u. sonst; εἰλίποδᾶς ἕλικας βοῦς Od. α, 92 u. sonst; so auch ἑλίκωψ, ἑλικῶπις, obwohl die Stellen Il. α, 98. 389. γ, 190. 234 nicht streng beweisen.

ἔλπομαι, vgl. l. volupe, ἀελπτέω, ἀελπής, ἄελπτος: καί lang in d. II. Senk. Od. φ, 157, ebenso b. ἐλπίς in d. IV. Senk. Od. π, 101. τ, 84; Hiat. b. ἔλπομαι im III. F. Il. ι, 40. ο, 288; ἔολπα (ϝέϝολπα) im II. F. Od. β, 275. γ, 275. ε, 379; über ἐώλπει s. d. Anm. S. 97.

ἕννυμι st. ἕς-νυμι, vgl. ϝήμα vestis Gortyn. Taf., sk. [root ] vas, vas-ê, induo mihi, l. ves-tio, Hiat. im V. F. sehr oft, als: Il. β, 261. ε, 905 u. s. w., am Ende des III. F. Od. τ, 327; Komp. ἐπιειμένος, καταειμένος (d. i. ἐπιϝειμένος, καταϝ.); Augm. ἑέσσατο, ἐϝέσσατο; aber undigamm. Il. γ, 57 λάϊνον ἕσσο; εἷμα, καί lang in d. III. Senk. Il. γ, 392, in d. IV. Od. ζ, 144, τοι in d. II. Il. χ, 510; Hiat. am Ende des IV. F. Il. β, 261; ἔσθος, Hiat. im V. F. Il. ω, 94; dahin auch ἑᾶνός (?), ε (ἱ) ανός, vgl. mit letzterem sk. vasanam, Curtius, Et.\textsuperscript{5} 376; doch sind keine stark beweisenden Stellen, u. ἑᾶνός ohne Dig. steht Il. ς, 352. 613. ψ, 254; Curtius ist auch wegen des ᾱ bezüglich der Zugehörigkeit von ἑᾶνός zweifelhaft.

ἕο, εὗ, ἕθεν, οἷ, ἕ u. ὅς, suus (aber nicht ἑός d. i. (ς) ἑϝός) 89) an unzähligen Stellen; alle Merkmale des ϝ finden sich bei diesem Pronomen.

ἔοικα (ϝέϝοικα), vgl. für das Digamma kypr. ϝεικόνα Dial.-Inschr. 76; davon εἰοικυῖαι (?) Il. ς, 418, st. ϝεϝοικ, ϝεϝικ., ἐῴκει (s. d. Anm. S. 97), εἰκώς (ϝεικώς, ϝεϝικώς), εἰκυῖα (ϝεικυῖα, ϝεϝικ.), Hiat. im V. F. Il. α, 119 u. s. w., im II. β, 190. ξ, 212; lange Vok. bleiben lang davor; εἴκελος, Hiat. am Ende des III. F. Od. τ, 384; Komp. ἐπιείκελος; so auch ἴκελος Il. δ, 86 ἡ δ ἀνδρὶ ϝικέλη. (Od. δ, 796 u. ν, 288 kann gelesen werden δέμας δὲ ϝέϝικτο st. δ ἤϊκτο.) Dazu auch ἐΐσκω d. i. ϝεϝίσκω (ϝεϝίκ-σκω), Hiat. Il. ε, 181 πάντα (ϝ) ε (ϝ) ίσκω u. sonst (doch widerstrebend Od. ι, 321. λ, 363).

ἕξ (aus σϝέξ), dor. ϝέξ Inschr.: τῶν οἱ ϝὲξ ἐγένοντο Il. ε, 270, vgl. Od. χ, 252. κ, 6. π, 248. Il. ω, 204 (ohne ϝ Il. ψ, 741. Od. ω, 497; ebenso ἕκτος Od. γ, 415, ἑξήκοντα ξ, 20, ἑξάετες γ, 115). Vgl. Flach, Dig. Hesiod. 39.

ἔργον s. ἔρδω. ἔργω, ἐέργω (woraus εἴργω), sondere ab, Pf. ἔεργμαι, sk. vṛ-n-aǵmi, l. urgeo, Hiat. am Ende des III. F. Il. λ, 437; Verlängerung auslautenden kurzen Vokals in d. II. Heb. Od. ξ, 411.

ἔρδω d. i. ἔρζω ἔρ (ς) δω, m. Metath. ῥέζω, St. ϝεργ ϝρεγ, goth. vaurkjan, wirke, Hiat. im V. F. Il. ξ, 261. Od. ο, 360; Pf. ἔοργα (ϝέϝοργα), Hiat. im V. F. Il. β, 272 u. sonst; über ἐώργει s. d. Anm. S. 97; ἔργον, vgl. Werk, Digamma auch inschr. bezeugt, wie eleisch ϝάργον, böot. Eigenn. Fεργαένετος, lang καί in der IV. Senk. Il. ο, 473. Od. ζ, 259. η, 26, in d. II. Heb. Il. ε, 432, αὐτοῦ in d. II. ε, 92, δή in d. II. Od. ρ, 226; Hiat. sehr oft am Ende des II. F. Il. α, 518, 573 u. s. w., am Ende des V. F. Il. β, 37, 137 u. s. w.; ἐργάζομαι, Hiat. am Ende des IV. F. Il. ς, 469. ω, 733 u. s.

ἔῤῥω, vgl. dor. βέρρης = δραπέτης, βερρεύω = δραπετεύω, eleische Inschr. ϝέρ (ρ) ην, Dial.-Inschr. 1153, 6. ϝάρρην 1152, 2, Hiat. im V. F. Il. θ, 239. ι, 364; vgl. auch ς, 421; Od. δ, 367.

ἕρση, sk. varsh-as, Regen, Tröpfeln, Curt.\textsuperscript{5} 345, b. Homer in der Bdt. Tau stets ἐέρση, aber ἑρσήεις u. ἐερσήεις (ἑρς. ohne Rücksicht auf das Digamma Il. ξ, 348. ω, 757, wo νῦν δέ μ ἐϝερσήεις van Leeuwen, Mnemos. N. S. XIII, 193; desgl. ἕρσαι Frischlinge, Od. ι, 222 χωρὶς δ αὖθʼ ἕρσαι).

ἐρύω, ἐρύομαι in der Bedeutung ziehen, entreissen sind digammiert (vgl. αὐέρυσαν aus ἀ (ν) ϝέρ.): Il. ξ, 76 πάσας δὲ ϝερύσσομεν, ο, 351 ἀλλὰ κύνες ϝερύουσιν (Fut.), χ, 67 ὠμησταὶ ϝερύουσιν (Fut.), ρ, 396 Τρωσὶν μὲν ϝερύειν, ε, 467 νεκρὸν γὰρ ϝερύοντο; ἐρυσσάμενος hat stets einen kurzen Vokal, sowie auch δέ vor sich; ferner lang οἷ in d. III. Heb. Il. ε, 298, καί in d. V. Od. γ, 65, 470. υ, 279; Hiat. im V. F. Il. π, 781 u. sonst, im II. F. Il. γ, 271 u. sonst; aber Od. τ, 481 ϝέθεν ἆσσον ἐρύσσατο, dann in der Redensart νῆα μέλαιναν ἐρύσσομεν, ferner Il. ρ, 635 ὅπως τὸν νεκρὸν ἐρύσσομεν, ψ, 21 δεῦρʼ ἐρύσας, ω, 16 τρὶς δ ἐρύσας, Od. ι, 77 u. μ, 402 ἱστία λεύκ ἐρύσαντες; ferner Od. α, 441 ἐπ-έρυσσε; mit ἐρύω hängt ῥυστάζειν, schleifen, zusammen: Il. ω, 755 πολλὰ ϝρυστάζεσκεν. Davon wollen manche trennen ἔρυσθαι, εἴρυσθαι in der Bedeutung schirmen, bewahren, schützen, schützend abwehren, als urspr. mit ς anlautend (σερύ-ομαι, vgl. serv-o),90) als: Il. α, 239 πρὸς Διὸς εἰρύαται, 216 ϝέπος εἰρύσσασθαι, δ, 138 ἥ οἱ πλεῖστον ἔρῦτο, die ihm Schutz gewährte, ε, 23 Ἥφαιστος ἔρῦτο σάωσε δέ u. s. w.; so auch Il. π, 411 αὐτὰρ ἔπειτʼ Ἐρύλαον, Volksschirmer; auch gehöre hierher das abgekürzte digammalose ῥύεσθαι, schützen, retten.91) Indes gerade Ἐρύλαος deckt neben dem Εὐρυσίλαος einer lesb. Inschrift den Sachverhalt auf: es bestand neben ϝερύω ϝρύω, ἐϝρύω, welches letztere lesb. als εὐρ., Homerisch neben εἰρ. (Dehnung, natürlich ohne Dig., Knoes 103) als ἐρύω mit Ausstossung des Dig. erscheint.92)

ἕσπερος, vesper, vgl. böot. ϝες (πέρας)? Dial.-Inschr. 801, lokr. Λορῶν τῶν ϝεσπαρίων das. 1478, Od. α, 422 μένον δ ἐπὶ ϝέσπερον ἐλθεῖν. ρ, 191 ποτὶ ϝέσπερα u.s.w. (Hes. op. 552; Pind. J. 7, 44).

ἔτης, inscr. Eliac. C. J. Gr. 11 = Dial.-Inschr. 1149 ϝέτας: Hiat. im V. F. Il. ζ, 239 u. sonst, im II. F. η, 295, ι, 464. Od. δ, 16.

ἔτος, Jahr, vgl. sk. vatsas, ϝέτος herakl. Tafeln u. a. Inschr.: Verlängerung kurzer m. Konson. ausl. Silben Il. ω, 765. Od. τ, 222. η, 261 u. s. w.; auslautende lange Vok. o. Diphth. lang vor ἔτος Od. α, 16. δ, 82 u. s. w.; Komp. αὐτόετες, ἑπτάετες u. s. w.

ἐτώσιος, vergeblich, scheint Dig. gehabt zu haben (Heyne, Thiersch, Bekker u. s. w.): Hiat. Od. χ, 253. 276. Il. ε, 854. ξ, 407. χ, 292 (dagegen δ vorher Od. ω, 283).

ἡδύς, s. ἁνδάνω.

ἧθος (vgl. l. suesco), Wohnort von Tieren, Stall, Hiat. am Ende des III. F. Od. ξ, 411; Il. ζ, 511 u. ο, 268 ist st. μετά τʼ ἤθεα mit Heyne u. Bekk. zu lesen μετὰ ϝήθεα. Hes. Th. 66. 167. 222. 525, Flach, Dig. d. Hes. 34. Vgl. διαλλάξαντο ἦθος Pindar Ol. 10, 21. (Aber ἠθεῖος ohne ϝ: Il. ψ, 94. Od. ξ, 147.)

[ἦκα (vgl. att. ἀήττητος)? s. Od. ρ, 254. Il. ω, 508, doch auch Od. ς, 92.]

ἦνοψ, funkelnd, καί lang in d. IV. Senk. Il. π, 408; Hiat. im IV. F. ς, 349. Od. κ, 360; ebenso d. Eigenn. Ἦνοψ Il. κ, 401. ψ, 634 (Ἠνιοπεύς, ου lang in der IV. Senk. θ, 120).

ἦρα, d. i. χάριν, ῳ lang in der II. Senk. Od. ς, 56, i. d. I. Il. ξ, 132; dazu ἐπὶ (nie ἐπ) ἦρα φέρων Il. α, 572. 578.

ἠρίον, Erdhügel, Hiat. am Ende des III. F. Il. ψ, 126.

ἠχή (vgl. ἰ-άχω) kommt nur im Anfange des Verses vor; die Ableitungen δυσηχής, ὑψηχής ohne ϝ; doch θάλασσά τε ἠχήεσσα (IV. F.) Il. α, 157. δώματα ἠχήεντα Od. δ, 72; κα ἠκέτα τέττιξ Hes. op. 582; b. dems. vor ἠχῇ ἠχώ Hiat. Sc. 438. 279. 348; s. auch Pind. Ol. 1.4, 21. Knoes a. a. O. 59 ff.

Ἰάνασσα Il. ς, 47, wegen des καί (§ 47, 6): καὶ Fιάνασσα; ebenso dann Ἰάνειρα das. (η vorhergehend).

ἰαχή, ἴαχον, vgl. αὐΐαχος Il. ν, 41 d. i. α copul. u. ϝιαχ-, γένετο̄ ϝιαχή Il. δ, 456 u. sonst, μέγᾶ ϝιάχων ξ, 421 u. sonst, s. § 17, 3; jedoch findet vor αχε, αχον oft die Elision statt, als: α, 482. ρ, 29 u. s. w., καί kurz υ, 62; man führt diese Quantität des ι auf Dehnung vor dem verlorenen ϝ wie in ἠείδη zurück, Knoes a. a. O. S. 60, oder möchte εἴαχον (aus ἐϝίαχον) schr., Wackernagel, K. Z. 25, 279; s. § 343.

ἰδ-εῖν, vid-ere, eleisch ϝειζώς = ϝειδώς u. s. w.: lang καί in d. V. Heb. Il. ρ, 179 u. sonst, ἐπεί in d. II. δ, 217. ο, 279, in d. IV. ε, 510 und sonst; Hiat. sehr oft, im V. F. α, 262 u. s. w., im II. F. γ, 217 u. s. w.; d. langen Vokale u. Diphthonge vor ἰδεῖν lang; οἶδα, weiss, lang τευ in d. II. Senk. Il. ς, 192, καί in d. IV. Od. ς, 228, υ, 309, ᾔδη (ϝείδη) in der II. ω, 407; Hiat. sehr häufig, im II. F. Il. β, 192 u. sonst, im V. π, 50 u. sonst, am Ende des IV. F. υ, 201 u. s. w., ᾔδεα (ϝείδεα) Hiat. im V. F. Il. β, 213 u. sonst, im II. θ, 366 u. sonst; Il. τ, 421 ist wohl st. εὖ νυ τοι mit Bekk. zu lesen εὖ νυ τὸ ϝοῖδα; doch bleiben andere Stellen ohne ϝ wie ς, 185; ἴδμεν, ἰδέω, ἰδυῖα, lang καί in der IV. Senk. Il. η, 281, που in d. II. Heb. α, 124; Hiat. im II. F. Il. β, 252, 301, im V. ς, 420, am Ende des III. ν, 273; b. ἰδυῖα im III. F. α, 608. ς, 380, 482. Od. η, 92; st. εἰδυῖα, das nur Il. ρ, 5 fester steht, ist sonst stets mit Ahrens (Rh. M. 2, S. 177 f.) ἰδυῖα zu lesen, wodurch auch das ϝ zu seinem Rechte kommt, also st. ταῦτ εἰδυίῃ Il. α, 365 ταῦτα ἰδυίῃ, st. ἔργʼ εἰδυίας Il. ι, 128 u. s. w. ἔργα ἰδυίας u. s. w. (vgl. La Roche, Hom. Textkr. 286 f.); εἰδώς, lang οὔπω in d. II. Senk. Il. ι, 440, εὖ oft in d. V.; Hiat. am Ende des III. F. λ, 710, im V. δ, 218 u. sonst; Konj. εἰδῶ (εἴδω) am Ende des III. F. Il. χ, 244; am Ende des IV. F. θ, 18. π, 19, im II. F. ν, 122, ψ, 322, im V. ο, 207; εἴσομαι, werde wissen, Hiat. am Ende des III. F. η, 226, ξ, 8; καί lang in der IV. Senk. Od. τ, 501; εἴδομαι, erscheine, bin ähnlich, εἰσάμην, lang δή in d. I. Senk. Il. ν, 98, οἷ in d. II. Heb. β, 215. μ, 103; Augm. ἐ-είσατο; Hiat. im Komp. διαείδεται ν, 277, διαείσεται θ, 535; εἶδος, καί lang in d. IV. Senk. Il. χ, 370 u. sonst, τοι in d. II. Heb. κ, 316, Hiat. im II. F. γ, 55; εἴδωλον, καί lang in d. II. Heb. ψ, 104; ἰδρείη, Kunde, Hiat. a. E. des III. F. Il. π, 359; ἴστωρ, Hiat. nach ἐπὶ ς, 501; ἰνδάλλομαι keine stark beweisenden Stellen (Il. ψ, 460. Od. γ, 246. τ, 224); Kompos. ἄϊστος, ἄϊδρις, ἀΐδηλος, ferner θεοειδής, ἰοειδής u. s. w.

ἵεμαι, ich strebe, beeile mich, stürme auf etwas los, ϝίεμαι, vgl. L. Meyer, K. Ztschr. 21, 355. Bzzb. Btr. I, 301 ff. (Vgl. Gr. I\textsuperscript{2}, 179, 289), Ahrens, Btr. I, 112 ff., lang in der I. Senk. πρόσσω, εἴα, ἔστη, αἰχμή Il. π, 382. ο, 543. π, 396. υ, 280, 399, πρόσσω in d. II. ν, 291; Hiat. am Ende des IV. F. λ, 537, ν, 386; am Ende des III. υ, 502; Il. μ, 274 ist mit L. Meyer πρόσω ἵεσθε z. lesen (widerstrebend ς, 501. Od. β, 327. λ, 346. ξ, 142 [v. l.]; aber ἵενται mit ι Il. δ, 77. Od. χ, 304 zu ἵημι); Aor. mit Augm. ε: ἐ-είσατο Il. ο, 415. Od. χ, 89, ἐ-εισάσθην Il. ο, 544; Hiat. in der Redensart διαπρὸ δὲ εἴσατο vom Speere oder Pfeile Il. δ, 138. ε, 538. ρ, 518. Od. ω, 524; Kompos. καταείσατο Il. λ, 538, ἐπιεισαμένη φ, 424; Fut. ἐπιείσομαι Il. λ, 367. υ, 454; aber vokalisch πάλιν εἴσομαι (werde gehen, also zu εἶμι) ω, 462, δεῦρʼ εἴσεται Od. ο, 213 (während τάχα εἴσομαι Il. ξ, 8 zu οἶδα gehört, Ahrens a. a. O. 115); leicht zu ändern ῥεῖα μετεισάμενος Il. ν, 90. ρ, 285 (ῥέα μεταεις.). Il. ν, 191 ist mit Bekker nach Zenodot ἀλλʼ οὔ πῃ χρὼς εἴσατο (ϝείσατο), d. i. apparuit, st. χροὸς εἴς. zu lesen. (Aber ἵημι hat kein ϝ; auch die Hiaten Il. δ, 75 ἀστέρα ἧκε u. ξ, 182 können durch die augmentierte Form ἕηκε entfernt werden.)

[Ἴκαρος, Ἰκάριος? vgl. Il. β, 145 πόντου Ἰκαρίοιο, Od. α, 329 u. ö. κούρη Ἰκαρίοιο, τ, 546 θάρσει Ἰκαρίου; doch δ, 797 μεγαλήτορος Ἰκαρίοιο.]

ἴκελος s. ἔοικα.

Ἶλιος, lang καί in d. III. Senk. Il. ζ, 493, οὔπω in d. IV. υ, 216, Verbalend. ῃ in d. IV. δ, 164 u. sonst, οἵ in der III. Od. θ, 495; Hiat. sehr oft, z. B. im I. F. Il. φ, 295, am Ende des III. θ, 131 u. sonst; keine Elision der elisionsfähigen Präp.; so auch Il. κ, 415 am Ende des Verses παρὰ σήματι Fίλου (wiewohl Ἴλου ohne ϝ λ, 166) u. ξ, 501 ἀγαυοῦ Fιλιονῆος.

ἰνίον s. ἴς.

ἴον u. Kompos., vgl. viola, Od. ε, 72 μαλακοὶ ϝίου. Il. ψ, 850 τίθει ϝιόεντα; so auch Hiat. vor ἰοδνεφές Od. δ, 135. ι, 426; vor ἰοειδής Il. λ, 298. Vgl. δῶρα ἰοστεφάνων Theogn. 250; auf Vasen Eigenn. Fιώ oft; Fιόλαϝος Dial.-Inschr. 3132; dazu Hes. sc. 77, 102, 323, 340, 467; Knoes II, 124.

Ἴρις, Hiat. im II. F. ὣς ἔφατ· ὦρτο δὲ Fῖρις Il. θ, 409. ω, 77, 159 u. im V. πόδας ὠκέα Fῖρις Il. γ, 129 u. sonst oder ποδήνεμος ὠκέα Fῖρις β, 786 u. sonst; aber Elision ε, 353, λ, 27, ψ, 198, οἷ (οἱ) kurz ε, 365; daher zweifelt Hoffmann l. d. II. p. 40, ob dieses Wort bei Hom. digammiert sei, und auch Knoes II, 126 möchte das Dig. auf feste, traditionelle Formeln beschränken. Vor dem Appell. ἶρις Elision Il. λ, 27. — Der Name Ἶρος in d. Odyssee erfordert Dig. ς, 334, vgl. 73 (75, 333), aber ohne ϝ 38, 56, 233. Kompos. Ἆϊρος ς, 73. Der Dichter leitet V. 6 offenbar den Namen von εἴρειν (ϝείρ.) = ἀπαγγέλλειν ab; es scheint somit die Schreibung Εἶρος richtiger, doch ist die Tradition für ι, vgl. Herodian I, 6. 191; II, 448. 526 L.

ἴς, vis, Hiat. nur am Ende des IV. F., wie Od. ι, 538; καί lang in d. IV. Heb. Il. μ, 320; ἐμοί desgl. λ, 668; vgl. auch ἱερὴ ἲς Od. π, 476. ς, 60 u. s. w.; aber Il. ρ, 739 ἐπιβρέμει ἲς ἀνέμοιο. φ, 356 καίετο δ ἴς; ἶνες, nervi, Hiat. am Ende des III. F. nur ψ, 191: χρόα ϝίνεσιν; ἰνίον, Genick, Hiat. am Ende des III. F. ε, 73, im I. F. ξ, 495. Ἶφι u. s. w. s. besonders.

ἶσος (aus ϝίσϝος, wie die Gortyn. Tafeln bieten) und die Derivata, vgl. sk. vishu = aeque Curt., Et.\textsuperscript{5} 378, lang μετηύδα in d. IV. Senk. Il. ψ, 569, καμινοῖ in d. II. Od. ς, 27, βροτολοιγῷ Il. λ, 295. μ, 130. ν, 802 u. s. w. (Knoes II, 129 f.); aber οι kurz vor ἴση Il. λ, 705. Od. ι, 42, 549; Hiat. vor ἶσος im II. F. Il. λ, 336 u. sonst. Vgl. b. Hom. ἐΐση.

ἰτέη, richtig εἰτέη (γιτέα Hesych.), Weide, althochd. wîda, sk. vê-tasas, eine Rohrart, Curt.\textsuperscript{5} 389, l. vi-tex, vi-men, lang καί in d. III. Senk. Od. κ, 510; aber Il. φ, 350 πτελέαι τε καὶ ἰτέαι, wo Bekk. liest: πτελέαι καὶ ϝιτέαι; verwandt ἴτυς, äol. ϝίτυς (Meister, Dial. I, 105 f.), Il. δ, 486 ὄφρα ϝίτσν.

ἶφι, ἴφιος, Ἰφιάνασσα (v. ἴς, l. vis), oft καί lang im V. F.: als: ἴφια μῆλα Il. ε, 556 u. s. w., ι, 145, 287; Hiat. am Ende des IV. F. α, 38 u. sonst, im I. F. ι, 466. ψ, 166; aber ἴφθιμος hat bei Hom. kein ϝ und muss, wenn es für ἰφίτιμος steht, mit dem zweiten ι auch das ϝ eingebüsst haben. — Die anderen Eigenn. mit Ἰφι- stehen meist so, dass ein ϝ hinzutreten kann, oft auch zum Vorteil des Verses (Od. λ, 296 βίη Fιφικληείη, Il. λ, 257 ὁ Fιφιδάμας); doch β, 518 υἱέες Ἰφίτου (υἷες Fιφ. Bentl., und Fίφιτος steht auf e. korinth. Vase Dial.- Inschr. 3133), λ, 261 ἐπʼ Ἰφιδάμαντι, Od. λ, 305 τὴν δὲ μέτʼ Ἰφιμέδειαν.

[ἰωή? Hiat. im V. F. Il. δ, 276. λ, 308. π, 127, immer nach der Genetivendung auf οιο; Il. κ, 139, Od. ρ, 261 ἤλυθʼ ἰωή ändert Hoffm. I<*>, p. 37 nach Bentley in ἦλθε ϝιωή.]

ἰωκή, Hiat. im V. F. Il. ε, 521, 740 (doch ohne ϝ λ, 601). Vgl. ϝιώκει = διώκει kor. Vase Dial.-Inschr. 3153.

οἶδα s. ἰδεῖν.

οἶκος, sk. vê[cnull ]as, Haus, l. vîcus (ϝ auch inschriftl. vielfach bezeugt): lang καί in der IV. Senk. Il. θ, 513 u. sonst, in d. II. Od. ψ, 7 u. sonst, μοι in d. II. Od. δ, 318, δή in d. II. Od. φ, 332, ψ, 36; καί in d. I. Heb. Il. ο, 498, in d. II. Od. ζ, 181; Hiat. im V. F. Il. ζ, 56. ρ, 738; am Ende des IV. F. μ, 221. Ebenso ϝοικίον, ϝοικεύς, ϝοικέω (doch οἰκωφελίη ohne ϝ Od. ξ, 223).

οἶνος, vinum, lang καί in d. IV. Senk. Il. ι, 489 u. s. w., in d. II. γ, 246, μοι in d. II. Od. β, 349, in d. I. Il. ζ, 264; καί in d. II. Heb. ι, 706 u. sonst; Hiat. im V. F. α, 462 u. s. w.; dazu οἶνοψ οἰνοχοέω u. s. w.; Οἰνόμαος, Hiat. im II. F. ν, 506, vgl. μ, 140 (doch ohne ϝ ε, 706 Αἰτώλιον Οἰν.); Οἰνεύς, Hiat. nach dem V. F. ι, 581. ξ, 117 (vgl. ι, 543; doch ohne ϝ β, 641 u. Οἰνεΐδης ε, 813. κ, 497); über ἐῳνοχόει s. d. Anm.

ὅς, qui, zeigt bei Homer keine genügenden Spuren des ϝ: in δάμᾶρ ὅς Od. δ, 126 ist wohl Naturlänge, s. § 120, Anm. 5; leicht erklärlich χωόμενο̄ς, ὅτι u. s. (lokrisch allerdings ϝότι, § 175, Anm. 2); deutlicher aber treten die Spuren des ϝ in dem Adverb ὡς, wie, hervor. In der Anastrophe stehend, macht es in der Regel eine vorhergehende kurze Silbe lang, als: θες ὥς, ὄρνιθε̄ς ὥς, φυτὸν ὣς, πέλεκῦς ὣς (Bekker, Hom. Blätter I, 204), und zwar im VI. Fusse Il. γ, 230. δ, 482. ι, 302. λ, 172 u. s., im IV. β, 190. γ, 60. ε, 476 u. s., im II. ζ, 443 (doch auch κτίλος ὣς γ, 196. θεὸς δ ὣς ε, 78. ὅλμον δ ὥς λ, 147. λέονθʼ ὥς 383, νιφάδες δ ὥς μ, 156 u. s. w., Knoes II, 167 f.). Die sich hieraus ergebende Nebenform ϝώς zu ὥς kann mit jenem ϝότι zusammengestellt werden; man vergleicht mit ὥς goth. svê. G. Meyer, Gr. 216\textsuperscript{2} f.

ὅς, suus, s. ἕο.

οὐλαμός s. εἴλω.

[ὠθέω § 198 b, 5, ohne Dig., trotz Il. π, 592 Τρῶε̄ς, ὤσαντο, und Od. λ, 596 ἄνω ὤθεσκε. Knoes II, 133.]

ὦλκα (Akk. v. d. ungbr. N. ὦλξ = αὖλαξ, Att. ἄλοξ) lautete anscheinend mit ϝ an, trotz sulcus: κατὰ ὦλκα Il. ν, 707, vgl. Od. ς, 375. Hesiod. op. 439. 443 (L. Meyer, Vgl. Gr. I\textsuperscript{2}, 178 f.; G. Meyer, 115\textsuperscript{2}).93)

ὡς, wie, s. ὅς, qui.

{\small\noindent\subparagraph{} Das Imperf. ἑήνδανον Il. ω, 25. Od. γ, 143 darf
nicht mit Bekker ἐϝήνδανον geschrieben werden; denn das η nach dem Digamma wäre
völlig unerklärlich; also ἐϝάνδανον oder allenfalls ἑήνδ. mit verlorenem und
durch die Dehnung gewissermassen ersetztem ϝ (Homer ἠείδη st. ἐϝείδη, att. ἑώρων
aus ἡόρων, ἐϝόρων). Ebenso ἐῳνοχόει Il. δ, 3. Od. υ, 255, das Bekker fälschlich
ἐϝῳν. schreibt; desgl. die Plusq. ἐῴκει, ἐώλπει, ἐώργει v. den Pf. ϝέϝοικα,
ϝέϝολπα, ϝέϝοργα, die ἐϝεϝοίκει, ἐϝεϝόλπει, ἐϝεϝόργει lauten müssten. S. § 198,
6.}

\section{Bemerkungen über das Digamma bei Homer}

\paragraph{} Dass bei Homer das Digamma oft ein vorgeschlagenes ε habe, dieses ε aber nicht digammiert sei (abgesehen vom Perf., als: ϝέϝοικα, ϝέϝολπα), haben wir § 16, 2, a, η gesehen. Man kann das ε als prothetischen Vokal, indes auch als Assimilation des ϝ an ε, ι fassen, da gerade vor diesen Vokalen und nicht vor α, ο sich dies ε findet. Durch Kontraktion der beiden ε erklärt sich εἴργω (att.); εἰαρινός, εἱανός möchte man kaum mit Recht hierher ziehen. Kühners ausführl. Griech. Grammatik. I. T.

\paragraph{} Der Übergang des ϝ in υ (§ 16, 2, b) findet sich bei Homer im Inlaut in der Hebung, etwa um eine lange Silbe zu gewinnen, oder infolge der Assimilation: αὐΐαχοι, zusammenschreiend [aus α copul. u. ϝιαχή],94) εὔαδεν st. ἔσϝαδεν = ἕαδεν, ταλαύρινος st. ταλάϝρινος, mit d. Schilde Stand haltend, καλαῦροψ st. καλά-ϝροψ (vgl. ῥόπαλον), Hirtenstab, nach Hoffm. I, p. 138 v. καλος, funis, fustis laqueo instructus, u. ϝρέπω, vergo, vgl. Curt.\textsuperscript{5} 351; αὐέρυσαν st. ἀναϝέρυσαν, ἀνϝέρυσαν, ἀϝϝέρυσαν, zogen zurück; Hesiod. op. 666, 693 καυάξαις st. κατϝάξαις v. ϝάγνυμι. 

{\small\noindent\subparagraph{} Dass das Hom. γέντο, er fasste, als äol. Form
  st. ϝέλτο (ἕλτο, ἕλετο) stehen sollte, ähnlich gebildet wie das dor. κέντο st.
  κέλετο, ist schwerlich anzunehmen, da der Übergang des ϝ in γ sich nirgends
  bei Homer findet. Man vergleicht jetzt Hesych. ἀπόγεμε, ἄφελκε, Κύπριοι,
ὕγγεμος, συλλαβή, Σαλαμίνιοι.95) 96) S. § 343. }

\paragraph{} Dass übrigens das Vau zu der Zeit, als die Homerischen Gesänge gedichtet wurden, bei den Ioniern nicht mehr in seinem ursprünglichen Umfange bestanden, sondern schon den Anfang des allmählichen Verschwindens gemacht habe, erhellt deutlich aus mehreren Erscheinungen. Vorerst zeigen mehrere Wörter, die in anderen Mundarten mit dem Digamma anlauteten, bei Homer keine Spur desselben. So Ἦλις ohne ϝ Il. β, 615, 626. λ, 671, 673 u. s. w. (Knoes II, 80), aber in Elis selbst Fᾶλις; ἴδιος Od. γ, 82 ἥδʼ ἰδίη (δ, 314 nach ἤ, was nichts beweist), aber böot. u. s. w. ϝίδιος, ἑστία: davon ἀνέστιος Il. ι, 63, ἐφέστιος β, 125 u. s. w., aber Hesych. γιστία, arkad. Fιστίας, Dial.-Inschr. 1203; dazu kommen ὁράω ὄρομαι u. s. w. (Knoes 141 f.), wo doch durch ἑώρων ἑόρακα das Digamma bezeugt ist, ἐμέω, lat. vomo, aber αἷμʼ ἐμέων Il. ο, 11; ἀπέμεσσεν ξ, 437. Ein ϝ hatte auch der Name Ἑλένη, wie die Alten (Dionys. A. R. I, 20; Priscian. I, p. 20 K.) bezeugen, unter Anführung des Verses (Bergk, Adesp. 31) ὀψόμενος Fελέναν ἑλικώπιδα; indes bei Homer ist die Spur des ϝ sehr schwach und unsicher und eine Menge Stellen widerstreben (Knoes II, 219). — Merkwürdig ist es auch, dass in einigen anscheinenden Derivatis von digammierten Wörtern das ϝ verschwunden ist, als: ϝῖφι, aber ἴφθιμος, ϝάγνυμι, aber ἀκτή. Insbesondere ist ϝ ziemlich überall geschwunden vor folgendem ο, ω (s. oben ὁράω), L. Meyer, K. Ztschr. XXIII, 53 ff. — Darnach kann man nicht wohl geneigt sein, das inlautende Digamma zwischen Vokalen (abgesehen von der Komposition und von Bildungssilben wie dem Augmente) dem Homer noch beizulegen, zumal da auch schon durch das Antreten einer Bildungssilbe das inlautend werdende Digamma bei ihm sichtlich leidet. Formen wie ἠείδη st. ἐϝείδη, Ἄϊδος st. αϝιδος, αχεν st. ϝιϝαχεν (vgl. § 16, 3, b; § 18 unter ἰαχή) scheinen Verlängerung zu zeigen als Ersatz des ausgefallenen Digamma. Ferner kann Kontraktion eintreten: χεῖσθαι, δῆσεν Il. ς, 100, ὄϊς οἰός aus ὄϝις ὄϝιος, sogar εἶδον Il. λ, 112. τ, 292 u. s. Δήϊος ist eigentlich δήϝιος; Homer kann aber δήιος auch spondeisch oder nach anderer Auffassung anapästisch gebrauchen, welche Verkürzung von η das Schwinden des ϝ voraussetzt. Ähnlich Πηλεΐδης von Πηλεύς, Πηλῆος, d. i. Πηλῆϝος; läge Πηλέϝος zu Grunde, so hätten wir in dem η den Beweis für das Schwinden des ϝ. 

{\small\noindent\subparagraph{} Dass aber δεῖσαι, δέος, δειλός, δεινός, δεῖμος
  (ἀδεής, Δεισήνωρ); δήν, δηρόν zu Homers Zeit ein ϝ hinter dem δ hatten, also
  δϝεῖσαι u. s. w., sieht man daraus, dass diese Wörter mit ihrem Anlaut
  Positionslänge bilden,97) vgl. sk. dvish (hassen), Δϝεινία korinth. Inschr.;
  so δεισας in der Senk. Il. χ, 19 u. sonst, ὑπο̄δείσας, περῖδείσας (nur Od. β,
  66 ὑποδείσατε), Il. ω, 116 εἴ κεν πως ἐμέ τε̄ δείσῃ (doch Od. μ, 203, ω, 534
  ἄρα δεισάντων, vgl. Il. ν, 163); daher δείδοικα mit ει zum Ersatze des
  weggefallenen ϝ (δέδϝοικα); Il. ο, 4 χλωροὶ ὑπ δείους, so κ, 376; ε, 817 οὔτε
  τί με̄ δέος. α, 515 ἢ ἀπόϝειπ, ἐπεὶ οὔ τοι ἔπῖ δέος; nie bleibt ein kurzer
  Vokal vor δέος kurz; ε, 574 τὼ μὲν ἄρᾶ δείλω βαλέτην, γ, 172 αἰδοῖός τέ μοί
  ἐσσι, φίλε ϝεκυρ, δεινός τε, λ, 10. κ, 272 τὼ δ ἐπεὶ οὖν ὅπλοισιν ἔνῖ
  δεινοῖσιν ἐδύτην (doch Il. ο, 626 δε δεινός. θ, 133 ἄρα δεινόν, Hartel, Hom.
  St. I\textsuperscript{2}, 7). θ, 423 κύον ἀδεές (ᾱ). ρ, 217 Ἀστεροπαῖόν τε̄
  Δεισήνορα. Das Adj. θεουδής ist entstanden aus θεοδϝεής. — Il. α, 416 οὔτι
  μάλᾶ δήν. π, 736 ἧκε δ ἐρεισάμενος, οὐδὲ δὴν χάζετο φωτός. ι, 415 ἐπ δηρὸν δέ
  μοι αἰών, vgl. Od. α, 203 (doch meist δηρόν ohne Dig. und Positionskraft, wie
Il. β, 435. ε, 885, 895 u. s. w.). }

{\small\noindent\subparagraph{} Die Wörter σείω und σαίνω scheinen mit σϝ
  angelautet zu haben; daher περῖσείω (περισσείω), ἐπῖσείω, ὑπο̄σείω, ἐσσείοντο
  (doch Il. ξ, 285 ποδῶν ὕπο σείετο), περῖσαίνω (περισσαίνω), nur Od. ρ, 302 μέν
  ὅ γ ἔσηνε.98) Ein Gleiches gilt von σάρξ, äol. σύρξ, vgl. Od. ι, 293 ἔγκατά τε̄
  σάρκας τε, λ, 219 ἔτῖ σάρκας, ς, 77 u. s. w. Σεύω dagegen (kurzer Stamm συ)
  hat nicht σϝ zum Anlaut, sondern σς aus τς, τj, κj;99) Verdoppelung ist hier
  in ἔσσευα, ἔσσυμαι, ἐπισσεύεσθαι, λαοσσόος, Positionslänge vor anlautendem ς
  in ὅτε̄ σεύαιτο Il. ρ, 463. τε̄ σεύαιτο ψ, 198 (ε, 293 Aristarch ἐξελύθη für
Zenodots ἐξεσύθη). }

{\small\noindent\subparagraph{} Betreffs des ursprünglich anlautenden ϝρ (in
  ϝρήγνυμι, ϝρήτωρ, ϝρηΐδιος u. s. w., Fröhde, K. Ztschr. 22, 264 ff.), ist es
  einerseits nicht unwahrscheinlich, dass Homer hier noch den Konson. gehabt
  hat; andererseits mangeln die sicheren Spuren davon, da die Positionskraft des
  anlautenden ρ und die Verdoppelung desselben, wenn es inlautend wird, auch den
  Attikern gemeinsam sind, und ferner bei diesen wie bei Homer durchaus nicht
  auf die Wörter sich beschränken, denen von Haus aus ϝρ zukommt. Vgl. § 75, 12.
  Dazu ist ein Zwang der Verlängerung für Homer bei ρ durchaus nicht vorhanden,
  daher z. B. ἔρρεξα und ἔρεξα; ὣς φάτο· ῥίγησεν δὲ κτἑ; ἔνθα κε ῥεῖα u. s. w.;
  also ist mindestens ῥέζω neben ϝρέζω, ῥέα neben ϝρέα u. s. w. vorhanden
gewesen. }

\paragraph{} Ferner findet Verlängerung einer kurzen auf einen Konsonanten
  ausgehenden Silbe in der Senkung nur von dem Pronomen ϝέο und (selten) vor
  Formen der Wurzel ϝιδ statt,100) z. B. in der III. Senk. Il. ε, 695 ἴφθιμος
  Πελάγων, ὅς ϝοι φίλος ἦεν ἑταῖρος, in der II. ζ, 157 ὤπασαν, αὐτάρ ϝοι Προῖτος
  κακὰ μήσατο θυμῷ, ι, 377 ϝεῤῥέτω· ἐκ γάρ ϝευ φρένας εἵλετο μητιέτα Ζεύς. Od.
  θ, 215 εὖ μὲν τόξον ϝοῖδα, in der I. Il. ε, 7 τοῖόν ϝοι πῦρ δαῖεν. Es erweist
  sich überhaupt die Positionskraft und die gesamte Bedeutung dieses
  absterbenden Konsonanten schwächer als selbst die des beweglichen ν.101)

  \paragraph{} An sehr vielen Stellen erscheint das anlautende Digamma bei Homer vernachlässigt.
  Nach Hartels Statistik (Hom. Stud. III, 62 ff.) zeigen sich in 3354 Fällen
  Wirkungen des ϝ, dagegen in 617 muss man, wenn die Lesart richtig, ein
  Schwinden desselben annehmen (wobei ein zugesetztes ν ἐφελκ., wie in πρόσθεν
  ἕθεν Il. ε, 56, als sofort zu beseitigen nicht gerechnet ist). Nun lässt sich
  an ungemein vielen Stellen mit Leichtigkeit das ursprüngliche ϝ wieder
  auffinden und herstellen, so dass z. B. in Bekkers 2. Ausgabe von jenen 617
  Stellen gegen 300 geändert sind. Statt ἔϝιδον in der Senkung wird oft εἶδον
  gelesen (wiewohl ει auch in der Hebung vorkommt, also die kontrahierte Form
  dem Homer nicht abgesprochen werden kann); st. ἐάνασσε in der Senkung ἤνασσε;
  zuweilen ist ein digammiertes Verb mit einem falschen Augmente versehen, als:
  Il. ο, 701 Τρωσὶν δ ἤλπετο θυμός st. Τρωσὶ δ ἐϝέλπετο. Ferner ἀνδρός τε
  προτέροιο καὶ ἄστεος Il. γ, 140 st. προτέρου καὶ ϝάστεος. Il. χ, 302 Ζηνί τε
  καὶ Διὸς υἱεῖ, Ἑκηβόλῳ st. Διὸς υἷϊ, Fεκηβ. Il. ο, 35 καί μιν φωνήσασʼ ἔπεα
  πτερόεντα προσηύδα st. φωνήσασα ϝέπεα (Synizese, die Christ Il. 160 gleichwohl
  für härter hält als ἔπεα ohne Digamma). Il. ε, 30 χειρὸς ἑλοῦσʼ ἐπέεσσι st.
  ἑλοῦσα ϝέπεσσι. ε, 166 τὸν δ ἴδεν Αἰνείας st. τὸν δὲ ϝίδʼ Αἰν. θ, 406 ὄφρʼ
  εἰδῇ st. ὄφρα ϝίδῃ. ξ, 383 αὐτὰρ ἐπεὶ ἕσσαντο st. ἐπεὶ ϝέσσαντο (ἐπεὶ ἕσς.
  Vindob.). μ, 48 τῇ τʼ εἴκουσι st. τῇ ϝείκουσι. Gerade solche Wörter wie τε, ῥα
  sind sehr oft zur Verdeckung des Hiats eingeschoben.102) 

  \paragraph{} Indes gibt es auch
  sehr viele Stellen, in welchen das Digamma von dem Dichter unbeachtet gelassen
  worden ist, und es ist eine reine petitio principii, wenn man (Bentley,
  Bekker, Nauck) überall auf Herstellung des ϝ durch noch so gewaltsame
  Konjekturen ausgeht. Die Hypothese, dass Homer das Digamma immer
  berücksichtigt, würde nur dann gerechtfertigt und erwiesen sein, wenn das
  Digamma sich bei allen digammierten Wörtern an allen Stellen leicht herstellen
  liesse, was so wenig der Fall, dass, wie wir oben gesehen haben, bei manchen
  ursprünglich digammierten Wörtern überhaupt kein ϝ mehr hervortritt, und bei
  ἕκαστος nicht weniger als 56 Stellen die Vernachlässigung zeigen. So
  unterscheidet denn auch Christ (Il. 158) eine ganze Klasse solcher Wörter, bei
  denen das anlautende Digamma nicht überall bewahrt sei, als ϝανδάνειν, ϝαρνός,
  ϝέτος (weil ihm Il. β, 328 τοσσαῦτα ϝέτη noch mehr unhomerisch scheint als
  τοσσαῦτʼ ἔτεα), ϝοῖκος, ϝοῖνος u. s. w. 

  \paragraph{} Es fragt sich nun, wie die
  Erscheinung, dass das Digamma in den Homerischen Gedichten meistenteils zwar
  als Konsonant behandelt, zuweilen jedoch unbeachtet gelassen ist, zu erklären
  sei. Wie Ludwich103) aufzeigt, ist Homerisch eben nicht Urgriechisch; es
  stehen ältere und jüngere Formen in dieser Dichtersprache friedlich
  nebeneinander, und zu diesen jüngeren, in des Dichters Zeit aber vorhandenen
  Formen gehörten auch ἔργον st. ϝέργον, ἔπος, εἶδον u. s. w., die nach
  Bedürfnis des Verses oder auch des darzustellenden Gegenstandes und Ethos sich
  in freier Weise einstellen, ohne dass dadurch das Verständnis des Wortes
  verdunkelt worden wäre. Man vergleiche damit die sonstige Beweglichkeit und
  Flüssigkeit der epischen Sprache (s. d. Einleit. S. 17), die ihr gestattete,
  in einzelnen Wörtern nach Bedarf des Verses auch andere Anlaute abzuwerfen,
  als: λείβω u. εἴβω, λαιψηρός u. αἰψηρός, σκίδναμαι u. κίδναμαι, σμικρός u.
  μικρός, γαῖα u. αἶα, ἐρίγδουπος u. ἐρίδουπος, oder eine Doppelkonsonanz zu
  vereinfachen, als: ὅππως u. ὅπως, ὅσσος u. ὅσος, ὅττι u. ὅτι u. s. w.104) Auch
  darf diese Erscheinung bei dem ϝ um so weniger befremden, da selbst in den
  Mundarten, in welchen sich der Gebrauch des Digamma länger erhalten hat, eine
ähnliche Unsicherheit im Gebrauche desselben stattfand. S. § 16. 

{\small\noindent\subparagraph{} Die Wirkungen des ϝ erstrecken sich bei einigen
  Wörtern bis auf die spätesten Zeiten, z. B. in dem α privat. st. ἀν in ἄοικος,
  ἀοίκητος, ἀόρατος, (freilich auch ἄοπλος, ἀόριστος u. a., und dagegen
  ἀνάλωτος), in der Zusammensetzung mit Präposition u. Nomen, als: ἐπιεικής, γῆν
  ἐπιέσασθαι Xen. Kyr. O. 4, 6, μελανοείμων Hippokr. VI, 658 L. (v. l.
  μελανείμ.); der Hiat bei dem Pron. οὗ findet sich bei Ioniern (Renner, Curt.
  Stud. I, 1, 149 f.) und selbst Attikern, als: ἁ δέ οἱ φίλα S. Tr. 650, ὅτε οἱ
El. 196 nach Herm. (codd. ὅτε σοι).}

\section[(c) Halbvokal j (§ 7)]{(c) Halbvokal j (§ 7)\protect\footnote{}}

\paragraph{} Der Halbvokal j, den fast alle indogermanischen Sprachen besitzen,
ist in der griechischen Sprache, welche auch den anderen Halbvokal w, ϝ
schliesslich allgemein beseitigt hat, schon von alters her gänzlich verdrängt,
so dass er in keinem Dialekte mehr vorkommt.106) Doch lässt sich sein
ursprüngliches Dasein teils aus der Vergleichung des Griechischen mit anderen
indogermanischen Sprachen, teils aus sehr vielen Erscheinungen im Griechischen
selbst auf unzweifelhafte Weise erkennen. Die Behandlung des j ist in dieser
Sprache eine sehr mannigfaltige gewesen: zum Teil lebt es als Vokal ι fort, für
sich oder in diphthongischer Verbindung; wiederum hat es in Verbindung mit einem
anderen Konsonanten diesen unter Umständen sehr modifiziert, u. s. w. 

\paragraph{} Die Verwandlungen, welche das ursprüngliche j im Griechischen erfahren hat, sind folgende: a) j wird ι, dem es unter allen Vokalen am Meisten verwandt ist, wie ϝ u. υ § 16, 3, b (vgl. Ἀχαΐα, Αἴας, Μαῖα, Τροία mit Achaja, Ajax, Maja, Troja, sowie Gajus, Pompejus, Trajanus, Aquileja mit Γάϊος, Πομπήϊος, Τραϊανός, Ἀκυληΐα).107) Indes ist eben wegen der engen Berührung von i und j sehr schwer zu bestimmen, inwieweit in den einzelnen Fällen der Halbvokal oder der Vokal ursprünglich ist. Vgl. Gen. S. der II. Dekl. ο-ιο, sk. a-sja, als: ἀγροῖο, sk. aǵrasja, ἵπποιο, sk. a[cnull ]va-sja, ς fiel aus (§ 15, 1), j blieb als ι nach; ebenso in ἀλήθεια für ἀληθεςjα u. s. w. Ferner gehören hierher die Bildungssuffixe: ιος, ία, ιον, sk. jas, jā, jam, als: ἅγ-ιος, sk. jaǵ-jas (colendus), πάτρ-ιος, sk. pitr-jas, l. patr-ius, ἐλευθέριος, κύριος (thessal. κῦρρος nach § 21, 6), μέτριος (lesb. μέτερρος mit eingeschobenem ε, für μέτερjος),108) παράλιος, χρόνιος u. s. w.; Kompar. ίων, ιον, l. ior, ius, sk. ījān, als: ἡδ-ίων, sk. svād-ījāns, l. suav-ior, in anderen Fällen im Griechischen als jων behandelt: μᾶλλον f. μάλjον, ἥττων f. ἥκjων u. s. w.; Verbalbildungen, als: ἰδ-ί-ω (σϝιδ-ί-ω), sk. svidjā-mi, δαίω brenne, aus δάϝjω, ναίω aus νάςjω, Hom. lesb. τελείω aus τελέςjω, εἴην (d. i. ἐς-ίην, sk. (a)s-jām, l. (e)s-jem). b) j wird scheinbar zu ε, d. h. es entwickelt sich vor dem j ein ε, welches nach Ausfall desselben übrig bleibt: so in dem dor. Futurum auf -σέω, sk. sjâmi, welche Gleichstellung indes vielfach bestritten wird, in πόλεως (aus πόληος) πόλεος = πόλεjος, ion. u. s. w. πόλιος. c) anlautendes j wird ζ (= sd, mit weichem s oder franz. z), d. h. es entwickelt sich vor j eine palatale, dann dentale Media, und dies δj wurde dann wie sonstiges δj (s. § 21, 1) behandelt; vgl. das ital. già (spr. dscha) aus dià mit l. jam, giacere aus diacere mit l. jacere; z. B. ζεύγ-νυμι, sk. ju-naǵmi, l. ju-n-go, ζυγ-όν, l. jug-um, d. Joch, ζέω ([root ] ζες), siede, sk. [root ] jas, nir-jas, ausschwitzen, ahd. jes-an, nhd. gären, ζειά, Dinkel, Spelt, sk. java-s, Gerste, ζώννυμι, gürte (f. ζώς-νυμι), vgl. Zend jāstō, gegürtet; zweifelhaft ist derselbe Vorgang im Inlaut, wo ihn Curtius für die Verben auf άζω, ίζω annimmt, sk. ajāmi. 

{\small\noindent\subparagraph{} In Dialekten wie dem böotischen ist von dem aus
  j hervorgegangenen ζ nur δ zurückgelassen; so böot. Δεύς st. Ζεύς, δυγόν st.
  ζυγόν; auch Homer hat statt des Präfixes ζα = σδα in einigen Wörtern nur δα:
  δα-φοινός, sehr rot (ζαφ. geht nicht in den Vers), δά-σκιος, sehr schattig
  (σδάσκ. übellautend).}

  d) anlautendes j wird zum Spir. asper, als: ὅς, ἥ, ὅ,
  sk. jas, jā, jad, ὡς, sk. jāt, ὑμᾶς, sk. jushmân, ἧπαρ, sk. jakrt, l. jecur,
  ἅγ-ιος, ἁγ-νός, ἅγ-ος, ἁγ-ίζω, sk. jaǵ-âmi (opfere, ehre), jaǵ-us (Gebet),
  jaǵjas (colendus), ὥρα, goth. jēr, ahd. jār (n. A. hatte ὥρα Digamma, doch
  ἄνωρος Gortyn. Taf., G. Meyer, Gr. S. 216\textsuperscript{2}), ὑσμίνη, Kampf,
  [root ] ὑθ, sk. judh-mas, Kampf, Kämpfer. e) Spurlos verschwunden ist
  anlautendes j in den Dialekten, die den Spir. asper nicht kennen, z. B. äol.
  ὔμμες; im Inlaute öfter, so wie man glaubt in den Verben auf άω, έω, sk.
  ajā-mi, als: φορέω, sk. bhārájāmi, Fut. auf σω st. ςjω, dor. σέω, σίω, sk.
  sjā-mi, Gen. Sing. auf ου (entst. aus οιο), als: ἵππου aus ἵπποιο, sk. a[cnull
]va-sja.

\section[Jod in Verbindung mit Konsonanten]{Jod in Verbindung mit Konsonanten\protect\footnote{}}

\paragraph{} δ mit j wird att., ion., dor., lesb. ζ = σδ (gleichwie auch das Altslovenische regelmässig dj zu žd werden lässt); im Anlaute, als: Ζεύς d. i. Δjεύς, sk. djāus, Himmel, Himmelsgott, altl. Diov-is, woraus mit pater Jū-piter wurde; lesb. auch ein gmgr. δι vor Vokal, so Ζόννυσος = Διόνυσος, ζά = διά, ζὰ νυκτός = διὰ νυκτός, ζάβατος = διαβατύς, vgl. ep. das ζα intensivum, als: ζάθεος, ganz göttlich; im Inlaute, wie bei den Verben auf ζω, deren Stamm auf δ ausgeht, als: κομίζω (vgl. κομιδή), φράζω ([root ] φραδ); aber ἑζόμην ist ἑς (ε) δόμην; ferner πεζός st. πεδ-ιός, κάρζα äol. st. καρδία, ῥίζα aus ϝρίδjα, σχίζα aus σχίδjα neben σχίδη. 

{\small\noindent\subparagraph{} Die anscheinende Verschmelzung von δj in σς (ττ)
  findet sich in einzelnen dialektischen Bildungen, als: πέσσον äol. st. πεδίον,
  (σαλπίσσω u.) φράσσω tarent. st. (σαλπίζω) φράζω; kret. ἐσπρεμμίττεν =
  ἐκπρεμνίζειν, Τῆνα, Ττῆνα = Ζῆνα; thessal. ἐνεφανίσσοεν, d. i. ἐνεφάνιζον. S.
  Curt. a. a. O. 672\textsuperscript{5} f. Es ist dabei δj in τj übergegangen,
oder es liegt ursprünglich τj, κj vor. }

\paragraph{} γ mit j wird gleichfalls ζ, indem γ vor
  j in δ übergeht; vgl. αζ-ομαι aus αγ-jομαι (sk. jaǵ-āmi, opfere, ehre) neben
  ἅγ-ιος, κρζ-ω (α lang, Herodian I, 442. 535, II, 929) neben κέ-κρᾶγ-α, σταζ-ω
  neben σταγ-ών, ὀλολύζω neben ὀλολῦγή u. s. w.; μείζων (ion. arkad. μέζων) aus
  μέγ-jων, v. μέγας (über d. ει st. ε vgl. Nr. 3, 4, 6), vgl. mag-nus, mājor st.
  magjor, ὀλίζων ep. aus ὀλίγjων, att. ὀλείζων, wo ebenfalls ι oder ε
  überflüssig erscheint, s. § 155, φύζ-α (aus φύγjα) neben φυγ-ή, l. fug-a. Ein
  Nasal vorher verschwindet: σαλπίζω d. i. -ί (ν) σδω — σάλπιγξ, πλαζω —
  ἔπλαγξα, κλαζω κλαγγή. S. Hdn. II, 399. Bei Verben ist nicht selten für γjω
  σσω, ττω eingetreten, als τάττω, St. ταγ, πράττω, St. πραγ (kret. aber regelm.
  πράδδω, δδ für att. ζ), σάττω St. σαγ (kret. σάδδω), πλήσσω, St. πληγ, πλαγ
  (lesb. πλζω), μάττω vgl. μάγειρος μᾶζα (bei welchem letzteren Herodian II, 937
  die Länge des α als Ausnahme hervorhebt, vgl. oben μείζων u. s. w.; lang war
das α auch in Ἀμᾶζών [aber μαζός] ἀλᾶζών, Hdn. I, 28 u. a. St.). 

{\small\noindent\subparagraph{} Diese Verschmelzung des γj in σς (ττ) ist z. T.
  auf Nebenformen mit κ zurückzuführen; vgl. πλήσσω (St. πληγ, πλαγ), A. P.
  ἐπλήγ-ην, πληγή, aber auch [root ] πλακ, vgl. πλάξ, πλακ-ός, σάττω (St. σαγ),
  σάγ-η neben σάκος, φράσσω (St. φραγ) neben l. farc-io, ῥήσσω b. Hippokr. (St.
ῥαγ) neben ῥάκος u. s. w.110) }

{\small\noindent\subparagraph{} In ἔρδω, ϝέρδω, [root ] ϝεργ, also aus ϝέργ-jω
  ist γj in δ übergegangen, indem ϝέρζω = ϝέρσδω sich nicht sagen liess; dagegen
nach Vokal ῥέζω ϝρέσδω aus ϝρέδjω ϝρέγjω.111) S. Hdn. II, 399.}

\paragraph{} κ, χ mit j
  werden σς (att. böot. ττ), indem κ und χ vor j zunächst in τ und θ übergehen;
  aus τj, θj wird dann nach Nr. 4 (τς) ττ oder σς, als: ἥσσων (ἥττων) aus
  ἥκj-ων, vgl. ἥκιστα, γλύσσων b. Xenophanes aus γλύκjων, φρίσσω (φρίττω) aus
  φρίκjω St. φρικ, πίσσα aus πίκjα, vgl. pix, pic-is, ἐλάσσων (ἐλττων) aus
  ἐλάχjων, vgl. ἐλάχιστος, θάσσων (θττων) aus τάχjων v. ταχύς (beide im
  Attischen mit einer in der Regel nicht bei diesen Übergängen vorkommenden
  Dehnung, vgl. μείζων u. s. w., Nr. 2, 4, 6); βράσσων b. Hom. aus βράχjων,
  βήσσω (βήττω) aus βήχjω, St. βηχ, vgl. βήξ, G. βηχ-ός, Φοίνισσα aus Φοίνικjα;
  so wird auch aus κτj σς: ἄνασσα aus ἄνακτjα (κj = ζ in βάζω, [root ] βακ, vgl.
  ἀβακ-έω ἀβακ-ής). Nach Konson. indessen entsteht κτ in φάρκτεσθαι (Phot.),
  φάρκτου (Hesych.) = φράττεσθαι φράττου, St. φραγ, φρακ, Siegismund, Curt.
  Stud. V, 159. 

  \paragraph{} τ oder θ mit j wird τς, σς (att. böot. ττ), indem τj und θj
  zunächst in τς übergehen, sodann entweder regressive oder progressive
  Angleichung eintritt, als: μέλιτjα wird μέλιτ-σα, dieses wird durch regressive
  Angleichung μέλις-σα, durch progressive μέλιττα, κρείσσων, κρείττων aus
  κρέτjων, vgl. κρατύς, κράτιστος (über d. überflüssige ι vgl. Nr. 2, 3, 6),
  Κρῆσσα aus Κρῆτjα, ἐρέσσω aus ἐρέτjω, vgl. ἐρέτ-ης, λίσσομαι aus λίτjομαι,
  vgl. λιταί; κορύσσω aus κορύθjω, vgl. ep. κε-κόρυθ-μαι, κόρυς, κόρυθος,
  βυσσός, ὁ, aus βυθjός, vgl. βυθός, βάσσων dor. aus βάθjων; ντ mit j wird (mit
  Verlust des ν) σς in der Femininform der Adjektive auf εις, εσσα, εν, als:
  χαρίεις, χαρίε (ν) τjα = χαρίεσσα; aber ς in der Femininform der übrigen Adj.
  und Partic. auf ντ, als: πάντjα = (πάνσα, so thessal. u. s. w.) πᾶσα, στάντjα
  = στᾶσα, βουλευθέντjα = βουλευθεῖσα, γράφοντjα = γράφουσα, δεικνύντjα =
  δεικνῦσα, ἑκόντjα = ἑκοῦσα. 

  \paragraph{} πj, φj werden πτ oder σς (att. ττ),112) entspr. βj
  (βδ oder) ζ, in folgender Weise: a) Entweder wird zwischen den Lippenlaut und
  j ein Zahnlaut eingeschoben, hinter dem j ausfällt; zunächst liegt der
  Zahnlaut δ, der sich gern mit j verbindet, s. § 20, c). Die Tenuis π konnte
  sich vor δ in β erweichen, wie viell. in ῥάβ-δ-ος aus ῥάπ-j-ος, [root ] ῥαπ,
  vgl. ῥαπ-ίς; aber gewöhnlich trat Assimilation der Media δ hinter π und φ (=
  π) ein, d. h. die Media δ wurde die Tenuis τ, die Aspiration aber verschwand
  wie sonst vor τ (κέκρυπται). So wurde aus τύπ-δj-ω erstens τύπ-τj-ω, sodann
  nach Wegfall des j τύπ-τ-ω, aus σκέπ-δj-ομαι, sk. pa[cnull ]j-âmi, l.
  spec-i-o, σπέπ-τj-ομαι, σκέπ-τ-ομαι, aus κρύφ-δj-ω κρύπ-τ-jω, κρύπτ-ω. Βλάπτω,
  St. βλαβ hat einen Nebenstamm auf π (kret. βλάπω, ἀβλοπές = ἀβλαβές), νίπτω
  ist späte Analogiebildung statt νίζω, aus ἔνιψα, νίψω. Für βδ aus βj führt man
  ῥοῖβδος an, das neben ῥοῖζος steht; Curtius freilich führt diese Formen auf
  ῥοῖϝjος zurück. b) Oder es wird aus πj τj (viell. durch πτj hindurch), aus βj
  δj, und weiter nach Nr. 3 aus τj σς, ττ, nach Nr. 2 aus δj, ζ, wobei indes
  nicht aus der Acht zu lassen, dass zu den betr. Worten grossenteils entweder
  im Griech. selbst oder doch in den verwandten Sprachen zugehörige Bildungen
  mit Guttural- (Palatal-) Laut existieren. So ὄσσε, ὄσσομαι, vgl. ὄψις, ὄψομαι
  u. s. w., ἀμβλυώσσειν, vgl. ἀμβλυωπός, aber auch böot. ὄκταλλος = ὀφθαλμός,
  sk. ak-sham, ak-shi, Auge, l. oculus; ὄσσα, ϝόσσα (ϝέπος, ϝόψ, ϝειπεῖν), vgl.
  sk. vi-va[kacute]-mi, rufe, va[kacute]-as, Wort, l. vox, G. vocis; φάσσα, vgl.
  φάψ, φαβ-ός, wilde Taube; πέσσω (πέπων, πέψω), sk. pa[kacute]-āmi, l. coqu-o,
  ἐνίσσω (ἐνιπή, ἠνίπαπε, ἐνένιπε); — νίζω, wasche, Fut. νίψω ([root ] νιβ,
  χέρνιψ, G. χέρνιβ-ος), vgl. sk. nê-nêǵ-mi, reinige; ion. u. Hom. λάζομαι,
  λάζυμαι (= λαμβ-άνω, [root ] λαβ, doch s. G. Meyer 198).113) 

  \paragraph{} λ mit j wird
  durch progressive Angleichung λλ, als: φύλλον aus φύλjον, vgl. folium, ἄλλος
  aus ἄλjος, vgl. alius, μᾶλλον aus μάλjον v. μάλα (wegen der att. Dehnung s.
  Nr. 2, 3, 4), vgl. melius; ἅλλομαι aus ἅλjο- μαι, vgl. salio, στέλλω aus
  στέλjω, wie stellan ahd. st. steljan; ebenso verschmilzt lesb. thessal. ρj
  nach ε, ι, υ zu ρρ und νj zu νν, als: φθέρρω (att. φθείρω) aus φθέρjω, κτέννω
  (att. κτείνω) aus κτένjω, κρίννω aus κρίνjω, οἰκτίρρω aus οἰκτίρjω, ὀλοφύρρω
  aus ὀλοφύρjω, vgl. goth. than-ja, dehne aus, ahd. dennan st. denjan. 

  \paragraph{} Nach αν
  oder αρ (ορ) tritt Epenthese ein, d. h. j tritt als Vokal ι in die
  vorangehende Silbe und verschmilzt mit dem Vokale derselben zu einem
  Diphthongen, als: μέλαινα aus μέλανjα, μάκαιρα aus μάκαρjα, μοῖρα aus μόρjα;
  φαίνω aus φάνjω, σαίρω aus σάρjω. Anscheinend ist dasselbe auch nach εν, ερ im
  Attischen, Ion., Dor. der Fall: χείρων, φθείρω, κτείνω u. s. w.; aber in
  φθείρω wird ει vielmehr Dehnung sein (unechtes ει), entspr. dem lesb. φθέρρω
  und dem arkad. φθήρω;114) ebenso ist für χείρων χέρρων äol., und wir werden
  auch κτείνω (äol. κτέννω) nicht anders beurteilen, auch nicht τέρεινα, wofür
  sich lesb. (Alc. 61) τερένας mit vereinfachter Gemination findet. Andererseits
  hat ἀμείνων nach Ausweis der alten Inschr. echten Diphthong; Δάειρα (Bein. der
  Persephone) wird zu Δαῖρα; oder ist dies aus Δάαιρα vgl. ἰοχέαιρα, und πρῷρα
  nicht aus πρώειρα (Herodian II, 410), sondern aus πρόαιρα vgl. νείαιρα? — Bei
  ιν, ιρ, υν, υρ ist Dehnung: κρνω, οἰκτρω, ἀμνω, φρω. Das Kyprische zeigt
  Epenthese auch bei αλ: αἶλος = ἄλλος, καιλαί = καλαί.115) Dehnung bei λ hat
  das att. ὀφείλω ΟΦΕ[lins ]Ο (aber ὀφέλλω, vermehre, in d. gew. Weise; bei Hom.
  auch ὀφέλλω, schulde). 

  \paragraph{} Aus μj ist in einzelnen Fällen anscheinend νj geworden
  und dies dann in üblicher Weise umgewandelt: βαίνω [root ] gam sk., χλαῖνα
  vgl. χλαμύς. Sodann aber wandelt sich wie Ahrens, Formenl. § 157, e wohl mit
  Recht annimmt, das μj auch in μν; d. h. es schiebt sich der für j bequemere
  N-Laut zwischenein, und j fällt dann aus: so ist auch Ῥωμαῖος, μία in
  neugriech. Volksaussprache romnjós, mnja. Dahin also τέμνω = τέμjω, κάμνω =
  κάμjω, während bei νέμω, δέμω, γέμω, τρέμω keine Form mit ν gebildet ist, d.
  h., wenn es ein νέμjω gab, das j einfach ausfiel. 

  \paragraph{} Aus den Verbindungen ςj, ϝj
  wird, durch Ausfall des ς u. ϝ, vokalisches ι, welches sich mit dem
  vorgehenden Vokale verbindet. So καίω aus κάϝjω, κλαίω aus κλάϝjω (att. mit
  Dehnung des α vor dem ausfallenden ϝ: κιω, κλιω, κω, κλω), γραῖα aus γράϝjα,
  γλυκεῖα aus γλυκέϝjα, δῖος aus δίϝjος; ναίω aus νάςjω (vgl. νάστης), ἀληθεςjα
  ἀλήθεια. Doch ἀήθεσσον Il. κ, 493 (St. ἀηθες), πτίσσω, St. πτις, νίσσομαι, St.
  νις νες (nach Osthoff, d. Verb. in d. Nominalkompos. S. 339 ff. von πτίνςjω,
  νί-νςjομαι), Λίβυσσα aus Λίβυςjα von Λίβυς, vgl. Λιβυστικός. (Daher auch böot.
Λίβυσσα, nicht -ττα.) 

{\small\noindent\subparagraph{} Der Prozess der Angleichung des ι = j nach
  Konsonant hat sich im Thessalischen fortgesetzt: es findet sich hier a)
  Angleichung, b) wo der Konsonant nicht verdoppelt werden kann, Ausstossung des
  ι, c) in der Schreibung auch Verdoppelung des Konson. mit Beibehaltung des ι,
  d. h. Übergang zur Assimilation. Beispiel für a): κῦρρος = κύριος, Eigenn.
  Ἄσσας = Ἀσίας, Ἀγάσσας, Ἀμείσσας wohl = Ἀμειψίας, Μνάσσας = Μνασίας (Bull. de
  corr. hell. 1889, 400); für b) Ἄντοχος nb. Ἀντίοχος, τρακάδι = τριακάδι (auch
  Διοννύσοι, Eigenn. Dial.- Inschr. 1329 muss = Διονυσίοι sein); für c)
  Παυσαννίαιος, D.-I. 1286, 12, Παυσαννίαο Mitt. d. arch. Inst. 1889, 59 f.,
  καθʼ ἱδδίαν D.-I. 361, Β, 12, προξεννιοῦν προξεννίαν aber προξένοις Bull. l.
  c., πόλλιος aber πόλι das. Vgl. auch das Epigr., Bull. de corr. VII, 61:
  σώιζων μὲν πίστιν, τιμῶν δ ἀρετὰν θάνες ὧδε Καλίας (—) Ἀρκὰς ἐὼν πατρίδος ἐκ
Τεγέας. Vgl. Prellwitz, Bzz. Btr. XIV, 298 ff.}

\section[Spiritus asper und lenis]{Spiritus asper und lenis\protect\footnote{}}\label{sec:22}

\paragraph{} Der Spiritus asper ist, soweit er überhaupt eine etymologische
Begründung hat (vgl. unten 11), aus dem Spiranten ς, sowie vereinzelt aus den
Halbvokalen ϝ und j hervorgegangen. (S. §§ 15, 1; 16, 3, a, δ; 20, d.) Sowie
aber die griechische Sprache eine grosse Abneigung gegen die eben genannten
Laute hatte, so sehen wir, dass auch der Spiritus asper selbst vielfach weichen
und in den Spiritus lenis übergehen musste. Diese Erscheinung zeigt sich in den
Dialekten von der frühesten Zeit ab, bis zuletzt der Spiritus asper in der
Kaiserzeit auch aus der Gemeinsprache allmählich verschwand, wie er denn im
Neugriechischen gänzlich fehlt. Schon bei Homer tritt bei mehreren Wörtern
gleicher Abstammung ein Schwanken zwischen dem Asper und Lenis hervor, s. Nr. 8.

\paragraph{} Der lesbische Aeolismus117) entbehrt von den frühesten bekannten
Zeiten an des rauhen Hauches, wie die Grammatiker lehren118) und die Inschriften
bestätigen. Beispiele aus den letzteren sind: κατεστακόντων, μετʼ Ἠρακλείτω,
κατείρωσις d. i. κατῖρ. = καθιέρωσις, κατιδρύσει; ferner aus den in Steinschrift
erhaltenen Epigrammen der Balbilla: τὀ, κὠ, κὦσσ, τʼ ὄσ, τότʼ ἄλις. Ahrens
irrte, wenn er ein Grammatikerzeugnis119) dahin deutete, dass die Aeolier nur ἐν
ταῖς ἰδίαις λέξεσιν (den speziell äolischen Worten) keinen Asper gehabt hätten,
und sich nun um die Aufstellung von Gesetzen und Regeln bemühte, nach denen der
Asper bald gefehlt hätte, bald nicht. Was in den Fragmenten der äolischen
Dichter sich Entgegengesetztes zeigt, muss korrigiert werden (das alte Berliner
Sappho - Fragment hat ΟΥΚὈΥΤΩ); was auf Inschriften der alexandrinischen und
römischen Zeit, gehört mit zu den vielen Verfälschungen durch die κοινή. 

{\small\noindent\subparagraph{} Nach den alten Grammatikern haben die Aeolier
auch bei ρ, ρρ die Psilose angewandt.}

\paragraph{} Hingegen haben die anderen äolischen Mundarten, das Böotische und
Thessalische, den Spiritus asper gehabt. Für das erstere120) bezeugen dies
zahlreiche Belege auf Inschriften, als ΗΑΓΕΣΑΝΔΡΟΣ, ΗΙΣΜΕΝΙΟΙ (in diesen Namen
von Ἱσμηνο-, als Ἱσμείνα, Ἱσσμεινίας u. s. w., steht immer der Asper [ausser der
verdächtigen Inschr. D. I. 698], so auch eine korinthische Vase ΗΥΣΜΕΝΑ, während
wir bei den Attikern Ἰσμήνη, Ἰσμηνίας schreiben, ohne Grund und Beweis); auch
die Grammatiker reden bei diesem Dialekte von einem Asper (s. über ἱών unten §
160). Über anlautendes υ s. unten 11; Ἀρίαρτος ist die echte einheimische Form
für Ἁλίαρτος. — Das Thessalische weist in den Inschriften καθʼ ἱδδίαν u. s. w.
auf.\footnote{} 

\paragraph{} Die dorischen Mundarten haben nicht alle den Asper, die meisten
indes wenden ihn an, wenn auch mit einigen Eigentümlichkeiten.122) Auf den
herakleischen Tafeln, die trotz des im übrigen ionischen Alphabets doch die
Zeichen ϝ und [rpress] bewahrt haben, findet sich der Asper im ganzen wie in der
gewöhnlichen Sprache, auch mit der Verwandlung der Tenuis in die Aspirata, als:
ὁ, ἁ, οἷος, ὅσος, ὡς, ἅμα, ἁμές (= ἡμεῖς), ἕκαστος, ἑκάτερος, ἱαρός (= ἱερός),
ὁρᾶν, ἔφορος, ἐφορεύω, ὕδωρ, ὑπό, ὑπέρ u. s. w.; ἕργω (aus Fέργω entst.) wie im
Att., ἀφέργω, ἐφέργω; auch in der Mitte des Wortes (in den Kompositis) ist er
fast stets hinzugefügt (ausser wo die Tenuis in die Aspirata verwandelt ist, s.
§ 23, 3), als: ἀνἑλόμενος (neben ποθέλωνται), συνἕργω; — aber Asper st. d.
Lenis: ἁκροσκιρίαι Ι, 65. 71 (so καθʼ ἅκρον Korkyra Dial.-Inschr. 3204), ἅρνησις
Ι, 156, ὁκτώ, ἑννέα (nach ἑπτά), οἱσόντι (= οἴσουσι) Ι, 150 (in ἵσος Ι, 175 ist
der Asper Vertreter des ϝ, ϝίσος, aber ἴσος Ι, 149. 170); [rpress]ιάσων bietet
eine Vase (Thumb p. 10); hingegen Lenis statt des Asper: ὄρος mit den Derivatis
(aber ΗΟΡFΟΣ Korkyra), ἀλία (Versammlung), ἀμαξιτός wie Homer (att. ἁμαξ.,
arkad. παρ-αμαξεύη ohne Θ geschrieben); über Fέξ st. ἕξ s. § 16, 2 a). 

\paragraph{} Der lakonische Dialekt bietet zahlreiche Beispiele des (in den
älteren Inschriften) als H geschriebenen Asper, im Anlaut wie auch im Inlaut
(vgl. § 23, 2). Darnach ist falsch unsere Schreibung Ἀγησίλαος (Inschr.
Ηαγηἵστρατος, Ηαγησιλα-, Papyrus des Alkman Ἁγησιχόρα, aber Ἀγιδώ); dagegen
ἀνιοχίων Inschr. des Damonon für ἡνιοχῶν; auch eine korinth. Vase (Dial.-Inschr.
3151) Ἀνιοχίδας; ὥιτ ἄλιον Alkman Papyrusfr. Col. II, 7. Umgekehrt Ὁπωρίς eine
wahrscheinlich lakonische Inschrift, Röhl, J. Gr. ant. nr. 61, vgl. χὠπάραν d.
i. καὶ ὀπώραν Alkm. fr. 76. Auf Alkman wird sich gleichwohl beziehen, was
Apollonius synt. 335, b bemerkt, dass “unzählige Male” im Dorischen der Spiritus
bei der Synalöphe vernachlässigt werde, unter Anführung von Beispielen wie κὠ,
κἀ; Bergk meint dies auf alte Schreibungen ΚΗΟ, ΚΗΑ zurückführen zu können (Lyr.
III\textsuperscript{4} p. 697), wie sich in der That Dial.-Inschr. 3170
(Phleius) ΑΙΤΗΟΡ [koopa ] ΟΝ findet. Ausserdem verzeichnen wir noch folgende
dorische Beispiele des Asper st. des Lenis oder des Lenis st. des Asper: a)
ἐφακείσθων u. ἐφακεῖσθαι (v. ἀκεῖσθαι) Delph. 1688, 37. 41; καθʼ ἑνιαυτόν Ther.
2448, VI, 25; κασάνεις b. Hesych. lak. st. καθάνεις v. ἅνω, vgl. att. ἁνύω st.
ἀνύω; ἐγκαθιδών (Spir. st. Dig.) und πένθἕτη Epidauros; ἕστε st. ἔστε Theokr. 1,
6. 6, 32. 7, 67 n. Ahrens; — b) auf einer alten argiv. Inschr. Röhl 30 steht der
Artikel Ηο, aber Ἰπομεδον (= Ἱππομέδων), vgl. tarent. ἴκκος = ἵππος, equus; das.
nr. 37 οπλίταν; auf e. theräisch. Inschr. Röhl 438 Ιαρον (= Ἱάρων), aber Ἱάρων
auf d. Helme des Hiero, Röhl 510; ἱαρός auf d. herakl. Taf. u. auf lak. Inschr.
R. 14, 38, 39\textsuperscript{a}, 39\textsuperscript{b} u. s. w. (arkad.
ἱερομνάμων u. s. w.); doch auch megar. ἐπʼ ἰαρεῦς D.-I. 3025 oft Rhodos, s.
Thumb S. 24, Akarnanien (das. 39); ferner Ἀρμονόα Ambrakia Röhl 331 (vgl.
Ἀρμοξίδαμος Petilia das. 544); in Messene Cauer, Del.\textsuperscript{2} 47 κατʼ
ἀμέραν (Z. 102), vgl. lokr. αμάρα D.-I. 1478, 1479, desgl. arkad. Bull. de corr.
hell. 1889, 281, Z. 9, 13, 16, ἐπάμερος, πεμπάμερος Pind., episch ἦμαρ; mess.
das. (Z. 55, 116) κατεσταμένοι, womit Thumb S. 12 das gew. ἔσταλκα neben
seltenerem ἕσταλκα vergleicht; vereinzelt Kalymna C. I. Gr. 2671 μετʼ ομονοίας;
Epidaur. Ἐφ. ἀρχ. 1885, 65 ff. (D.-I. 3342), Z. 26 κἀμῶν.123) 

\paragraph{} Von den dorischen Mundarten hat das Kretische von alter Zeit her den Asper ganz eingebüsst, wenn auch dies sich nicht auf alle Städte gleichmässig erstreckt; in Hierapytna wenigstens wurde, wie Herodian zu Od. γ, 444 aus der Schrift eines von dort gebürtigen Grammatikers anführt, noch späterhin αἱμνίον mit Asper gesprochen. In den ältesten Inschriften, die wir aus Kreta haben, wie den gortynischen Tafeln, findet sich das H entweder überhaupt nicht, oder nur in vokalischer Geltung; auch in der Synalöphe bleibt die Tenuis. Vgl. Thumb S. 24 ff. Ferner ist in den altlokrischen Inschr. zwar das H als Spiritus gebraucht, aber der Hauch äussert in der Synalöphe keine Wirkung; er möchte also dem Erlöschen nahe gewesen sein (das. 39). Hierzu stimmt, dass in der dem Lokrischen nahe verwandten eleischen Mundart, wie die alten olympischen Inschr. beweisen, der Hauch von alters her nicht mehr vorhanden war (das. 32 f.). Im Arkadischen dagegen ist Spiritus asper gewesen (Meister, Dial. II, 103; hervorzuheben Ἀγεμώ = Ἡγεμόνη; auf der Inschr. von Tegea, Bull. de corr. a. a. O. (oben 5) ἀμέρα s. o.; einmal Ηάν = ἄν Z. 9, doch ἄν 5; ὄτι 5, 9, ΟΣ = ὅς od. ὡς 9, παρΗεταξαμένος 20), während das mit dem Arkadischen so nahe verwandte Kyprische ihn nicht schreibt und ein h wie es scheint erst aus s neu entwickelt hat (das. 240 ff.). 

\paragraph{} Mit dem asiatischen Aeolismus stimmt in betreff der Psilosis auch
die neuionische Mundart Asiens überein.124) Aber auch schon in dem alten
Ionismus Homers zeigen sich ziemlich viele Spuren der Psilose, indem aspirierte
Wörter in gewissen Formen den Asper in den Lenis übergehen lassen.125) Übrigens
ist zu beachten, dass mehrere Wörter, die in unserem Texte mit dem Asper
anlauten, bei Homer mit dem ϝ anlauteten, als: ἁνδάνω, ἕκηλος, und ferner, dass
die Setzung von Spir. asper oder lenis in diesen Gedichten von den Grammatikern
herrührt, die nur in den Fällen von Synalöphe hie und da einen handschriftlich
überlieferten Anhalt hatten, übrigens aber entweder nach Analogien entschieden,
oder darnach, ob ein Wort etwa aus diesem oder jenem Dialekte zu stammen schien.
Einzelnes: ἁθρόος Aristarch u. Herodian wie att. (La Roche, Hom. Textkr. 180);
desgl. ἁραιός (aber ἀραῖος), das. 201; andere schrieben ἀθρόος, ἀραιός; Streit
war auch über ἅδην — ἄδην ἄδδην, ἁδινός — ἀδινός (das. 178 ff.). Den Lenis hat
ἠέλιος (lak. ἄλιος oben 5, ὑπʼ ἀλίωι Korkyr. C. J. Gr. 1907 = Kaibel, Epigr.
185); ἀλέη Sonnenwärme, att. ἁλέα εἵλη; ἤμβροτον ἀβροτάζω nb. ἁμαρτάνω (ἀμβρότην
äol.); von ἅλλομαι lautet der Aor. II. M. α<>λσο (ἄλσο), ἆλτο, ἄλτο (ἐπᾶλτο Il.
φ, 140), ἄλμενος; ἐπ ἄμαξαν Il. μ, 448, κατʼ ἀμαξιτόν χ, 146 (La Roche 187),
att. θἀμάξιον Eust. 1387, 10126; neben ἅμα ἄμυδις (äol.) sehr oft; ἄμμες (=
ἡμεῖς), ἄμμε (= ἡμᾶς), ἄμμι (ν) (= ἡμῖν), desgl. äol.; neben ἁνδάνω (ϝσανδάνω)
ἦδος (als äolisch, La Roche 270, doch Tryphon ἧδος); neben ἕκηλος (ϝέκηλος)
εὔκηλος (aus ἐϝέκηλος, s. § 18); neben Ἑωσφόρος ἠώς, neben ἡμέρη ἦμαρ; — neben
ἱδρώς ἴδιον (Impf. v. ἰδίω) Od. υ, 204; ἴρηξ (Spitzner ad Il. ο, 237, ς, 616);
ἴστωρ (ϝίστωρ) Spitzner ad Il. ς, 501; ἐπίστιον Od. θ, 250; — neben ὁμοῦ ὄ-αρ,
ὀ-αρ-ίζειν, ὀ-αρ-ιστύς, ὀ-αριστής, ὄ-θριξ, οἰ-έτης (d. i. ὀ-ϝέτης) Il. β, 765,
ὄ-πατρος; neben ὁδός οὐδός, ἡ, Weg, Od. ρ, 196, αὐτόδιον, selbigen Weges, θ,
449; st. ὅλος (sk. sarvas = omnis, altlat. sollus) οὖλος; οὖρος, Grenze, st.
ὅρος. — Bei den nachfolgenden Epikern findet sich: Hymn. Cer. 88, Hes. Sc. 341
ὑπ ὀμοκλῆς st. ὑφʼ ὁμ.; Hes. Th. 830 u. Hymn. Hom. 27, 18 ὄπʼ ἰεῖσαι st. ὄφʼ
ἱεῖσαι (s. Goettling ad Hes.); Hes. Op. 559 τὤμισυ st. θὤμισυ v. τὸ ἥμισυ, ἐπ
Ηφαίστοιο θύρῃσιν ein Dichter bei Hdn. II, 839, mit besonderer Entschuldigung. 

\paragraph{} Dass in dem neueren Ionismus, wie er in Herodot und andern
ionischen Prosaikern (ausser Hippokrates,127) Renner, Curtius Stud. I, 1, 151,
Littré, Hipp. I, 494 f., 499) vorliegt, der Asper nicht mehr vorhanden gewesen
sei, sieht man daraus, dass er auf eine vorangehende Tenuis mit nur sehr wenigen
Ausnahmen in Kompositis, die wir anführen werden, keine Wirkung äussert, als:
οὐκ ὁμολογέουσι Her. 1, 5, ἀπαμμένους 2. 121, 4, ἐπέδρης 1, 17, ἐπέδρην 5, 65
(aber ἔφεδρον 5, 41, ἀφεῖτο 8, 49, was man beides ändert; doch 7, 193 ἐντεῦθεν
γὰρ ἔμελλον ὑδρευσάμενοι ἐς τὸ πέλαγος ἀφήσειν, ἐπὶ τούτου δὲ τῷ χώρῳ οὔνομα
γέγονε Ἀφέται scheint ἀφήσειν notwendig wegen Ἀφέται, das Herodot stets so
schreibt), κατύπερθε 2, 5, κατάπερ 1, 118, 131, 169, κατά (st. καθʼ ἅ) 1, 208
(aber 9, 82 κατὰ ταυτὰ καθώς, pariter ac, wofür Dind. κατώς, Bred. S. 93 κατά d.
i. καθʼ ἃ, Schäfer καὶ, Stein ὡς καὶ lesen will, aber Athen. IV. 138 c hat auch
καθώς), κατίσαι 2. 121, 5 (aber μέθες 1, 37, καθεύδει 2, 95, καθεύδουσι 4, 25 in
den codd.). Die Komposita, namentlich die alten und festen, haben immerhin ihre
Ausnahmestellung; denn auch in Elis finden wir ποθελομένω, und auf ionischen
Inschriften καθημένου (Teos), κάθοδον (Halikarn.), μεθέληι (Chios), dagegen
πέντʼ ἠμέρηισιν (Chios), ἀπʼ ἐκάστου (Milet), indem mit dem Verluste des
Spiritus von ὁδός, ἕληι nicht notwendig die Ersetzung des θ durch τ in den
Kompositis verbunden war. Die handschriftliche, von uns fortgepflanzte
Schreibung des Asper bei Herodot ist widersinnig und sollte aufgegeben
werden.\footnote{}

\paragraph{} Betreffs des Ionismus der westlicheren Inseln im ägäischen Meere
liegt die Sache wesentlich anders. Euböa hatte den Spiritus so gut wie Attika,
wie die Inschriften beweisen. Für den Parier Archilochos bezeugt Athenäus III,
107, f die Schreibung ἐφʼ ἥπατι (fr. 131\textsuperscript{4} Bgk.), aus welcher
er den Asper von ἧπαρ erweisen will; auch die sonstigen Fragmente stimmen dazu,
nur 70 ἐπ (v. l. ἐφ) ημέρην, 115 ἐπʼ ήβης. Vgl. Fick Bzz. Btr. XI, 246 f. Die
Inschr. der Kykladen bewahren gleichfalls Zeugnisse des Asper: Delos und Naxos
ὁ, ἑκηβόλωι mit H, Keos ἐφίστια, Amorgos Ἱπποκράτης, Ἱπποκλῆς, Siphnos ἱερόν. 

\paragraph{} Der Atticismus bildet zu dem Aeolismus und zu dem Ionismus Asiens einen strengen Gegensatz, indem er eine grosse Vorliebe für die Aspiration hat. So haben mehrere Wörter im Attischen den Asper, die in anderen Dialekten (oder in der κοινή) den Lenis haben, als: ἁνύω, ἁνύτω nach der Vorschrift der alten Grammatiker,129) doch steht Eur. Bacch. 1100 οὐκ ἤνυτον; ebenso verhält es sich mit ἁθρόος, Moschop. p. 33 Titz. Eustath. p. 1387, 7 (Herodian. I, 538 L.) u. ἁθροίζω (daher hat b. Dem. 27.35 Dind. für οὐκ ἄθρουν οὐχ ἅθρουν hergestellt), mit ἅδην,130) ἁμίς, ἁλέα, ἁλεαίνω, ἁλύω, ἅρκυς (Eustath. ad Od. 1535, 20), ἅθυρμα (Moeris p. 5), ἁμόθεν (alicunde), ἁμοῦ u. s. w., ἅσμενος (zu ἥδομαι; Usener N. Jahrb, 1865, 255 nach Bodl. u. Par. A des Plat.),131) αὗος, αὕω, αὑαίνω (Ar. Eq. 394 ἀφαύει, Eccl. 146 ἀφαυανθήσομαι), εἱρκτή εἱρχθῆναι (Herodian I, 538 L.), ἕνη s. Passow Lex., u. a. m. Die altattischen Inschriften (Meisterhans, Gr. d. att. Inschr. 65\textsuperscript{2} ff.) zeigen zwar keine Konstanz in der Setzung des H, welches vielmehr oft genug fehlt, sie setzen den Spiritus zuweilen auch da, wo die geregelte Sprache ihn wegen eines h in der folgenden Silbe weglässt, als ἕχω, ἱσχύς (vgl. ἱχθῦς att. nach Gellius N. A. 2, 3); doch sind folgende Wörter als im Attischen aspiriert aus den Inschr. anzuführen: Ἅβδηρα oft, Αἷσα Αἵσων Αἵσωπος, ἁκούσιος (aus ἀἑκ. wie Ἅιδης aus Ἀἵδης, φροῦδος aus πρὸ ὁδοῦ), einmal ἑλπίς (auch in der κοινή einmal), ἕνος vgl. Gramm., Ἱλείθυα (an ἵλεως angelehnt), Ἱλισός (aber ἴδιος und ἴσος); vereinzelt ὁγδόη wie in Heraklea, ὁπίσθιον, Bull. d. corr. hell. XII, 284; dagg. auffallend oft ημέρα, Thumb 63. 

\paragraph{} Aber auch in dem gemeingriechischen Gebrauche kommen einzelne Erscheinungen vor, welche den Schwestersprachen gegenüber als besondere Eigentümlichkeiten des Griechischen hervortreten. So z. B. hat jedes anlautende υ (= ü) den Asper, während das alte υ = ου u der Böotier nach allem Anschein den Lenis hatte, als: ὕδωρ, böot. οὔδωρ, lokr. Inschr. ὐδρία, sk. ud-am, l. unda, ὑπέρ ὕπατος (böot. Υπατόδωρος, Upatod. ohne H Dial.-Inschr. 1130), sk. upari; vgl. Thumb S. 41 f.; der Spir. scheint mit dem Übergange des u in ü angetreten zu sein.132) Fernere Unregelmässigkeiten: att. ἕως, ion. ἠώς, dor. ἀώς, äol. αὔως, sk. ušas, l. aurora (ausosa); ἡγεῖσθαι ἁγεῖσθαι neben ἄγειν; ἵππος neben tarent. ἴκκος (Et. M. p. 474, 12), sk. açvas, l. equus, daher λεύκιππος S. El. 706 (λεύχιππος überall nicht), Γλαύκιππος, böot. Ἀντιππίδας und Ἄνθιππος u. s. w.;133) vgl. § 23, 3. 

{\small\noindent\subparagraph{} Übrigens finden sich auch bei attischen
  Schriftstellern einzelne anscheinende Aeolismen oder Ionismen, als: Aesch. Ag.
  528 ἀντήλιος st. ἀνθ., ebenso S. Aj. 805; ἀπηλιώτης Eur. Cycl. 19, auch in
  Prosa; Ar. Av. 110 ἀπηλιαστής st. ἀφ.; S. Ant. 251 ἐπημαξευμένη: Xen. Hell 4.
  4, 10 ἱππαρμοστής ([root ] ἀρ, dor. Ἀρμονόα, Ἀρμοξίδαμος oben 5) u. s. w.134)
  — Umgekehrt hat die κοινή nachmals in mehreren Wörtern die Aspiration, welche
  im Attischen den Lenis tragen (Giese, a. a. O., 404 ff.; G. Meyer,
  Gr.\textsuperscript{2}, 244; Thumb, S. 59, 70 ff.): C. J. Gr. 2329, 7 (Tenos)
  καθʼ ἱδίαν, ebenso 2335, 3 (Tenos). 2347 c, 8 (Syros), u. s. w. (καθʼ ἱδδίαν,
  Thessal. oben 3); das. 2347 c, 48 καθʼ ἕτος, vgl. πενταἑτηρίδα, Tafeln von
  Heraklea, δεχέτης u. s. w. reichlich in der κοινή, Wagner, Quaest. de epigr.
  gr., p. 91; C. J. Gr. 2448, VI, 25 u. Bullet. de corr. hell. VIII, 24, 10. 16
  (Amorgos) καθʼ ἑνιαυτόν; 3137, 75 (Smyrna) ἐφʼ ἵσῃ, u. so oft (Dittenberger,
Syll., p. 781); Papyr. Louvr. I μεθοπωρινός, vgl. oben 5.}

\section{Interaspiration oder die Aspiration in der Mitte der Wörter}\label{sec:23}

\paragraph{} In dem Gebrauche des rauhen Hauches in der Mitte der Wörter sind
zwei Fälle zu unterscheiden: a) der rauhe Hauch ist in einfachen Wörtern
Vertreter eines geschwächten ς; b) er erscheint in zusammengesetzten Wörtern.
Der erstere Fall gehört nur einigen Mundarten, der letztere der griechischen
Sprache überhaupt an. 

\paragraph{} Dass das ursprüngliche ς als Anlaut vor Vokal und als
intervokalischer Inlaut bei allen Griechen sich in der Regel in den Spiritus
asper verwandelt bezw. ausfällt, haben wir § 15 gesehen; einige dorische Stämme
aber, nämlich die Lakedämonier und Argiver, vereinzelt auch die Eleer und (nach
den Glossen, weniger nach den Inschriften) die Kyprier verwandeln das von
anderen Stämmen zwischen Vokalen (die Kyprier auch das im Anlaut)
zurückgelassene ς in den Spiritus asper.135) Der ältere Lakonismus, wie der des
Dichters Alkman, scheint diesen Gebrauch noch nicht zu kennen; Alkman sagt μῶσα.
Aber recht früh, jedenfalls lange vor Aristophanes, trat diese Verflüchtigung
ein. So findet sich auf Inschriften regelmässig: ἐποιεἑ, d. i. ἐποίηἑ st.
ἐποίησε, Ποοἱδάν Ποσειδών, Ἁγηἵστρατος; in junglakon. Inschriften ohne
geschriebenen Spir. σαάμων, d. i. σαἅμων st. σησάμων, Σώανδρος = Σώσανδρος; in
der Lysistrata des Aristophanes: μῶα (d. i. μῶἁ) st. μῶσα, att. μοῦσα, πᾶα =
πᾶσα, ἐκλιπῶα = ἐκλιποῦσα, ὅρμαον = ὅρμησον u. s. w.; nur in wenigen Wörtern
findet sich in diesem Stücke ς, wie in παυσαίμεθα, und jedenfalls mit Recht da,
wo ς aus einem T-Laute + ς entstanden ist: πείσομες v. πείθ-ω. Argivische
Inschr. bieten Θράϋλλος (C. J. Gr. 1120), ἐποίϝεἑ, u. a. m. Das Auffallendste
sind zwei lakonische Aufschriften eines Grenzsteines (Röhl J. Gr. ant. Add. nova
p. 184): Διοκέτα | Διολευθερίο, d. i. Διὸ (ς) ἱκέτα, Διωλευθερίω m. Kontraktion
aus Διο (ς) ελ. — Aus dem Atticismus wird das Wort ταὧς, entstanden aus ταϝῶς
(lat. pavo), sicherlich ein Fremdwort, von den Grammatikern Tryphon und Seleukos
b. Athen. p. 397, e. u. 398, a. als einziges Beispiel eines inlautenden Asper
angeführt; im übrigen spreche man λεὤς, νεὤς, βαιὄς, θοὄς u. s. w. Doch kommen
noch hinzu die Interjektionen εὐοἵ, εὔἁν, εὐαἵ, Herodian Ι, 547; Apollon. Synt.
p. 319. Kühners ausführl. griech. Grammatik. I. T. 

\paragraph{} In zusammengesetzten Wörtern wird auf altattischen Inschriften der
Asper in der Mitte des Wortes zuweilen ebenfalls bezeichnet,136) als: ΑΗΟΡΙΟΣ
ἀὥριος, ΕΝΗΙΔΡΥΕΣΘΑΙ, προσἡκέτω, εὔὁρκον (att. Inschr.), ebenso meistens auf den
Herakleischen Tafeln.137) Dazu stimmt auch die lateinische Umschrift, als
exhedra (exedra), Panhormus, parhippus, Euhemerus; der Spir. erlosch also
jedenfalls in der Aussprache nicht. Auch die alexandrinischen Grammatiker
bedienten sich bei zusammengesetzten Wörtern der “Interaspiration”, wie Eustath.
ad Il. p. 524, 2 berichtet, machten indes ihre Ausnahmen und Vorbehalte.138)
Aristarch unterschied die wirklich aus zwei Begriffen zusammengesetzten und die
(der Bedeutung nach) nur abgeleiteten Wörter durch den Spiritus, indem er die
ersteren, da sie die Bedeutung zweier hätten, mit dem Asper, die letzteren, da
die Bedeutung des zweiten Teils zurücktrete, mit dem Lenis schrieb. So ὠκύἀλος
νηῦς, weil der Sinn von ἅλς nicht gefühlt werde und das Beiwort einfach gleich
ὠκεῖα sei; ebenso ταλαύῤινος (κοτυλήῤυτον Il, ψ, 34, weil von ἀρύω, nicht von
ῥέω). Darnach unterblieb also die Aspiration auch in Eigennamen (Ptolem. Ascal.
Herodian. II, 48 L.; Ael. Dionys. schol. Il. ο, 705), als Εὐρύἀλος, Ἀγχίἀλος,
Φίλἰππος (aber φίλἱππος), Μελάνἰππος; doch hebt Herodian als Gegeninstanz
Πάνὁρμος und den Eigenn. Ἔφιππος hervor. Die Sache wird dennoch eine gewisse
Richtigkeit haben, da es ja auch Λεύκιππος, Γλαύκιππος u. s. w. heisst, während
wo das Wort ἵππος als solches gefühlt wird, der Spir. nicht wegbleiben kann,
ausser in altüberlieferten poetischen Worten wie λεύκιππος. Ἔφιππος aber ist das
zum Eigennamen gemachte Adjektiv ἔφιππος. — Irrig ist Gieses (S. 333) Meinung,
dass der Spiritus in der Elision (ausser bei Tenues) verschwunden sei; denn
hiergegen zeugen nicht nur die Grammatiker (πάρἁλος), sondern auch auf att.
Inschriften die Schreibung ΠΑΡΗΕΔΡΟΙ, wiewohl gemäss der Seltenheit derselben
anzunehmen ist, dass der Spiritus in diesem Falle noch weniger als sonst gehört
wurde. In Fällen, wie ὕφαλος, δεχήμερος u. s. w., hat die Tenuis die Aspiration
aufgenommen. — Nach den Scholien ad Dionys. in Bekkeri An. II. p. 693 setzten
die alten Grammatiker (d. h. die Alexandriner) auch in der Mitte eines einfachen
Wortes über ρ mit vorangehender Aspirata den Asper und über ρ nach Tenuis den
Lenis, als: χῥόνος, ἀφῥός, θῥόνος; Ἀτῤεύς, κάπῤος. 

