% TeX root=../../main.tex

\chap{Einteilung der Sprachleute}

\section{Artikulation der Sprachlaute}

\paragraph{} Die Sprachlaute sind artikulierte Laute (ἔναρθροι, Ggstz.\ ἄναρθροι
unartikulierte, wie die der Tiere), d.~h.\ solche, welche durch die Einwirkung
der Sprachwerkzeuge eine bestimmte Gestalt erhalten. Unter Artikulation der
Laute versteht man daher die Bildung der Stimme durch die Sprachwerkzeuge zu
Lauten von bestimmter Gestalt. Sprachwerkzeuge sind ausser der Mundhöhle die
Kehle, die Zähne, die Zunge und die Lippen.

\paragraph{} Diejenigen Sprachlaute, welche bloss durch eine grössere oder
geringere Erweiterung oder Verengerung der Mundhöhle hervorgebracht werden und
am ungehindertsten durch den Mund gehen, heisst man Vokale (φωνήεντα sc.\
στοιχεῖα), die übrigen, welche unter stärkerer Einwirkung der Kehle, der Zähne,
der Zunge oder der Lippen gebildet werden, Konsonanten (σύμφωνα sc.\ στοιχεῖα).
Jene tönen für sich allein hell und voll, sie sind φωναί; diese sind für sich
höchstens Geräusche (ψόφοι), und haben an einer φωνή nur mit Hülfe eines Vokales
teil.


\section*{Vokale}
\addcontentsline{toc}{section}{Vokale}

\section{Einfach Vokale}

\paragraph{} Die Griechen hatten, wie wir
\pararef{sec:kurzgeschischte}{par:heta} gesehen haben, anfänglich nur fünf Vokalzeichen: Α, Ε, Ο, Ι, Υ, welche als kurz (βραχέα) und als lang (μακρά) gebraucht wurden. Nachher kamen für das offene (lange) Ε das Zeichen Η und für das offene (lange) Ο das Zeichen Ω hinzu, und noch später wurden Ε und Ο auf die Geltung kurzer Vokale beschränkt, während Α, Ι und Υ nach wie vor als kurz und als lang gebraucht und daher δίχρονα oder ἀμφίβολα genannt wurden.

\paragraph{} Das Verhältnis der Vokale zu einander wird am besten durch die
bekannte Vokalpyramide dargestellt, an deren Spitze \ipa{a}, und an deren beiden
unteren Ecken \ipa{i} und \ipa{u} stehen, während die verschiedenen \ipa{e} und
\ipa{o} auf der Linie zwischen \ipa{a} und \ipa{i} bezw.\ \ipa{a} und \ipa{u}
Platz finden, \ipa{ü} aber zwischen \ipa{i} und \ipa{u}.

\begin{center}
	\begin{tikzpicture}[scale=1]
		% Coordinates
		\coordinate (A) at (0.00, 0.00);
		\coordinate (B) at (7.00, 0.00);
		\coordinate (C) at (3.50, 3.25);
		\coordinate (D) at (3.50, 0.00);
		\coordinate (E) at (1.25, 0.9);
		\coordinate (F) at (6.0, 0.9);
		\coordinate (G) at (0.75, 2.2);
		\coordinate (H) at (4.50, 2.2);

		% Drawing
		\draw (A) -- (B);
		\draw (A) -- (C);
		\draw (C) -- (B);
		\node[above left, black] at (A) {i (ι)};
		\node[above right, black] at (B) {u};
		\node[above, black] at (C) {a (α)};
		\node[above, black] at (D) {ü (υ)};
		\node[above, black] at (E) {e geschlossen (ε)};
		\node[above right, black] at (F) {o geschl. (ο)};
		\node[above right, black] at (G) {e offen (η)};
		\node[above right, black] at (H) {o offen (ω)};
	\end{tikzpicture}
\end{center}

A, i, u stellen sich im Griechischen wie im \Langlong{sanskrit} und in den semitischen
Sprachen deutlich als die drei Grundvokale dar, und zwar gehören die E- und
O-Laute im Griechischen zum Bereiche des \ipa{a}, nicht zu dem des \ipa{i} und
\ipa{u}.

\paragraph{} Der dritte Grundlaut ist im Griechischen kein reiner, sondern aus
dem U-Laute durch Annäherung an ι getrübter; aber ohne Zweifel hat er
ursprünglich den reinen Laut u, wie im Lateinischen und Deutschen, gehabt, und
dieser Laut ist insbesondere für Homer noch anzunehmen, bei welchem εὖ als εὖ
und ἐΰ (\ipa{eu} und \ipa{e-u}) erscheint, αὔω im Aorist ἤϋσα bildet (\ipa{auo} — \ipa{ē-ūsa}). Auch
haben namentlich die Böotier diesen ursprünglichen Laut treu bewahrt, indem sie
ihr υ wie u, und zwar als kurzes und langes u, aussprachen; also σύν, τύχα,
κᾶρυξ, Πῡ́θιος, ὗς lautete bei ihnen wie \ipa{sun}, \ipa{tucha}, \ipa{karux},
\ipa{Pūt'ios},
\ipa{hūs}.\footnote{S.\ Ahren, \emph{Dial.} I, 196sq.\ u.\ p.\ 180sq.; Meiter.
	\emph{Gr. Dial.} I, S.\ 231ff. Vgl.\ Dietrich in Kuhns Zeitschr. 1865, S.\ 64.}
Nachdem aber im \Langlongdecl{att}{en} und \Langlongdecl{ion}{en}
(\Langlongdecl{dor}{en}) sich die Bezeichnung ου für
einen dem langen \ipa{u} wenigstens nahe verwandten Laut gebildet hatte; nahmen auch
die Böotier im 4.\ Jahrh.~v.~Chr.\ dieses ου an und gebrauchten es nicht nur für
das lange, sondern auch für das kurze \ipa{u}, als: κούνες st.\ κύνες, οὔδωρ
st.\ ὕδωρ,
σούν st.\ σύν, κοῦμα st.\ κῦμα, welche Schreibung auch in die Gedichte der
Korinna eingeführt wurde, daher in deren Fragmenten: τού, οὐμές, οὐμίων,
πουκτεύι, ῶνούμηνεν (= ὠνύμαινεν), γλουκού, λιγουράν u.~a.
Jedoch schwankt auf den
\langlongdecl{böot}{en} Inschriften die Schreibung zwischen ου und υ, während andererseits
die Böotier in späterer Zeit das lange υ (= \ipa{ȳ}) häufig für οι (ῳ)
verwendeten, als: τῦς ἄλλυς st.\ τοῖς ἄλλοις, ἵππυς st.\ ἵπποις, προβάτυς st.\
προβάτοις; τῦ δάμυ st.\ τῷ δάμῳ.\footnote{S.\ Ahrens l.\ d.\ p.\ 191sqq.;
	Meister, S.\ 236.}
Eine dem \ipa{ü} ähnliche Trübung stellte sich mit der Zeit auch bei ihnen ein, zu ü
sich verhaltend wie das \langlongdecl{engl}{e} \ipa{ū} (~\ipa{i͜u}~) zum
\langlongdecl{franz}{en} \ipa{u}, dem es entspricht
(\emph{duc} \glsxtrshort{engl} \emph{duke}); die Böoter schreiben ιου, was sich besonders nach Dentalen und
nach λ findet: Πολιούστρατος, τιούχα, Διωνιούσιος.\footnote{Meister, S.\ 233f.
	(Ahrens Add.\ II, 519).}
Unter den \langlongdecl{dor}{en} Stämmen sind die Lakonier die Einzigen, in deren Glossen das
ου sowohl für ῡ als fur υ vorkommt.
So findet sich bei \authorlong{Hesychius} z.~B.\ διφοῦρα =
γέφυρα, κάρουα = κάρυα, μουσίδδει = μυθίζει, τούνη = τύνη (σύ). Auf den sehr
späten lakonischen Inschriften 1347 und 1388 findet sich ο st.\ υ in Κονοουρεῖς
st.\ Κυνοσουρεῖς;\footnote{S.\ Ahrens, II, p.\ 124--126.}
sonst geben die Inschriften nur υ wie gewöhnlich, und es scheint daher das \ipa{u} für υ auf die vulgäre Sprache Lakoniens beschränkt gewesen zu sein.

\paragraph{} Hinsichtlich der Kürze und Länge der Vokale ist zu bemerken, dass
weder die kurzen noch die langen von den alten Grammatikern alle als gleich kurz
oder lang angesehen wurden. Dass das ε der kürzeste Vokal sei, schloss man aus
der sogenannten \langlongdecl{att}{en} Deklination, in der es auf den Accent nicht einwirkt,
indem die Stimme über dasselbe leicht hingleitet, als: Μενέλεως, ἵλεῳ, πόλεως,
selbst χρυσόκερως, φιλόγελως. Dass es insbesondere kürzer sei als ο, entnahm man
aus dem Vokative, der die kurzen Vokale liebt, als: λόγος λόγε;\footnote{S.
	\authorlong{Herodian} in Bekk.\ \emph{Anecd.} II, p.\ 798sq.
	\authorlongdecl{Herodian}{s} Vater \glslink{ApolloniusDyscolus}{Apollonius}
	behauptete dagegen, ο sei kürzer als ε. S.\ \glsxtrshort{Theodosios}, p. 33sq.}
dass aber ω kürzer sei als η, daraus, dass man Μενέλεων, πόλεων u.~s.~w.\
proparoxytonisch betont, was nie der Fall ist, wenn η in der letzten Silbe
steht.\footnote{Bekk.\ \emph{Anecd.} II, p.\ 979.}


\section{Diphthonge}

% TODO: Acertar referência cruzada
\paragraph{} Sämtliche Diphthonge (αἱ δίφθογγοι scil. συλλαβαί),\footnote{%
	Das Wort ἡ δίφθογγος zeigt schon durch sei Genus an, dass es eig.\ Adjektiv
	und dass ein weibliches Substantiv zu ergänzen sei; nun werden aber die
	Diphthonge sowohl von Griechen (τὴν ου συλλαβήν \glsxtrshort{DionysHalic} oben
	\pararef{sec:buchstabenausprache}{sec:ausprachekonsonanten} p.~\pageref{pvau}) als von Lateinern (\emph{ae syllaba}~\passagem{QuintilianusI718})
	öfters συλλαβαί \emph{syllbae} gennant und es it daher dieses Wort als
	ursprünglich zu ergänzen anzunehmen.
	Vgl.\ \glsxtrshort{Theodosios} p.\ 34: ἡ συλλαβὴ ἡ ἐκ φωνηέντων δύο
	συνεστηκυῖα δίφθογγος καλεῖται, was dann damit gerechtfertigt wird, das im
	eig.\ Sinne (κυρίως) die Bezeichnung φθόγγος nur den Vokalen zukomme.
} mit
Ausnahme von υι, sind aus der Verschmelzung eines der Vokale α, ε, η, ο, ω mit ι
oder υ (im Werte von u) zu einem Mischlaute entstanden, als:

\begin{center}
	\begin{tabular}[c]{ll}
		α + ι = αι, als: αἴξ    & α + υ = αυ, als: παύω                \\
		ε + ι = ει, als: δεινός & ε + υ = ευ, als: ῥεῦμα               \\
		ο + ι = οι, als: κοινός & ο + υ = ου, als: βοῦς                \\
		ᾱ + ι = ᾳ, als: δᾴς     & η + υ = ηυ, als: ηὖξον (im Augmente) \\
		η + ι = ῃ, als: λῃστής  &                                      \\
		ω + ι = ῳ, als: ᾠδή     &
		ω + υ = ωυ, als: ἑωυτοῦ                                        \\
	\end{tabular}
\end{center}


\noindent Der Diphthong ωυ findet sich im \Langlongdecl{att}{en} nur in der Krasis, und auch
da selten (ωὐριπίδη ὦ Εὐριπίδη \passagem{Thesm4}, πρωυδᾶν προαυδᾶν
\passageme{Aves556}); auch im \Langlongdecl{ion}{en}, wo er mehr hervortritt, ist in den
sichern Fällen Krasis der Entstehungsgrund (ἑωυτοῦ aus ἕο αὐτοῦ), und ebenso im
\Langlongdecl{dor}{en} (ωὑτός \passagem{Theokr1134}, s. Ahrens II, 222).

\paragraph{} Ist der erste Vokal ein langes α oder ein η oder ein ω, so wurde
das in älterer Zeit daneben gesetzte (προσγραφόμενον, \emph{iota adscriptum}) ι in der
Minuskelschrift seit dem 12.\ Jahrh.\ unter den langen Vokal gesetzt (iota
subscriptum, ἔχει τὸ ι ὑποκάτω γραφόμενον \glsxtrshort{Theodosios}
108).\footnote{%
	Eine den Übergang von ι adscriptum zum ι subscriptum anzeigende Schreibweide is
	die, wo der Buchstabe zwar seitwärts, aber entweder höher oder tiefer als die
	Zeile gesetzt wird, als α\textsuperscript{ι}, α\textsubscript{ι}. S.\
	Gardthausen, \emph{Gr.\ Paleogr.}, S.\ 193, 203.
}
Bei der Unzialschrift jedoch wird das ι immer noch neben den ersten Vokal gesetzt; ΑΙ, ΗΙ, ΩΙ, Αι, Ηι, Ωι, als: ΤΗΙ ΧΩΡΑΙ, ΤΩΙ ΚΑΛΩΙ.

\paragraph{} In dem Diphthongen υι vereinigen sich υ (ursprünglich und
dialektisch \ipa{u}, gew.\ \ipa{ü}) und ι zu einer Silbe, doch geschieht dies in der
gewöhnlichen Sprache nur vor Vokalen, als: μυῖα, ἅρπυια. Vor Konsonanten kommt
υι auch in Dialekten fast gar nicht vor, eher am Ende, wie in den Dativen ἰξυῖ
(\glsxtrshort{Homerus}), Δέρμυι (\glsxtrshort{böot} Inschr., \emph{Dial.- Inschr.} 875).

	{\small\noindent\subparagraph{} Da die Vokale α, ε, η, ο, ω bei den
		Diphthongen dem ι und υ vorangehen, so werden sie προτακτικά, ι und υ hingegen
		ὑποτακτικά genannt; in dem Diphthonge υι ist jedoch υ προτακτικόν. S.\
		\glsxtrshort{DionysThrax} in Bekk.\ \emph{Anecd.}\ Il, p.\ 631, Schol.\ ad
		\glsxtrshort{DionysThrax} ib.\ II, p.\ 801, \glsxtrshort{Theodosios} Canon.\ ib.\ III, p.\ 1187, wo der merkwürdige Schluss gemacht wird: εἰ ἄρα οὖν τὸ ι καὶ τοῦ ὑποτακτικοῦ
		ὑποτακτικόν ἐστι, δῆλον, ὅτι ἀσθενέστερόν ἐστι πάντων τῶν φωνηέντων. -- Dass ᾳ,
		ῃ, ῳ ursprünglich Diphthonge waren, später aber zu Einzellauten herabsanken,
		haben wir~\ref{sec:buchstabenausprache} gesehen. Über die zwiefache
		Entstehung von ου s.\ oben~\pararef{sec:kurzgeschischte}{par:heta}; das.\ über die entsprechend zwiefache von ει.}

	{\small\noindent\subparagraph{} Inschriften und Handschriften (insonderheit
		die Volumina Herculanensia) aus der römischen Zeit verwenden, wie wir oben
		sahen (\pararef{sec:buchstabenausprache}{par:betreffdesditp}) das ει als Bezeichnung jedes langen ι: πολείτης, μεισεῖν,
		μειμεῖσθαι. Dass gelegentlich ein ει für ι aus Unkunde oder Versehen mit
		unterläuft, kann den Nutzen nicht hindern, den wir aus dieser Schreibung für die
		Erkenntnis der Quantität ziehen; denn wo sie häufig und stehend wiederkehrt, wie
		in πείπτω st. πίπτω, ἔτρειψα st. ἔτριψα, ist der Schluss auf Länge des ι
		berechtigt und sicher.\footnote{%
			Vgl.\ Dittenberger in \emph{Hermes} I, S.\ 415; A.~von Bamberg,
			\emph{Zeitschr.\ f.\ Gymnasialwesen} 1874, S.\ 13ff.
		}}

	{\small\noindent\subparagraph{} Unter allen Diphthongen müssen οι und αι für
		die kürzesten gelten, da sie rein, d.~h.\ ohne antretenden Konsonanten
		auslautend, in Beziehung auf die Betonung in der Flexion (mit Ausnahme des
		Optativs) und in den Adverbien πρόπαλαι und ἔκπαλαι als kurz behandelt werden,
		als: τράπεζαι, γλῶσσαι, τύπτεται, ἄνθρωποι, οἶκοι (die Häuser, zu unterscheiden
		von dem Adverb οἴκοι, zu Hause, \emph{domi}). Sodann sind αι und οι die einzigen Diphthonge, welche in der Dichtersprache elisionsfähig sind.}

	{\small\noindent\subparagraph{} In den Diphthongen αυ und υι kann, a priori
		betrachtet, der erste Vokal entweder kurz oder lang sein, und man kann somit,
		einschliesslich des ᾶυ und des ῦι, zu der Zahl von 14 Diphthongen
		gelangen.\protect\footnote{%
			Die Theorie der 14 Diphthonge entwickelt G.~Hermann, \emph{de emend.\
				rat.\ graeace gramm.}, p.\ 48sqq.
		} Nachweisbar ist indes weder ᾶυ noch ῦι; im Gegenteil finden wir im
		\Langlongdecl{att}{en}
		ναῦς für das \langlongdecl{ion}{e} νηῦς mit offenbar kurzem α; denn das lange hätte zu η
		werden müssen. Erscheint aber hier für ᾶυ αυ, so wird auch im attischen γραῦς,
		wo ρ ein ᾱ schützen würde, vielmehr α gesprochen worden sein. Ganz unklar bleibt
		die Quantität in dem \langlongdecl{dor}{en} αὖξον, \glsxtrshort{att} ηὖξον.}
\bigskip

\paragraph{} Die alten Grammatiker (\authorlong{Choreoboskos} in Bekkeri \emph{Anecd.}
III.\ p.\ 1214 sq., \glslink{Theodosios}{Theodosios} p.\ 34 sq.\ ed.\ Göttl., die
Scholien ad \glsxtrshort{DionysThrax} in Bekk.\ An.\ II.\ p.\ 804,
\authorlong{Moschopulos} p.\ 24 sq.\ ed.\ Titze), die aber alle aus einer Quelle
geschöpft zu haben scheinen, teilen die Diphthonge in folgende Klassen ein:
\begin{compactenum}[(a)]
	\item δίφθογγοι κατʼ ἐπικράτειαν, d.~h.\ solche, in welchen der eine Vokal ein
	solches Übergewicht über den anderen hat, dass er allein gehört wird, der
	andere ἀνεκφώνητον ist, nämlich ᾳ, ῃ, ῳ, als: Μηδείᾳ, Ἑλένῃ, καλῷ. So lehrt
	\authorlong{Choreoboskos}; die anderen Grammatiker fügen noch ει hinzu, als:
	Νεῖλος. Es ist dies gemäss der Aussprache in römischer Zeit, wo das ι in ᾳ, ῃ,
	ῳ verstummt, das ει zu \ipa{ī} geworden war.
	\item δίφθογγοι κατὰ κρᾶσιν, d.~h.\ solche, in welchen die beiden Vokale zu einem
	Mischlaute verschmelzen und Einen Laut bilden, der zu beiden Vokalen stimmt
	(ἁρμόζει), nämlich: αυ, ευ, ου, als: αὐλός, εὔχομαι, οὗτος.
	\item δίφθογγοι
	κατὰ διέξοδον, d.~h.\ solche, in welchen der Laut beider Vokale getrennt
	(χωρίς) gehört wird, nämlich: ηυ, ωυ, υι, als: νηυσίν, ἑωυτοῦ, υἱός.
	\item Die Diphthonge αι und οι werden als besondere, zu keiner der angegebenen
	Klassen gehörige angeführt.
	\authorlong{Choreoboskos}, mit dem die Anderen
	übereinstimmen, sagt: ἐπειδὴ οὖν ἡ αι δίφθογγος ἡ ἐκφωνουῦσα τὸ ι καὶ ἡ οι
	δίφθογγος οὔτε κατʼ ἐπικράτειάν εἰσιν οὔτε κατὰ διέξοδον οὔτε κατὰ κρᾶσιν,
	ὥσπερ ἐστερήθησαν τοῦ ἰδιώματος τῶν διφθόγγων, ἐστερήθησαν καὶ τοῦ χρόνου
	τοῦ παρεπομένου ταῖς διφθόγγοις, καὶ τούτου χάριν αὗται μόναι ἐκ τῶν
	διφθόγγων τῷ τονικῷ παραγγέλματι ἀντὶ κοινῆς παραλαμβάνονται καὶ πρὸς ἕνα
	ἥμισυν χρόνον ἔχουσιν. Der Grund, weshalb die Grammatiker die Diphthonge αι
	und οι nicht zu den διφθόγγοις κατὰ κρᾶσιν gerechnet und ihnen sogar die
	Eigentümlichkeit der Diphthonge abgesprochen haben, scheint kein anderer zu
	sein, als weil dieselben in Beziehung auf die Betonung als kurz angesehen
	werden.
\end{compactenum}

\setcounter{subparagraph}{4}
{\small
	\noindent\subparagraph{} Nach \glslink{Theodosios}{Theodosios (Gramm.\ p.\
		35)} werden die Diphthonge eingeteilt (a) in eigentliche (κύριαι): αι, αυ, ει,
	ευ, οι, ου, und in uneigentliche (καταχρηστικαί): ᾳ, ῃ, ῳ, υι, ηυ, ωυ,
	wahrscheinlich, weil bei diesen nicht beide Laute zu einem Mischlaute
	verschmelzen, sondern entweder (ᾳ, ῃ, ῳ) nur der eine, oder (υι, ηυ, ωυ) beide
	in einer Silbe gehört werden.
	Diese Einteilung kann älteren Ursprungs sein, da ει in der Reihe der
	eigentlichen erscheint.
	In den Scholien ad \glsxtrshort{DionysThrax} (Bekk.\ \emph{Anecd.}\ II, p.\ 803) werden
	αι, αυ, ει, ευ, οι, ου εὔφωνοι, ηυ, ωυ, υι κακόφωνοι und ᾳ, ῃ, ῳ ἄφωνοι
	genannt.
	Eine andere Dreiteilung, der im Text gegebenen ziemlich entsprechend, findet
	sich bei dem Musiker \passagem{ArQuintPeriMousike121}:
	αἱ δίφθ., ἃς ἤτοι κατὰ κρᾶσιν ἢ κατὰ συμπλοκὴν ἢ κατʼ ἐπικράτειαν γίγνεσθαί
	φαμεν.
	Es wird indes nicht ganz klar, in welcher Weise die Diphthonge sich in diese
	drei Klassen verteilen.
	Zu vermuten steht, dass in der ursprünglichen Theorie der Musiker, welche sich
	von Alters her mit der Lehre von den Sprachlauten beschäftigten
	(\passagem{Cratylus424c}), nur δίφθ.\ κατὰ κρ.\ u.\ συμπλοκήν unterschieden
	wurden, indem die ἐπικράτεια bei ᾳ u.~s.~w.\ erst viel später eintrat, ja auch
	nachmals von den Musikern geleugnet wurde (s.\ oben
	\pararef{sec:buchstabenausprache}{par:diphthongeaihiwi} N.~\ref{foot:3131}).
	Beim eigentlichen Diphthonge lautet die Stimme während der Bewegung aus einer
	Vokalstellung in die andere und nur während dieser Bewegung, so dass eine
	wirkliche Mischung (κρᾶσις) ist wie zwischen Wasser und Wein; bei uneigentlichen
	Diphthongen dagegen bestehen die Laute neben einander, wie in einer Verflechtung
	(συμπλοκή).
	S. Rumpelt, \emph{das natürliche System der Sprachlaute}, S.\ 47.
}


\section{Die Konsonanten}

\paragraph{} Die Konsonanten (σύμφωνα sc.\ στοιχεῖα, der Name bereits bei
\glsxtrshort{DionysThrax}) zerfallen:
Erstens nach den Sprachwerkzeugen, durch deren Einwirkung sie gebildet werden, in:
\begin{compactitem}
	\item Kehllaute (gutturales): κ, γ, χ;
	\item Zahnlaute (dentales): τ, δ, θ, ν, ς, λ, ρ;
	\item Lippenlaute (labiales): π, β, φ, μ.
\end{compactitem}

\noindent Die Konsonanten, welche durch dasselbe Sprachwerkzeug hervorgebracht werden, heissen gleichnamige Konsonanten.

	{\small\noindent\subparagraph{} Den Zitterlaut (\emph{consonans tremula}) ρ haben wir
		nach \passagem{DeCompVerb14}: “τὸ δὲ ρ
		(ἐκφωνεῖται) τῆς γλώσσης ἄκρας ἀπορραπιζούσης τὸ πνεῦμα καὶ πρὸς τὸν οὐρανὸν
		(\emph{palatum}) ἐγγὺς τῶν ὀδόντων ἀνισταμένης” zu den Zahnlauten zu
		rechnen, während er anderweitig in den Sprachen vielfach guttural ist.}

\paragraph{} Zweitens nach ihrer Lautbeschaffenheit in:
\begin{compactenum}[(a)]
	\item halblaute (\emph{semivocales}, ἡμίφωνα), welche den Vokalen zunächst
	stehen: λ, ρ, ν, nasales γ (\pararef{sec:buchstabenausprache}{sec:ausprachekonsonanten}), μ, ς, welche wieder zerfallen in:
	\begin{compactenum}
		\item[(α)]
		flüssige (\emph{liquidae}, ὑγρά, \glsxtrshort{DionysThrax} in
		\glsxtrshort{AnecdII} p.\ 632, \passagem{MarVictArsGrammatica6618}, so
		benannt \emph{quando hae solae inter consonantem et vocalem immissae non asperum
			sonum faciunt}; auch ἀμετάβολα [das.], weil sie in der Flexion, z.B.\ im
		Futurum, nicht umgewandelt werden): ρ, λ und die Nasallaute: das dentale ν,
		das gutturale γ (= dem \glsxtrshort{latin} \emph{n adulterinum}) vor
		Kehllauten (\pararef{sec:buchstabenausprache}{sec:ausprachekonsonanten}) und das
		labiale μ;
		\item[(β)] die Spiranten oder Hauchlaute: den Kehlspiranten \ipa{h}, der im
		Griechischen durch den Spiritus asper bezeichnet wird (\ref{sec:digammahomer}) und den
		Zahnspiranten ς;
		\item[(γ)] die Halbvokale \ipa{v} und \ipa{j}, von denen jener (ϝ, Digamma, Vau)
		dialektisch im Griechischen fortbestand, aus dem \Langlongdecl{att}{en} aber und aus der
		Gemeinsprache verdrängt war, während das j überall nur in seinen Spuren
		erkannt werden kann;
	\end{compactenum}
	\item stumme (\emph{mutae}, ἄφωνα): \\
	hauchlose (ψιλά, \emph{tenues}) π, κ, τ,\\ gehauchte (δασέα,
	\emph{aspiratae}) φ, χ, θ,\\ mittlere (μέσα, \emph{mediae}) β, γ, δ.
\end{compactenum}

Die Konsonanten, welche gleiche Lautbeschaffenheit haben, heissen gleichstufige Konsonanten.

{\small\noindent\subparagraph{} Der Name stumme ist daher genommen,
weil sich diese Laute ohne Beihülfe eines Vokals nicht aussprechen lassen. Die
Gesamteinteilung der Laute in φωνήεντα, ἡμίφωνα und ἄφωνα geht, wenn auch
nicht ganz mit diesen Namen, bis weit in die attische Zeit zurück, da sie
Plato schon geläufig ist. S.\ \passageme{Cratylus424c}{}: ἆρ’ οὖν καὶ ἡμᾶς οὕτω δεῖ πρῶτον
μὲν τὰ φωνήεντα διελέσθαι, ἔπειτα τῶν ἑτέρων κατὰ εἴδη τά τε ἄφωνα καὶ ἄφθογγα
(\emph{mutae}, ohne φωνή und ohne φθόγγος) · οὑτωσὶ γάρ που λέγουσιν οἱ δεινοὶ περὶ
τούτων· καὶ τὰ αὖ φωνήεντα μὲν οὔ, οὐ μέντοι γε ἄφθογγα (mit φθόγγος,
semivocales); Vgl.\ \passageme{Philebus18b}f., wo diese Dreiteilung auf den mythischen
Erfinder der Schrift, den Ägypter Theuth, zurückgeführt wird, \passageme{Theaetet203b},
wo für φθόγγος ψόφος gesagt wird (τὸ σῖγμα τῶν ἀφώνων ἐστί, ψόφος τις μόνον,
οἶον συριττούσης τῆς γλώττης· τοῦ δ αὖ βῆτα οὔτε φωνὴ οὔτε ψόφος),
\passageme{Sophista253a}. In dem Namen φωνήεντα, der dem
\langlongdecl{att}{en} Dialekte nicht gemäss ist, zeigt
sich der nicht attische Ursprung dieser Einteilung; man kann an Prodikos von
Keos oder an Hippias von Elis denken, welcher letztere sich nach
\passagem{HippiasMaj285cd} viel mit der Theorie der Buchstaben beschäftigte. — S.\ ferner
\passagem{DeCompVerb14}: πρώτη μὲν (διαφορὰ τῶν γραμμάτων),
ὡς Ἀριστόξενος ὁ μουσικὸς ἀποφαίνεται, καθ’ ἣν τὰ μὲν φωνὰς ἀποτελεῖ, τὰ δὲ
ψόφους· φωνὰς μὲν τὰ λεγόμενα φωνήεντα, ψόφους δὲ τὰ λοιπὰ πάντα.\ δευτέρα δὲ,
καθ’ ἣν τῶν μὴ φωνηέντων ἃ μὲν καθ’ ἑαυτὰ ψόφους ὁποίους δή τινας ἀποτελεῖν
πέφυκε, ῥοῖζον ἢ συριγμὸν ἢ ποππυσμὸν —· ἃ δ’ ἐστὶν ἁπάσης ἄμοιρα φωνῆς καὶ
ψόφου, καὶ οὐχ οἷά τ’ ἠχεῖσθαι καθ’ ἑαυτά. διὸ δὴ ταῦτα μὲν ἄφωνά τινες
ἐκάλεσαν, θάτερα δ ἡμίφωνα.\ οἱ δὲ τριχῇ νείμαντες τὰς πρώτας τε καὶ
στοιχειώδεις τῆς φωνῆς δυνάμεις, φωνήεντα μὲν ἐκάλεσαν, ὅσα καὶ καθ’ ἑαυτὰ
φωνεῖται καὶ μεθ’ ἑτέρων, καὶ ἔστιν αὐτοτελῆ· ἡμίφωνα δὲ, ὅσα μετὰ μὲν
φωνηέντων αὐτὰ ἑαυτῶν κρεῖττον ἐκφέρεται, καθ’ ἑαυτὰ δὲ χεῖρον καὶ οὐκ
αὐτοτελῶς· ἄφωνα δὲ, ὅσα οὔτε τὰς τελείας, οὔτε τὰς ἡμιτελεῖς φωνὰς ἔχει καθ’
ἑαυτά, μεθ’ ἑτέρων δ ἐκφωνεῖται. — Die Neueren sind der Sache tiefer auf
Kühners \emph{ausführl.\ griech.\ Grammatik}.\ I.\ T.\ den Grund gegangen, und
teilen die Konsonanten zunächst in Explosivlaute (= \emph{mutae}), bei denen im Munde
ein Verschluss gebildet wird und mit der plötzlichen Lösung dieses
Verschlusses der Laut hervorgeht, und Reibelaute, bei denen nicht ein
Verschluss, sondern nur eine Verengerung stattfindet, als bei \ipa{s},
\ipa{f}, \ipa{ch}. Bei
den Liquidae ist zwar (ausser bei \ipa{r}) Verschluss, aber der Luftstrom geht auf
anderem Wege ungehindert durch; diese Laute stehen also in der Mitte.}

{\small\noindent\subparagraph{} Die Einteilung der \emph{mutae} in \emph{Tenues},
\emph{Aspiratae} und \emph{Mediae} hat mit Härte und Weichheit von Haus aus
nichts zu thun,
sondern nur mit dem Hauche, der entweder stark ({Aspir.}) oder schwach ({Med.})
oder gar nicht (\emph{Tenues}) mit dem Laute verbunden ist. Die
\langlong{latin}{e}
Übersetzung von ψιλά mit \emph{tenues} ist schlecht und irreführend. S.\
\glsxtrshort{DionysThrax}
in \glsxtrshort{AnecdII} p.\ 631; \passagem{DeCompVerb14};
\passagem{ArQuintPeriMousike120}. Es muss also bei β γ δ eine gelinde Aspiration vernommen
worden sein, und demgemäss sind im \Langlongdecl{neugrc}{en} die Medien gerade so gut
wie die Aspiraten zu Spiranten (\ipa{v}, \ipa{f} {u.~s.~w.}) geworden.
-- Die Neueren scheiden \ipa{b} und \ipa{p} u.~s.~w.\ entweder als tönende und tonlose Laute, weil bei
ersteren die Stimmritze mittönt, oder als \emph{fortes} und \emph{lenes}, harte und weiche;
diese Scheidungen gehen auch durch die Reibelaute hindurch, und es verhält
sich \glsxtrshort{franz} \ipa{s} zu \glsxtrshort{franz} \ipa{z}, \ipa{f} zu \ipa{v}
gerade wie \ipa{t} zu \ipa{d}, \ipa{p} zu \ipa{b}.}

{\small\noindent\subparagraph{} Unter allen Konsonanten steht ρ den
Vokalen am nächsten. Schon griechische Grammatiker (\glsxtrshort{CommentDionysThrax} in
\glsxtrshort{AnecdII}, p.\ 693sq.\ 806sq., \glsxtrshort{Theodosios} p.\ {27sq.}) bemerken, ρ habe die
δύναμις φωνήεντος, indem es (a) mit dem Spiritus bezeichnet werde, (b) in der I.~Dekl.\ ᾱ nach ρ bleibe,
während es bei den übrigen Konsonanten in η übergehe,
(c) die \langlongdecl{äol}{e} Mundart den Vokalen bei folgendem ρ wie bei folgendem Vokale
ein υ zusetze, als: αὐώς, εὔαδε, αὔρηκτος. Das \Langlong{sanskrit} hat einen vokalischen
R-Laut, der ri heisst (ṛ geschrieben); dazu auch einen freilich selten
gebrauchten vokalischen L-Laut; auch \langlongdecl{slavisch}{e} Sprachen, wie das
\Langlongdecl{czechisch}{e},
besitzen vokalisches r und l.}

\paragraph{} Die drei Doppelkonsonanten: ξ, ψ, ζ vereinigen in sich einen
stummen Konsonanten und den Spiranten ς, nämlich κς, πς, σδ.
S.~\pararef{sec:buchstabenausprache}{sec:ausprachekonsonanten}.

\section{Spiritus asper und lenis}

\paragraph{} Ausser dem \emph{Spiritus asper} (πνεῦμα δασύ), der zu den Spiranten
gehört (Kehlkopfspirans) und dem \langlongdecl{latin}{en} und
\langlongdecl{deut}{en} \ipa{h} entspricht {(\hspace{0.3em}  ̔\ )}, bezeichnet die grammatische Schrift der Griechen und demgemäss unsere griechische Schrift auch den Explosivlaut des Kehlkopfes, welcher vor Vokalen im Wortanfang, bei uns besonders auch in der Komposition vor vokalisch anlautendem zweiten Teile (“Mundart”) zu hören ist, und den die Semiten mit Aleph schreiben.
Man nennt diesen Laut {(\hspace{0.3em} ̓\ )} \emph{spiritus lenis}, πνεῦμα ψιλόν, obwohl eigentlich diese
Bezeichnung, “hauchloser Hauch”, eine \emph{contradictio in adiecto} ist.\footnote{
	Korrekt \passagem{PhilodemosFr94}  (Fl.\ Jahrb.\ Suppl.\ XVII, 247):
	ἀνέσει καὶ ἐπιτάσει καὶ προσπνεύσει καὶ ψιλότητι.
}
Oder er heisst προσῳδία ψιλή, gemäss der abusiven Ausdehnung des Wortes προσῳδία
= \emph{accentus} auf die sonstigen verwandten Lesezeichen.
Jedes mit einem Vokale anlautende Wort hat eines dieser beiden Hauchzeichen, als: Ἀπόλλων, ἱστορία.
Bei Diphthongen nimmt das Hauchzeichen üblichermassen seine Stelle über dem zweiten Vokale ein, als: οἷος, εὐθύς, αὐτίκα; bei den uneigentlichen Diphthongen: ᾳ, ῃ, ῳ aber in der Unzialschrift links oben von dem ersten Vokale, als: Ἄισσω (ᾄσσω), Ἧι (ᾗ), Ὠιδή (ᾠδή).
Es beruht dies nicht auf Lehren der Grammatiker, sondern hat sich als praktisch
bei uns herausgebildet, damit man nicht ἀίξ \ipa{a-ix}, Αἴσσω \ipa{aisso} spreche.
Die Liquida ρ wird anlautend mit einem starken Hauche gesprochen und erhält
daher zu Anfang des Wortes den Asper, als: ῥήτωρ (\emph{rhetor}).
Treffen in der Mitte des Wortes zwei ρ zusammen, so erhält das erstere den
Lenis, das letztere den Asper, als: Πύῤῥος (\emph{Pyrrhus}).
S.\ \glsxtrshort{CommentDionysThrax} in \glsxtrshort{AnecdII} p.\ 693.
Diese Schreibung ῤῥ ist indes in neuerer Zeit abgekommen und hat auch kaum einen
Zweck, obwohl sie, wie \langlong{latin}{es} \ipa{rrh} zeigt, eines Grundes keineswegs entbehrt.
Die Steinschrift der alten Griechen kannte alle diese Lesezeichen nicht; auch in
der Bücherschrift noch der römischen Zeit fügte man höchstens hie und da um der
Unzweideutigkeit willen den Asper zu, wenn etwa z.~B.\ οὗ von οὐ zu scheiden war.
Nur bei Dichtern nichtattischen Dialekts kamen die Lesezeichen seit der Zeit der
Alexandriner regelmässig zur Verwendung.

{\small\noindent\subparagraph{} In den vor ionischen Alphabeten, so dem alten
attischen, wurde der rauhe Hauch durch den Buchstaben Heta (\epigraphic{Ͱ},
\epigraphic{}) bezeichnet. Nach Annahme des ionischen Alphabets bildeten die
Tarentiner und Herakleoten in Italien für den Hauch ein neues Buchstabenzeichen,
wozu sie die erste Hälfte des Η verwandten: \epigraphic{}. Anderswo, so in
Athen vielleicht schon zu Platos Zeit, wurde dies selbe Zeichen als Lesezeichen
übergeschrieben: {\thirdiarygreek{} α }, und dieser Gebrauch ging auf die
alexandrinischen Grammatiker über. Der \emph{Spiritus lenis} wurde in älterer
Zeit gar nicht bezeichnet; erst die alexandrinischen Grammatiker benutzten dazu
das Zeichen {\thirdiarygreek{}}, d.~h.\ die andere Hälfte des Η\@.
\glsxtrshort{CommentDionysThrax} in \glsxtrshort{AnecdII}, p. 692: τὸ σημεῖον
τῆς δασείας, ἤτοι τὸ διχοτόμημα τοῦ Η τὸ ἐπὶ τὰ ἔξω ἀπεστραμμένον {.} {..}, τὸ
δὲ ἕτερον τοῦ αὐτοῦ στοιχείου διχοτόμημα τὸ ἐπὶ τὰ ἔσω ἐστραμμένον. p. 706: ἡ
δασεῖα συναπτομένη τῇ ψιλῇ τύπον τοῦ Η ἀποτελεῖ, οἷον {\thirdiarygreek{}𐅂 };
noch deutlicher p. 780 extr. Sehr bald wurden die Zeichen zu \epigraphic{} und
\rotatebox[origin=c]{180}{\epigraphic{}} verkürzt; aus diesen beiden eckigen
Figuren entstanden später in der jüngeren Minuskelschrift die abgerundeten
Zeichen: \hspace{0.3em}  ̔\ und \hspace{0.3em} ̓\hspace{0.3em}.}

\paragraph{} Inlautend kam der Spiritus asper im allgemeinen nur in der
Komposition vor; doch wurde er in diesem Falle gewiss noch schwächer als sonst
gehört. Die Inschriften, die das Η = \ipa{h} verwenden, lassen das inlautende
mehrenteils weg; das \Langlongdecl{latin}{e} indes gibt in der Regel auch den
inlautenden
Hauch wieder: \emph{exhedra} (\emph{exedra}), \emph{parhippus},
\emph{Panhormus}, \emph{Euhemerus}.\footnote{S.~K.~L.~Schneider, \emph{Ausf.\
		lat.\ Gr.} I, S.\ 192.}
Näheres über die “Interaspiration” S.~unten~\ref{sec:23}.

\setcounter{subparagraph}{1}
{\small\noindent\subparagraph{} Über den Gebrauch der Aspiration in den Dialekten
	S.~\ref{sec:22},~\ref{sec:23}.}

