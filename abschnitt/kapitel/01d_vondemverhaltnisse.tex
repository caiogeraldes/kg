% TeX root=../../main.tex

\renewcommand{\chaptitlefont}{\centering\large\bfseries\scshape}
\chap{Von dem Verhältnisse der Sprachlaute zu einander in den Dialekten}
\renewcommand{\chaptitlefont}{\centering\LARGE\bfseries\scshape}

\section*{A. Vokale}
\addcontentsline{toc}{section}{A. Vokale}
\section{(a) Kurze Vokale}

Der Wechsel der Vokale sowohl als der Konsonanten in den verschiedenen Mundarten erstreckt sich nie auf alle Wörter einer Mundart, sondern ist immer auf einzelne Kategorien oder gar auf einzelne Fälle beschränkt. Einige Wandlungen kommen jedoch in dieser oder jener Mundart vorz<*>gsweise häufig vor und müssen daher als besondere Eigentümlichkeiten (Idiome) der Mundart angesehen werden. Wir bemerken aber vorweg, dass es sich bei der folgenden Aufzählung mit nichten stets darum handelt, das Ursprüngliche oder aus einem Anderen Entstandene als solches hervorzuheben, sondern zunächst nur darum, die Thatsache zu verzeichnen, dass in diesem Dialekte in diesen Fällen dieser Laut dem anderweitigen Laute des gewöhnlichen Griechischen entspricht. Wechsel der drei Vokale: α, ε, ο (A-Vokale, § 9, 1): α u. ε: Böot. in einigen Wörtern, als: γά (γέ), κά (κέ), ἅτερος (ἕτερος), Ἄρταμις, ἱαρός (ἱερός); alles dies auch dor., ἄτερος auch lesb. und überhaupt ursprünglich, s. § 157, 8, während lesb. thessal. κε, thessal. ἱερός (ἱαρουτοῖς Krannon, Dial.-Inschr. 361, Β, 24); thessal. Ortsname Inschr. Κιάριον, jüngere Münzen Κιεριείων; asiat. äol. (lesbisch) in einigen Orts- und Zeitadverbien auf θεν (Gramm.), als: ἔνερθα (ἔνερθεν), ἐξύπισθα (ἐξόπισθεν) Adesp. 67 Bgk., πρόσθα u. a., doch ἄλλοθεν u. κήνοθεν Alc. 86 u. a., und die Grammatiker selbst bezeugen, dass nicht alle auf -θεν diesem Wechsel unterlagen; auf die Frage “woher” hatten auch die Aeolier θεν, vgl. § 72, 2, i, für πρόσθεν aber bietet eine lesbische Inschrift πρόσθε, auch κατύπερθεν und πάροιθεν Alk. 15. 9 (Meister, --116-- Dial. I, 40); fest dagegen ist τα auf die Frage “wann”, dor. -κα, als πότα, ὄτα, ἄλλοτα, wo das gew. τε für τεν stehen könnte (vgl. § 48, 1), so dass auch hier der Wechsel von α und εν vorläge, § 68, 4; — dor. Ἀρταμις (auch Ἀρτεμις auf späteren Inschr.), Ἄπταρα, e. kret. Stadt (auf späteren Münzen Ἄπτερα), χάραδος Flussgeröll = χέραδος (vgl. χαράδρα), Ahrens, Dial. II, 118, Meister, C. St. IV, 367, ἅτερος, γά, (κ f. κέν), ἱαρός (ἱερός auf späteren Inschr.; ἱαρός u. ἱερός in d. Beschlusse der Amphiktyonen 380 v. Chr.; ἱερᾶς Sophr. 98 hält Ahrens II, S. 116 für verderbt), ἱάραξ, σκιαρός Pind. O. 3, 24. 32, πιάζω Alkm. 44; es möchte hier überall ε unter dem Einflusse des ι aus α entstanden sein, vgl. ion. χλιερός st. -αρός, in der κοινή ὕελος, φιέλη (G. Meyer, 109\textsuperscript{2} f.); ferner φρασί Pind. z. B. Ol. 7, 24 u. sonst oft (neben φρήν, φρενός u. s. w.), wo α = εν, s. § 68, 4, desgl. in d. Adv. ἄνωθα tab. Heracl. 1, 17. 87 (von oben, ἄνωθεν), πρόθθα f. πρόσθα, Gortyn. Taf. (doch Selinus πρόσθε); aber ἔνδοθεν, Gort., vgl. o.; fest ist α in den Zeitadv. wie πόκα, ἄλλοκα (ἄλλοτε); i. Heraklea Dat. Pl. III. Dekl. auf ασσι, als: ἔντασσι (ἐόντεσσι nach dem weniger strengen Dorismus), ὑπαρχόντασσι, πρασσόντασσι, ποϊόντασσι; τάμνω (τέμνω), τράπω, F. τραψῶ u. s. w., τράφω, στράφω, τράχω (bei diesen 4, glaubt Ahrens II, p. 119, stehe das α wegen des vorhergehenden ρ; vgl. unten S. 118 lesb. τρόπω, στρόφω); doch auch στρέψαι, τρέχω auf einzelnen Inschr.; — eleisch (mit einigen Schwankungen, Meister II, 29 f.) und grösstenteils lokrisch (Allen, C. Stud. III, 219) regelm. αρ für ερ, als ϝάργον = ἔργον, πὰρ πολέμω = περὶ πολέμου, φάρην f. φέρειν (el.), ἀμφόταρος, ϝεσπάριος, πατάρα, φάρειν (vgl. φαρέτρα) lokr., doch πέρ f. περί, μέρος; ausserdem eleisch Opt. συνέαν = συνεῖεν, vgl. § 25 üb. ᾱ st. η; auch ἀποτίνοιαν, παρβαίνοιαν; vereinzelt γνῶμαν = γνῶναι, Dial.-Inschr. 1150 (Meister); εὐσαβέοι 1151, μάν = μέν das.; ἔστα “bis” das. wie kret. μέστα (ion. ἔστε, lokr. ἔντε); Augm. ἀ Dial.-Inschr. 1176 [koopa ]οῖός μἀπόησεν (eleisch?) nach Ahrens I, 229, der aus Hesych. vergleicht: ἄδειρεν = ἔδειρεν, ἄβραχεν st. ἔβρ.; — arkad. θύρδα = θύραζε (vgl. Meister II, 117, 282, 320); — alt- und neuionisch: τάμνω (τέμνειν Od. γ, 175, τάμν. Bk.); neuion. τράπω (an einigen Stellen b. Herod. in allen Codd. τρέπω), ἐτράφθην (aber ganz überwiegend τρέψω, ἔτρεψα, τρέψομαι [selten v. l. mit α], nicht, wie im Dor., τραψῶ u. s. w.); μέγαθος. ε u. α: Lesb. in κρέτος (κράτος) Alc. 25, ἐπικρέτει = -κρατεῖ 81 (nach Bgk.), vgl. ion. att. κρέσσων, κρείσσων f. κρέτjων; θέρσος (θάρσος), θέρσεισ Theokr. 28, 3 = θαρσοῦσα (Bergk), Θέρσιππος (auch böot. Θέρσανδρος, Homer Θερσίτης, Πολυθερσείδης); ἔρσην, Inschr. (auch --117-- neuion., kret., epidaur.), δρέκων Gramm. (böot. Eigenn. Δρέκων; vgl. δέρκομαι); γελάνα (vgl. γελᾶν) für γαλήνη, vgl. dor. γελανής, Pind. O. 5, 2. P. 4, 181; εν für α n. d. Gramm. in d. Verbalendung μεθεν, als: λεγόμεθεν (λεγόμεθα), φερόμεθεν (aber φορήμεθα Alc. 10); — thess. διέ für διά; — arkad. Θερσίας nb. Θρασέας, -κρέτης und κράτης in Eigenn. (so auch kypr.), Ἐρίων = Ἀρίων, δέλλω f. βάλλω, δέρεθρον f. βάραθρον; — altion. βέρεθρον (βάραθρον); — neuion. ἔρσην (ἄρσην), τέσσερες, τεσσεράκοντα; εν f. α in εἶτεν, ἔπειτεν, s. Eust. 1158, Stein, Herod. p. LXVI, εἵνεκεν (auch Pind. ἕνεκεν, εἵνεκεν nb. ἕνεκα, Mommsen zu Ol. 14, 19; ἕνεκεν auch i. d. κοινή); — neuion. Verb. auf έω st. άω, als; φοιτέω, ὁρέω; s. § 251, 3; ἵλεος? s. § 111, 5; vor α s. § 41; — att. ἔγχουσα (ἄγχουσα) Xen. Oec. 10, 2, auch Ar. Lys. 48 χἤγχουσα = καὶ ἡ ἔγχ.; ἐρρηφορεῖν nb. ἀρρηφ. (Meisterhans, Gr. d. att. Inschr. 12\textsuperscript{2}); — in der κοινή φιέλη, καταπέλτης (πάλλω; in d. att. Inschr. καταπάλτης), ὕελος (v. l. bei Herodot; vgl. Phryn. Rutherford 364; umgekehrt att. πύελος, μυελός f. πύαλος, μυαλός der κοινή, s. das.), σίελος st. σίαλον (Moeris, Cohn, Philonis lib. de opificio mundi, p. XLIX),140) ψίεθος (Moeris), χλιερός (bei Kratin, fr. 143 K. in -αρός geändert), μιερός; auch τέσσερες, τεσσεράκοντα (Nov. Test.), ψεκάς (b. Aesch. Agam. 1534, Eur. Hel. 2 jetzt korrigiert). α u. ο: böot. selten: ϝίκατι (εἴκοσι), διακάτιοι (διακόσιοι); lesb. in ὐπά (ὑπό, auch eleisch ὐπά), ὑπαδεδρόμακεν Sapph. 2, 10; — arkad. τριακάσιοι (Stymphal. τριακόσιοι) u. s. w., vgl. böot. dor.; — dor. ϝίκατι, ϝείκατι, ἴκατι, εἴκατι (εἴκοσι), διακατίοι, τριακατίοι u. s. w., sonst sehr selten, als: ἄναιρον (ὄνειρον) kret. Hesych., vgl. b. dems. ἄναρ (ὄναρ); κάῤῥα (nach Ahrens II, p. 120 u. 102, not. 4) vielleicht für κόρση Alkm. 44; — neuion.: ἀῤῥωδέειν. ο u. α: lesb. in nicht wenigen Wörtern vor einer Liquida und nach einer Liquida mit einer Muta, als: ὄν (ἀνά), ὄνω, ὀνεκρέμασσαν Alc. 32, ὀμνάσθην (ἀναμνησθῆναι) Theokr. 29, 26, ὀνέλων 30, 32; auf Inschr.: ὀντέθην, ὄνθεντα, ὀνθέμεναι, ὀγκαρυσσέτω, vgl. Hesych. ὀσκάπτω (ἀνασκ.), ὄστασαν (ἀνέστησαν); ὀνία (ἀνία) S. 1, 3, Alc. 88, ὀνίαρον (ἀνιηρόν) Alc. 98; γνόφαλλον (γνάφαλλον, att. κνέφαλλον) Alc. 34, τομίας (ταμ.) id. 67, χόλαισι (χαλῶσι) id. 18; ὄλοχος Theokr. 28, 9; b. Hesych. δόμορτις (δάμαρ), σπολεῖσα (σταλεῖσα); ferner besonders ρο, ορ nach Kons. st. αρ, ρα: στρότος Gr. στρόταγος u. s. w. Inschr., θροσέως Gr., βροχέως S. 2, 7, βρόσσονος (βραχυτέρου) Hesych., τετόρταιος Theokr. 30, 2; μέμορθαι (εἱμάρθαι), ἔφθορθαι, μορνάμενος, κόρτερα u. a. Auf den späteren Inschr. --118-- sind manche Vulgärformen, als: ἀναγράψαντας, ἀναγράψαι, ἀνηκόντων, ἀποσταλέντα, στρατάγοις; die Stellen bei Dichtern, als: ἀμπέτασον Sapph. 29, ἂν τὸ μέσσον Alc. 18, ἀμμένομεν Alc. 41, στράτος Alc. 66, halten Ahrens I, p. 78, Meister I, 50 für verderbt; in Balbillas äol. Gedichten findet sich δέκοτος (arkad.), λόχον (λαχόην S. 9, ἔλαχον d. älteste Inschr. v. Mytilene), ὀΐοισα, γρόπτα (?) und γρόππατα = γράμμ.; — thessal.: ὀνέθεικε = ἀνέθ., (doch ἀν-Pharsal. Kierion), κόρνοψ b. d. Oetäern = πάρνοψ (auch böot. πόρνοψ, desgl. äol., Meister I, 49); — böot. στροτός in Eigenn., ἐροτός desgl. (Ἐροτίων), desgl. thessal. Ἐροτοκλίας; — arkad.: ἑκοτόν in Ἑκοτόνβοια, δέκοτος, δυώδεκο, ἐφθορκώς; kypr. στροπά ἀστραπή, ὄν = ἀνά, κορζία καρδία; — dor.: τέτορες (τέσσαρες, wohl Einfluss des ϝ von τέτϝαρες), κοθαρός (καθαρός), auch eleisch κόθαρσις; ἀνεπιγρόφως tab. Heracl. I, 84 neben γράφω, doch auch γρόφων Partic. Melos (Röhl J. gr. ant. 12. 412), ἀπογρόφονσι Kreta (γροφεύς Elis neben τὸ γράφος), γροφά γροφίς Epidaur., aber immer ἔγραψα, auch Aor. Pass. ἀγγραφῆμεν Kret., wonach ο auf das Präsens und seine Ableitungen (G. Meyer 27\textsuperscript{2}) beschränkt scheint; kret. ἀβλοπές (ἀβλαβές) ἀβλοπία (Oaxos); — altion.: πόρδαλις v. l. Il. ν, 103. φ, 573, wo aber Aristarch πάρδαλις, wie παρδαλέη, vgl. Spitzner ad ν, 103; — att.: ὀστακός (ἀστακός), nach Athen. 3, p. 105, b., ὀσταφίς (ἀσταφίς, σταφίς), ἄλοξ (αὖλαξ, Hesych. auch ὄλοκες); μολάχη Vase, Kretschmer K. Z. 29, 410. ε u. o,ο u. ε: lesb.: ἔδοντες (ὀδόντες), ἐδύνα (ὀδύνη), aber ὀρράτω st. ἐρράτω εἰράτω v. εἴρω necto, ἐπιτρόπης = -τρέπεις, Theokr. 29, 35 (dor. τράπειν, was der äolischen Form zunächst zu Grunde liegt), στρόφω f. στρέφω (dor. στράφω); — böot.: Ἐρχομενός, Τρεφώνιος; Ἐρχομενός hiess auch das arkad. Orchomenos b. d. Einw.; mit Ε auch att. Inschr., mit Ο erst im 3. Jahrh. v. Chr.; vgl. jungatt. Ὀρχιά f. Ἐρχιά unten; ὀβελός neben ὀβολός; auch attisch beide Formen, und zwar scheint ο aus Assimilation an die Endung ός hervorgegangen, daher (Inschr.) stets ὀβελίσκος, ὀβελεία, διωβελία, ἡμιωβέλιον (Meisterhans, Gr. d. a<*>. Inschr. 18\textsuperscript{2}); in der ursprünglichen Bedeutung Spiess hielt sich das ε immer; dorisch, arkadisch ist ὀδελός; — ferner dorisch ἑβδεμήκοντα, also auch ἕβδεμος (Ahrens II, 281), woher ἑβδεμαῖος Epidauros; Ἀπέλλων f. Ἀπόλλων weit verbreitet, wiewohl auch Ἀπόλλων dorisch; in Eigenn. wie Ἀπελλῆς, Ἀπελλίκων, Ἀπελλίων auch ausserhalb des Dorismus (G. Meyer 32\textsuperscript{2}); γεργύρα (γοργ.) Alkm. fr. 132, ϝέργανον (γέργ.) Hesych.; περτί pamphyl. in περτέδωκε, vgl. πρές lesb. n. d. Gramm. für πρός, Meister, Dial. I, 44; umgek. Κόρκυρα d. einheimische Name, wofür attisch im 4. Jahrh. Κέρκυρα (Meisterhans 17\textsuperscript{2}); κρέμυον = κρόμυον kennen die Gramm., daher Κρεμμυών Flecken bei --119-- Korinth; ἔνυμα lakon. f. ὄνομα, s. § 44, ὀλινύει Hesych. = ἐλινύει; — attisch: τριακόντορος u. -ερος Inschr. (Meisterhans das.), b. d. Autoren mit ο, was auch auf d. Inschr. häufiger, Herodot τριακόντερος πεντηκόντερος (ἐρέσσω); Πυανοψιών, erst nachchristl. -εψιών Inschr.; ebenso Ὀρχιεύς i. röm. Zeit für Ἐρχιεύς; ἑρκάνη Ael. Dionys. Eust. 969, 1, in unsern Texten ὁρκάνη; die Inschr. auch Κερσεβλέπτης für Κερσοβλ. der Autoren; im 5. Jahrh. nebeneinander Ἀλωπεκοννήσιοι und (mit Assimilation) Ἀλωποκονν. (wie Τριπτόλομος Vasen); bei Autoren schwankend ὀχυρός u. ἐχυρός, jenes älter (Hesiod, s. G. Meyer 9\textsuperscript{2}); — ionisch ist ἑξάπεδος Herodot. 2, 149 für att. ἑξάπους. Ausserdem kommen noch folgende Fälle vor: ε u. ι: Lesb. in τέρτος (τρίτος), vgl. lat. tertius; κέρναν Inschr. Alc. 41 = κιρνάναι, κεραννύναι (ε urspr., vgl. πίτνημι-πετάννυμι, σκίδνημι-σκεδάννυμι u. s. w., § 41); — thessal. starkes Schwanken, als Ὑβρεστάς, ἀπελευθερεσθές (-σθείς) wie von ἀπελευθερίζω, Mitt. d. arch. Inst. 1889, 59 f. (Pherai), κρεννέμεν (κρίνν., κρίνειν), ἀνεθείκαεν und -ιν st. ἀνέθηκαν; πατρουέαν πατρωΐαν; — dor. Σεκυών einheim. Namensform, vgl. Apollon. Adv. p. 555, Dial.-I. 3162, 3167, 3169; — b. Hom. ἀγχέμαχοι (neben ἀγχιμαχητής, ἀγχίμολος), auch att. Καλλένικος, und so Schwanken zw. ἀρχε- und ἀρχι-, Χαιρε- und Χαιρι- (Meisterhans, Gr. d. att. Inschr. 90\textsuperscript{2} f.); att. μελέϊνος nb. -ίινος von μελία (Dissimilation); ι u. ε: Böot. vor einem Vokale in θιός (θεός), Τιμασίθιος, Θιογίτων, χρίος (χρέος), κλίος (κλέος), νίος (νέος), ϝίαρ (ἔαρ), in den obliquen Kasus der III. Dekl. von Wörtern auf εις (= ης), ος n., υ n., als: Πραξιτέλιος (Πραξιτέλους) v. Nom. Πραξιτέλεις (Πραξιτέλης), Ἀλκισθένιος u. s. w., ϝέτια (ἔτεα, ἔτη), ϝάστιος (ἄστεος), ϝικατιϝέτιες, in den Pron. ἱών (ἐγών), τιοῦς (doch τεοῦς Corinn. fr. 11, ἑοῦς 2, ἁμίων u. οὐμίων (ἡμέων u. ὑμέων), τιός (τεός, σός); in der Konjug.: ἴει = ἔῃ ᾖ, ἰών (ἐών), ἴωνθι (ἔωσι), ἀνέθιαν (ἀνέθεαν = ἀνέθηκαν), besonders in den Verbis contractis auf έω: ἐπολέμιον, ἀσεβίοντας, πολεμαρχιόντων, αὐλίοντος u. s. w., δοκίει (δοκέῃ); der Wechsel ist also durchgehend, nur dass εε und εει (= εη) gewöhnlich ει, ε + ι (ε + ει) ῑ wird, s. § 50; in Thespiä aber (z. T. auch in Theben) bleibt ε; vermittelnde Schreibung ει in ἀνέθειαν, Θειογίτα; ausserdem vor ς mit Konson. ἱστία (wie dor., arkad., ion.), πρισγεῖες d. i. πρεσβῆες, πρέσβεις, ει in Θεισπιεύς, vgl. § 27; — thessal. Λίων, Κλιόμαχος (Krannon), doch andere Orte ε; λιθίας Larisa, nach Fick für λιθέας, vgl. χρύσιος; — arkad. ἰν = ἐν, Τηλίμαχος vgl. oben ἀγχι- u. ἀγχε- u. s.; — kypr. ἰν; vor Vok. ϝέπιjα, κατέθιjαν, so vor α stets, vor ο dagegen auch ε, als θιός und θεός; — lesb. in den Derivatis auf ιος (= εος), ία, ιον, als: φλόγιον (φλόγεον) Alc. 39, πορφυρίαν Sapph. 64, χάλκιαι und --120-- κυνίαισι Alc. 15, σιδάριος Theokr. 29, 24, δενδρίῳ ib. 12, u. in ὄψι (ὀψέ) Adesp. 55 Bgk., vgl. ὀψιμαθής u. s.; Inschr. vereinzelt γλύκιος D.-I. 272; aber in den Stoffadj. ιος auf d. Inschr. fest, wonach Meisters Vertheidigung des ειος εος b. lesb. Dichtern unhaltbar ist; — dor. a) in ἱστία (ἑστία), ἱστιῶ (ἑστιῶ), ἱστιῶντ Epich. 19 (auf Inschr. auch ἑστία, Ἑστία, mit ι auch böot. arkad.); b) vor folgendem Vokale, allgemeiner bei Adjekt. auf ιος (εος): ἀργύριον Epich. fr. 5, φοινίκιαι 12, βόϊον 77, φοινικίῳ v. l. Theokr. 2,2, aber gew. Theokr. εος (auch auf Inschr. v. Delphi, Rhodos u. a. O. χρύσεος, χάλκεος), ὄστιον u. ὄστια Theokr., συκία = συκῆ tab. Heracl.; ausgedehnter strengdor.: θιός (θεός) kretisch, σιός (θεός) u. σιά (θεά) lakon., θιήϊον (θέειον, θεῖον) kret., περιστεριών desgl.; γαλλιῶται (γαλεῶται) lakon. b. Hesych.; Gortyn. Taf. ἀδελφιός ἀδελφιά, πλίας πλίασι vgl. hom. πλέες = πλείονες, θῖνος d. i. θίινος θέϊνος göttlich, sonst kret. ψούδια ψεύδη, ἐμμανίας ἐμμανεῖς, Κρητογενία = -νῆ, συγγενίεν = συγγενεῖς; Gen. Τιμοκράτιος tab. Heracl. 1, 166 st. -εος; c) (strengdor.) Gen. Pron. pers. b. d. Tarent. Rhinthon: ἐμίο, ἐμίω, ἐμίως, τίω, τίος, τίως = ἐμέο, τέο; d. desgl. in dem strengeren Dorismus bei den Verben auf έω vor ο und ω: Ar. Lys. 198 ἐπαινίω, μογίομες = μογέομεν, ἀδικίομες, ὑμνίωμες = ὑμνέωμεν, λυχνοφορίοντες = -έοντες, so auch im Fut.: ὀμιώμεθα141) 183 = ὀμεόμεθα, ὀμούμεθα; auf den Herakl. Tafeln ἀδικίων, ἐξεπόϊον ἐξεποίεον, ποϊόντασσι = ποιεόντεσσι, ποΐων, ποΐωντι = ποιέωσι, Fut. ἀνανγελίοντι = ἀναγγελέουσι u. a.; mit ω st. ο142): II. 18. 45 ἐμετρίωμες = ἐμετρέομες; auf kret. Inschr. κοσμίοντες, ὁρμιόμενοι v. ὁρμίω = ὁρμέω st. ὁρμάω, Fut. ἐμμενίω, βοαθησίω, προλειψίω, πραξίομες, χαριξιόμεθα, φυλαξίομεν (doch auch πωλέοντα, ὠνεόμενον, ἐπαινέομεν u. a.); — alt- und neuion. in ἱστίη Hom. u. Her. (ἑστίη, ἑστία), Ἱστίαια Hom., ἱστιητορίου, ἐπίστιος (att. ἐφέστιος), ἱστία (Imperfekt), ἱστιῆσθαι, Ἱστιαιεύς, Ἱστιαίην (Alles b. Herod., an wenigen Stellen ἑστ., die Bredov., p. 146 korrigieren will); att. ἴσθι f. ἔσθι sei, ἔσθι Hekataeus b. Hdn. II, 355 (Hom. u. äol. ἔσσο). ι u. υ: lesb. anlautend vor p in ἰψήλων (ὑψήλων) Adesp. 60 Bgk., ἴψος (ὕψος), ἴπαρ ἰπέρ (ὕπαρ, ὑπέρ). So die Grammatiker; es mangelt die Bestätigung auf Inschr. oder in Fragm., ausser ἴψοι Sapph. fr. 91 (so cod. A corr. des Hephästion, Studemund, Anecd. p. 117). Indes sind die Zweifel unberechtigt, s. Thumb, Spir. asp., 46 f. --121-- — Über Schwanken zw. ι u. υ im Attischen und in der κοινή s. § 9, 5. — Singulär πτέον att. für πτύον Ael. Dionys. Eust. 948, 19. υ u. α: S. § 9, 4. Lesb. in σύρκες (σάρκες) u. πές (ς) υρες, Hom. πίσυρες (τέσσαρες); υ ist hier (Ahrens I, pag. 79) aus ϝα entstanden: πέτϝαρες, σϝάρκες (vgl. § 19, Anm. 3); Βύκχις Eigenn. (zu Βάκχος); arkad. κατύ f. κατά. υ u. ο: Lesb. ziemlich oft als An-, In- und Auslaut, als: δύσσευς (Οδυσσεύς), ὔσδος (ὄζος) Sapph. 4, ὔμοιος (ὁμοῖος) Theokr. 29, 20, ὐμάρτη ib. 28, 3, ὐμαλίκων hergest. 30, 20, ὔμοι (ὁμοῦ) Balbilla; ὔμφαλος, ὔπισθα, ἐξύπισθα; — δύνει (δονεῖ) Sapph. 40, μύγις, ὄνυμα (auch dor.; thessal. Ὀνύμαρχος, böot. ὠνούμηνε Corinn. 2, ὄν (ι) ουμα Inschr.; in Kompositis auch in anderen Dialekten, als: ἀνώνυμος u. s. w.), στύμα Theokr. 29, 25 (Στυμάργου Hipp. V, 84); ἄγυρις Gramm. (vgl. ὁμήγυρις, πανήγυρις, aus -γυρρις -γυρσις, arkad. πανάγορσις, ὁμήγορις kret. Epigr. Bull. de corr. hell. 1889, 59 f.), Μεγαλάγυρος b. Strab. 13, p. 617 (auch att. ἀγύρτης, συναγυρμός Plat.); doch ἀγόραν Dial.-Inschr. 311; — ἀπύ Alc. 33, 84, ἀπὺ Φωκάας Sapph. 44, ἀπυστρέφονται Sapph. 78, auch Inschr. öfter, wiewohl früh das vulgäre ἀπό eindringt (auch thess., arkad., kypr. ἀπύ); δεῦρυ (δεῦρο); — arkad. ausser ἀπύ auch ἄλλυ; — kypr. ἀπύ, -τυ für το 3. Pers. Med., als γένοιτυ; — dor. in ὄνυμα Epich. fr. 27, ὄνυμα u. ὀνυμάζω Pind., vgl. oben; wie ἀνώνυμος, πανήγυρις ist ὑπωρυφία nb. ὀροφά Epidaur. Dial.-I. 3325 v. 42 [auch att. πευτώρυγος διώρυγος u. s. w. von ὀρ (ό) γυια, Meisterhans 20\textsuperscript{2}, Wackernagel, Dehnungsges. d. gr. Kompos. 49]; — episch in ἄλλυδις v. St. ἀλλο, ἄμυδις (Hom. neben ἅμα), beides Aeolismen, wie bei ἄμ. der Spiritus anzeigt; — neuion. in ὐπέατι Herod. 4, 70 st. ὀπέατι v. d. Nom. ὔπεας st. ὄπεας (Lobeck. Pathol. p. 218, not. 32); ῥυφεῖν ῥοφεῖν Hippon. 132, auch Hippokr. nach Hdschr. (VI, 198 cod. θ; Ermerins I, 658 hat mit Recht ῥυφ. aufgenommen; VII, 20. 26 u. ο. ῥοφανέτω v. l. ῥυμφανέτω; auch dor. m. υ Eust. 1430); chalkidisch (Kyme) ὑπύ = ὑπό. — Vulgär. τρυφαλίς st. τροφαλίς, Hdn. I, pag. 91. ο u. υ: Lesb. in πρότανις, προτάνιος auf Inschr. (auf späteren auch mit υ; προτανεία προτανεύω auch auf einigen att. Inschr. um 300, Meisterhans, Gr. d. att. Inschr., 19\textsuperscript{2}). Vgl. § 9, 5.
\section{(b) Lange Vokale und Diphthonge}

Wechsel der zwei langen Vokale: ᾱ und η und Bemerkungen über das kurze α. Die langen Vokale η und ω stehen mit ᾱ in engster Verbindung, s. § 9, 3. Der Gebrauch des η statt des langen α ist eine ganz besondere Eigentümlichkeit der ionischen Mundart, und hierin zumeist scheidet sich diese von der dorischen sowie den äolischen und pseudäolischen, --122-- welche das ᾱ rein bewahrt haben. Dazwischen steht die attische, welche im Gebrauche des ᾱ und η eine schöne Mitte hält, indem sie durch Abwechslung dieser beiden langen Vokale die Eintönigkeit sowohl des sich so oft häufenden langen α als des η vermeidet. Man vergleiche das attische ἡμέρᾶ mit dem dorischen ἁμέρα und dem ionischen ἡμέρη: jenes hat etwas breites, wie die Alten selber fanden,143) dieses ist allzu dünn. Aber der Gebrauch des ᾱ erstreckt sich bei Doriern und Aeoliern nicht so weit, wie der des η bei den Ioniern; denn |während dieses sowohl aus einem ursprünglichen α als auch aus ε hervorgeht, beschränkt sich das äolische und dorische ᾱ auf die Fälle, in welchen ein α zu Grunde liegt; wo aber ein ε zu Grunde liegt, gebrauchen die asiatischen Aeolier (Lesbier), die Arkadier und die Dorier ebenso wie die Ionier η, die böotischen und thessalischen Aeolier ει,144) z. B. äol. u. dor. λθᾶ, [root ] λαθ, vgl. λαθ-εῖν, ion. u. att. λήθη; aber: lesb., arkad. u. dor. μάτηρ (Stamm ματερ- in ματέρες), ionisch att. μήτηρ, böot. thessal. μάτειρ. Die elische Mundart indes gebraucht das ᾱ auch in solchen Fällen, wo die Dorier, Arkadier und Lesbier η, die Böotier und Thessalier ει haben, als: μά = μή, ϝράτρα st. ϝρήτρα ῥήτρα, εἴα = εἴη, πατάρ = πατήρ; als Kürze entspricht zum Teil α, als in den obliquen Kasus der Wörter auf ήρ145) und im Optativ (συνέαν, s. § 24, 1), doch geht der Gebrauch des ᾱ anscheinend weiter als der des α, und lässt nicht viele η übrig.146) Hervorzuheben ist noch, dass auf einigen Inseln des ägäischen Meeres (nam. Keos und Naxos) das speziell ionische ē und das (mit Ausnahme der Eleer) gemeingriechische in der Aussprache und darnach auch in der Schrift unterschieden wurden: nur jenes war è (offenes ē) und wurde mit Η geschrieben, während dieses é (geschlossen) war und durch Ε mitbezeichnet wurde: also ΜΗΤΕΡ dor. μάτηρ spr. mètér.147) In diesen Dialekten also fällt gemeingriechisches η mit der Dehnung von ε (ion. att. ει) zusammen, indem dies (unechte) ει von Haus eben ē´ ist, und sie berühren sich eng mit dem Böotischen und Thessalischen, deren ει urspr. auch mit blossem Ε bezeichnet wird; dagegen in den übrigen --123-- ionischen Mundarten und im Attischen ist gemeingriech. ē wie ionisches ē ununterschieden è gewesen, und ει (Ε) blieb für sich. Das äol.-dorische ᾱ und das ionische η findet sich a) in Stämmen, als: ἆδυς lesb., ἁδύς, ἅδομαι dor., ἡδύς, ἡδονή ion. und att., [root ] ἁδ-, vgl. ἁδ-εῖν; ἁγεῖσθαι dor., ἡγεῖσθαι ion. u. att., ἄγ-ω, daher στρατᾶγός dor., στρατηγός ion. und att.; μᾶκος dor., μῆκος ion. und att., [root ] μακ, vgl. μακ-ρός; στάλα dor., στήλη ion. und att., [root ] στα, vgl. ἱ-στά-ναι; θνατός dor., θνητός ion. u. att., [root ] θαν, vgl. θαν-εῖν; χάν dor. Epidaur. D.-I. 3340, Z. 134, χήν ion. u. att.; — b) in Flexions- und Ableitungsformen, wie in der I. Dekl. und allen davon abgeleiteten Wörtern und Wortformen, als: νίκα, ᾶς, ᾳ, ᾶν, νικαφόρος, ἐνίκασα, νικασῶ u. s. w. dor., = νίκη, ης, ῃ, ην, νικηφόρος, ἐνίκησα, νικήσω u. s. w. ion. u. att., Μοῦσα, ᾶς, ᾳ dor., ης, ῃ ion. u. att.; Ἀτρείδας, δᾳ, δαν dor., Ἀτρείδης, ῃ, ην ion. u. att.; Ableitungen γάϊος von γᾶ, γῆ, ἀλκάεις von ἀλκά, ἀλκή, σιγαλός von σιγά, σιγή, ὀδυνᾶρός von ὀδύνα, ὀδύνη; so auch die Adv. auf ᾳ u. ᾶν dor., ῃ u. ην ion. u. att., als: πᾷ, παντᾷ, ἁσυχᾷ, κρυφᾷ, κρύβδαν; πῇ, πάντῃ, ἡσυχῇ, κρυφῇ, κρύβδην; in Verbalendungen, als: ἐρρύαν dor., ἐρρύην ion. u. att., namentlich in denen auf μᾶν, τᾶν, σθᾶν dor., μην, την, σθην ion. u. att., als: συνεθέμαν συνεθέμην, ἠχθόμαν ἠχθόμην, ὀλοίμαν ὀλοίμην, ἐποιησάταν ἐποιησάτην, ἐκτησάσθαν ἐκτησάσθην, ὀλοίσθαν ὀλοίσθην; in der Tempusbildung der V. liquida, als: ἔσᾶναν, ἐκύδᾶνεν, ἔφᾶνας, ἔσφᾶλε Pind.; im Augmente von Verben, die mit α anlauten, als: ἀρχόμαν (zu ἄρχομαι), ἆγον, ἄγγειλα, desgleichen bei Diphthongen, als: αὔξησα, αὔδασα (wohl mit Verkürzung des ᾱ im Diphth.) dor., ἠρχόμην, ἦγον, ἤγγειλα, ηὔξησα, ηὔδησα ion. u. att.; ferner in Bildungssilben, als: in der Endung τᾶς, G. τατος (Lat. tās, G. tātis), ion. u. att. της, G. τητος, der Substantiva abstracta, als: ταχυτάς, G. ταχυτᾶτος, ταχυτής, G. ταχυτῆτος, νεότας, νεότης; in der Endung ᾶξ, G. ᾶκος, ion. ηξ, G. ηκος, att. nach ρ ᾶξ, ᾶκος, der Subst., als: μύρμηξ ion. att., μύρμᾶξ dor., θώρᾶξ, ᾶκος, dor. u. att., θώρηξ, ηκος, ion.; in der Endung ᾶν, G. ᾶνος, ion. u. att. ην, G. ηνος, der Volksnamen, als: Ἕλλᾶν, ᾶνος, Ἕλλην, ηνος, aber nach ι auch att. α, als: Αἰνιάν, ion. Αἰνιήν; fast immer in der Endung ᾶνᾶ, ion. ηνη, der Subst., als: Ἀθάνα, Ἀθήνη, Ἀθᾶναι, Ἀθῆναι, Μεσσάνα, Μεσσήνη; in dem ersten Teile der Komposita, in denen dor. ᾱ statt des gewöhnlichen ο steht, als: πολεμᾶδόκος, στεφανᾶφορία st. πολεμηδ., στεφανηφ., u. dieses st. πολεμοδ., στεφανοφ.; τριταμόριον, πεμπταμόριον (Archimed.); so auch im Att. βιβλιαγράφος; im 2. Teile bei der Dehnung des α, als εὐνεμος εὐήνεμος, φιλνωρ φιλήνωρ von ἀνήρ. Hingegen stimmen der Aeolismus und der Dorismus in dem Gebrauche des η (böot. thessal. dafür ει) mit dem Ionismus und Atticismus überein, wenn η aus ε hervorgegangen ist (Nr. 2), wie in --124-- dem Nom. III. Dekl. auf ης, G. εος, ηρ, G. ερος (ρος), ηρ, G. ηρος, Vok. ερ u. Fem. ειρα, ην, G. ενος, als: εὐγενής, böot. thess. εὐγενείς; πατήρ, böot. thess. πατείρ; σωτήρ (Vok. σῶτερ, Fem. σώτειρα), σωτηρία u. s. w. u. nach dieser Analogie δικαστήρ δικαστήριον u. s. w.; ποιμήν, φρήν; — ferner in γῆρας, ἦθος, ἀκήρατος, κρημνός, vgl. γέρας, ἔθος, κεράσαι, κρεμ-άσαι; θήσω, συνθήκα (συνθήνα, ἐπιθήνα), ἄρνησις, κινήσω, ἀκίνητος, γνήσιος, σκληρός, v. [root ] θε, ἀρνέ-ομαι, κινέ-ω, [root ] γεν, σκελ; in den Konjunktivendungen, als: βλάπτῃ, γίνηται, vgl. Indik. βλάπτει, γίνεται; in den Indikativ- und Optativendungen auf ην, als: ἐδικάσθην, εἴην, δηλωθείη, vgl. ἐδίκασθεν, εἶεν; in den Endungen ημαι, ήθην, ησθαι, die dem Stamme angesetzt werden, als: γενήθην (= γενηθῆναι) Inschr. Kyme, Dial.-Inschr. 311; im Augmente von Verben, die mit ε anlauten, als: ἠρχόμαν v. ἔρχομαι. Anm. 1. Bei den Verben findet vielfach ein Schwanken zwischen der Bildung auf -άω und der auf -έω statt, und daher ist (Ahrens II, p. 147 sq.) das -ασα, -άσω bei solchen Verben zu erklären, die nach der gewöhnlichen Bildung auf -έω ausgehen, als: ἐπτοάθην Eur. Iph. A. 584, vgl. d. äol. ἐπτόασεν Sapph. 2, 6; v. ποτάομαι ἐκπεποταμένα Sapph. 68, dagegen v. ποτέομαι ποτέονται Alc. 43, πότῃ st. πότησαι Sapph. 41, ποτήμενα Theokr. 29, 30 (Ahrens I, p. 85, Meister I, 180). Besonders schwanken solche Verben, welche von einem Substantive der I. Dekl. abgeleitet sind, als: δινέω (v. δίνη) ἐδινάθην Pind. P. 1.1, 38 (v. l. -ήθην), ὠκυδινάτοις J. 4, 5 (-ήτοις Mommsen), δίνασεν Eur. H. F. 1459, φωνέω (v. φωνή) φώνασε Pind. (doch auch φωνήσαις, ἀφώνητος), ὠνέομαι (v. ὠνή) ὠνασεῖται Sophr. 89; aber auch πονέω (v. πόνος) ἐξεπόνασαν Sapph. 98, ἐξεπόνασεν Eur. Iph. A. 209, ποναθῇ u. πεποναμένον Pind. wie von πονάω (doch auch ἐπόνησα, ἐξεπόνησεν). Umgekehrt: κτάομαι κτήσασθαι (Pind. P. 9, 52) κτῆμα (Mytil. Dial.-Inschr. 214), κτῆσις (Kyme das. 311), böot. Κτεισίας, vgl. κτέαρ, κτέανον; χράομαι ebenso durchgängig mit η, vgl. χρέος, χρεία, aber auch Präs. ἀποχρέω Epich. 114, καταχρεῖσθαι καταχρείσθωσαν Delphi, Dittenb. Syll. 233, 37. 58, χρηείσθω Kalchedon das. 369, 7 u. s. (s. § 343); λάω λῶ ich will, λῆμα Pind., λῆϊς (= λῆσις, βούλησις) lakon., aber auch im Präsens statt λῶ λε (ί) ω kret. u. s., s. § 343. Anmerk. 2. Von Verben, die ihren Stamm für die Ableitung der Tempora mit Synkope oder Metathesis umwandeln, sind hier hervorzuheben: βάλλω, βλη- auch äol.-dor., also βεβλήμεναι Alcae fr. 15, 5, vgl. βλείης Epicharm. fr. 154, βέλος, arkad. δέλλω; καλέω, κλη auch äol.-dor. (vgl. κέλομαι), daher κέκλημαι, κικλήσκω Pind., κατάκλητος u. ἐκκλησία Inschr. Dagegen von δέμω, baue, findet sich bei Pindar u. d. Tragg. mehrfach δμᾶ, in θεόδματος, εὔδματος; doch mangeln nicht die Varianten mit η, s. Mommsen zu Pind. Ol. 3, 7. Ferner kommt von τέμνω (dor. τάμνω) τμᾶ, als ἐτμάθην, τμᾶμα Archimedes, wiewohl τέτμηνθ Pind. J. 5 (6), 22. Τέθνᾶκα, κέκμᾶκα, δέδμακα (θαν, καμ, δαμ) bedürfen kaum der Hervorhebung. Anmerk. 3. Das η bleibt dor. in mehreren Fällen, wo der Ursprung des Vokales nicht deutlich ist,148) nämlich: a) in den Subst. auf ης, G. ητ-ος, als: Κρής (Κρήτα; daf. Κρεήτη Archiloch.), Κωρῆτες, λέβητες Epich. (λέβεις böot.), Μαγνής Pind.; b) in den Adj. auf ηρος u. ηλος, als: πονηρός, ὀκνηρός, ὑψηλός u. s. w.; vgl. indes oben 3, b); c) in den Zahlw. auf ήκοντα u. ηκοστός, als: πεντήκοντα, πεντηκοστός (πεντακοστός Archimedes), ἑβδεμήκοντα (ἑβδομείκοντα böot.); d) in den Verbalformen, die --125-- an die Wurzel η ansetzen, samt den entsprechend gebildeten Derivatis, als: γεγενημένος (St. γεν), ἐκελήσατο Epich. 48, ἐθελήσω, μέλημα Pind., σχήσω, εὐσχήμων; auffällig μεμενακός Archimedes, Heiberg, Fl. Jahrb., Suppl. XIII, 549; e) in mehreren besonderen Wörtern, als: ἀρήγω, βληχρός, βροτήσιος, δή, ἐπειδή (ἐπιδεί böot.), δῆλος (aber Δᾶλος die Insel), ἤ, ἤδη, ἥβα (εἵβα böot. thessal.; b. Theokrit u. A. ἅβα, junglesb. Inschr. ἔφαβος, vgl. Ahrens II, 151; Meister I, 64; J. Weidgen, Qua ratione Euripides in carm. mel. Doridem temperaverit [Jena 1874], p. 14), ἧμαι, ἥμερος tab. Heracl. I, 172, vgl. ἀνήμερος Eur. Hec. 1057, ΕΜΕΡΟΣ mit E = é Keos, Bechtel, Inschr. d. ion. Dial. 47, S. 49, s. oben 2) (ἅμερος Pind., Aesch. Ag. 721), ἥμισυς und ἡμι- in Kompos., als ἡμιλίτριον (Theokr. ἅμιου 29, 5; über lesb. αἴμισυς s. § 26; es wird auch b. Theokr. αἴμιου zu schr. s.; aber ἅμισυς, ἁμιόλιος auch Archimedes, Heiberg, Fl. Jahrb. Suppl. XIII, 549 f.; Ἥρα, ἥρως, ἥσσων, Θῆβαι (böot. Θειβῆος = Θειβαῖος), θῆλυς, θήρ (θησαυρός zu τίθημι), θρῆνος, κάπηλος, κηρός (Dial.-Inschr. 3325, v. 271), κρηπίς, λήγω, μή, μήδομαι, μῆλον Schaf (μεῖλον böot.; dagegen μᾶλον Apfel), μην-ός, lesb. μῆνν-ος von μής, μείς; μηρός (böot. μειρός, Meister I, 222); μῆτις, νήπιος, ξηρός, πῆμα, τηρέω (τηρεῖ Alkm. 23, col. III, 9), χῆρος u. a. Besonders hervorzuheben sind: πλη- trotz πιμπλάναι, vgl. πλείων, πλεῖστος, daher ἐνέπλησαν Sophr. 30, πλήθω, πλῆθυς, πλήθα lokr. (eleisch πλαθύω, πλᾶθος kret. Inschr., auch Kyme spät D.-I. 311), πλήρης (böot. πλειάρειν Akk. Sg. Etym. M.) u. s. w., G. Meyer 41\textsuperscript{2}, Meister, Dial. I, 69; πρη- trotz πιμπράναι, so lesb. ἐνέπρησε, Herakl. ἐμπρησόντι; ῥήγνυμι trotz ῥαγῆναι, daher ῥῆξαι, ἐρρηγεῖα Herakl. = ἐρρωγυῖα, lesb. ϝρῆξις, αὔρηκτος = ἀϝρ., ἄρρηκτος Herakl. Von ῥη, ϝρη kommt ῥήτωρ, ϝρήτα (kypr.), ῥήτρα (doch ϝράτρα eleisch, und auch der kret. Ζεὺς Ὀράτριος scheint hierher zu gehören, = ϝράτριος), ῥησίαρχος (Epicharm.), ἄρρητος (Alkm.) u. s. w. Ferner auf ήνα (oben 3): Μυτιλήνα die einheimische Namensform, Dial.- Inschr. 213, vgl. Meister, Dial. I, 70; εἰρήνα s. das. 69; II, 93, so Pindar nach fast einhelliger Überl., Peter, dial. Pind. 9 f.; desgl. Alkman 23, III, 23 εἰρήνας, vgl. dens. b. Prisc. I, 22 (indes ἰράνα böot. oft, auch arkad. ἰράνα). Σελήνα steht bei Archimedes, Heiberg Fl. Jahrb., Suppl. XIII, 549. Bei ἥσυχος schwankt die Überlieferung: bei Pindar ist öfters in allen Hdschr. η überliefert und wird von Hsg. wie Mommsen durchweg hergestellt (dagegen α Bergk); für η die theban. Inschr. Philol. 1889, 418 ΙΣΟΥΧΙΟΣ = Εἱσούχιος. Στῆθος steht bei äol. Dichtern und Pind. frg. 218 (239); ςτᾶθος Dial.-Inschr. Sikyon 3163. Über Ζάν, Δάν G. Ζανός nb. Ζήν, Δήν, Δηνός s. § 130. Über das Verhältnis der attischen Sprache zu der ionischen ist Folgendes zu bemerken: a) Statt des ionischen η gebrauchen die Attiker, wenn demselben einer der Vokale ε, ι oder ein ρ vorangeht, das lange α, als: ion.: χώρη, ης, ῃ, ην, νεηνίης, ἰητρός, θώρηξ, ηκος, κρητήρ, τρηχύνω, τρηχέως, πρήσσω, πειρήσομαι, θεήσεσθαι, θέητρον, ἀπέδρη, ἐθυμίησε, θυμίημα, Συρήκουσαι, γρηῦς, κέκρημαι, λάθρῃ, λίην, πέρην u. s. w., att. χώρᾱ, ᾱς, ᾳ, ᾱν, νεανίᾱς, θώρᾱξ, ᾱκος, κρᾱτήρ u. s. w.; so auch in den metr. Inschriften Attikas, Kirchhoff, Herm. V, 54, wo sogar ἆνορέαν f. ion. ἠνορέην; Τρᾶρας sagte Theopomp für Τρῆρας, Hdn. II, 593; in Kompos., als: Ion. διήκονος, διηκονεῖν, att. δικονος; γενεηλογεῖν, att. γενεᾶλ.; sogar διᾶνεκής att. Inschr., Kom., Platon (Meisterhans 13\textsuperscript{2}) f. διηνεκής des Ion.; wenn ἐνεγκεῖν darin steckt, wohl att. Umformung des im Ion. gebildeten Wortes [ebenso dor. διανεκής, Byzanz D.-I. --126-- 3059; aber hellenistisch mit η, s. Bechtel z. d. Inschr.]; — b) die Abstrakta von Adj. auf ης u. ους gehen ion. aus auf είη u. οίη, altatt. auf είᾶ u. οίᾶ nach Aelius Dionys. b. Eustath. ad Od. η, p. 1579, 27, der anführt: ἀναιδεία u. προνοία aus Aristoph., ἀγνοία (so Soph. Tr. 350), εὐκλεία (so Aesch. S. 685); Buttm. I, § 34, A. 4 fügt hinzu: ὑγιεία Ar. Av. 604, ἀνοία Aesch. S. 402. Eur. Andr. 519; bei den jüngeren Attikern aber gehen sie auf εια und οια aus, als: ἀληθείη, ης, ῃ, ην ion., ἀλήθεια, ᾱς, ᾳ, αν att., εὐνοίη ion., εὔνοια att., παλιῤῥοίη ion., παλίῤῥοια att.; — c) vereinzelt ναυᾶγός ναυᾶγεῖν ναυᾶγία ναυγιον (zu ἄγνυμι) dor. att. (-ηγός att. zu ἄγω), ion. mit η ναυηγός; κλη (καλη Bezzenberger, Btr. VII, 66), Bruch im medizin. Sinne, ion. (u. später) κήλη, Cobet, Misc. 416; (ὀπαδός für ion. ὀπηδός wie Tragg. auch Plato); ferner ν st. ἐάν, ion. ἤν ἐπν st. ἐπεάν ἐπήν nicht gut attisch, indem ausser bei Xenoph. ἐπειδαν dafür gesetzt wird]; das α steht hier um der Deutlichkeit willen, wie in ὀστᾶ aus ὀστέα. Anmerk. 4. Ausnahmen: attisch scheinbar: χρῆσθαι, χρῆμα; ῥήγνυμι, ἄῤῥηκτος, ῥῆγμα u. s. w., was auch dem Dor. u. Aeol. gemeinsam, gleichwie ῥῆμα u. s. w.; ausserdem zuw. auf Inschr. in fremden Eigenn., als Αὐλιῆται, Ἰουλιῆται (Meisterhans 13\textsuperscript{2}); über die Kontraktion ὑγιῆ s. § 123, Anm. 8; ἰήλεμος Θρῇξ u. s. w. Tragg., vgl. S. 32; ion. b. Herodot mehrere dor. und fremde Eigennamen, als: Ἀρχέλαοι, Θήρας, α (Gen.), αν (aber d. Insel Θήρη, ης, ῃ, ην), Θαννύρας, α, Ἀμίλκας, ᾳ, αν, Ἀριστέας, Ὀνεᾶται, Ὑᾶται, Χοιρεᾶται, Τιθορέα 8, 32, Κρᾶθις; mit kurzem α μεσαμβρίη, att. μεσημβρία (von ἡμέρα), ἀμφισβατέειν, ἀμφισβασίας (auch Inschr. Zeleia Bechtel, Inschr. d. ion. Dial. 113 ἀμφισβατῆι), λαξεσθαι 7, 144, λελαμμαι, att. εἴλημμαι; aus euphon. Grunde ᾶήρ (b. Hippokr. auch ἠήρ) 149) aber ἠέρος u. s. w., ἐάσας 1, 90 (wie auch Hom., der ebenso ἑᾶδότα, ἑᾶνός hat);, auffällig καραδοκεῖν 7, 163; dazu kommen aber sehr viele Wörter und Formen, wo aus Ersatzdehnung oder Kontraktion auch im Ionischen ᾱ hervorgegangen ist (Harder, de α vocali ap. Hom. producta, D.-I. Berl. 1876): πᾶς, πᾶσα aus πάνς, πάνσα (doch ἔμπης Hom., ἔμπᾶς Tragg.), νικήσας, -ασα, Akk. plur. τάς (Nom. acc. Du. ᾱ), ἐνίκα, νικᾶσθαι, ἄτη (aus ἀϝάτη), δαλός Hom. (aus δαϝελός, neuion. nach Schol. V Il. 15, 421 δαυλός), κᾶλον Holz (καίω, St. καϝ) Hymn. Merc. 112, Hes. op. 427, δανός (δαίω, St. δαϝ) Hom., vgl. bei den jüngeren Ioniern κέκρᾶγα, κεκράκτης Hippokr. VI, 388 (mit υ haben wir κραυγή); λᾶρός (λαύω); ριστον Frühmahlzeit Hom. u. Sp. (aus ἀϝέριστον, vgl. ἠερίη in der Morgenfrühe; αὔριον), s. Curtius, Stud. II, 175. Ferner κᾶλός schön Hom. (ᾱ auch sp. ion. Dichter wie Archilochos, Harder, S. 22 f.; G. Meyer 78\textsuperscript{2} will diesen wie dem Hom. καλλός aufnötigen), aus καλjός; ἆρή, ἆρᾶσθαι aus ἀρϝά Hom., ders. ἱκνω, κιχνω, νομαι, φθνω (?), vgl. ἐλαύνω; b. Κρ scheint Zusammenziehung zu sein, Fem. Καειρα, Lugebil, Bzz. Beitr. X, 303 f.; ᾱ vor ρ auch in φᾶρος (Hom.; att. Dichter φάρος u. φᾶρος), θυμαρέα Hom. nb. θυμῆρες (v. l. θυμαρές) Od. ι, 362. ρ, 199 (Harder, S. 72, bringt θυμαρής mit ἆράομαι zusammen, herzerwünscht), Λάρισα Hom. (Λήρισαι, Ληρισαῖος Herod.). Vor Vokal ἆΐσσω, ἀκρᾶής u. dgl. s. § 38, 4; vgl. auch unten Anm. 7; λᾶας, κρατι u. s. w., s. § 140 u. 130; nach Vok. ἀᾶγής wie ἐγη. Im --127-- späteren Ionismus auffällig φαρμᾶκός Hipponax frg. 5 ff. (n. Eustath. φάρμᾶκος ion., s. Bergk, Lyr. II\textsuperscript{4}, 462); dieselbe Quantität scheint auch Demosth. 25, 80 beobachtet zu sein, s. Blass z. St. Bei den Nr. 5 angeführten Abstraktis schwankt bei Herodot der Gebrauch zwischen beiden Formen: προνοίη, εὐνοίην, παλιῤῥοίην neben εὔνοιαν 3, 36 (εὐνοίην Stein), διάνοιαν 1, 46. 90. 2, 162. 9, 45, διάνοια 2, 169, ἀληθείη, ἀτελείη, ὑγιείη, προμηθείη, μεγαλοπρεπείη, ἀτρεκείη, εὐμαρείην u. s. w. neben εὐμένεια 2, 45, ἄδειαν 2. 121, 6, ἐπιμέλειαν 6, 105, ἐμμέλειαν 6, 139, περιφάνεια 4, 24 (είη überall Stein). Anmerk. 5. Über das dor. ᾱ bei den attischen Dichtern s. Einleit. S. 32 f. Anmerk. 6. Das kurze α bleibt auch ionisch; daher die Subst. auf υια, als: μυῖα Ὠρείθυιαν u. Ὠρειθυίην in d. Hdschr. schwankend Hdt. 7.189] (aber die Oxytona mit langem α haben υιη, als: μητρυιή), auf αια, εια, οια, als: Νίσαια, Ἐλάτεια, Εὔβοια (aber immer Ἱστιαίη b. Herod. in allen codd., b. Hom. aber Ἱστίαια, s. Bredov. p. 129; ferner Herodot Φωκαίη (z. B. 1, 165 dreimal) neben Φώκαια; Μηδείην 1, 2 (Μήδειαν Bekker). Von den Femininis auf εια von Mask. auf εύς u. ης, als: βασίλεια, regina, macht nur ἱρείη eine Ausnahme, das nach der Lehre der alten Grammatiker (s. Pierson. ad Moerid., p. 191) auch att. ἱερείᾶ, in der κοινή aber wie bei| Homer ἱέρεια lautete, vgl. § 106, 1, γ). Wo bei Herod. μίη, οὐδεμίη, μηδεμίη st. μία u. s. w. gelesen wird, ist die Lesart verderbt; ebenso wird sich die Sache beim Hippokr. verhalten; ingleichen findet sich bei Subst. auf ρα an sehr wenigen Stellen Herodots η, offenbar verderbt, als: μοίρην 1, 204. 2, 17, da an fast allen Stellen sowohl dieses Substantiv als andere α (αν) haben. S. Bredon., p. 132 sq.; ebenso ist πρῴρην 1, 194 (vgl. 7, 180) gewiss verderbt, obwohl es Lehrs auch bei Apoll. Rh. I, 372 herstellt. Ferner gebraucht Herodot πρύμνη, σμύρνη (diese beiden auch bei den Trag.), Σμύρνη, aber τόλμα 7, 135, wie zumeist b. d. Attikern, s. § 105, 1, b), aber dor. τόλμᾶ). — Statt der Endung ασιος der Zahladjektive sagt Herodot ήσιος, als: διπλήσιος, πολλαπλήσιος, πενταπλήσιος, ἑξαπλήσιος; für πεντακόσιος hat Homer, Od. γ, 7 aus metr. Bedürfnis nach gew. Lesart πεντηκόσιος, nach Aristarch u. Herodian aber πεντᾶκός. wie πᾶναπάλῳ u. dgl., s. § 75. Verdächtig ist auch ἀναπλήσσουσι für ἀναπλάσς. Hippokr. II, 58 L. Anmerk. 7. Über das Homer. ᾱ in gewissen Wörtern der I. Deklination s. § 103, 1. Homer hat auch (gegen d. ion. Dial.) vor ο, ω in weitem Umfange ᾱ, als Gen. I. Dekl. Mask. ᾶο, Plur ων; λᾶός, νᾶός, Ἀμριάρᾶος (dafür Zenodot Ἀμφιάρηος, vgl. Düntzer, Zenodot p. 50; Zen. schrieb sogar Ἀριήδνη f. Ἀριάδνη). Umgekehrt findet sich im Dor. Ἀμφιάρηος, Ἀμφιάρης, s. § 109, Anm. — Aus euphon. Gründen ψῆρας neben ψαρῶν, s. § 41; in Eigenn. Λάρισα (s. o.), σωπός, Φᾶρις, σιος, Θεᾶνώ u. a., Harder, de α vocali 93 f. — Endlich heisst es bei Hom. μν nb. μήν (Il. α, 302, β, 291 u. s.) und μέν (ἦ μέν, οὐ μέν, so auch Herodot, Krüger, Gr. II, 2, 189), att. μήν, dor. μν. Aber für πολυπμονος Il. δ, 433 ist bessere Lesart πολυπάμμονος, vgl. Πάμμονα ω, 250; Brugmann, C. St. IV. 100.
\section{Fortsetzung}

ᾱ u. ω: Böot. u. dor. πρᾶτος, ion., att., lesb. (thessal., kypr.) πρῶτος, aus πρόατος, s. § 50, 4; im weitesten Umfange dor. u. s. w. ᾱ aus αο, αω, wofür att. meist ω, als Ἀτρείδα, Ποτειδάν, Ἀτρειδᾶν, s. das.; θῶκος ion. (Hom. auch θόωκος), att. θᾶκος (lakon. θάβακος, d. i. θάϝακος), das Vb. θάσσω (θαάσσω Hom.) u. θοάζω Tragg., vgl. --128-- § 56, 1, a). Συναγαγαί f. συναγωγαί kret. Inschr. Bull. de corr. hell. IX, 17. η (= ᾱ) u. ω: neuion. in einigen Substantivis gentilibus, als: Μαιῆτις (Μαιῶτις), G. Μαιήτιδος, A. Μαιῆτιν, Μαιήτην, Μαιητέων (aber Herod. 4, 3 Μαιῶτιν in allen codd., u. so Hippokr.), Ἱστιαιήτιδος v. l. -ώτιδος 8, 23 das Gebiet von Ἱστίαια (aber Ἱστιαιώτιδος alle Hdschr. 7, 175, vgl. 1, 56), Ἀμπρακιητέων, -ῆται 9, 28 u. 31, v. l. -ωτέων, -ῶται, wie in allen Hdschr. 8, 45. 47 steht (immer Πελασγιῶτις, Φθιῶτις, Φθιῶται, Θεσσαλιῶτις, Ἰταλιωτέων). ω u. αυ· ὦλαξ dor. (αὖλαξ), vgl. ep. ὦλκα § 18 (att. ἄλοξ); ferner dor. αὐσωτοῦ f. αὐ (τὸ) ς αὐτοῦ s. § 168 Anm. 5; Καππώτας, Benennung eines gew. Steinblocks in Lakonien (Pausan. III, 22, 1), von καταπαύω; Ῥωκίονς d. i. Ῥαυκίους kret. Inschrift; neuion. διαφωσκούσῃ (v. l. διαφαυσκ.) Her. 3, 86, 9, 45 διαφαυσκούσῃ (v. ll. mit ω u. mit α), 7, 36 ὑπόφαυσιν; τρῶμα u. seine Derivata Her., Hippocr. = τραῦμα (u. so auch att. τιτρώσκω, τέτρωμαι, ἔτρωσα), θῶμα, θωμάζειν u. s. w. neben θωῦμα oder θώϋμα, welche Form Struve, Quaest. de dial. Herod. spec. III. p. 11 ff. u. Bredov. p. 142 sq. als die allein richtige anerkennen, indem sie meinen, dass in diesem Worte nicht wie in τρῶμα αυ einfach ω, sondern das α in dem Diphthonge αυ in ω verwandelt und daraus ωυ entstanden sei. Es möchte aber doch eher θῶμα richtig u. θωῦμα wie τρωῦμα nach der irreführenden Analogie von ἑωυτοῦ ἑαυτοῦ daraus verfälscht sein (Lindemanm, dial. Ion. recent. 29 f.), vgl. den dorischen E. N. Θωμάντας (von θαυμαίνω) Inschr. Phleius, D.-I. 3172a (III, p. 190). Bei Hippokr. θαυμάζω, Littré I, 499; doch θωμ. VI, 496 nach θ. η u. ει: ω u. ου: η und ω statt des gewöhnlichen nicht (echt diphthongischen) ει und ου wird von den Lesbiern, ω st. ου auch von den Böotiern gebraucht, wenn Dehnung oder Kontraktion stattfindet, a) η st. ει als: χήρ (G. χέῤῥος) = χείρ, κῆνος (= κεῖνος) Sapph. 2, 1 u. ö. Alc. 86, κῆ (= ἐκεῖ), τρῆς aus τρέες (τρεῖς); Infin. Akt. der V. auf ω, als: φέρην = φέρειν, ἀρκέην, συνέχην auf Inschr., εἴπην Alc. 55, Sapph. 28, ἄγην Sapph. 1, 19, ἐπιδεύσην 2, 15, κρέκην 90, φροντίσδην 41; vgl. § 210, 9; ebenso 2. Pers. Sing. Akt., s. § 209, 2, als: πώνης (πώνεις = πίνεις) Alc. 52, ἔχης S. 99, ναίης Melinn. 3, vgl. Choerob. Dict. 497, 5, Apoll. Synt. p. 92 (wonach Ahrens' [I, p. 91 sq.] Zweifel an der Richtigkeit dieser Form nicht zulässig sind); aber die 3. Pers. Sing. hat (echtdiphthongisches) ει auch im Lesbischen; — im Augmente, als: ἦπον besser ᾖπον = ἔειπον, εἶπον (echtes ει), ἦχες Sapph. 28 (= εἶχες). — b) ω st. ου: Gen. S. II. Dekl., als: ἀνθρώπω (aus ἀνθρώποο); Gen. --129-- v. αἰδώς u. ἱδρώς und derer mit Nom. auf ω, als: αἴδως (aus αἴδοος) st. αἰδοῦς, ἴδρως, Σάπφως v. Σαπφώ; so auch in der Krasis, als: τὦπος aus τὸ ἔπος; δίδων Theokr. 29, 9, vgl. oben φέρην st. φέρειν; ὦν (auch neuion., so b. Herod. ὦν, οὔκων u. οὐκῶν, γῶν, τοιγαρῶν, ὁσονῶν 2, 22, desgl. böot., dor.) st. οὖν; ὤρανος Alc. 17, Sapph. 1, 11 neben ὄρανος (οὐρανός); böot. Μῶσα = Μοῦσα, θέλωσα = θέλουσα Cor. 19, Akk. Pl. auf Inschr. ἐσγόνως, σουγγράφως, Ar. Ach. 879 αἰελούρως, entst. aus ονς; ferner: βωλά st. βουλή, Εὔβωλος. Auch das Arkadische, (Kyprische,) Eleische hat η und ω entsprechend dem ion.-att. ει und ου. In Beziehung auf den dorischen Dialekt ist zu bemerken, dass der strengere Dorismus η u. ω, der mildere dagegen ει und ου hat, wenn Kontraktion oder Ersatzdehnung stattfindet, als: φιλήτω = φιλεέτω φιλείτω, im Augm., als: ἦχον = ἔεχον = εἶχον; die Silbe κλη, entst. aus κλεε, in Eigennamen, als: Κλησθένης = Κλεισθένης, Ἡράκλητος; — ω (entst. aus οο) = ου im Gen. S. II. Dekl., als: ποντίω = ποντίου, τῶ = τοῦ, γλυκυτάτω (auch lokr. ΔΑΜΟ = δάμω, wiewohl das. τούς u. so im übrigen d. mildere Dorismus, vgl. v. Wilamowitz, Zeitschr. f. Gymn.-W. 1877, S. 642), im Gen. S. der Subst. auf ώ, als: Σαπφώ, G. Σαπφῶς (aus όος) = Σαπφοῦς, ἐλάσσως (aus οες = ονες) Arist. Lys. 1260 = ἐλάσσους, von der Konjug. auf όω: μισθῶντι = μισθοῦσι, in Kompositis, als: δαιδῶχορ lak. (aus δαιδόοχος) st. δᾳδοῦχος, ζευγῶχος Hermion Dial.-Inschr. 3385. — Ersatzdehnung: ἧς = εἷς t. Heracl. u. tarentin., καταλυμακωθής t. Heracl. st. καταλυμακωθείς, μής t. Heracl. = ion. u. att. μείς = μήν; Dat. Pl. III. Dekl. auf ωσι, als: διδῶσι aus διδόνσι (= διδοῦσι), μετέχωσιν Kret.; die Endung ωσα (aus ονσα) = ουσα, als: ἄγωσα = ἄγουσα t. Heracl., ἔωσα kret., Μῶσα lak.; Akk. Pl. II. Dekl., als: νόμως = νόμους (aus ονς); ἦμεν, mild. Dor. εἶμεν, aus ἔς-μεν; ἦμεν steht auch auf Inschr. aus dem Gebiete des mild. Dor., so Argos Dial.-Inschr. 3277, Kos Bull. de corr. hell. VI, 254 ff., Kalymna (nb. εἶμεν) das. X, 240 f., Rhodos ἐξήμειν = ἐξεῖναι; entspr. ἠμί f. εἰμί Thera, Röhl, I. Gr. ant. 449 (vgl. 446), Rhodos das. 473. Ferner steht η b. Vb. liqu., als παραγγήλωντι (Aor.) kret., ἔστηλαν desgl., δήληται Praes. Kos Bull. de corr. hell. V, 239, von δήλομαι = milddor. (lokr.) δείλομαι = βούλομαι. Poet. Dehnung ist in Πηρίθοος = Πειρίθοος (f. Περίθ.). Infinitiv ην (Kontraktion aus εεν nach Curtius) seltener, indem der streng. Dor. mehrenteils mit Kürze εν hat (§ 210, 9): ἁνδάνην Alkm., χαίρην Theokr. 14, 1, ἕρπην 15, 26, εὑρῆν 11, 4; dazu entspr. dem μειν der Rhodier u. s. w. kret. μην: η<>μην, δόμην Gortyn. Einzelne Wörter: ἄπηρος = ἤπειρος, χήρ, G. χηρός = χείρ (ἐκεχηρία auch Delph. 1688 v. 48 f., was Ahrens nicht mit Recht Kühners ausführl. griech. Grammatik. I. T. --130-- bezweifelte), σηρά f. σειρά Hdn. II, 579, βωλά = βουλή (auch Argos D.-I. 3277, Nemea 3320), βώλομαι = βούλομαι, κῆνος = κεῖνος, κῶρος = κοῦρος Theokr. Kallim., u. so kret. Inschr. Κώρα, Κωρῆτες; κωραλίσκος kret. nach Phot., auch lakon.; sonst aber mit Kürze: κόρα Aristoph. Lys. 1308 (urspr. κόρϝα); ὠρανός; nur b. Theokrit μῶνος (μοῦνος ion.) und τὸ ὦρος (οὖρος ion.), nach Ahrens II, p. 162 poet. Dehnungen nach Analogie. Ferner steht ω in ὦς, ὤατος st. οὖς, οὔατος, s. § 130. Dagegen für echtdiphthongisches ου hat auch im streng. Dor. ου zu stehen, ebenso für echtes ει ει; damit hat nichts zu thun die von den Grammatikern als dorisch (und äolisch) angegebene Ersetzung des ει durch η in Wörtern und Wortformen, in denen auf η (= ει) ein Vokal folgt, als: ὄρηος = ὄρειος, Λύκηος = Λύκειος, ὀξῆα = ὀξεῖα, πέληα = πέλεια, πασιχάρηα = πασιχάρεια, πλήων = πλείων, μήων = μείων, äol. Κυπρογένηα, Κυθέρηα, Τυρραδήῳ, παχήᾳ u. s. w. (Meister, Dial. I, 92), wo überall ι ursprünglich ist, vgl. ὄρειος aus ὀρέ-ϊος, ὀξεῖα aus ὀξέια, πλείων aus πλε-ίων. Soweit nämlich hier die Überlieferung richtig und nicht vielmehr ῃ zu schreiben ist (ἐπιμεληίας Inschr. Kyme, D.-I. 250, 5, spät), liegt doch ηι zu Grunde, wie auch im arkad. πλῆστος = πλῆιστος, πλεῖστος; wir werden diese Erscheinungen einerseits § 43, 5, andererseits, da sich ῃ von ηϊ schwer trennen lässt, bei der Lehre von der Diäresis § 55, 4 behandeln. Oder aber, wenn auf dor. Inschr. römischer Zeit sich solches η zeigt, ist dies der allgemein damals erfolgte Lautübergang von ει vor Vokal zu ē, so auf einer Inschr. von Byzanz, D.-I. 3059 χρήας, πλήονας, ἀσαμήωτον, ἐπιτάδηον. Was ου betrifft, so gebraucht zwar Theokr. 9, 7 βῶν (= βοῦν), 8, 48 Akk. Pl. βῶς u. sehr oft βωκόλος, βώτας (auch b. Hom. Il. η, 238 βῶν), aber nicht nur Epicharm. 97 βοῦς, sondern auch tab. Heracl. Βουβῆτις; βῶς ist also wie dor. νᾶς st. ναῦς zu fassen, § 128, 3; Schubert, Misc. z. Dial. Alkmans 63 f. Statt χοῦς sollen die Argiver nach Athen. 8, 365, d. χῶς gesagt haben, aber auf d. t. Heracl. 1, 103 steht Akk. Pl. χοῦς, s. Ahrens II, p. 165 sq. Δοῦλος lautet auch böotisch so, und ΔΟΥΛΙΟΝ hat eine altattische Inschrift; gleichwohl wird in Gortyn ΔΟΛΟΣ δῶλος geschrieben, und diese Form steht bei Theokrit, Kallimach., Hesych. (Ahrens II, 163). — ὦν durchweg b. d. älteren Dor., sowie auch b. d. Aeol. und Ion. (s. oben) st. des att. οὖν, das sich auch bei den späteren Doriern findet; auch πώλυπος (att. πουλύπους) kommt ausser bei Epicharm fr. 33 bei dem ionischen Iambendichter Semonides aus Amorgos b. Athen 7, 318 f. (Bergk, fr. 29) vor150), ist also --131-- gleichfalls ein Wort besonderer Art, s. Ahr. II, p. 167 sq., unten § 148, II. Böotische Wandlungen der Diphthonge (vgl. Einl. S. 9): αε u. αι: οε u. οι: Auf alten böotischen Inschriften, insbes. von Tanagra, wird für ΑΙ (gemeingr. αι und ᾳ) ΑΕ, für ΟΙ (gemeingr. οι u. ῳ) ΟΕ geschrieben, welche böotische Schreibung auch Priscian kannte und mit der lateinischen Weise (comoedia, tragoedia f. κωμωιδία, τραγωιδία) verglich (I, § 53). Beispiele: Αἐσχρόνδας (= -ώνδας), Ἀβαεόδορος (d. i. -δωρος), ἐπὶ Ἀμεινοκλείαε (Dat.), Μοέριχος, Πολυαράτοε (Dat.). Auch auf altkorinthischen Denkmälern findet sich dergleichen: ΑΘΑΝΑΕΑ; doch bedeutet in diesem Alphabet E das att. ει, so dass Ἀθαναεία zu transkribieren ist, vgl. auf lat. Inschr. der Übergangszeit zw. altlat. ai u. neuerem ae die Schreibungen wie quaeistor. Anderweitig, so auf att. Vasen, ist ΑΕ, ΟΕ äusserst selten. Terentianus Scaurus VII, 16 Keil: antiqui qu<*>que Graecorum hanc syllabam per ae scripsisse traduntur. Blass, Ausspr.\textsuperscript{3}, 55 f.; Meister, Dial. I, 235, 238. η u. αι (ᾳ): Im jüngeren Böot., doch bereits seit Ende des 5. Jahrh. (und zwar hat dieses η nach den Gramm. wie das gewöhnliche αι am Ende des Wortes in der Flexion keinen Einfluss auf die Betonung, als: τύπτομη = τύπτομαι): I. Dekl. als: ἱππότη (Dat. Sg. u. Nom. Pl.), εὐεργέτης, τῆς, φίλης ἀγκάλης Corinna (die indes selbst noch nicht so geschrieben haben kann) = φίλαις ἀγκάλαις, λιγουροκωτίλης ἐνοπῆς dies.; in d. Adj. auf ηος (= αιος) v. Subst. d. I. Dekl. als: Θειβῆος = Θηβαῖος, den Patronymika, als: Καλλιῆος = Καλλιαῖος; über die Endung εῖος st. ῆος s. ει u. αι; in d. Konjug., als ὀφείλετη = ὀφείλεται, κεκόμιστη, δεδόχθη = δεδόχθαι, ἀπογράφεσθη = ἀπογράφεσθαι; endlich überall sonst, als: ἠ = lesb. u. dor. αἰ (εἰ), κή = καί, χῆρε = χαῖρε, πῆδα Cor. = παῖδα, Ἠολεῖα = Αἰολέα u. s. w., Meister, das. 238 ff. ῑ u. (echtdiphthongisches) ει: Böot. schon seit alter Zeit: ἄρχι = ἄρχει, ἀπέχι; Subst. auf ια = εια, als: ἀσφάλια, Adj. auf ιος = ειος, als: Ἀργῖος = Ἀργεῖος, Patronym., als Φιλοκρατῖος, auf ίδας = είδας, als: Καλλικλίδας, ἠΐ = αἰεί, Θεογίτων, ἰράνα, πλίονα, ἶμι = εἶμι, ἀΐδων Cor. 18, κίμενος u. a. In anderen Dialekten ist dieser Übergang in klassischer Zeit entweder gar nicht oder nur ganz vereinzelt zu konstatieren, während in nachklassischer jedes ει (ausser vor Vokal, s. oben η und ει) mindestens in der Aussprache in ῑ überging. In dem attischen Monatsnamen Ποσιδεών (ion. Ποσιδηϊών Anakr. 6) ist --132-- Verkürzung (vgl. § 27<*> ι u. ει), die auch für das dor. Ποτιδάν, Ποτιδᾶς anzunehmen; das Verhältnis von dor. ϝίκατι (tab. Heracl.) zu ϝείκατι (das.), εἴκατι ist unklar (nach Ahrens, Philol. XXIII, 202 ist der Diphth. hier missbräuchlich; s. auch G. Meyer S. 375\textsuperscript{2}). Über χίλιοι nb. χέλλιοι (lesb.) χείλιοι (böot., ion. Chios) χηλίοι (streng dor.) aus χίσλιοι, χέσλιοι s. § 66, 3. 184, 1. ῡ u. οι: Böot. spät, erst im 3. u. 2. Jahrh., u. auch da ohne Konsequenz (wenn ῡ am Ende des Wortes steht, nach den Gramm. ohne Einwirkung auf die Betonung), als: ϝυκία = οἰκία, καλύ = καλοί, ἐμύ = ἐμοί, Ὅμηρυ = Ὅμηροι, τύ = τοί (οἱ), τύδε = τοίδε (οἵδε), Dat. Pl. τῦς ἄλλυς, ἵππυς; doch auch οι, als ϝοικία, u. so fast immer Βοιωτοί; βριμώμενοι Cor. 18, λευκοπέπλοις 20, doch scheint die Schreibung ῡ auch in Corinnas Gedichte nachmals eingedrungen zu sein, da die Gramm. ihre Kenntnis eben aus Cor. haben werden, u. so Bergk λευκοπέπλυς. ῡ u. ῳ: Böot. Inschr. in gleicher Weise wie ῡ für οι, als: τῦ δάμυ (τῷ δάμῳ), τῦ, αὐτῦ, ὁδῦ, indem der Diphth. ῳ von Alters her fehlte. Die Gramm. lehren πατροῖος, ἡροῖος als böotisch, Meister, Dial. I, 249 f.; Blass, Ausspr.\textsuperscript{3}, 57. Lesbische ι-Diphthonge durch Epenthese und durch Schwund eines ν vor folgendem ς:151) αι und ᾱ (η): Die Epenthese eines ι (j) aus der folgenden Silbe in die vorige, wodurch Diphthong entsteht (§ 21, 7), hat im Lesbischen noch folgende Belege: ἴσταιμι, νίκαιμι (§ 284, 2), αἴμισυς st. ἥμισυς (ἥμ. auch dor., § 25, Anm. 3) Gramm. u. Inschr. Mytil. D.-I. 213; αἰμίονος, Αἰσίοδος (Ἡσίοδος) Gramm., μαῖνις (μᾶνις, μῆνις), παίτρα f. πατρα (dies beides nur Tzetzes). Die letzteren eigentümlichen Erscheinungen sind noch wenig aufgeklärt.152) Ferner gebrauchen die Lesbier αι st. ᾱ, wenn ν vor ς ausgefallen ist, a) Nom. S. III. Dekl., als: τάλαις st. τάλᾶς (τάλανς), μέλαις st. μέλᾶς, παῖς Adesp. Bgk. 59 st. πᾶς (aus πάντς); b) Fem. v. πᾶς παῖσα st. πᾶσα (πάντσα); c) Partic. im Mask. und Fem., als: ἴσταις, ἴσταισα, γέλαις (v. γέλαιμι = γελάω), γέλαισα, κέρναις Alc. 34. 41 st. κιρνάς, so auch dor. Lyr. χαλάξαις Pind. P. 1, 6, ῥίψαις 45, τελέσαις 79, συντανύσαις 81, θρέψαισα 8, 26, s. Hermann, Opusc. I, p. 259; Mommsen, Fleckeis. --133-- Jahrb. 1861, 40 ff.; Peter, dial. Pind. 57 f.; d) Akk. Pl., als: ταὶς δίκαις = τὰς δίκᾶς (aus τὰνς δίκανς), ὄχθαις (= ὄχθας) Alc. 9, κυλίχναις μεγάλαις u. πλέαις 41, νύμφαις, ταίς, τετυγμέναις 85, ἀπάλαις, πλέκταις Sapph. 46, λύγραις Theokr. 28, 20, αὐλεΐαις θύραις 29, 39; e) 3. Pers. Pl., als: φαῖσι st. φασί (aus φάντι, φάνσι) S. 66, δίψαισι v. δίψαιμι Alc. 39, wonach man auch Alk. 34 πεπάγασιν in πεπάγαισιν korrigiert. οι u. ου: Lesb. analog dem αι für ᾱ: a) Nom. Partic., als: ὔψοις v. ὔψωμι = ὑψόω (aus ὕψοντς), ὄρθοις; b) Femin. Partic., als: παθοίσας Alc. 42, πνέοισα (πνεύοισα) 66, πλήθοισα Sapph. 3, λίποισα 84, δοῖσαι 10, ἔχοισα 85; so auch Μοῖσα st. Μοῦσα (f. Μόνσα); Akk. Pl. II. Dekl., als πασσάλοις Alc. 15 (= πασσάλους), aus πασσάλονς), στεφάνοις Sapph. 78, ἀνδρεΐοις πέπλοις, μαλάκοις πόκοις, δόμοις, νόσοις Theokr. 28, 10. 12. 16. 20; d) 3. Pers. Pl., als: κρύπτοισιν Alc. 15, φορέοισι Theokr. 28, 11 st. φορέουσι aus φορέονσι; ἐμμενέοισι, οἰκήσοισι Inschr. Das Fem. der Partic. auf οισα st. ουσα findet sich auch bei dor. Lyrikern, wie καχλάζοισαν Pind. O. 7, 2, θέοισαν 6, 12, ἀΐοισα 26, παπταίνοισα 28, ἔχοισα 30, πταίοισα 7, 26, αἰθοίσας 48; ebenso die 3. Pl. auf οισι (ν), als φιλέοισιν Pind. P. 3, 18, besonders wo das ν parag. erfordert wird, welches an -οντι nicht antreten kann, Peter, dial. Pind. 55 f. Die Endung οισι zeigt sich auch auf dem ion. Chios: πρήξοισιν Bechtel, Inschr. d. ion. Dial. 174 (für πρήξουσιν, Konjunkt. Aor. mit kurzem Modusvokal). ῳ und ω: Lesb. im Konjunktiv, doch nur auf den älteren Inschr. (später ω ohne ι): γινώσκωισι D.-I. 304, Α, 39, γράφωισι 213, 3, aus γινώσκωνσι, γράφωνσι. Ebenso auf dem ion. Chios: λάβωισιν Bechtel a. a. O. Bei εις für ενς, als τίθεις, τίθεισα trifft das Lesbische mit dem Ion.-Att. scheinbar zusammen; doch ist das ει im Lesb. wirklicher Diphthong, im Att.-Ion. nur verlängertes ε. Ausserdem kommen noch folgende Fälle vor: αι u. ᾱ (dor.) od. η: Lesb. θναίσκω, μιμναίσκω, dor. θνσκω, μιμνσκω, besser θνᾴσκω, μιμνᾴσκω, s. Inschr. Mitt. d. arch. Inst. VI, 304 ΘΝΑΙΣΚΩΝ, att. θνῄσκω, μιμνῄσκω (so mit ι zu schr., aus θνη-ίσκω, μιμνη-ίσκω). Ferner lesb. μαχαίτας Alk. 33, μαχατάς dor., μαχητής Hom., unklarer Entstehung. ᾱ u. αι: Att. ἐλα, Ölbaum und Olive, κω, κλω, gew. ἐλαία, καίω, κλαίω. Diesem attischen ᾱ wird ᾳ zu Grunde liegen: κάϝjω, κjω, κᾴω, vgl. § 21, 9, so auch ἐλᾴα aus ἐλαι-ία von ἔλαιον (Cauer, Curt. Stud. VIII, 270), indem αιι zu ᾳ verschmilzt (Wackernagel, Kuhns Zeitschr. XXVII, 278), als Κωπᾴδων (Aristoph.) aus Κωπαίδων, ματᾴζειν aus ματαιίζειν, ὡρᾴζεσθαι aus ὡραιίζ., Φιλᾴδης (Riemann, --134-- Revue de philol. IX, 178) aus Φιλαιίδης von Φίλαιος, σπηλᾴδιον aus σπηλαιίδιον von σπήλαιον. Die Grammatiker lehren κάω, κλάω als attisch (s. d. St. bei Voemel, Dem. Cont. p. 36, Wecklein, Cur. epigraph. 63 ff.); die hdschr. Überlieferung ist sehr schwankend, bietet aber überwiegend αι;153) inschr. Zeugnisse mangeln, während für ἐλάα (im 5. Jahrh. auch noch ΕΛΑΙΑ, Meisterhans, Gr. d. att. Inschr. 25\textsuperscript{2}) solche vorhanden sind. (Bei Homer wird vor ι ἐλινος, ἐλᾶΐνεος geschrieben.) Für den Verlust des ι von ᾳ vor Vokal (vgl. § 39, 2) sind zu vergl. λῶον, σωῶ (att. Inschr.) für λῷον, σῳῶ, Ἀμφιαρᾶον f. Ἀμφιαρά-ιον (Meisterhans das.). Die Sache scheint hiernach keine übergrosse Bedeutung zu haben, indem in Athen sowohl ΚΑΙΩ (d. i. κᾴω) als ΚΑΩ beliebig geschrieben sein wird; in der hellenistischen u. ionischen Form καίω war α kurz und Diphthong αι. Noch stellen die Gramm. mit κάω, κλάω att. ἀετός für αἰετός zusammen; indes bieten hier die Inschr. der klassischen Zeit durchweg αι (Meisterhans das.), und das α in ἀετός (so Delos, Dittenberger Syll. 367, v. 191; ἀέτωμα Athen Afg. 3. Jahrh., Ἀετίων Jasos, Bechtel, Inschr. d. ion. Dial. = ep. Ἠετίων?) kann als kurz genommen werden. Ferner ἀεί für αἰεί (urspr. αἰϝεί), s. Voemel, a. a. O. 28 ff., und zwar behauptet Apollon. adv. p. 600 die Länge des α, die sich aber schwer erweisen lässt. (Vgl. § 27 unter α und αι.) αι u. ει: Lesb., dor., episch αἰ = εἰ, αἴθε = εἴθε, so auch eleisch αἰ u. αἴτε (böot. aus αἰ ἠ, s. o.); dor. κύπαιρος (κύπειρος), κυπαιρίσκω Alkm. 38; ἄναιρον kret. (ὄνειρον), φθαίρω Gramm. (φθείρω, doch auch lokr. Inschr. φθείρω); lesb. κταίνω (κτείνω) Meister, Dial. I, 181; dagegen dor. κλᾷξ (κλείς) hat ᾳ entspr. dem ion. ηι (κληίς); unklar λαία Pind. O. 1.0, 44 (λᾴα?), ion ληίη, att. λεία; vgl. λαῖον, Saatfeld, Theokr. 10, 21. 42, λᾷον Bergk mit Ahrens n. Apollon. adv. 567, ion. λήιον. Endlich Kompos. von γῆ: dor. μεσόγαιον (μεσόγειον), κατώγειον, ἀνώγαιον (Gramm.), ἔγγαιος (Inschr. Thera); att. ει ist auch hier aus ηι hervorgegangen. ει u. αι: Böot. spät in d. Endung αῖος, als Θειβεῖος (= Θηβαῖος), vordem Θειβῆος; das ει vor Vokal wird in hellenistischer Weise ein ē bezeichnen. Meister I, 241. Aber ziemlich ausgedehnt ist der Übergang von αι zu ει im Thessalischen: Verbalendungen Med. --135-- 3. Sg. τει (βελλειτει = βούληται), 3. Plur. νθειν, Infin. σθειν; ferner Εἵμουν = Αἵμων, Ἀνδρείμουν = Ἀνδραίμων. ει u. η: Böot., thessal. ist die Ersetzung des dor.-lesb. η durch ει, welches in älterer Zeit E geschrieben wird (also das geschlossene, allmählich nach i übergehende ē statt des offenen; Mittelstufe zum Itacismus), so böot. εὐγενείς, εὐσεβείς, πατείρ, μάτειπ, εἵρως = ἥρως, πονειρός = πονηρός; τίθειμι = τίθημι, φίλειμι, ἔθεικα = ἔθηκα, ἐπόεισε = ἐπόησε (ἐποίησε), ποειτάς, ἑβδομείκοντα, εἴ = ἤ, ἐπῖδεί = ἐπειδή, μεί = μή, Θειβῆος, Φωκεῖος = Φωκῆος v. Φωκεύς, εἶμεν = dor. ἦμεν (att. εἶναι); das unechte att.-ion. ει lautet demnach auch böot. so (strengdor., lesb. η), so auch χείλιοι, strengdor. χηλίοι (χίλιοι), ὀφείλω, παρμείναντα, Φαεινός, Χειρίας u. s. w., während das echte ει böot. zu ι wird (s. o.). Auch für ῃ (Konjunkt.) steht böot. ει, als ἴει (ἔῃ, ᾖ), δοκίει (δοκέῃ, δοκῇ); vgl. arkad. η, als νέμη, was auch im Böot. vorausliegen wird. — Thessal. ὀνέθεικε = ἀνέθηκε, ἱερομναμονείσας, οἰκοδόμειμα, μειννός = μηνός, Κιεριείων u. s. w.; Inf. Aor. Pass. -θεῖμεν, Konj. -θεῖ. Vgl. unten den analogen thessal. Übergang des ω in ου. Eine merkwürdige Ausnahme bildet in beiden Dialekten der Name Herakles mit seinen Ableitungen, als Ἡρακλίδας böot., Ἡρακλείδας thessal. (doch Ἑρακλίος, Bull. de corr. hell. 1889, 400, Εἱρακλεῖ das. 435 = Dial.-I. 1286), während die von Ἥρα gebildeten Namen den gew. Übergang zeigen, als Εἱρόδοτος; ebenso die von ἥρως: Εἱρωίδας böot., Εἱρουίδας thessal. — Dor. steht ει mit η wechselnd oft für ηι in der 3. Pers. Konj., s. § 213, 2. ει u. υι: Das Femin. Perf. hat bei den meisten Doriern die Endung εῖα statt υῖα, als: ἐρρηγεῖα, ἐπιτετελεκεῖα, ἑστακεῖα, συναγαγοχεῖα auf Inschr. (ion. -οῖα Gramm. Hippokr., s. § 145, Anm. 7). So auch neuattisch γεγονεῖα (vom 3. Jahrh. v. Chr. ab); man kann neuatt. ει für οι in δυεῖν, οἴκει = οἴκοι (Menander) vergleichen (Herodian I, 504. II, 463); dazu τοῖς λοιπεῖς, C. I. Att. II, 467, 12 f. (100 v. Chr.), Blass, Ausspr. 56\textsuperscript{3}f. οι u. αι: Arkad. -τοι für -ται im Medium 3. Pers., als βόλητοι = βούληται. οι u. ει: ὄνοιρος lesb. st. ὄνειρος; arkad. Ποσοιδάν, lakon. Ποοἱδάν; auch böot. Ποτοιδάϊχος, Prellwitz, Bzz. Btr. IX, 329, Dial.-I. 474, 12. ου u. ευ: Kret. ψούδια = ψεύδη Phot., so auch auf einigen kret. Inschr. βωλουομέναις, ἐξοδούσαντες, ἐπιτάδουμα, ἐλούθερον (Bull. de corr. hell. IX, 11). Der erste Laut hat sich dem zweiten angeglichen, gleichwie im Lat. altes eu durchgängig zu ou (ū u. s. w.) geworden ist. ου u. ω: Thessalisch, dem Übergange von η in ει (s. o.) entsprechend, als: Σουσίπατρος, Κραννουνίουν, γνούμα, ὀνάλουμα; Dat. II. Dekl. (mit Verlust des ι), als τοῦ κοινοῦ, ἱαροῦ st. τῶ u. s. w., G. Plur. κοινάουν --136-- ποθόδουν, τοῦν, πολιτάουν, s. Meister, Dial. I, 297 f. In Pharsalos indes findet sich auf etwas älteren Inschr. noch Ω, als Ἀφθονε (ί) τω, D.-I. 328, vgl. Bull. de corr. hell. 1889, p. 403. ῡ u. υι: Allgemein vor Konsonanten (§ 43, 2), indem υι nur vor Vokal (bei Homer u. s. w. auch am Ende) vorkommt (ausser dor. υἷς für υἷ, οἷ, lesb. τυῖδε): ἰχθδιον st. ἰχθυ-ίδιον, ἐκδῦμεν Hom. st. ἐκδυῖμεν (§ 214, 1; § 281, Anm. 3). In Athen ist aber schon im 4. Jahrh. auch das υι vor Vokal regelmässig ῡ geworden, als ός, κατεαγῦα, s. Cauer, Curt. Stud. VIII, 275. Riemann, Rev. de philol. I, 35. Meisterhans 46\textsuperscript{2} ff. (der irrig das υ als kurz ansieht, während kein att. Dichter ὑός mit kurzer 1. Silbe gebraucht). Allen, Arch. Inst. of America IV, 71 f. ῡ u. ω: Lesb. (vgl. lesb. υ für ο § 24) n. d. Gramm. in χελύνα χελώνη, τέκτυν τέκτων, s. indes Meister, Dial. I, 75 f. (τέκτονες Sapph. 91; zu τέκτυνες wäre τέκτῦν analog). Doch zeigt sich dieser Übergang in ἀμύμων Hom. nb. ἀμώμητος (μῦμαρ, ψόγος Hesych.); Κύμη d. i. κώμη.
\section{Kurze Vokale und lange Vokale oder Diphthonge}

α u. αι: Die ι-Diphthonge neigten vor Vokal zur Abwerfung des ι, welches leicht halbkonsonantisch wurde und dann ausfiel; daher die prosodischen Verkürzungen wie τοιαῦτα, § 75, 13. Es gehört dahin auch ω, ᾶ st. ῳ, ᾳ vor Vokal, § 26 unter ᾱ u. αι. Der asiatische Aeolismus nun gebrauchte oft α st. αι, als: Ἄλκαος, ἄκμαος, ἄρχαος, Θήβαος, πάλαος, βεβαώτερος, Ἀθαναα Alc. 9, Theokr. 28, 1 n. Emend., Φωκάας Sapph. 44, μάομαι 25, ἄϊ st. αἰεί (s. Ahrens I, p. 100, Meister I, 89 ff.), vgl. unten ο u. οι, ε u. ει; doch kommt auf Inschr. wie bei den Dichtern ebenso auch αι vor. — Böot. selten (Πλαταεῖος = Πλαταιέως, Ἀϊκλίδας); thessal. δικαοῖ st. δικαιοῖ, Γεννάος (s. Ahrens, Add. II, p. 533, Meister I, 299). Ionisch Ἀθηνάης Delos, Bechtel nr. 54, öfter Euböa, s. Fritsch, Vok. d. Herod. Dial. 37 ff.; attisch Πειραεύς u. dgl., Ἀθηναα u. daraus Ἀθηνᾶ; die ursprüngliche Form von ἀεί ist αἰϝεί, wie sie sich in einer krisäisch. Inschr. erhalten hat (s. Ahrens II, p. 378); vgl. sk. êva-s, gehend, beweglich, 1. aevum, goth. aiv-s, Zeit (s. Curt. Et., p. 385\textsuperscript{5}); daraus entstand die Form αἰεί (vgl. αἰών), die sich im ionischen Dialekte (bei Herodot fast durchweg, sehr selten ἀεί, aber d. Komp. ἀείναος 1, 93. 145, wie auch Hom. Od. ν, 109 sogar ἆενάοντα steht, mit v. 1. αἰεν. bei Eustath., u. Hes. Op. 295 ἆενάου, Harder de α voc. 62 ff.) und in der Dichtersprache, zuweilen auch in der attischen Prosa154) neben ἀεί erhalten hat; auch --137-- die att. Inschriften kennen αἰεί nb. ἀεί, Meisterhans 25\textsuperscript{2}. Man wird darnach bei den Attikern αἰεί schreiben, sowie der Vers die Länge fordert (wie in der That der Med. des Aesch. u. Soph. in der Regel bietet), auch gegen Apollonios, der (adv. 600) den Attikern ἆεί beilegt, vgl. § 26 unter α<> u. αι). — Singulär ist tarent. ἄνεγμα f. αἴνιγμα. — Umgekehrt hat sich αι für α eingeschlichen in παλαιστή, wofür die att. Inschr. stets παλαστή haben (Meisterhans 14\textsuperscript{2}),155) Γεραιστός u. Γεραστός (Riemann, Bull. de corr. hell. III, 497), also vor στ, vgl. Τροιζήν st. Τροζήν, οι st. ο vor ζ = σδ. ο u. οι: Dor. v. πο<>ω auf Inschr. ἐπόησε, ἐποησάταν, πεπόηνται s. Ahrens II, p. 188; so auch lesb. ἐπόησε, ποήσασθαι u. s. w., πόης Theokr. 29, 21, ἐπόησε das. 24; att. Inschr. ποεῖ, ποητής u. s. w., doch nicht vor folgendem O-Laut, Meisterhans 44\textsuperscript{2}; auch in Hdschr. bewahrt, so sehr oft Demosth. or XLIV in cod. S, s. Blass z. das. § 20, doch ebenfalls nur vor η ει; vgl. lat. poeta, poema; die ion. Wörter ποίη, ῥοιή, στοιή, χροιή lauten att. πόα, ῥοά, στοά, χρόα (doch auch ποία, χροιά Aristoph., Eur., στοιά Aristoph. Eccl. 684 u. 686); aber für att. χλόη (so mit η, also ohne Anzeichen eines ι vor der Endung) steht auch Hdt. 4.34 χλόη, wiewohl Fritsch S. 46 χλοίη verlangt; ders. hält 1, 74 das ὁμοχροιίη der Hdschr., als von χροιή mit ίη abgeleitet (Stein ὁμοχροίη); att. auch ὄα (Aristoph. frg. 228 K.) οἴα ᾤα Schafpelz, Ὄα (Ὤα) att. Demos, vgl. οἴα = κώμη Herodian Ι, 302, Οἰιᾶται d. i. Ὀϊᾶται (a. Rhodos) C. I. A. I, 226, 7, b; τρίττοα u. τρίττοια Inschr.; dor. πνοιά u. πνοά Pind., στοιά u. στοά Inschr. (στωΐα lesb.); so auch öfters lesb.: πόας Sapph. 54 (aber ποίας 2, 14), ἐπτόασεν Sapph. 2, 6 (ἐπτοήθη Anakr. 51), εὐνόας u. εὔνοαν Inschr., λαχόην st. λαχοίην Et. M. 558, 30 (s. Ahrens I, p. 101, Meister, Dial. I, 89 ff.). Bei den Derivatis der Eigennamen auf οια, als: Εὔβοια, wird das ι in unseren Texten in der Regel ausgelassen, als: Εὐβοεύς, Εὐβοΐς (Εὐβοῖδα S. Tr. 74, Eur. Heracl. 83, El. 442, aber Εὐβοιίς im Nomin. S. Trach. 237, 401), Εὐβοϊκός (Eur. Hel. 767); so auch Hdt. Hdschr. Εὐβοεύς u. s. w., Fritsch 45 f. Umgekehrt οι für ο in dem späteren Τροιζήν für Τροζήν, Blass, Ausspr. 53\textsuperscript{3}, vgl. oben unter α f. αι; sodann vor η att. vom 4. Jahrh. ab, als βοιηθεῖν, ὀγδοίη, das. 52\textsuperscript{3}, Meisterhans 45\textsuperscript{2}. ε u. ει: Die Abschwächung des (echtdiphthongischen) ει in ε vor einem Vokale ist ebenfalls überall verbreitet. Neuion. die Adj. auf εος, εη, εον st. ειος, εια, ειον, als: βόεος (auch Hom. neben βόειος u. Pind. P. 4, 234 βοέους), αἴγεος, οἴεος, χήνεος (aber nach Stein ἡμιόνειος, --138-- μήλειος; noch weiter geht im Eintreten für ει Fritsch S. 43 ff.); ferner ἐπιτήδεος, τέλεος (so auch Aesch. Suppl. 515 u. ö., Plat. häufiger als τέλειος; b. Her. 9, 110 τέλειον, sonst -εος), (ὑπώρεος), daher ἡ ὑπωρέη od. ὑπώρεα (oft mit ει überl.); dagegen ἐπέτειος annuus mit ει nach Stein (Praef. LXII), Ἡράκλεος (Ἡρακλέοισιν Inschr. Teos), Ὑπερβόρεος (Ὑπερβόρειοι Hellanikos), Ἀριμάσπεος; (ὀθνέην ὁδόν ein Dichter b. Hdn. II, 558 v. ὀθνεῖος); dann πλέος, πλέη, πλέον st. πλεῖος Hom.; Kompar. v. πολύς b. Herod. πλέων, πλέον od. πλεῦν, G. πλεῦνος, πλέονι, πλέονα, πλεῦνα, πλέω, πλεῦνες u. πλέους, πλέοσι, πλεόνων u. πλεύνων, πλεῦνας, πλεόνως156) (aber 1, 192 πλεῖον, 1, 167. 2, 120. 121, 4 πλείους in allen Codd.); Fem. auf εα st. εια s. § 126 v. Adj. auf υς, als: θῆλυς, θήλεα, θῆλυ, θηλέης, θηλέῃ, θήλεαν, θήλεαι, ἡμίσεα (v. ἥμισυς), ἡμίσεαι, ἡμισέας, τρηχέα (v. τρηχύς), βαθέα, εὐρέα, ἰθέα, βραχέα, βαρέα, δασέα (auch Inschr. Milet), ταχέα, ὀξέα, πλατέα (daher auch die Insel Πλατέα); die Iambographen indes -εῖα, auch Demokrit ἰθείῃ (Renner, Curt. Stud. I, 175); ἔωθα Hdt. Hippokr. (II, 284. VI, 160); die Inschr. bieten auch ποιήσεαν (Teos); νικηθέη (Zankle Röhl 518); ferner vor e. Konson.: alle Formen des Verbs δείκνυμι ([root ] δικ, sk. di[cnull ]-âmi, zeige, l. indĭc-o) nebst seinen Kompositis ausser Praes. u. Impf., also: δέξω, δέξομαι, ἔδεξα, ἐδεξάμην, δέδεγμαι, ἐδέχθην (Herod. 2, 30. 4, 79. 6, 61. 9, 82 δείξαντα, δείξω, δεῖξαι, ἐπιδεῖξαι, δεῖξαι ändert Bredov. p. 153 und ebenso Stein in δέξαντα u. s. w.); auch im Präs. ἀποδεκνύντες Inschr. Chios; aber Hippokr. ἀπόδειξις u. s. w., Littré Ι, 499; κύπερος st. κύπειρος; vgl. αἴγερος für αἴγειρος Hdn. II, 411 mit einem Senar als Beleg; ἔρεγμα att., ἔριγμα ion. v. ἐρείκω, [root ] ἐρικ; ἔργω (auch b. Hom. neben ἐέργω [u. Il. 23, 72 überl. εἴργω]), dränge, v. ϝέργω ἐέργω, aus welchem letzteren εἴργω entstanden scheint, s. § 343. — Lesbisch: ἀλάθεα st. ἀλήθεια Theokr. 29, 1 (εια Hdschr.), πλέαις st. πλείαις Alc. 41; — dorisch bei Sophron fr. 39 ὤψεον st. ὤψειον v. ὀψείω, ἀσάλεα st. ἀσάλεια (Et. M. 151, 47), γενεᾶτις (von γένειον) fr. 55, ἁδέαι Epich. 34, Theokr. 3, 30 ἁδέα, 7, 78 εὐρέα; ἡμίσεα öfter b. Archimed., sodann auf dor. Inschr. ἀτέλεα st. ἀτέλεια, ὑγιέᾳ, ἱαρέαι st. ἱέρειαι, Ἡράκλεα, Name der Stadt, st. Ἡράκλεια, Νικοκράτεα, Εὐκράτεα, Νικόκλεα157) u. a.; πλέων, doch nicht gleichmässig, s. Ahrens II, p. 188; vor e. Kons.: ἀποδεξάντω für -δειξ. Inschr. Kos, nach Bechtel, Gtg. Nachr. 1890, 31 ein importierter Ionismus; μέζων Epich. 32, κρέσσων Pind., Theokr. (beides auch ion., doch ist μείζων, κρείσσων das weniger Regelrechte, vgl. § 21; übrigens dies ει wohl gedehntes ε). Im Attischen findet sich ausser τέλεος πλέον (§ 156, 3) --139-- vereinzelt ἡμίσεα, θρασέα u. dergl. b. Autoren (§ 125 Anm. 12); ferner ist ὄστρειον ursprünglicher (Athen. 3, 44) als ὄστρεον (bei Platon beides, Schanz, Prolegom. Phaedr. p. VI), ἐπιθειάζω (von θεῖος) als ἐπιθεάζω (letzteres Aesch., Eur., Pherekr., bei Plato schwankt die Lesart, Schanz das. VII, Rutherford, Phryn. 275, der auch περιθεοῦν von θεῖον Schwefel aus Menander anführt; gerade die Kompos. und Ableitungen neigen zur Verkürzung, woher auch Ποσιδεών aus Ποσιδειών Ποσιδηιών [letzteres altatt.], Αἰνεᾶται von Αἴνεια u. dgl., Hdn. II, 278, Meisterhans 42\textsuperscript{2}, Ἀρεοπαγίτης von Ἄρειος πάγος); δωρειά (att. Inschr. in klassischer Zeit überwiegend, Meisterhans 31\textsuperscript{2} ff.) älter als δωρεά, welches in unseren Texten ausschliesslich erscheint (Herodian kennt beide Formen, I, 285. II, 601). Vgl. v. Bamberg, Ztschr. f. G.-W. 1874, S. 620. Riemann, Rev. de phil. IX, 52 u. A. (bei den Tragg. ist überall δωρειά zulässig, nicht mehr in der neueren Komödie). Auf den att. Inschr. aber begegnen seit ältester Zeit zahlreiche Schreibungen wie ἐπιμελέας, ἱέρεα, πολιτέα, γραμματέον, Θησέον (ἐν τῷ Θησέῳ d. Kom. Pherekrates, Nauck, Mél. III, 116, Kock fr. 49) u. s. w., Meisterhans 31\textsuperscript{2} ff. Umgekehrt aber wird nam. im 4. Jahrh. v. Chr. nicht minder häufig st. ε vor Vok. ει geschrieben, als εἰάν, εἱαυτοῦ, ἱερείως u. a. m., das. 35\textsuperscript{2} ff., Blass, Ausspr. 33\textsuperscript{3} ff., jedenfalls ohne Änderung der Quantität u. mit der Aussprache des ι als eines schwachen j; auch ausserhalb Athens findet sich diese Schreibung (vgl. βοιηθέω u. dergl., oben unter ο u. οι). ε u. ευ: Analog vor Vokal, asiat.-äol. u. dor., aber spät: ἐπισκεάσαντα Kyme, Dial.-I. 311, ἐπισκεάζειν, σκεοθήκα Korkyra das. 3195; auch in der κοινή; vgl. G. Meyer 137\textsuperscript{2} ff., Blass, Ausspr. 78\textsuperscript{3}. α u. αυ: ἀτοῦ ἑατοῦ vom 1. Jahrh. v. Chr. ab nicht selten, Blass, Ausspr. 77\textsuperscript{3} f.; ἄλοξ — αὖλαξ — ὦλαξ s. § 26 ω und αυ. ει (unechtes, d. i. gedehntes ε) u. ε: Alt- u. neuion. ξεῖνος (entst. aus ξένϝος, ξέννος; auch b. d. Tragikern, s. Wunder, Exc. ad Soph. O. C. 925, Gerth, C. Stud. I, 2, 239)158) mit seinen Derivatis, doch ξένιον ξενίη achtmal in der Odyssee; στεινός, κεινός (aber Od. 22, 249 κενά, Herod. 4, 123 κεκενωμένον) wohl ebenso zu erklären (Nbf. κενεός d. i. κενεϝός, vgl. kypr. κενευϝός; über d. att. Komp. στενότερος s. d. Lehre v. d. Kompar., § 154 Anm. 2); εἴνατος (aber ἔνατος Il. β, 313), εἰνακόσιοι, εἰνάκις st. ἔνατος (ἔνϝατος) u. s. w.; εἵνεκα Hom. (doch ἕνεκα Il. α, 110, ἕνεκʼ α, 94, ἕνεκεν Od. 17, --140-- 288. 310), εἵνεκεν Herod.; εἴριον (aber ἐρίοιο Od. δ, 124), εἰρίνεος Herodot; besonderer Art ist ἤνεικα, ἐνεῖκαι u. s. w. § 343 st. ἤνεγκα (att. auch ἤνειγκα geschr., in anderen Dial. mit ι, als dor. ἀνηνίκαμες, ἀνήνικε, Mylonas, Bull. de corr. hell. X, 143 f., Baunack, Inschr. v. Gortyn 56 ff.); — altion. εἰν f. ἐν (ί), auch S. Ant. 1241 εἰν Ἅιδου δόμοις, εἰνάλιος Pind., Christ, Philol. XXV, 619; ferner Hom. ὑπείρ f. ὑπερ (ί); πειραίνω (Pind., Soph. Tr. 781), πεῖραρ πεῖρας (Pind.); δειρή δειράς (letzteres auch att., W. Schulze, Qu. Hom. 23; kret. δηράς); — neuion. δείρω st. δέρω, als Präsensverstärkung, vgl. φθείρω, doch auch Arist. z. B. Av. 364 δεῖρε, ebenso Kratin. fr. 361, Kock (δαῖρε Lobeck); — im Anlaut bei folgenden alt- und neuion. Verben: εἰλίσσω (auch b. d. Tragikern nach Bedarf des Metrums, Gerth, C. Stud. I, 2, 243; auch att. Inschr. u. Plat., so Polit. 270, d. 286, b), neuion. (§ 343), vgl. volvo (ἑλιγμοί Herodot 2, 148); εἰλύω (Hom., doch ἐλύσθην), εἵνυμι, att. ἕννυμι, ει<>ρωτάω, εἰρύω (Hom. auch ἐρύω) aus ἐϝρ., § 343, auch Soph. Tr. 1034 εἴρυσον; — böot. vor ς mit Konson. in Θεισπιεύς u. s. w. st. Θεσπ., Θιόφειστος st. Θεόφεστος Θεόθεστος; auch vulgär εἴσχηκα εἴσχημαι, Blass, Ausspr. 33\textsuperscript{3}; G. Meyer 123\textsuperscript{2} f. ι u. ει: Alt- u. neuion. ἴκελος (ι) st. εἴκελος (beide Formen bei Hom.) d. i. ϝίκελος; doch προσεικέλην Herod. 2, 12; Ποσιδήιον Hom. u. Herod., Ποσείδιον und Ποσίδειον att. Inschr. (Meisterhans 42\textsuperscript{2}), Ποσιδηϊών Monatsn. Anakreon 6, att. Ποσιδεών, dor. Ποτιδάν Ποτιδᾶς, vgl. § 26 unter ῑ u. ει, Ahrens, Philol. XXVIII, 193 ff. Dagegen heisst es Ποτείδαια, Ποσειδανία. ου (unechtes, d. i. gedehntes ο) u. ο: Alt- u. neuion. in einzelnen Wörtern vor Liquidis und ς: νοῦσος (auch Pind.) st. νόσος (b. Hom. Beides), aber immer νοσέειν νόσημα (so in π. ἱερῆς νούσου cod. θ stets; s. auch die v. l. bei Littré II, 224 ff. u. s. w., Lindemann, dial. Ion. rec. 7 f.); μοῦνος (auch Pind. u. zuweilen b. d. Tragik., s. Wunder, Exc. ad Soph. O. C. 925, Gerth, C. Stud. I, 2, 238)159) st. μόνος (b. Hom. Beides; οὐ μόνον codd. Herod. 7, 9), u. dessen Derivata, als: μούναρχος u. s. w.; οὐλόμενος, verderblich, ep. u. poet.; πουλύς altu. neuion. st. πολύς, s. § 146; Οὔλυμπος Οὐλυμπία auch Pind., b. Hom. auch Ὄλυμπος, so regelm. Herod. 1, 43. 46; 7, 128 u. s. w., wie auch Ὀλυμπίη, τὰ Ὀλύμπια, Ὀλυμπιάς, Ὀλυμπιονίκης, Ὀλυμπιόδωρος; οὔνομα und ὄνομα Homer (vgl. § 38, 5), bei Herodot ist grosses Schwanken d. Hdschr. (G. Meyer 94\textsuperscript{2}; auch Fritsch Vok. d. herod. Dial. p. 8 leugnet οὔνομα und erkennt nur τοὔνομα an; so ist auch Hippokr. II, 190. VI, 392 L. οὔν. aus τοὔνομα verdorben); κοὐνομα- --141-- κλυτον Semon. 7, 87 beweist nichts (κὠνομ. leicht herzustellen); für οὐνομάζειν auch Stein wie b. Hom. ὀνομάζ., ebenso ὀνομαστί 5, 1. 6, 79. Ὀνομαστός 6, 127, Ὀνομάκριτος 7, 6; κοῦρος (auch Pind. κούρα), adolescens, st. κόρος urspr. κόρϝος, κουρίδιος, Διόσκουροι, dies zuweilen auch att., wie Thuc. 3.75. 4, 110 Διοσκούρων, Διοσκούριον (Herod. 4, 33 u. 34 κόρας, κόραι, mit ο auch Hippokr. IX, 44. 48); ὁ οὖρος (Inschr. Chios u. dor. Thera) st. ὅρος, altdor. noch ὅρϝος, dazu πρόσουρος, ὅμουρος, ὁμουρέειν, οὐρίζειν (auch Tragg. in Komp. stets ου, als πρόσουρος, ἄπουρος, Eichler, de form. epicarum in trag. Aesch. atque Soph. usu, p. 35); τὸ οὖρος st. ὄρος (b. Hom. u. Herod. Beides, obwohl Stein für Her. οὖρος verwirft; οὔρεσιν Semon. 14; auch b. Pind.; ὀρέων Anakr. 2, ὄρεα ὄρεσι Hippokr. II, 58. 70. 72 u. s. w.), οὔρειος b. Hom. u. den Trag. (ὀρείας Hippon. 35); οὖλος altion. b. Hom. statt ὅλος (urspr. ὅλϝος), vgl. (altlat. sollus, ganz?) sk. sarvas, omnis (Curt. Et., p. 551\textsuperscript{5}); οὐλαί Gerstenkörner b. Opfer, att. ὀλαί; κουλεόν st. κολεόν (b. Hom. Beides, κολεόν Hekataeus b. Hdn. I, 61); die Kasus von γόνυ u. δόρυ: γούνατος, Hom. auch γουνός u. s. w.; δούρατος (Hom. auch δουρός) u. s. w., s. § 130 (b. Herod. auch δόρατα, δόρασι; δοριαλώτου 8, 74. 9, 4); auch Pind. gebraucht ἐπιγουνίδιος, δούρατος, δουρί; über d. Trag. s. Gerth, C. Stud. I, 2, 242; einzeln vor δ ὁ οὐδός, Schwelle, st. ὀδός, aber ἡ ὁδός, Weg (nur Od. ρ, 196 ἡ οὐδός); vor Vokal τὰ οὖα Hippokr. II, 500 L. = att. ὄα (Arlesbeeren). ο u. ου: Die Verlängerung des ο zu ου unterbleibt in den Dial. zuweilen auch da, wo die gew. Sprache sie hat: lesb. nb. ὤρανος ὄρανος Sapph. 64. Alk. 34 (so auch κόρα S. 62. 65. Alk. 14); βόλομαι arkad. nb. βωλά f. βουλή. Ausstossung st. Kontraktion (vor zwei Konsonanten) zeigt das dor.-arkad. (eleische) δαμιοργός, dor. auch δαμιεργός, nb. dor. (phok.) δαμιουργός, ion.-att. δημιουργός, doch samisch δημιοργός, aus δημιο (ϝ) εργός Hom., s. Meister, Dial. II, 41 f.; vgl. die Lehre v. d. Kontraktion § 50, 4. 6. — Ferner: Συρηκόσιος ion., Συρακόσιος dor. u. att. st. Συρακούσιος (dor. auch Συρακόσαι Συρακόσσαι, Pind., s. § 50, 4). ο u. ω: Neuion. in ζόη (paroxyt.) st. ζωή (lesb. mit ι ζοΐα Theokr. 29, 5); ζοός n. Emend. Archil. 63; dor. Epich. fr. 158 ζοός, Theokr. ep. 17 (18), 9 ζοάν; in einigen Wörtern der II. att. Dekl.: ion.-dor. λαγός st. att. λαγῶς, Hom. λαγωός, alt- u. neuion. κάλος st. κάλως, Κέον Herod. 8, 76, att. Κέων, πλέος Hdt., ἔκπλεον dor., att. πλέως, Hom. πλεῖος, s. § 109, Anm., 111, 5. ω u. ο: Böot., ep. u. b. Pind. Διώνυσος st. Διόνυσος, wofür lesb. Ζόννυσος, thessal. Διόννυσος, ion. Διένυσος (Amorgos) oder Δεύνυσος (Anakr. 2, 11) aus Δεόν., vgl. G. Meyer 284\textsuperscript{2}. (Auch Pind. Διόνυσος I. 7, 5.) — Über dor. κῶρος u. s. w. s. § 26 ω u. ου. --142-- — Ein bes. Fall ὀτίς — ὠτίς (Vogelart), Hippokr. VI, 356 und dazu Littré. Anmerk. Über die Verkürzung und Dehnung, Weglassung und Hinzufügung, Kontraktion und Krasis der Vokale, über die Diäresis der Diphthonge in den Dialekten s. im übrigen d. Wohllautslehre.
\section*{B. Konsonanten}
\addcontentsline{toc}{section}{B. Konsonanten}

Das verschiedene Verhältnis der Konsonanten zu einander in den verschiedenen Mundarten beruht ziemlich überall auf dem Gesetze, dass gleichstufige Konsonanten (§ 7) mit gleichstufigen und gleichnamige (§ 7) mit gleichnamigen wechseln.
I. Wechsel der gleichstufigen Konsonanten unter einander. A. Mutae.

\section*{I. Wechsel der gleichstufigen Konsonanten unter einander}
\addcontentsline{toc}{section}{I. Wechsel der gleichstufigen Konsonanten unter einander}
\section{A. Mutae}

a) Tenues. κ u. π: Die Pronomina interrogativa und indefinita lauten neuion. (d. i. bei den asiat. Ioniern, dagegen nicht auf Euböa)160) κοῦ, ὅκου, κόθεν, ὁκόθεν, κῶς, ὅκως, κώ, οὔκω, κῄ, κότε, κοτέ, ὁκότε, οὐδέκοτε, κοῖος, κοίη, κοῖον, ὁκοῖος, κόσος, ὁκόσος, κότερος, ὁκὁτερος, aus κϝο- vgl. § 16, 3, c (vgl. sk. kas, wer, kutas, woher, kadâ, wann, kataras, wer von zweien, katamas, wer von vielen, lat. (ali)cubi, (ali)cunde, quis, (ali)quis u. s. w., λύκος u. lupus, ἴκκος tarent. in Etym. M. p. 474, 12 u. ἵππος aus ἴκϝος, sk. a[cnull ]-vas, l. equus, durch Angleichung, s. § 64, 4); in allen übrigen Mundarten steht π, also: ποῦ, ὅπου u. s. w.; thessal. aber κίς für τίς, πόκκι f. ὅτι, s. unter κ und τ; ferner thessal. Κιέριον Κιάριον = Πιέριον, Κύδνα urspr. für Πύδνα nach Steph. Byz. v. K., bei den Oetäern (Strab. 13, p. 613) κόρνοψ = πάρνοψ (πόρνοψ lesb.-böot.); böot. ὄκταλλος = ὀφθαλμός, dor. ὄπτιλλος; in der gew. Spr. ist ἀρτοκόπος Korruption aus ἀρτοπόπος (vgl. πεπτός, πόπανον, Phrynich. Rutherf. 303, Cobet, Misc. 148. π u. κ: πύανος b. Pollux 6, 61, woher att. Πυανοψιών, aber auf Samos Κυανοψιών, G. Meyer 191\textsuperscript{2}, Kirchhoff, Berl. Ak. Ber. 1859, 751, lak. πούανος b. Hesych. (πυάνιος Alkm. 75) = κύαμος? — eleisch ὀπτώ f. ὀκτώ, wohl nach ἑπτά, Meister, Dial. II, 56. --143-- κ u. τ: Dor. die Adverbien der Zeit auf οκα: πόκα, ποκά, οὔποκα, οὐπώποκα, ὅκα, τόκα, ὁπόκα, ἄλλοκα = πότε, ποτέ, ὅτε, τότε, ὁπότε, ἄλλοτε; ὅκκα = ὅτε κα, ὅταν. Thessal. κίς = τίς, πόκκι = ὅτι, s. § 175 Anm. 2. τ u. κ: Als dorisch wird von Schol. Theokr. 1, 1 τῆνος = κεῖνος, ἐκεῖνος angeführt, aber Ahrens II., p. 270 leitet es von dem Demonstrativstamme το (vgl. τοσσῆνος v. τόσος) ab mit der Bedeutung iste; τ st. κ ist nicht dor. S. § 173, 3. π u. τ:161) Böot. in πέτταρα = τέσσαρα, πετταράκοντα = τεσσαράκοντα, πετράς = τετράς, πέτρατος = τέτρατος τέταρτος; beides aus κϝ vgl. quattuor; lesb. πέμπε (G. πέμπων Alc. 33) = πέντε (auch Od. δ, 412 πεμπάσσεται), vgl. quinque (aber böot. πέντε πεντακάτιοι); πές (ς) υρες = τέσσαρες, πήλυι = τηλόσε (πῆλε auch im Böot. Πειλεστροτίδας, doch auch Τειλεφάνειος böot.), σπέλλω = στέλλω, Fut. κασπολέω Sapph. 80, σπόλα = στολή; doch ἀπέστελλαν, ἀποστέλλαντα, ἀποστσλέντα d. Inschr.; thessal. πέμπε, πεισάτου = τεισάτω v. τίνω, so auch kypr. Fut. πείσει (vgl. ποινή); böot. ποταποπισάτω D.-I. 488, 85 nach Baunack, Philol. 1889, 411; dor. σπάδιον (vgl. l. spatium) = στάδιον. b) Mediae. γ u. β: Böot. u. dor. γλάχων, γλαχώ Ar. Ach. 861, 874, Theokr. 5, 56, alt- u. neuion. γληχών, γληχώ; att. βληχών, βληχώ; dor. γλέφαρον Pind. = βλέφαρον att.; bei Alkm. 23, col. III, 7 wie es scheint ποτιγλέποι = προσβλέποι (doch epidaur. Inschr. βλέφαρον βλέπω). Ferner πρισγεῖες böot. = πρεσβῆες, πρέσβεις; kret. πρεῖγυς, πρεισγευτάς, πρειγευτάς, πρισγευτάς, πρεγγευτάς, b. Herodian σπέργυς; dazu kret. πρείγιστος = πρέσβιστος, Komp. πρείγονα. Vgl. § 11. β u. γ: Böot. βανά, G. βανηκός Cor. 21 = γυνή, γυναικός st. γϝανά (vgl. Goth. qvinô); aus γϝ hat sich β gebildet, während in der gew. Form aus ϝα υ entstand. Als dor. wird γανά von Gramm. angeführt. S. Ahrens I, p. 172. Curt., Et.\textsuperscript{5}, S. 175 u. 479 und oben § 11, 1. Thessal. Βύλιππος, Βυλιάδας vgl. Γύλιππος, Γύλων, Γῦλις. β u. δ:162) Böot. u. lesb. βελφίς (βέλφις), Βελφοί = δελφίς, Δελφοί vgl. l. bis aus duis, bellum aus duellum, doch scheint bei den griech. Wörtern wiederum γϝ zu Grunde zu liegen. Vgl. Curtius Et., S. 479\textsuperscript{5}. Ferner lesb. βλῆρ nb. δέλεαρ; nach Ahr. I, p. 41 f., --144-- Curt. Et.\textsuperscript{5}, S. 237. 483 steht βλῆρ f. δλῆρ, da δλ sich nicht sprechen liess, vgl. Fick, Bzzb. Btr. 6, 211; lesb. u. anderweitig σάμβαλα Sapph. 98. Eumel. b. Pausan. 4. 33, 3. Hipponax 18. Anakr. 14 (ποικιλοσάμβαλος); nach Schwalbe de Deminutivis p. 83 stammt das Wort σάνδαλον aus dem Persischen sandal (calceus); thessal. Βωδών od. Βωδώνη = Δωδώνη, womit der thessal. Eigenn. Βούδουν zusammenzuhängen scheint: Βωδωναῖε war Il. π, 223 v. l. f. Δωδωναῖε. Vgl. § 11, 3. δ u. β: Dor. ὀδελός Epich. 58, Ar. Ach. 762 = ὀβελός (Bratspiess); dazu auf e. delph. Inschr. sowie in Gortyn u. arkadisch ὀδελός = ὀβολός; aber auf jüngeren dor. Inschr. ἡμιώβελον, ὀβελίσκος (vgl. § 24 unter ε u. ο). Ferner dor. δήλομαι (Kos) oder δείλομαι (lokr.) = böot. βείλομη, thess. βελλειτει = βούληται (arkad. βόλομαι, in anderen dor. Mundarten βώλομαι, lesb. βόλλομαι Theokr. 28, 15); arkad. δέλλω = βάλλω Inschr. Tegea, während die Gramm. als arkadisch vielmehr ζέλλω, ἔζελεν angeben; ebenso für δέρεθρον (Hesych. ohne Angabe des Dial., d. i. βάραθρον βέρεθρον) als arkad. ζέρεθρον. Meister II, 105 f. Die Natur dieses ζ ist wenig klar; vielleicht entstammen die Formen einem arkad. Lokaldialekte, der wie das Eleische für jedes δ ζ schrieb. Noch wird ἐπιζαρεῖν (Eur. Phoen. 45, Rhes. 441) als arkad. für ἐπιβαρεῖν angegeben (Eustath. 909, 27; Meister das.). δ u. γ: Die alten Grammatiker führen für diesen Wechsel die dor. Wörter δᾶ = γᾶ, γῆ, Δᾶμάτηρ, δάπεδον = γάπεδον, ferner δνόφος u. δνοφερός (b. Hom., Hippokr., auch b. d. Trag., Lyrik. u. spät. Att.) = γνόφος äol.) an; allein dieses δᾶ kommt nur in Interjektionen, als: φεῦ δᾶ vor, während anderweitig auch dor. γᾶ erscheint. Kypr. indes ζᾶ; Hesych. auch dialektisches δῆ = γῆ, Meister II, 254. Andererseits überall Δημήτηρ, Δαμάτηρ mit δ, nirgends mit γ.163) Auch δάπεδον mit kurzem α kann nicht von γάπεδον (ᾱ) herkommen (δπεδα Aesch. P. 829 ist von Porson in γαπ. emendiert, vgl. Hdn. I, 378); Curt. Et.\textsuperscript{5}, S. 621 f. meint, die Silbe δα sei wie in δα-φοινός δά-σκιος von ζα = διά abzuleiten, und so steht ζάπεδον Xenophan. 1, 1; Epigr. Paros Kaibel 750 a, 3. Die Wörter γνόφος, γνοφερός sollen nach Ahr. I, p. 73 des Wohllautes wegen für δνόφος, δνοφερός gesetzt sein, wie γλυκύς st. δλυκύς (dulcis), vgl. δεῦκος = γλεῦκος, ἀδευκής (Hom.) = ἀγλευκής (sicil., Epicharm., Rhinthon, auch Hippokr. Xenoph., lakon. ἀγλευκέρ = ἀηδές Hesych., Ahrens II, 101). Wahrscheinlicher ist die Ansicht von Curt., Et.\textsuperscript{5} --145-- S. 535, in γνόφος sei γ eine Erweichung von κ (κνέφας), δ aber durch den Einfluss des ν entstanden, so ἁδνόν kret. st. ἁγνόν, Ἀριάγνη Vasenaufschrift st. Ἀριάδνη. — Ferner dor. δέφυρα Kreta (Comparetti, Mus. Ital. II, 635), lakon. δίφουρα Hesych., = γέφυρα; als theban. erscheint in den Hdschr. Athen. 14, 622, a βλέφυραν, wofür βέφ. Meineke, Kock (Com. I, 725). c) Aspiratae (vgl. § 12, 2). φ und θ: Aeolisch wird genannt φήρ = θήρ, φηρίον = θηρίον, und damit das Homerische φηρσὶν ὀρεσκῴοισι Il. α, 246 (vgl. β, 743) erklärt, was richtig sein wird trotz Meister I, 118 f.; πεφειράκοντες = τεθηρακότες und Φιλόφειρος = Φιλόθηρος thessal. Inschr. Ferner böot. Θιόφειστος (= -φεστος) zu θέσσασθαι = εὔξασθαι, Θέστωρ, ἀπόθεστος = ἄπευκτος, vgl. lat. festus (wiewohl Curtius, Et. 520\textsuperscript{5} die Wörter nicht mehr so auffasst); böot. Φετταλός, thessal. Πετθαλός = Θετταλός; Alkman fr. 22 φοίναις = θοίναις. So [root ] φεν (ἔπεφνον) u. θείνειν, ferire; φλᾶν b. Pind., Theokr., Hippokr., Aristoph. u. φλίβειν Od. 17, 221, Theokr. 15, 76, Hippokr. (so VI, 292. 300) = θλᾶν, θλίβειν werden von den alten Gramm. als äol. angegeben (vgl. Meister I, 119). φ u. χ: Lesb. αὔφην Ioann. Gramm. = αὐχήν, aber ἀμφήν nach Hesych., u. so ἄμφενα Theokr. 30, 28. χ u. θ: Lesb. πλήχω = πλήθω in Cramer, Anecdot. Oxon. 1. 149, 6; aber Sapph. 3 πλήθοισα; dor. nur ὄρνιχος u. s. w. (= ὄρνιθος) v. ὄρνις (kret. indes ὄννιθα). χ u. φ: Thessal. ἀρχιδαυχναφορείσας auf e. Inschr. = ἀρχιδαφνηφορήσας.
\section{B. Liquidae und σ}
a) Liquidae unter einander. λ u. ρ: Dor., neuion. Her. 2, 92, selbst att. b. Aeschyl. fr. 309 D. κλίβανος, att. κρίβανος; κριβανίτας u. κλιβανίτας Sophron 56. 57. κριβανίτας Epich.; arkad. κρᾶρος = κλῆρος, vgl. att. ναύκραρος neben ναύκληρος, Meister II, 104. 319, G. Meyer 172\textsuperscript{2}. Im att. Ἄγλαυρος für Ἄγραυλος haben ρ u. λ ihren Platz gewechselt; auf e. att. Inschr. steht Κλωπίδης f. Κρωπίδης. Meisterhans 63\textsuperscript{2} f. — Kret. λάκη = ῥάκη; aber λακίς, λακίζω u. die Derivata sind att. Vgl. § 13. 67, 4. ν u. λ: vor τ u. θ dor. in Κένται = Κέλται, δέντα = δέλτα (Et. M. 503, 47), φιντάται Epich. 31 = φίλταται, Φιντίας tab. Her., Φίντις Sicilier b. Pind. = Φιλτίας, Φίλτις, κέντο Alkm. 141 = κέλετο, βέντιστος Theokr. 5, 76 = βέλτιστος, ἐνθεῖν oft Theokr. = ἐλθεῖν, ebenso ἐνθοίσα Alkm. 23, col. III, 5; ἐνθών D.-I. Korkyra 3188; aber daneben im Dorismus βέλτιστος, βέλτιον, ἐλθεῖν (dies z. B. Kreta; auch lakon. Kühners ausführl. Griech. Grammatik. I. T. --146-- Aristoph. Lysistr. ἔλσοιμι, ἔλσῃ mit ς für θ); Curtius (Et.\textsuperscript{5} S. 450) sieht diesen Übergang als eine seltene Art der Assimilation an, da die dentalen Mutae dem dentalen Nasal näher stehen als dem λ. λ u. ν: Der Ort Νάπη auf Lesbos hiess b. Hellanikos (Hdn. I, 338) Λάπη; sodann λίτρον st. νίτρον; ϝίτρον ist ein Lehnwort, hebr. neter (s. Curt., Et.\textsuperscript{5}, S. 450), das b. Hippokr. neben λίτρον, sowie bei Sappho vorkommt, aber erst seit Alexanders Zeit allgemein gebräuchlich wird, s. Lobeck ad Phryn. p. 305, Rutherford, Phryn. 361; Herodot 2, 86, 87, 92 und die ächt att. Schriftsteller gebrauchen nur λίτρον. Ferner νίκλον Hesych. = λίκνον ([root ] νικ); ἄλλος alius, sk. anjas. L. Meyer, Bzz. Btr. II, 105; Vgl. Gramm. I\textsuperscript{2}, 2, 848. G. Meyer 177\textsuperscript{2}f. Über πλεύμων st. πν. s. § 14. ν u. μ: Dor. νίν, alt- u. neuion. μίν, nach Döderlein ältere Form, aus ἰμ-ιμ entstanden, wie das altlat. em-em, v. d. Pronominalstamme ἰ; vgl. Curt. Et.\textsuperscript{5}, S. 543 u. oben § 14, 2. b) Liquida ρ und Spirant ς. Vgl. § 15. In der vulgären lakon. und in der eleischen Mundart wird im Auslaute, z. T. auch im Inlaute vor Konson. ρ statt des ursprünglichen ς gebraucht;164) so b. Hesych. lakon. ἐπιγελαστάρ st. ἐπιγελαστής, ἀδελφιόρ st. ἀδελφός, ἀκκόρ st. ἀσκός, δαιδῶχορ st. δᾳδοῦχος, πίσορ st. πίθος, σιόρ st. θεός, τίρ st. τίς, νέκυρ st. νέκυς, ἀβώρ st. ἠώς, πόρ st. πούς, ζύγωνερ st. ζύγωνες (d. i. βόες ἐργάται), φουλλίδερ st. φυλλίδες, βίωρ (d. i. ϝίωρ) st. ἴσως, einmal in der Lysistr. παλεόρ γα 988, auf Inschr. erst sehr spät; im Inlaute z. B. πούρτακος st. πύστακος, μίργωσαι st. μίσγουσαι; [kret. τεόρ b. Hesych. st. τεός = σός; auf Inschr. noch nirgends derartiges, ausser κόρμος = κόσμος, doch überwiegend κόσμος]; eleisch: jüngere Inschr. ausnahmslos ρ statt ausl. ς, als Δαμοκράτηρ, Ἀγήτορορ, Τενέδιορ, Διονυσιακοῖρ, τᾶρ, πλείονερ, auf älteren wenigstens vielfach, als μάντιερ, τὶρ; inlautend nach Pausan. 5, 15, 4 Ἀπόλλων Θέρμιος = θέσμιος, welche Deutung indes zweifelhaft, Meister II, 51 Anm. In anderen Dialekten sporadisch vor Media, als Πελαργικόν att. (Meisterhans 63\textsuperscript{2}) = Πελασγικόν, Θεόρδοτος thessal. = Θεόσδοτος. Besonderer Art aber ist der Rhotacismus von Eretria (Oropos), den Platon (Kratyl. 434, c) irrig durch ein angebliches σκληροτήρ st. σκληροτής kennzeichnet; die Inschr. zeigen vielmehr, dass ausser vor weichen Kons. (Eigenn. Μίργος, Bechtel, Inschr. d. ion. Dial., S. 10. 13) das ς auch zwischen Vokalen in ρ übergeht; dagegen am Ende nirgends, also δημόριος, παραβαίνωριν. Vgl. lat. generis für genesis, floris für floris u. s. w., d. kiesen u. küren. --147-- ῤῥ st. des älteren ρς gebraucht ein Teil der Dorier (auch die Eleer) und mit ihnen die Attiker, doch so, dass die Tragiker u. die älteste Prosa diesen Atticismus nicht annahmen, gleichwie derselbe auch in die κοινή nicht überging, als: Megara Ὄρριππος, Χερρίας, Helm des Hieron Τυρ (ρ) άν d. i. Τυρσηνά, Thera Θαρ (ρ) υπτόλεμος, Alkman 44 κάρρα = κόρρα Theokr. 14, 34, κόῤῥη att., κόρση ion.; κάρρων dor. (aus καρτίων, κάρσσων) = κρείσσων, κρείττων, θάῤῥος m. den Derivatis st. θάρσος (θαρρεῖ Epich. 153), ἄῤῥην st. ἄρσην (ἔρσην kret. Epidaur.), μυῤῥίνη att. st. μυρσίνη, Χεῤῥόνησος st. Χερσόνησος, Τυῤῥηνός st. Τυρσηνός; att. Inschr. Φερρέφαττα, ταρρός u. a., Meisterhans 76\textsuperscript{2} f. Doch bleibt das ς att. in βύρσα (Inschr. Aristoph.), Ἕρση, fremden Namen wie Πέρσης, Μαρσύας, sowie in Flexionen und Ableitungen, s. § 64, 5. Von Schriftstellern der Prosa hat Thukyd. ρς, ρρ und ρς Xenoph., ρρ die Redner (seit Andokides) und Plato.165) Anmerk. Πυρσός, rötlich, hat Euripides, so πυρσαῖς γένυσι Phoen. 32, was Hesych. u. Photius (dieser mit πυρραῖς) citieren; aber die Form ist falsch, da πυρϝός (Korinth D.-I. 3119, h) zu Grunde liegt; mit Unrecht also hat man Aesch. Pers. 316 πυράν (Med. pr.) πυρράν in πυρσήν emendiert. Πυρρός steht auch Herodot 3, 139, Hippokr. II, 74. VI, 74 L.; vgl. die Heroennamen Πύρρος, Πύρρα (letzteres auch Ortsname), die Eigenn. Πύρρος, Πυρρίας, Πύρρων (Πύρων Thessalier Isokr. 17, 20) auch in Thessalien u. Böotien. c) Liquida ν und Spirant ς: Statt des ς, das die Dorier in der 1. Pers. Pl. Akt., als: φέρομες, sk. bhárâmas, l. ferimus, καλέομες, πεπόνθαμες, und in ἦς = erat bewahrt haben, gebrauchen die Aeolier, Ionier und Attiker ν, als: τύπτομεν u. s. w., ἦν. Ferner: dor. αἰές, ἀές, lesb. αἶι (ν) ἄϊ, thess. ἀΐν, ep. poet. αἰέν, welches indes auch dor. ist, wie αἰέ u. ἀέ; dor. πέρυτις u. πέρυτι = πέρυσι (ν), ἔνδος (u. ἔνδοι) = ἔνδον; umgekehrt αὖτιν Gortyn (αὖθιν die Rheginer) = αὖθις, ἔμπᾶν u. ἔμπα Pind. nb. ἔμπας (ion. ἔμπης); τετράκιν u. s. w. lakon. Inschr. (auch lesb. nach Theokr. 30, 27 ὀπποσσάκιν), ohne Kons. πολλάκι poet., τουτάκι nb. -κις, ἑξηκοντάκι, τετράκι, θαμάκι (u. -κις) Pind., vgl. Herodian I, 506; ἑξᾶν Inschr. Rhod. Kos Thera f. ἑξῆς.
\section*{II. Wechsel der gleichnamigen Konsonanten unter einander}
\addcontentsline{toc}{section}{II. Wechsel der gleichnamigen Konsonanten unter einander}
\section{(a) Kehllaute}
γ u. κ: Att. γναφεῖον, Γνίφων nb. älterem κναφεύς, Κνίφων (Meisterhans 58\textsuperscript{2}), so auch κνάπτω altatt., ἐκνάπτετ Soph. Ai. 1031 nach Laur. pr., aber γναπτόμενοι Aesch. P. 576 [κν. Dindorf];166) Herodot --148-- κναφεύς, κναφήϊον, Hippokr. γναφεύς (II, 666 L.). Ferner γνά (μ) πτω (Hom.) κνάπτω κνάμπτω κάμπτω beuge, Siegismund, Curt. Stud. V, 192 f.; καμψώνυχες u. γαμψώνυχες für γναμψ., Adj. γαμψός u. s. w.; hellenist. γράστις f. att. κράστις, Hdn. II, 537. Vgl. cygnus st. cycnus, grabātus κράβατος. S. § 10, 1. κ u. γ: κλάγος b. Hesych. kret. st. γλάγος Hom. (= γάλα); att. κωλακρέται (so auch Inschr.) st. κωλαγρέται. κ u. χ: Lesb., dor. u. neuion. (doch nicht Hippokr.) δέκομαι, ep., att. δέχομαι, das sich zuweilen auch auf dor. Inschr. findet; in Ableitungen auch ep. att. κ, als ἱστοδόκη, δωροδοκεῖν, ξενοδοκεῖν, πανδοκεῖον (Lobeck ad Phryn., p. 307, Rutherford, Phryn. 362), nachklass. πανδοχεῖον, πανδοχεύς u. s. w.; neuion. οὐκί = οὐχί; Hom. τετυκεῖν, τετυκέσθαι v. τεύχω; dor. (sicil.) κιτών (Sophr. 62) u. κύτρα, wahrscheinlich nur sikel. Solöcismus, da Epicharm χύτρα sagt, s. Ahrens II, p. 82; vgl. den Skythen in Aristoph. Thesmoph., der für jede Aspirate die Tenuis setzt. — Über σχ st. σκ, χμ st. κμ, γμ, χν st. γν u. s. w. s. § 63, 1. 2; für das vulgäre ῥέγχω ist die att. Form ῥέγκω; umgekehrt att. θυηχοῦς f. θυηκόος, Wecklein, Cur. epigr. 42 f., Roscher, C. St. I, 2, 80, der eine Menge sonstiger Belege nam. aus Inschr. beibringt.

\section{(b) Zahnlaute}

τ u. δ, δ u. τ: τρύφακτος f. δρύφ. Hdn. II, 595; dor. Ἀρτάμιτος = Ἀρτέμιδος v. Ἄρταμις = Ἄρτεμις, θέμιτος, att. Θέμιδος, ion. Θέμιος, s. § 130. Δάπις att. st. τάπις, Ar. Plut. 527, Suid. v. δάπιδας; nach Ael. Dionys. (Eust. 1369) ist auch δάπης d. alte Form für das τάπης d. Jüngeren (welches indes auch in unserm Homer steht). Schwanken zw. δ u. τ ist ferner in Ἀτραμυτηνός Ἀδραμ. (Ἀδραβυτ.) auf att. Inschr., wie auch bei Autoren in diesem Stadtnamen Schwanken; ähnl. ἀδράφαξυς nb. ἀτράφαξυς (-ις) (ψευδατράφαξυς Ar. Eq. 630) und ἀνδράφ. (Hippokr. VI, 560 L.); att. Inschr. κρατευτής (Il. ι, 214) u. κραδευτής; regelm. die Inschr. ἐνῴδιον Ohrgehäng (man erwartete ἐνωτίδιον), nicht ἐνώτιον, wonach Aesch. frg. 101 zu berichtigen. Wackernagel, Philol. Anz. 15, 199; Meisterhans 61\textsuperscript{2}, nach Riemann, Rev. de philol. IX, 56. Πελιτνός att. st. πελιδνός, Thuk. (2, 49) nach Ael. Dionys. b. Eustath. 735. — Anderes G. Meyer 202\textsuperscript{2}. τ u. θ: τίριος (b. Hesych.) kret. st. θέρεος, auf d. Gortyn. Tafeln ἄντρωπος, τετνακός, τνατῶν (aber θάνῃ wie gew.); eleisch Dial.-Inschr. 1149 ἐνταῦτα st. ἐνταῦθα oder ion. ἐνθαῦτα, ferner eleisch nach ς, als προστιζίων = προσθιδίων, und in den Endungen σται, στᾶν, στω, στων f. σθαι, σθαν, σθω, σθων (Meister, Dial. II, 54), ebenso lokrisch (Allen, Curt. Stud. III, 241 ff.; Blass, Ausspr. 111\textsuperscript{3}); alt- u. --149-- neuion. αὖτις = αὖθις (auch Polyb. αὖτις, Kälker, Lpz. Stud. III, 228), kret. αὖτιν; über Ταργήλιος (Anakreon) s. Roscher, Curt. Stud. I, 2, 114 ff. θ u. τ: Böot. 3. Pers. Pl. ἔχωνθι st. ἔχωντι (att. ἔχωσι), ἴωνθι st. ἴωσι, ἀποδεδόανθι st. ἀποδεδώκασι, so auch im Med. -νθη (= νται), -νθο, -νθω, u. thessal. -νθι, -νθειν (= νται), -νθο, s. § 63, 2, u. über σθ nb. στ das. 1. Verschreibungen auf Inschr. Roscher, S. 85 f. θ u. δ: Att. seit Alexanders Zeit ganz gewöhnlich und schon erheblich früher auftauchend οὐθείς, μηθείς st. οὐδείς, μηδείς, indem die Media mit dem Hauche von εἷς sich zur Aspirata vereinigt hat (s. § 187, 1); so scheint auch sonst auf att. Inschr. zuweilen οὔθ vor οἱ, ὑγιές st. οὐδ zu stehen, Meisterhans 80\textsuperscript{2}; auch dor. μηθαμεῖ = μηδαμοῦ (μηδʼ ἁμεῖ) Inschr. Epidauros. λ u. δ: λάφνη b. Hesych. pergäisch = δάφνη, Ὀλυσσεύς (l. Ulixes) b. Eustath. 289, 38; so auch altatt. Vasen Ὀλυττεύς, Meisterhans 64. 77; Οὐλιξεύς kennt Prisc. VI, 92, vgl. Bergk zu Ibyc. 11 A. 73, Οὐλίξης sicil. Plut. Marc. 20, s. Jordan, krit. Btr. z. Gesch. d. lat. Spr. 39 ff., G. Meyer 179\textsuperscript{2}, Kretschmer, K. Z. 29, 430 ff.; vgl. l. lacrima u. δάκρυ, levir u. δαήρ. ν u. δ: νύναμαι νυνατός Gortyn. Taf. für δύναμαι δυνατός, wo entweder hier Dissimilation oder dort Assimilation im Spiele ist. δ u. ς vor μ: Alt- u. neuion. ὀδμή = ὀσμή, ἴδμεν = ἴσμεν; b. Hom. auch Infin. ἴδμεναι, wie ἔδμεναι v. ἔδω; es wird dies äolisch genannt, Meister, Dial. I, 151. Ferner φράδμων Il. π, 638, προπεφραδμένα Hes. Op. 655, πεπυκάδμενος Sapph. 56; κεκαδμένος Pind. O. 1, 27; ὀδμή auch b. Eurip. Hipp. 1391; Aesch. Prom. 115, vgl. Dindorf, Thes. 5, 1733; Xenophon wird wegen des ion. ὀδμή von Phrynichus getadelt, Rutherford 160 ff.; Pollux 2, 76 führt ὀδμάς u. εὐοδμία (so L. Dindorf, Hdschr. mit ς) aus Antiphon an. Bei Aesch. Pers. 417 schwankt die Lesart zwischen ἀφρασμόνως (Med.) u. ἀφραδμόνως, aber Lobeck. ad Aj. 23 zieht wegen des gewöhnlichen Gebrauches der Tragiker ἀφρασμ. vor. Vgl. Kretschmer, K. Z. XXIX, 429 f. (arkad. Ὁπλοδμία Phyle nb. Ἥρα Ὁπλοσμία u. a.; Μεδμαίων u. Μεσμαίων d. Münzen von Medma in Italien); § 61, Anm. θ u. ς vor μ: Bei Hom. εἰλήλουθμεν, κεκορυθμένος, ἐπέπιθμεν; ion. ἀναβαθμός Herod. 2, 125, att. καταβασμός Aesch. Pr. 817; βαθμός Soph. fr. u. Sp., βαθμίς Pind., aber ἀναβαζμός d. i. -σμός att. Inschr.; κλαυθμός u. die Derivata allgemein gebräuchlich (aber ἀνακλαυσμός, Dion. Hal.), σταθμός, ῥυθμός167) (doch ῥυσμός Archil. fr. 66 Bergk u. --150-- Demokrit), δυθμή Callim. (fr. 539, Hymn. 6, 10) st. δυσμή. Vgl. oben δ u. ς u. § 61, Anm. ς u. τ (vgl. § 63, 3):168) Das ursprüngliche τ, das die Boötier, Thessalier und Dorier samt den Eleern gemeiniglich treu bewahren, und wodurch diese einen Gegensatz zu den anderen Mundarten bilden, erweichten die Lesbier, die Arkadier und Kyprier, die alten und neuen Ionier und die Attiker, insbesondere vor ι, in ς, s. § 10, 3, was man Assibilation nennt. Im Anlaut zeigt sich dies kaum, desto mehr in der Mitte der Wörter, und zwar a) in den Adj. auf τιος dor. = σιος, als: πλούτιος = πλούσιος, ἐνιαύτιος = ἐνιαύσιος, πλατίος = πλησίος; in den substantivierten Adjektiven dieser Endung, als: Ἀρταμίτιον (v. Ἄρταμις, dor. G. Ἀρτάμιτος), = Ἀρτεμίσιον, Ἀφροδιτία, e. Stadt, = Ἀφροδισία; Σελινουντίοι = Σελινούσιοι u. s. w.; in den Zahlwörtern der Hunderte, als: διακατίοι = διακόσιοι, τριακατίοι = τριακόσιοι u. s. w., doch sind auch die Formen auf όσιοι schon frühzeitig bei den Doriern im Gebrauche und auf den Inschr. nach Alexanders Zeit allein üblich, s. Ahrens II, p. 61 sq. u. p. 281, gleichwie die Inschr. dieser Zeit auch ἐνιαύσιος aufweisen (Kretschmer, K. Z. XXX, 584); — b) in den Abstraktis auf τία (bei einigen auch att., s. Lobeck, Parerg. p. 505 sqq., Cobet, Misc. 215 f., als δημοκρατία u. andere auf -κρατία) als: ἀδυνατία (v. ἀδύνατος) dor. = ἀδυνασία, ion. ἀδυνασίη, γεροντία, lakon. Wort b. Xenoph. R. L. 10, 1 u. 3, das Amt eines spartanischen Senators, v. γέρων, οντ-ος, = γερουσία (auf jüngeren dor. Inschr. die gewöhnl. Form, als: εὐεργεσία); — c) die dor. Abstrakta auf τις sind selten, als: ϝοινάρυτις = οἰνήρυσις n. Ahrens 55 (die Amphiktyoneninschr. C. I. Gr. 1688 hat nicht δῶτις = δόσις, sondern λῶτις, ein unerklärtes Wort); gewöhnlich auch dor. σις; ebenso herrscht σι in den Komposita vor Verben, als Ἁγησίλαος, wiewohl Ὀρτίλοχος Paus. 4, 30, 2 die Ursprünglichkeit des τ auch in diesen Bildungen zeigt, G. Meyer 289\textsuperscript{2}, Müllensiefen de titul. lac. dial. p. 182; — d) böot. ϝίκατι, dor. ϝείκατι ϝίκατι εἴκατι = εἴκοσι, wie auch dor. vom 3. Jahrh. ab; dor. πέρυτις od. πέρυτι = πέρυσι, vorigen Jahres, ποτί u. kret. πορτί (Hom. προτί u. ποτί) = πρός; — e) dor. in der 3. Pers. S. u. Pl., als: φατί = φησί, φαντί = φασί, δίδωτι = δίδωσι, τύπτοντι = lesb. τύπτοισι, att. τύπτουσι, τιθέντι = τιθεῖσι; so auch böot. τίθειτι u. s. w., im Plur. aber νθι wie auch thessal. (s. oben θ u. τ); — f) Ποτειδάν dor., aber auch Ποσειδάν und mit Verhauchung --151-- des ς lakon. Ποοἱδάν; dazu mit ι Ποτιδάν Ποτιδᾶς; Ableitungen Ποτείδαια Ποσειδανία, vgl. § 122 Anm. 5; böot. Ποτ (ε) ιδάων Kor. 1, thessal. Ποτειδοῦνι (Abltg. Ποσιδίουν Eigenn.); lesb. Ποτίδαν und Ποσείδαν. Zu πίπτω, d. i. πι-π (ε) τω, dor. lesb. Aor. ἔπετον (wiewohl ἐμπέσων Sapph. 42 überl.); hier ist kein nachfolgendes ι Grund der Assibilation, sondern vielleicht die Analogie des Fut. πεσοῦμαι, s. § 226 Anm. 2. — Assibilation im Anlaute vor ι kypr. σίς σὶς = τίς τὶς (sonst überall hier und in anderen Wörtern τ erhalten); vor υ in σύ σοί σέ u. s. w., dor. τύ τοί τέ u. s. w., auch böot. τού u. s. w., aber lesbisch scheint ς gewesen zu sein, s. § 160 f.; ferner ist böot. τῦκον für σῦκον, vgl. (nach Ahrens) Τυκῆ = Συκῆ, e. Teil von Syrakus (doch συκία = συκῆ tab. Heracl.); τυρίσδω (st. συρίζω) b. Theokr. in den schlechteren Codd. — Dieser Dorismus des τ = ς erhält sich nur in der Konjugation und in der Präposition ποτί zu allen Zeiten; in allen übrigen Wörtern und Wortformen wich er seit Ende des 4. Jahrh. dem gewöhnlichen Gebrauche mit ς. S. Ahrens II, p. 59 sqq. Hingegen gebrauchen statt eines anscheinend ursprünglichen τ merkwürdiger Weise die Dorier ς in σάμερον u. σᾶτες, während die Attiker das τ bewahrt haben: τήμερον (aus τό u. ἡμέρα?) und τῆτες (aus τὸ ἔτος?), heuer; in dem Dor. σᾶτες fällt auch das α auf; ion. immer σήμερον, σῆτες, so dass die Scheidung der Dialekte hier wie bei σς — ττ ist (thessal. τᾶμον = τήμερον n. Prellwitz, Dial.-I. 345, 44; Prellwitz, dial. Thess. 48 verweist auf Apoll. Rh. 4, 252 τῆμος); ebenso in dem ion. und gem. σηλία, att. τηλία, vgl. σάω (σήθω) siebe, aber att. δια-ττάω ἐττημένα, Wackernagel, K. Z. XXVIII, 121; ferner heisst es im Ion. und gemein σεῦτλον, σύρβη, att. τεῦτλον (auch Hippokr., so VI, 248. 252; v. l. 560; II, 482) τύρβη; über σίλφη u. att. τίλφη od. τίφη Ar. Ach. 920. 925 s. Lobeck ad Phryn. p. 300, Rutherford 359; nach G. Meyer 258\textsuperscript{2} Anm. ist indes τίφη (vgl. lat. tipula) von σίλφη zu trennen. Das megar. σά st. τίνα gehört zu ion. ἅ-σσα att. ἅττα, also gls. σσά f. τjά τία, s. unten ττ u. σς. ς u. θ: Lakon. σιός = θεός, σιά Alkm. = θεά; ναὶ τὼ σιώ = θεώ (Dioskuren), b. Alkm. auch σαλασσομέδοισα (= θαλ.), σάλεσσιν = θάλεσιν, σάλλει = θάλλει, ἔσηκε = ἔθηκε, παρσένος = παρθένος u. s. w., in Aristoph. Lysistrata: σέλει = θέλει, σέτω = θέτω, σιγῆν = θιγεῖν, σιά, ἀγασός = ἀγαθός, μυσίδδω = μυθίζω, ἔλσῃ = ἔλθῃ u. a.; viele sonstige lakonische Wörter mit ς st. θ werden von den alten Grammatikern und bei Hesychius angeführt, s. Ahrens II, S. 68 sq., der p. 70 bemerkt, dass dieses ς st. θ nicht zu jeder Zeit von den Lakedämoniern gebraucht worden sei, indem in der lakon. Kolonie Tarent sich nicht die geringste Spur davon zeige. Auch die --152-- Inschr. haben erst in hellenistischer Zeit etwas davon, in Eigenn. wie Ἐλευσία = Ἐλευθία (= Ἐλευθώ, Ἱλείθυα), während übrigens in vorchristl. Inschr. nichts als θ erscheint (u. nach ς anscheinend τ, vgl. oben τ und θ). Das θ bleibt ferner (Spiess, C. Stud. X, 362) in den Texten (Alkman, Lysistr.) nach ς (ποτήσθω), ν (ἐπανθεῖ u. a. Alkm., πεπόνθαμες, Κορινθία Lys.), vor λ, ρ (ἀεθλοφόρον, ὀρθρίαι), nach φ (φθέγγεται) u. jedenfalls auch χ, endlich, was auch die Grammatiker hervorheben, wenn die nächste Silbe mit ς beginnt (θωστήρια); anderweitiges θ ist in den Frg. Alkmans für entstellt, in der Lysistr. (θείκελος, ἴθι, θάγοντας) für nicht echt lakonisch zu nehmen. S. Blass, Ausspr. 108\textsuperscript{3} f. Die Sache kann nun nicht anders als so sein, dass die Lakonier schon zur att. Zeit statt t' (engl.) th sprachen, welchen Laut die anderen Griechen, wenn sie Lakonisches wiedergaben, mit ς ausdrückten; so kam auch in Alkmans Gedichte dies ς hinein, während der Dichter selbst jedenfalls θ schrieb. (Blosse Korruptel ist νεὶ τὼ σιώ im Munde des Böoters Aristoph. Ach. 905, st. νεὶ τὼς θιώς τοὺς θεούς gl. Vict.]; der Böoter kann nicht bei den Dioskuren schwören. Meister I, 260 hätte dies nicht als Beweis für den vermeintlich spirantischen Laut des θ bei den Böotern benutzen sollen.) — Ein besonderer Fall ist ἄννηθον (Ar. Th. 486, codd. ἄνηθον) ἄνητον äol. ἄνησον ἄνν. Hippokr. II, 274. VIII, 170 (mit θ VI, 558), lat. anisum. θ u. ς: Nach Strab. 13, p. 912 sagten die Rhodier ἐρυθίβη st. ἐρυσίβη und nannten daher den Apollo Ἐρυθίβιος. ττ u. σς: Statt der aus einem K- oder T-Laute mit j oder ς entstandenen Lautgruppe σς, welche die meisten Dorier, die Lesbier, Arkadier, Kyprier und der grössere Teil der Ionier gebrauchen, haben andere Stämme mit umgekehrter Assimilation ττ; gemeinsame Grundlage für beides ist τς, was auf den ältesten kretischen Inschriften in der Gestalt von ζ noch zuweilen erscheint (Blass, Ausspr. 120\textsuperscript{3}). Die seit Ascoli (Krit. Stud. 324 ff.) herrschende und auch von Curtius statt der dargelegten früheren angenommene Erklärung, wonach σς überall zuerst entstand und daraus sich erst ττ bildete, entbehrt durchaus des genügenden Grundes und wird s. Z. wohl wieder aufgegeben werden. Am weitesten durchgeführt ist das ττ im Böotischen: oft θάλαττα, dann φυλάττι, πίττα (aus Guttural mit j, § 21, 3), aber auch ὁπόττα f. ὁπόσσα, ὁπόσα (τ mit j, § 178, Anm. 3), im Aor. ἐκόμιττα u. s. w. (aus δ-ς, τς) für ἐκόμισσα, ἐκόμισα; so auch ἐπεχαρίττω (-α) γὦ ξένε Ar. Ach. 867, wie st. ἐπιχαρίττω Rav. ἐπιχαρίττως vulg. zu lesen, = ἐπεχαρίσω, u. das. 884 κἠπιχάριτται (so zu schr.) = ἐπιχάρισαι, während das Attische in ὅσος, ὁπόσος u. s. w., in ἐκόμισα u. s. w. das τ von τς --153-- gleichwie in χάρισι, παισί ausgestossen hat (μέσος aus μέθjος auch böot. Dial.-I. 491). Im übrigen ist ττ auch attisch, u. zwar von jeher (Meisterhans 77\textsuperscript{2}); sogar Κατ (τ) άνδρα, Ὀλυτ (τ) εύς = Ὀδυσσεύς findet sich auf att. Vasen, nur einmal τές (ς) αρα. Die Tragödie aber hat diesen Atticismus so wenig wie ρρ st. ρς angenommen, sondern wandte das ion. σς an; ihr folgte die älteste Prosa (Gorgias, Antiphon, Thukydides), während die Komödie und die Spätere Prosa (Lysias, Isokrates, Plato u. s. w.) den Dialekt folgerichtig wiedergab.169) Somit att. πράττω, τάττω, γλῶττα, θάλαττα, μέλιττα, κρείττων, ἥττων u. s. w., auch καττύω aus κατ-σύω; ein besonderer Fall ist att. τέτταρες, böot. πέτταρες, ion. τέσσερες, dor. τέτορες, wo τϝ zu Grunde liegt, § 16, 3 c). Ausgenommen sind im Att. die Wörter πτήσσω, πτίσσω, πτύσσω (wegen πτ, indem πτήττω zu hart, Lobeck, Paralip. 31); aber βασίλισσα ist nicht echt attisch (Phrynich. Rutherford 306); ebensowenig βυσσός (βύσσος Fremdwort), χαρίεσσα (dagegen μελιττοῦττα aus -τόεσσα, οἰνοῦττα). Mit dem Böotischen u. Attischen teilt auch das benachbarte Euböische samt dem Oropischen diese Eigentümlichkeit: Κιττίης Styra, ἔλαττον Oropos, dazu ἐκπρηττόντων Bechtel Nr. 22, dessen Zweifel (S. 13. 37) an euböischem ττ nicht genügend begründet erscheinen. Doch mögen die Chalkidier σς gehabt haben (ὅσσα Rhegion B. 5, πίσσης Olynth 8, b). Im Thessalischen finden wir: Πετθαλοί = Θετταλοί wohl besonderer Art; Μολόσσειος Pherai (D.-I. 328), aber Μολοτοῖ Larisa Bull. de corr. hell. 1889, 381 f. (die Schreibung mit einem τ auch dem Steph. Byz. bekannt); wiederum Lar. (D.-I. 345) ὅσσα u. πρασσέμεν, dazu ἐνεφανίσσοεν = ἐνεφάνιζον, Phalanna Μέλισσα D.-I. 1331. Die Grammatiker schreiben ττ für σς auch den Thessaliern zu (Meister I, 265, 1), desgleichen den Kitiern auf Kypros. Endlich findet sich ττ auf Kreta: ὁπόττοι, δάτταθθαι = δάσσασθαι, κάρτων d. i. κάρττων (anderweitig dorisch κάρρων aus κάρσων) = κρείττων Gortyn (doch das. früher ζ: ὄζος ὅσος, ἀνδάζαθ (θ) αι, s. o.); auf anderen späteren kret. Inschr. θάλαττα und auch θάλαθθα, doch auch das gew. dorische θάλασσα. Vgl. G. Meyer 273\textsuperscript{2}. Dies ττ steht dann kret. auch für δδ ζ, als φροντίττοντας (Inatos) Bull. de corr. hell. XIII, 73, ἐσπρεμμίττεν = ἐκπρεμνίζειν (Gortyn) das. IX, 9; sogar im Anlaut, wie Ττῆνα = Ζῆνα; G. Meyer 256\textsuperscript{2}, Blass, Ausspr. 120\textsuperscript{3}. ττ u. στ: Böot. in der Redensart ἴττω Δεύς, per assimilat. st. ἴστω, Ar. Ach. 911, Plat. Phaed. 62, a, epist. VII, 345, a; gewöhnlich aber --154-- wird στ beibehalten, als: κεκόμιστη, ἔστω u. s. w.; lakon. βεττόν (Kleid) = ϝεστόν (ἑστόν) v. ἕννυμι, vestio, ἄττασι = ἄ (ν) σταθι, ἀνάστηθι, desgl. wohl lakon. ἐττία = ἑστία, ἔττασαν = ἔστασαν; tarent. in Ἄφραττος b. Hesych. st. Ἀφραστος. τθ (θθ) u. σθ: ὀπιτθοτίλα f. σηπία böot. nach Strattis (Meister, Dial. I, 265), Inschr. indes überall σθ; ferner θθ Gortyn in den Verbalendungen, als χρήθθαι, ὠνήθθαι, ἀμφαινέθθω (nie τθ geschr.), auch in πρόθ (θ) α, und zwischen Auslaut und Anlaut in τὰθ θυγατέρας u. s. w., Baunack, Inschr. v. G. 18; G. Meyer, 261\textsuperscript{2}f. In späteren kret. Inschr. findet sich θθ auch für στ: ἱθθᾶντι, ἱστῶσι (s. das.). (Spir. asper u. ς: s. § 23, 2; hier sind Laute, die weder gleichnamig noch gleichstufig sind, vertauscht.)
\section{(c) Lippenlaute}

π u. β: Kret. ἀβλοπές st. ἀβλαβές, auch Präs. βλάπω, vgl. § 21, 5, a; böot. πούλιμος b. Plut. Symp. 6. 8, 1 st. βούλιμος, von Plut. aus πολύλιμος abgeleitet. β u. π: Delph. nach Plutarch, Q. Gr. 2. p. 292, e βατεῖν u. βικρός st. πατεῖν, πικρός; Μηκύπερνα u. -βερνα att. Inschr.; Schwanken zwischen β u. π auch in den Verbindungen μπρ μβρ st. μρ, μπλ μβλ st. μλ, als Ἀμβρακία u. Ἀμπρακία (die Münzen d. St. überwiegend mit β, s. Dial.-I. 3185, ebenso Xen. u. A., doch mit π Herodot Thuk., att. Inschr. beides, Meisterhans 59\textsuperscript{2}); ἀμβλακεῖν u. ἀμπλακεῖν § 343, s. auch § 69, 1. φ u. π, π u. φ: Dor. (aber auch in anderen Dial.) in ἐφιορκέω durch Hauchverschiebung st. ἐπιὁρκέω, u. so auch in a. dor. Inschr. (Kreta) ἐπιορκήσαιμι ἐπιορκόντι (Cauer, Del.\textsuperscript{2} 116 f.), s. § 53, 4, C; auf lesb. Münzen Φίττακος st. Π.; σπόνδυλος, λίσπος, ἀσπάραγος, σπυράς, σπογγία gew., σφόνδυλος (Inschr.), λίσφος, ἀσφ., σφυράς, σφογγία att.; hier scheint das ς aspirierenden Einfluss gehabt zu haben, s. § 62, 1. Für τράπηξ att. τράφηξ, Meisterhans 60\textsuperscript{2}, f. πιθάκνη πιθάκνιον att. φιδάκνη φιδάκνιον, Lobeck, Phryn. 113. Moeris p. 393. Wecklein Cur. epigr. 42. Meisterhans 80\textsuperscript{2}; üb. φανός u. (früher) πανός b. d. Attikern s. Roscher, Curt. St. 1, 2, 72; lokrisch φρίν f. πρίν. — π für φ in μεσοπέρδην f. μεσοφέρδην Hes. φ u. β: Dor. κολυμφᾶν st. κολυμβᾶν. β u. φ: Ἄμβρυς (ς) ος st. Ἄμφρυσος (Ahrens II, 84 f.; Bull. de corr. hell. V, 431 u. Dial.-Inschr. 1520 mit β); maked. allgemein: Βερενίκη, Βίλιππος, βαλακρός, γαβαλά u. s. w. st. Φερ., Φίλ., φαλ., κεφαλή, s. Einl. S. 23 f. μ u. π: πεδά lesb. u. böot. st. μετά, das in beiden Dialekten ebenfalls vorkommt, doch jedenfalls nur aus der κοινή eingeschleppt ist; --155-- πεδά auch dor., als argiv. πεδαϝοίκοι (Dial.-I. 3265. 3269), πεδαφορᾶς Epidaur. 3325, 276 u. s. w., besonders kret., s. § 325, 6; auch πετά in Πεταγείτνυος Monat in Kos u. Kalymnos, Πεταγείτνιος in Kalchedon, rhod. Πεδαγείτνιος, Bull. de corr. hell. VIII, 42, Dittenberger, Syll. p. 534; Ahrens I, p. 152 glaubt mit Pott, Et. F. II, S. 515 (I\textsuperscript{2}, 517 f.), dass beide Formen von verschiedenen Wurzeln herkommen, und erklärt πεδά als verwandt mit πούς, so auch Osthoff u. A., § 325; ματεῖν (μάτεισαι Sapph. 54 = πατοῦσαι) wird als äol. st. πατεῖν von Grammatikern angegeben; so auch μάτης Theokr. 29, 15 ἐξ ἐτέρω δἔτερον (scil. κλάδον) μάτης (die Zweifel von Ahrens I, 45, Meister I, 125 nicht gerechtfertigt); lakon. Θεράμναι (= Θεράπναι) b. Steph. Byz.; ἄμακις, das Hesych. als kret., u. ἄματις, das er als tarent. für ἅπαξ erklärt, stellt Ahr. II, p. 85 mit der [root ] ἁμ (lat. sem, vgl. semel, simplex) zusammen, vgl. § 188 Anm. 2. μ u. β: Aeol. κυμερνήτης (richtig -άτας) st. κυβερνήτης, auch kypr. mit μ, Meister II, 254; aber βάρμιτος (= βάρβιτος) stellt Ahr. I, p. 45 m. βάρμος od. βάρωμος (dies Sapph. 154) zusammen; lakon. b. Hesych. ἀμάκιον = ἄβαξ, abacus. β u. μ: Epidauros βόλιμος rhod. βόλιβος = μόλυβδος, vgl. § 69, 1; βαρνάμενοι att. Epigramm Meisterhans 59\textsuperscript{2}, auch dor. Epigr. D.-I. 3175. 3189, vgl. sk. mrnâmi kämpfe (μαρνάμενοι in e. anderen att. Epigramm, 749 Kaibel)170); ferner att. Inschr. Σερμυλία und (seltener) -βυλία, Ἀδραμυτηνός und (seltener) -βυτηνός, Meisterhans 60\textsuperscript{2}; b. Hesych. βόρμαξ = μύρμηξ, Meister, D. II, 219; Antiphan. fr. 44 K. (II, 28) βύσταξ f. μύσταξ, u. a. m., Roscher, C. Stud. III, 129 ff. IV, 201; Angermann, Dissimilation (Lpz. 1873), S. 35. Die Verwandlung des μ in β in βλίττειν st. μλίττειν (vgl. μέλι), in βλάξ st. μλάξ (vgl. μαλακός), in βλώσκειν v. μολεῖν gehört nicht den Dialekten an, sondern ist in der Wohllautslehre zu erklären, s. § 58, 5. ππ u. μμ: Aeol., wenn μμ aus πμ (βμ, φμ) entstanden ist, als: ὄππατα v. [root ] ὀπ st. ὄμματα, ἄλιππα st. ἄλειμμα v. [root ] ἀλιφ. Anmerk. Für den Wechsel nicht verwandter Konsonanten in den Dialekten lassen sich keine Beispiele aufweisen; denn μόλις u. μόγις, κοῶ b. Epich. 19 u. νοέω (das Ep. gleichfalls gebraucht, sowie auch νόος), μινύρεσθαι u. κινύρεσθαι u. dgl. gehören verschiedenen Wurzeln an. Über den Gebrauch von μόγις u. μόλις ist Folgendes zu bemerken: μόγις wird von den Ioniern gebraucht, so bei Hom. (auch Il. χ, 412 hat d. Ven. μόγις) und stets b. Herod., μόλις bei den Attikern ungleich --156-- häufiger als μόγις, so bei Thukyd. (nur 7, 40. 8, 27. 34. 92 μόγις in allen Codd.), s. Poppo I, 1, p. 208, III, 1, p. 132; bei Sophokles stets μόλις, bei Aeschylus und Euripides auch μόγις;171) μόλις durchaus vorherrschend bei Xenophon,172) so auch bei Demosthenes173) u. Aristoteles; aber bei Aristoph. u. Platon ist μόγις vorherrschend.174)
\section{(d) Die Doppelkonsonanten}

δι u. ζ: Lesb. ζά (entst. aus δjά) st. διά, als: ζὰ νύκτος, ζαβάλλειν (Hesych.) st. διαβάλλειν, ζάβατος (Sapph.) st. διάβατος, ζαελεξάμαν (Sapph. 87) st. διελ., ζὰ τὰν σὰν ἰδέαν Theokr. 29, 6; inschr. (spät) ζά D.-I. 255, Ζόννυσος st. Διόνυσος 271 (die älteren Inschr. nur διά); ζηνεκές st. διην. Callimach.; so auch das ζα intensivum, entst. aus διά = durch u. durch, d. i. sehr, als: ζάδηλος Alc. 18, ep. poet. ζάθεος, ζάκοτος, ζατρεφής, ζάλευκος, ζαμενής, ζάπεδον, ζαπίμελος, ζαθερής, ζάπυρος, ζάπλουτος; als Inlaut in κάρζα äol. Et. M. 407, 18 st. καρδία; so auch kypr. κόρζα (überl. κορζία) Hesych., ζάει = διάει ders., s. Meister I, 127 ff., II, 253. Κάρζα hat des Metrums wegen Dindorf b. Aesch. Sept. 288, Suppl. 71. 799 geschrieben. Das ζ möchte im Aeolischen weiches s (ds) bedeutet haben, Blass, Ausspr.\textsuperscript{3} 118. σδ u. ζ: Andererseits lösen die Lesbier nach den Grammatikern ζ (d. i. σδ) in seine Bestandteile auf, so b. den Gramm. Σδεύς, σδυγός, μελίσδω, κωμάσδω, βρίσδα st. Ζεύς, ζυγός, μελίζω, κωμάζω, ῥίζα; παρίσδων, Alc. 52, ἀχνάσδημι 124, ὔσδων Sapph. 4, ὔσδῳ (ὄσδῳ) 93, φροντίσδην 41, ἐϊκάσδω 104, σδεύγλα Melinno, χθίσδον Balbilla; oft wird aber ζ beibehalten, als: Ζεύς, Ζεφύρῳ, κωμάζοντα Alc., ὑποζεύξαισα, ἰζάνει, μείζων Sapph., so auch auf allen Inschriften, die freilich in die ältesten Zeiten nicht zurückführen; nur auf der kymäischen Inschr. aus röm. Zeit D.-I. 311 steht προσονυμάσδεσθαι (archaisierend). Meister, Dial. I, 129 ff. Es möchte dies σδ nur graphisch von dem gew. ζ verschieden sein, indem im Altlesb. der Buchstabe Z in anderem Werte verwandt wurde (s. oben δι u. ζ), und die damit in Zusammenhang stehende getrennte Schreibung des gew. ζ sich in den Hdschr. fortpflanzte, wiewohl natürl. ohne Konsequenz. Blass, Ausspr. 118\textsuperscript{3}; vgl. auch unten σκ u. ξ). Auch bei dor. Dichtern, als Alkman und anderen Lyrikern (aber nicht b. Pindar), und besonders bei den Bukolikern finden sich Beispiele dieses Gebrauches (sogar b. Xenophan. Eleg. 1, 6 ὀσδόμενος), aber nie in dem reinen Dorismus, der mit Ausnahme der lakonischen, kretischen (und megarischen) Mundart überall ζ --157-- unverändert bewahrt. Hieraus schliesst Ahrens II, p. 95 mit Recht, dass der Gebrauch des σδ st. ζ nicht dor., sondern aus der lesbischen Mundart entlehnt sei. δ u. ζ: Böot., lakon., kret., eleisch δ statt anlautendem ζ (entst. aus δj), als Δεύς b. lak., b. Δάν = Ζεύς, b. u. l. δυγόν st. ζυγόν, b. Δῆθος st. Ζῆθος, l. δωμός = ζωμός; böot. Inschr. Δεύξιππος, Δωίλος, δώει = ζώῃ, ζῇ, δαμιώοντες ζημιοῦντες; kret. Inschr. Δῆνα (nb. Τῆνα, Ττῆνα, s. u.), Gortyn δώῃ = ζώῃ; eleisch ὐπαδύγιον. In anderen Dial. Δάνκλη, Münzen d. St. Zankle (Messene) u. Inschr. Röhl 518, vgl. Hesych. δάγκολον (δάγκλον Ahrens) δρέπανον; δορκάς, δόρξ und ζορκάς (Herodot nb. δορκάς) ζόρξ, b. Homer neben ζα δα- in δαφοινός st. σδαφ. ζ. (metr. Not), δάσκιος (euphon. st. σδάσκ. ζάσκ.), δάπεδον, Blass, Ausspr. 116\textsuperscript{3}. Vgl. auch μέδεα (Archil.) μέζεα (Hesiod) μήδεα (Homer), Hdn. II, 572. δδ u. ζ: Böot., lakon., kret. in der Mitte der Wörter (vgl. § 64), so b. Aristoph. Ach. 958 θερίδδω st. θερίζω, in der Lysistrata: γυμνάδδομαι st. γυμνάζομαι, μυσίδδω st. μυθίζω, ποτόδδει st. προσόζει, ψιάδδω st. ψιάζω, θυρσάδδω, παίδδω, παραμπυκίδδω; ausserdem finden sich mehrere (lakon.) Wörter mit δδ b. Hesychius, s. Ahrens II, p. 96 sq.; ὀπιδ (δ) όμενος lakon. Inschr., κομίδδεσθη, τράπεδδα böot. Inschr.; kret. Gortyn δικάδδω, Oaxos (Bull. de corr. hell. VI, 460) φροντίδδοντες, Gortyn auch πράδδω, σάδδω, als von γ-Stämmen (gleichs. πράζω, σάζω); Elis βραΐδδει = ῥαΐζει Hesych., Inschr. χραΐδ (δ) οι, δικάδ (δ) ωσα u. s w., Meister II, 53; so auch bei den Megareern nach Aristoph., als: μᾶδδαν Ar. Ach. 732 st. μᾶζαν, χρῄδδετε 734 st. χρῄζετε, während die megar. Inschr. gleich den korinth. und sonstigen dorischen ζ haben. In Kreta wird auch ς δ zwischen Auslaut u. Inlaut oft zu δδ assimiliert, als τᾶδδίκας, G. Meyer 262\textsuperscript{2}. ττ u. ζ: Statt φράζω soll Korinna φράττω gesagt haben, s. Ahrens I, p. 176; vgl. att. ἁρμόττω f. ἁρμόζω, σφάττω f. σφάζω, u. das kret. ἐσπρεμμίττεν u. s. w., oben unter ττ u. σς; ferner unten σς und ζ. σθ u. ζ: μασθός (dor. n. Heraklid.) Xen. An. 1. 4, 17 (v. l. μαστός), Aesch. Ch. 545 (-στ- Blomfield); Dial.-Inschr. 3246 (Akrai b. Syrakus). στ u. ζ: μαστός (ion. Hdn. I, 144) Aristotel., Pind., die Trag., u. μαζός Hom.; b. Herod. Hippokr. beides;175) μαζός auch Aesch. Ch. 531 (-στ- Blomf.), Eur. Ba. 700, Hec. 144 (hier v. l. -στ-). ζ u. σς: Statt σς gebrauchen die Lesbier in einigen Wörtern ζ, als: ἔπτᾶζον Alc. 27 st. ἔπτησσον, ἐπιπλζοντα Sapph. 17 statt ἐπιπλήσσοντα --158-- (ᾱ Herodian II, 929), aber σταλάσσον Sapph. 116. Es sind dies Schwankungen im Kennlaut des Verbalstammes, vgl. oben kret. πράδδω u. dgl. σς u. ζ: Statt ζ gebrauchten die Tarentiner in einigen Wörtern σς, als: σαλπίσσω, λακτίσσω, φράσσω; über die denselben Tarentinern beigelegte Verbalendung άζω f. άσσω, als ἀνάζω f. ἀνάσσω, s. Ahrens II, 101. — Thessal. ἐνεφανίσσοεν st. ἐνεφάνιζον. ζ u. γ: ὄλιζον st. ὀλίγον wird als thessal. od. äol. von den Grammatikern angeführt; Ahrens I, S. 219 hält es für eine Erfindung der Grammatiker, die den Namen der thessalischen Stadt Ὀλιζών davon ableiteten; doch hat es auch Lykophr. 627. δ (d. i. δδ) u. σς: Auf einer kret. Inschr. διαφυλάδων (gls. -ζων) st. διαφυλάσσων. ξ u. σς (ς): Die Dorier bilden von den Verben auf ζω (d. i. δjω) das Fut. u. den Aor. mit ξ st. ς, als: χωριξῶ, ἐχώριξα, ἐγδικαξαμένοι, mit Ausnahme von δανείζω (δανεῖσαι) u. σῴζω = σωΐζω, das beide Formen hat (σῷξαι u. σῷσαι); so auch Pindar, der indes auch ς u. zuw. σς gebraucht, Peter, dial. Pind. 59 f.; ξ auch in den von V. auf ζω abgeleiteten Abstraktis auf ξις st. σις, als: χείριξις, ἐγδίκαξις, und in dem ersten Teile der Komposita, als: Ἀρμοξίδαμος; auch arkad. παρετάξωνσι von παρετάζω; aber in anderen Verbalformen, sowie in Derivatis findet sich Gutturallaut st. ς nur selten und nur in der lakon. und sikel. (ital.) Mundart, daher bei Theokrit, nicht bei Pindar, als: ἅρμοκται = ἥρμοσται u. νενομίχθαι = νενομίσθαι Pythag., ἐκεκρατηρίχημες Sophr. 71 = ἐκεκρατηρίκειμεν v. κρατηρίζω, ἐλυγίχθης Theokr. 1, 98; μελικτάς Theokr. 4, 30 v. μελίζω; auf Inschriften kommt in diesen Fällen nur ς vor, als κατεδικάσθεν, νενόμισμαι, λογισταί, οἱ ἁρμοσταί, auch lakon. ἁρμοστήρ. Darnach kann man nicht füglich bei diesen Verben ein Schwanken zwischen gutturalem u. dentalem Kennlaut annehmen (Cauer, Sprachw. Abh. a. Curt. Gesellsch. 127 ff., G. Meyer 466\textsuperscript{2}), sondern wird glauben, dass aus altem τς (s. § 31 ττ u. σς) unter den Einflüssen der Analogien von Aoristen auf -αξα u. s. w. ξ geworden ist, statt wie sonst σς. (Δικασσέω Kalymna Bull. de corr. hell. Χ, 240 nb. ψαφίξηται, offenb. aus euphon. Gründen; so δικασσαίεν ἐδικάσσαν Argos D.-I. 3277; desgl. (γ) ἐργά (ς) σασθαι Epidaur. D.-I. 3325; von σχίζω nicht nur Pind. σχίς (ς) ε, sondern auch die epidaur. Inschr. D.-I. 3339, 99 ἀνσχίσσαντα.) Ein bes. Fall ist κλαιγω (d. i. κλᾴγω) schliesse (κλείω) auf den tab. Heracl., wo das Präs. den Guttural mit sämtlichen anderen Formen teilt: κλᾳκτοί, Andania, κλᾳξῶ ἀπόκλᾳξον Theokr., Subst. κλᾷξ, κλᾳκός. Morsbach, C. Stud. Χ, 18 ff.; G. Meyer 218\textsuperscript{2} Anm. — Die Sikuler sollen nach Heracl. b. Eust. 1654, 18 eine Neigung gehabt haben, die V. --159-- auf άω in V. auf άζω umzubilden, als: σιγάζω, ἀνιάζω, ἀτιμάζω, so ἀκροαζομέναEpich. 75 v. ἀκροᾶσθαι; daher b. Theokr.: ἐγέλαξε, γελάξας v. γελάω (Morsbach, C. Stud. Χ, 21 f.), χαλάξαι (χαλάξαις auch Pind.) v. χαλάω (aber auch ἔφθαξα (v. l. ἔφθασσα) Theokr. 2, 115 v. φθά-ν-ω);176) auch in der gew. Sprache erklärt sich hieraus ἐγέλας (ς) α, ἐχάλα (ς) σα. — Ferner steht ξ für σς (ττ) in neuion. διξός, τριξός, att. διττός, τριττός, gew. (auch ep.) δισσός, τρισσός. ς u. ξ: Böot., thessal., arkad. wird die Präp. ἐξ vor einem folgenden Konsonanten in ἐς, böot. auch vor einem Vokale in ἐσς verwandelt, als: B. ἐς Μωσάων, ἐσσάρχι st. ἐξάρχει, ἔσγονος = ἔκγονος; thess. ἐσγόνοις, ἐσδόμεν, aber ἐξεργασθεισεσθειν, ark. ἐσδοθέντων, ἔστεισιν, aber ἐξέστω; so auch auf einer böot. Inschr. sonst vulgären Dialektes ἐν ταῖς πέρις πόλεσιν st. πέριξ; ebenso ἕς für ἕξ (ἑσκηδεκάτη), vgl. § 72, Anm. 4. Die Präpos. ξύν wird im Attischen etwa um 410 von σύν abgelöst (Meisterhans 181\textsuperscript{2}), in anderen Dial. aber ist σύν schon althergebracht, s. § 325, 5. Im Ausl. ς für ξ italiot. βάννας (Hesych.) st. ἄναξ, dor. κόϊς st. κόιξ, Akk. κόϊν Epicharm. b. Poll. 10, 174; eleisch κύλλας (κύλλα Is. Voss, Meister II, 59) = σκύλαξ. σκ u. ξ: Statt ξίφος dor. σκίφος, sc σκιφύδρια, σκιφίας Epich. 23. 29, σκιφατόμος lakon. Inschr. 55b, v. 62 Müllensiefen; aber ξίφος Ar. Lys. 156. Die alten Grammatiker legen scheinbar diese Eigentümlichkeit auch den Aeoliern bei (aber Alc. 33 ξίφεος), ebenso σπ st. ψ, als σπέλιον, σπαλίς f. ψέλιον ψαλίς; in der That aber haben die betr. Angaben (s. Ahrens I, p. 49) nur so einen Sinn, wenn man κσίφος (überl. Bk. An. 815, 32), πσέλ (λ) ιον (überl. cod. Barocc. b. R. Schneider, Bodleiana p. 43) liest; denn es soll gezeigt werden, dass ξ ψ ζ Doppelkonsonanten sind, und dafür wird die äolische Schreibung (γράφουσι) angezogen. Die Aeolier also, wie sie vor alters nicht ζ schrieben sondern σδ (s. o.), so gleichzeitig κς für ξ u. πς für ψ. Vgl. die anderen, hierfür unzweideutigen Stellen Ahrens I, 48 (Meister I, 126), wo ἱέρακς, Πέλοπς, Ἄραπς als Bsp. erscheinen. ξ u. κτ, κ: Im Zusammenhang mit dem Wechsel zw. σκ und ξ steht der zwischen ξ und κτ, κ, auch σκ, s. darüber G. Meyer 247\textsuperscript{2} f., unten § 57, 1. Κτ ist ein speciell griechischer, in den verwandten Sprachen sich nicht findender Anlaut; das Sanskrit hat dafür k[sbreve]. Ebenso χθ, πτ, φθ, s. das. 255. 249 ff.; dem πτ φθ entspricht in den verwandten Sprachen im allgem. sp. Im Griech. vgl. man: ion. ξυνός (ξύν, σύν), gew. κοινός, rhod. κτοίνα Dittenberger, Syll. 305; --160-- κτανεῖν κανεῖν (Präs. καίνω) kret. σκενέν in κατασκένηι Gortyn (Blass, Rh. Mus. 1886, 313; Baunack, Stud. I, 4), κτείς u. ξαίνω (ξάνιον = κτένιον Hesych.). Im gortyn. Dial. kommt auch inlaut. χτ nicht vor (daf. ττ). ψ u. σπ, σφ, π, φ, πτ, φθ: σφίν σφέ, syrakus. ψίν ψέ, lakon. (Empedokl. 188 St.) φίν; πτύω, dor. ψύττω, ἐπιφθύσδω Theokr., ψυττόν πύελον Hesych., lat. spuo, d. speien (spiuwan althochd.); ψίσις ψείρει Hesych. st. φθίσις, φθείρει (Roscher, Curt. Stud. 1, 2, 125, der freilich dies anders erklärt, wie auch G. Meyer 212\textsuperscript{2}); im Lakon. (Dor.) ψίλον = πτίλον (Pausan. 3, 19, 6) vgl. ἄψιλον = ἄπτιλον Hesych. ([root ] πετ fliegen) scheint Übergang von πτ in ψ vorzuliegen, vgl. Roscher, Curt. Stud. II, 423 ff. Für den Wechsel zw. πτ und π vgl. πτόλις πτολίεθρον Hom., auch kypr. πτόλις, desgl. arkad. Pausan. 8, 12, 7, οἱ ττολίαρχοι (ττ aus πτ) thess. D.-I. 1330; πτόλεμος Hom. u. nach Gramm. kypr., s. auch § 57, 1; πτελέα Ulme u. πελέα Epidaur. D.-I. 3325, Z. 44 (s. Prellwitz z. St., Meister II, 260); πυκνός und πτύσσω Hdn. II, 233; anderes b. Meister a. a. O. ψ u. ς: Ψάπφοι Sapph. 59, Ψάπφ 1, 20 (aber Σάπφοι Alk. 55). ς u. ψ: Tarent. ἄσεκτος Rhinthon b. Hesych. st. ἄψεκτος.
\section{(e) Wechsel der Vokale und Konsonanten}

υ u. λ: Gewisse Kreter verwandelten λ vor einem Konsonanten in υ, so b. Hesych. αὐκάν = ἀλκάν, αὐκύονα = ἀλκύονα, αὕμα = ἅλμη, αὖσος = ἄλσος, θεύγεσθαι = θέλγεσθαι, αὐγεῖν = ἀλγεῖν, εὐθεῖν = ἐλθεῖν; vgl. im Franz. autre aus alt(e)rum, chaux aus calcem, beaux aus bellos. Auf kret. Inschr. hat sich noch nichts derart gefunden, ausser viell. auf den Gortyn. Taf. 5, 18 ἀδευφιαί (sonst das. stets mit λ). G. Meyer 179\textsuperscript{2} f. ι u. ς, ρ: Kret. πρεῖγυς = πρέσβυς, πρείγιστος, πρειγήϊα, πρειγευταί u. s. w. auf Inschr., s. § 28, b; kret. μαῖτυς st. μάρτυς; umgekehrt Ἀπόλλων Αἰγλήτης und Ἀσγελάτας (D.-I. 3430) auf Anaphe. γ u. ι: Böot., lesb. u. dor. ἀγρέω (thessal. ἁνγρ.) = αἱρέω, ἄγρει u. κατάγρει Sapph. 2, 14. 43; b. Hom. ἄγρει, fass an = auf denn, ἀγρεῖτε (Antimachos mit Umstellung ἀργεῖτε, Herodian II, 383), αὐτάγρετος = αὐθαίρετος, παλινάγρετος (Theokr. 29, 28), ζωγρεῖν; ferner ἄγρα, ἀγρεύειν177). ν u. α: in d. 3. P. Plur. ion. u. att., als: ἐστάλατο, τυπτοίατο, πεπτέαται u. s. w. st. ἔσταλντο u. s. w. S. § 214, 4—8, u. über andere hierher gehörige Erscheinungen § 68, 4. Anmerk. Über die Assimilation, Verdoppelung, Umstellung, Abwerfung und Einschiebung der Konsonanten in den Dialekten siehe die Wohllautslehre.
