% TeX root=../../main.tex

\chap{Einteilung der Sprachleute}

\section{Artikulation der Sprachlaute}

\paragraph{} Die Sprachlaute sind artikulierte Laute (ἔναρθροι, Ggstz.\ ἄναρθροι
unartikulierte, wie die der Tiere), d.~h.\ solche, welche durch die Einwirkung
der Sprachwerkzeuge eine bestimmte Gestalt erhalten. Unter Artikulation der
Laute versteht man daher die Bildung der Stimme durch die Sprachwerkzeuge zu
Lauten von bestimmter Gestalt. Sprachwerkzeuge sind ausser der Mundhöhle die
Kehle, die Zähne, die Zunge und die Lippen.

\paragraph{} Diejenigen Sprachlaute, welche bloss durch eine grössere oder
geringere Erweiterung oder Verengerung der Mundhöhle hervorgebracht werden und
am ungehindertsten durch den Mund gehen, heisst man Vokale (φωνήεντα sc.\
στοιχεῖα), die übrigen, welche unter stärkerer Einwirkung der Kehle, der Zähne,
der Zunge oder der Lippen gebildet werden, Konsonanten (σύμφωνα sc.\ στοιχεῖα).
Jene tönen für sich allein hell und voll, sie sind φωναί; diese sind für sich
höchstens Geräusche (ψόφοι), und haben an einer φωνή nur mit Hülfe eines Vokales
teil.


\section*{Vokale}
\addcontentsline{toc}{section}{Vokale}

\section{Einfach Vokale}

\paragraph{} Die Griechen hatten, wie wir
\pararef{sec:kurzgeschischte}{par:heta} gesehen haben, anfänglich nur fünf Vokalzeichen: Α, Ε, Ο, Ι, Υ, welche als kurz (βραχέα) und als lang (μακρά) gebraucht wurden. Nachher kamen für das offene (lange) Ε das Zeichen Η und für das offene (lange) Ο das Zeichen Ω hinzu, und noch später wurden Ε und Ο auf die Geltung kurzer Vokale beschränkt, während Α, Ι und Υ nach wie vor als kurz und als lang gebraucht und daher δίχρονα oder ἀμφίβολα genannt wurden.

\paragraph{} Das Verhältnis der Vokale zu einander wird am besten durch die
bekannte Vokalpyramide dargestellt, an deren Spitze \ipa{a}, und an deren beiden
unteren Ecken \ipa{i} und \ipa{u} stehen, während die verschiedenen \ipa{e} und
\ipa{o} auf der Linie zwischen \ipa{a} und \ipa{i} bezw.\ \ipa{a} und \ipa{u}
Platz finden, \ipa{ü} aber zwischen \ipa{i} und \ipa{u}.

\begin{center}
	\begin{tikzpicture}[scale=1]
		% Coordinates
		\coordinate (A) at (0.00, 0.00);
		\coordinate (B) at (7.00, 0.00);
		\coordinate (C) at (3.50, 3.25);
		\coordinate (D) at (3.50, 0.00);
		\coordinate (E) at (1.25, 0.9);
		\coordinate (F) at (6.0, 0.9);
		\coordinate (G) at (0.75, 2.2);
		\coordinate (H) at (4.50, 2.2);

		% Drawing
		\draw (A) -- (B);
		\draw (A) -- (C);
		\draw (C) -- (B);
		\node[above left, black] at (A) {i (ι)};
		\node[above right, black] at (B) {u};
		\node[above, black] at (C) {a (α)};
		\node[above, black] at (D) {ü (υ)};
		\node[above, black] at (E) {e geschlossen (ε)};
		\node[above right, black] at (F) {o geschl. (ο)};
		\node[above right, black] at (G) {e offen (η)};
		\node[above right, black] at (H) {o offen (ω)};
	\end{tikzpicture}
\end{center}

A, i, u stellen sich im Griechischen wie im Sanskrit und in den semitischen
Sprachen deutlich als die drei Grundvokale dar, und zwar gehören die E- und
O-Laute im Griechischen zum Bereiche des \ipa{a}, nicht zu dem des \ipa{i} und
\ipa{u}.

\paragraph{} Der dritte Grundlaut ist im Griechischen kein reiner, sondern aus
dem U-Laute durch Annäherung an ι getrübter; aber ohne Zweifel hat er
ursprünglich den reinen Laut u, wie im Lateinischen und Deutschen, gehabt, und
dieser Laut ist insbesondere für Homer noch anzunehmen, bei welchem εὖ als εὖ
und ἐΰ (\ipa{eu} und \ipa{e-u}) erscheint, αὔω im Aorist ἤϋσα bildet (\ipa{auo} — \ipa{ē-ūsa}). Auch
haben namentlich die Böotier diesen ursprünglichen Laut treu bewahrt, indem sie
ihr υ wie u, und zwar als kurzes und langes u, aussprachen; also σύν, τύχα,
κᾶρυξ, Πῡ́θιος, ὗς lautete bei ihnen wie \ipa{sun}, \ipa{tucha}, \ipa{karux},
\ipa{Pūt'ios},
\ipa{hūs}.\footnote{S.\ Ahren, \emph{Dial.} I, 196sq.\ u.\ p.\ 180sq.; Meiter.
	\emph{Gr. Dial.} I, S.\ 231ff. Vgl.\ Dietrich in Kuhns Zeitschr. 1865, S.\ 64.}
Nachdem aber im \Langlongdecl{att}{en} und \Langlongdecl{ion}{en}
(\Langlongdecl{dor}{en}) sich die Bezeichnung ου für
einen dem langen \ipa{u} wenigstens nahe verwandten Laut gebildet hatte; nahmen auch
die Böotier im 4.\ Jahrh.~v.~Chr.\ dieses ου an und gebrauchten es nicht nur für
das lange, sondern auch für das kurze \ipa{u}, als: κούνες st.\ κύνες, οὔδωρ
st.\ ὕδωρ,
σούν st.\ σύν, κοῦμα st.\ κῦμα, welche Schreibung auch in die Gedichte der
Korinna eingeführt wurde, daher in deren Fragmenten: τού, οὐμές, οὐμίων,
πουκτεύι, ῶνούμηνεν (= ὠνύμαινεν), γλουκού, λιγουράν u.~a.
Jedoch schwankt auf den
\langlongdecl{böot}{en} Inschriften die Schreibung zwischen ου und υ, während andererseits
die Böotier in späterer Zeit das lange υ (= \ipa{ȳ}) häufig für οι (ῳ)
verwendeten, als: τῦς ἄλλυς st.\ τοῖς ἄλλοις, ἵππυς st.\ ἵπποις, προβάτυς st.\
προβάτοις; τῦ δάμυ st.\ τῷ δάμῳ.\footnote{S.\ Ahrens l.\ d.\ p.\ 191sqq.;
	Meister, S.\ 236.}
Eine dem \ipa{ü} ähnliche Trübung stellte sich mit der Zeit auch bei ihnen ein, zu ü
sich verhaltend wie das \langlongdecl{engl}{e} \ipa{ū} (~\ipa{i͜u}~) zum
\langlongdecl{franz}{en} \ipa{u}, dem es entspricht
(\emph{duc} \glsxtrshort{engl} \emph{duke}); die Böoter schreiben ιου, was sich besonders nach Dentalen und
nach λ findet: Πολιούστρατος, τιούχα, Διωνιούσιος.\footnote{Meister, S.\ 233f.
	(Ahrens Add.\ II, 519).}
Unter den \langlongdecl{dor}{en} Stämmen sind die Lakonier die Einzigen, in deren Glossen das
ου sowohl für ῡ als fur υ vorkommt.
So findet sich bei \authorlong{Hesychius} z.~B.\ διφοῦρα =
γέφυρα, κάρουα = κάρυα, μουσίδδει = μυθίζει, τούνη = τύνη (σύ). Auf den sehr
späten lakonischen Inschriften 1347 und 1388 findet sich ο st.\ υ in Κονοουρεῖς
st.\ Κυνοσουρεῖς;\footnote{S.\ Ahrens, II, p.\ 124--126.}
sonst geben die Inschriften nur υ wie gewöhnlich, und es scheint daher das \ipa{u} für υ auf die vulgäre Sprache Lakoniens beschränkt gewesen zu sein.

\paragraph{} Hinsichtlich der Kürze und Länge der Vokale ist zu bemerken, dass
weder die kurzen noch die langen von den alten Grammatikern alle als gleich kurz
oder lang angesehen wurden. Dass das ε der kürzeste Vokal sei, schloss man aus
der sogenannten \langlongdecl{att}{en} Deklination, in der es auf den Accent nicht einwirkt,
indem die Stimme über dasselbe leicht hingleitet, als: Μενέλεως, ἵλεῳ, πόλεως,
selbst χρυσόκερως, φιλόγελως. Dass es insbesondere kürzer sei als ο, entnahm man
aus dem Vokative, der die kurzen Vokale liebt, als: λόγος λόγε;\footnote{S.
	\authorlong{Herodian} in Bekk.\ Anecd. II, p.\ 798sq.
	\authorlongdecl{Herodian}{s} Vater \glslink{ApolloniusDyscolus}{Apollonius}
	behauptete dagegen, ο sei kürzer als ε. S.\ \glsxtrshort{Theodosius}, p. 33sq.}
dass aber ω kürzer sei als η, daraus, dass man Μενέλεων, πόλεων u.~s.~w.\
proparoxytonisch betont, was nie der Fall ist, wenn η in der letzten Silbe
steht.\footnote{Bekk.\ Anecd. II, p.\ 979.}


\section{Diphthonge}

% TODO: Acertar referência cruzada
\paragraph{} Sämtliche Diphthonge (αἱ δίφθογγοι scil. συλλαβαί),\footnote{%
	Das Wort ἡ δίφθογγος zeigt schon durch sei Genus an, dass es eig.\ Adjektiv
	und dass ein weibliches Substantiv zu ergänzen sei; nun werden aber die
	Diphthonge sowohl von Griechen (τὴν ου συλλαβήν \glsxtrshort{DionysHalic} oben
	\pararef{sec:buchstabenausprache}{sec:ausprachekonsonanten} p.~\pageref{pvau}) als von Lateinern (\emph{ae syllaba}~\passagem{QuintilianusI718})
	öfters συλλαβαί \emph{syllbae} gennant und es it daher dieses Wort als
	ursprünglich zu ergänzen anzunehmen.
	Vgl.\ \glsxtrshort{Theodosius} p.\ 34: ἡ συλλαβὴ ἡ ἐκ φωνηέντων δύο
	συνεστηκυῖα δίφθογγος καλεῖται, was dann damit gerechtfertigt wird, das im
	eig.\ Sinne (κυρίως) die Bezeichnung φθόγγος nur den Vokalen zukomme.
} mit
Ausnahme von υι, sind aus der Verschmelzung eines der Vokale α, ε, η, ο, ω mit ι
oder υ (im Werte von u) zu einem Mischlaute entstanden, als:

\begin{center}
	\begin{tabular}[c]{ll}
		α + ι = αι, als: αἴξ    & α + υ = αυ, als: παύω                \\
		ε + ι = ει, als: δεινός & ε + υ = ευ, als: ῥεῦμα               \\
		ο + ι = οι, als: κοινός & ο + υ = ου, als: βοῦς                \\
		ᾱ + ι = ᾳ, als: δᾴς     & η + υ = ηυ, als: ηὖξον (im Augmente) \\
		η + ι = ῃ, als: λῃστής  &                                      \\
		ω + ι = ῳ, als: ᾠδή     &
		ω + υ = ωυ, als: ἑωυτοῦ                                        \\
	\end{tabular}
\end{center}


\noindent Der Diphthong ωυ findet sich im \Langlongdecl{att}{en} nur in der Krasis, und auch
da selten (ωὐριπίδη ὦ Εὐριπίδη \passagem{Thesm4}, πρωυδᾶν προαυδᾶν
\passageme{Aves556}); auch im \Langlongdecl{ion}{en}, wo er mehr hervortritt, ist in den
sichern Fällen Krasis der Entstehungsgrund (ἑωυτοῦ aus ἕο αὐτοῦ), und ebenso im
\Langlongdecl{dor}{en} (ωὑτός \passagem{Theokr1134}, s. Ahrens II, 222).

\paragraph{} Ist der erste Vokal ein langes α oder ein η oder ein ω, so wurde
das in älterer Zeit daneben gesetzte (προσγραφόμενον, \emph{iota adscriptum}) ι in der
Minuskelschrift seit dem 12.\ Jahrh.\ unter den langen Vokal gesetzt (iota
subscriptum, ἔχει τὸ ι ὑποκάτω γραφόμενον \glsxtrshort{Theodosius}
108).\footnote{%
	Eine den Übergang von ι adscriptum zum ι subscriptum anzeigende Schreibweide is
	die, wo der Buchstabe zwar seitwärts, aber entweder höher oder tiefer als die
	Zeile gesetzt wird, als α\textsuperscript{ι}, α\textsubscript{ι}. S.\
	Gardthausen, \emph{Gr.\ Paleogr.}, S.\ 193, 203.
}
Bei der Unzialschrift jedoch wird das ι immer noch neben den ersten Vokal gesetzt; ΑΙ, ΗΙ, ΩΙ, Αι, Ηι, Ωι, als: ΤΗΙ ΧΩΡΑΙ, ΤΩΙ ΚΑΛΩΙ.

\paragraph{} In dem Diphthongen υι vereinigen sich υ (ursprünglich und
dialektisch \ipa{u}, gew.\ \ipa{ü}) und ι zu einer Silbe, doch geschieht dies in der
gewöhnlichen Sprache nur vor Vokalen, als: μυῖα, ἅρπυια. Vor Konsonanten kommt
υι auch in Dialekten fast gar nicht vor, eher am Ende, wie in den Dativen ἰξυῖ
(\glsxtrshort{Homerus}), Δέρμυι (\glsxtrshort{böot} Inschr., \emph{Dial.- Inschr.} 875).

	{\small\noindent\subparagraph{} Da die Vokale α, ε, η, ο, ω bei den
		Diphthongen dem ι und υ vorangehen, so werden sie προτακτικά, ι und υ hingegen
		ὑποτακτικά genannt; in dem Diphthonge υι ist jedoch υ προτακτικόν. S.\
		\glsxtrshort{DionysThrax} in Bekk.\ Anecd.\ Il, p.\ 631, Schol.\ ad
		\glsxtrshort{DionysThrax} ib.\ II, p.\ 801, \glsxtrshort{Theodosius} Canon.\ ib.\ III, p.\ 1187, wo der merkwürdige Schluss gemacht wird: εἰ ἄρα οὖν τὸ ι καὶ τοῦ ὑποτακτικοῦ
		ὑποτακτικόν ἐστι, δῆλον, ὅτι ἀσθενέστερόν ἐστι πάντων τῶν φωνηέντων. -- Dass ᾳ,
		ῃ, ῳ ursprünglich Diphthonge waren, später aber zu Einzellauten herabsanken,
		haben wir~\ref{sec:buchstabenausprache} gesehen. Über die zwiefache
		Entstehung von ου s.\ oben~\pararef{sec:kurzgeschischte}{par:heta}; das.\ über die entsprechend zwiefache von ει.}

	{\small\noindent\subparagraph{} Inschriften und Handschriften (insonderheit
		die Volumina Herculanensia) aus der römischen Zeit verwenden, wie wir oben
		sahen (\pararef{sec:buchstabenausprache}{par:betreffdesditp}) das ει als Bezeichnung jedes langen ι: πολείτης, μεισεῖν,
		μειμεῖσθαι. Dass gelegentlich ein ει für ι aus Unkunde oder Versehen mit
		unterläuft, kann den Nutzen nicht hindern, den wir aus dieser Schreibung für die
		Erkenntnis der Quantität ziehen; denn wo sie häufig und stehend wiederkehrt, wie
		in πείπτω st. πίπτω, ἔτρειψα st. ἔτριψα, ist der Schluss auf Länge des ι
		berechtigt und sicher.\footnote{%
			Vgl.\ Dittenberger in \emph{Hermes} I, S.\ 415; A.~von Bamberg,
			\emph{Zeitschr.\ f.\ Gymnasialwesen} 1874, S.\ 13ff.
		}}

	{\small\noindent\subparagraph{} Unter allen Diphthongen müssen οι und αι für
		die kürzesten gelten, da sie rein, d.~h.\ ohne antretenden Konsonanten
		auslautend, in Beziehung auf die Betonung in der Flexion (mit Ausnahme des
		Optativs) und in den Adverbien πρόπαλαι und ἔκπαλαι als kurz behandelt werden,
		als: τράπεζαι, γλῶσσαι, τύπτεται, ἄνθρωποι, οἶκοι (die Häuser, zu unterscheiden
		von dem Adverb οἴκοι, zu Hause, \emph{domi}). Sodann sind αι und οι die einzigen Diphthonge, welche in der Dichtersprache elisionsfähig sind.}

	{\small\noindent\subparagraph{} In den Diphthongen αυ und υι kann, a priori
		betrachtet, der erste Vokal entweder kurz oder lang sein, und man kann somit,
		einschliesslich des ᾶυ und des ῦι, zu der Zahl von 14 Diphthongen
		gelangen.\protect\footnote{%
			Die Theorie der 14 Diphthonge entwickelt G.~Hermann, \emph{de emend.\
				rat.\ graeace gramm.}, p.\ 48sqq.
		} Nachweisbar ist indes weder ᾶυ noch ῦι; im Gegenteil finden wir im
		\Langlongdecl{att}{en}
		ναῦς für das \langlongdecl{ion}{e} νηῦς mit offenbar kurzem α; denn das lange hätte zu η
		werden müssen. Erscheint aber hier für ᾶυ αυ, so wird auch im attischen γραῦς,
		wo ρ ein ᾱ schützen würde, vielmehr α gesprochen worden sein. Ganz unklar bleibt
		die Quantität in dem \langlongdecl{dor}{en} αὖξον, \glsxtrshort{att} ηὖξον.}
\bigskip

\paragraph{} Die alten Grammatiker (\authorlong{Choeroboskus} in Bekkeri Anecd.
III.\ p.\ 1214 sq., \glslink{Theodosius}{Theodosius} p.\ 34 sq.\ ed.\ Göttl., die
Scholien ad \glsxtrshort{DionysThrax} in Bekk.\ An.\ II.\ p.\ 804,
\authorlong{Moschopulos} p.\ 24 sq.\ ed.\ Titze), die aber alle aus einer Quelle
geschöpft zu haben scheinen, teilen die Diphthonge in folgende Klassen ein:
\begin{compactenum}[(a)]
	\item δίφθογγοι κατʼ ἐπικράτειαν, d.~h.\ solche, in welchen der eine Vokal ein
	solches Übergewicht über den anderen hat, dass er allein gehört wird, der
	andere ἀνεκφώνητον ist, nämlich ᾳ, ῃ, ῳ, als: Μηδείᾳ, Ἑλένῃ, καλῷ. So lehrt
	\authorlong{Choeroboskus}; die anderen Grammatiker fügen noch ει hinzu, als:
	Νεῖλος. Es ist dies gemäss der Aussprache in römischer Zeit, wo das ι in ᾳ, ῃ,
	ῳ verstummt, das ει zu \ipa{ī} geworden war.
	\item δίφθογγοι κατὰ κρᾶσιν, d.~h.\ solche, in welchen die beiden Vokale zu einem
	Mischlaute verschmelzen und Einen Laut bilden, der zu beiden Vokalen stimmt
	(ἁρμόζει), nämlich: αυ, ευ, ου, als: αὐλός, εὔχομαι, οὗτος.
	\item δίφθογγοι
	κατὰ διέξοδον, d.~h.\ solche, in welchen der Laut beider Vokale getrennt
	(χωρίς) gehört wird, nämlich: ηυ, ωυ, υι, als: νηυσίν, ἑωυτοῦ, υἱός.
	\item Die Diphthonge αι und οι werden als besondere, zu keiner der angegebenen
	Klassen gehörige angeführt.
	\authorlong{Choeroboskus}, mit dem die Anderen
	übereinstimmen, sagt: ἐπειδὴ οὖν ἡ αι δίφθογγος ἡ ἐκφωνουῦσα τὸ ι καὶ ἡ οι
	δίφθογγος οὔτε κατʼ ἐπικράτειάν εἰσιν οὔτε κατὰ διέξοδον οὔτε κατὰ κρᾶσιν,
	ὥσπερ ἐστερήθησαν τοῦ ἰδιώματος τῶν διφθόγγων, ἐστερήθησαν καὶ τοῦ χρόνου
	τοῦ παρεπομένου ταῖς διφθόγγοις, καὶ τούτου χάριν αὗται μόναι ἐκ τῶν
	διφθόγγων τῷ τονικῷ παραγγέλματι ἀντὶ κοινῆς παραλαμβάνονται καὶ πρὸς ἕνα
	ἥμισυν χρόνον ἔχουσιν. Der Grund, weshalb die Grammatiker die Diphthonge αι
	und οι nicht zu den διφθόγγοις κατὰ κρᾶσιν gerechnet und ihnen sogar die
	Eigentümlichkeit der Diphthonge abgesprochen haben, scheint kein anderer zu
	sein, als weil dieselben in Beziehung auf die Betonung als kurz angesehen
	werden.
\end{compactenum}

\setcounter{subparagraph}{4}
{\small
	\noindent\subparagraph{} Nach \glslink{Theodosius}{Theodosius (Gramm.\ p.\
		35)} werden die Diphthonge eingeteilt (a) in eigentliche (κύριαι): αι, αυ, ει,
	ευ, οι, ου, und in uneigentliche (καταχρηστικαί): ᾳ, ῃ, ῳ, υι, ηυ, ωυ,
	wahrscheinlich, weil bei diesen nicht beide Laute zu einem Mischlaute
	verschmelzen, sondern entweder (ᾳ, ῃ, ῳ) nur der eine, oder (υι, ηυ, ωυ) beide
	in einer Silbe gehört werden.
	Diese Einteilung kann älteren Ursprungs sein, da ει in der Reihe der
	eigentlichen erscheint.
	In den Scholien ad \glsxtrshort{DionysThrax} (Bekk.\ Anecd.\ II, p.\ 803) werden
	αι, αυ, ει, ευ, οι, ου εὔφωνοι, ηυ, ωυ, υι κακόφωνοι und ᾳ, ῃ, ῳ ἄφωνοι
	genannt.
	Eine andere Dreiteilung, der im Text gegebenen ziemlich entsprechend, findet
	sich bei dem Musiker \authorlong{AristidesQuintilianus} (p.~29 Jahn, 44 Meibom):
	αἱ δίφθ., ἃς ἤτοι κατὰ κρᾶσιν ἢ κατὰ συμπλοκὴν ἢ κατʼ ἐπικράτειαν γίγνεσθαί
	φαμεν.
	Es wird indes nicht ganz klar, in welcher Weise die Diphthonge sich in diese
	drei Klassen verteilen.
	Zu vermuten steht, dass in der ursprünglichen Theorie der Musiker, welche sich
	von Alters her mit der Lehre von den Sprachlauten beschäftigten
	(\passagem{Cratylus424c}), nur δίφθ.\ κατὰ κρ.\ u.\ συμπλοκήν unterschieden
	wurden, indem die ἐπικράτεια bei ᾳ u.~s.~w.\ erst viel später eintrat, ja auch
	nachmals von den Musikern geleugnet wurde (s.\ oben
	\pararef{sec:buchstabenausprache}{par:diphthongeaihiwi} N.~\ref{foot:3131}).
	Beim eigentlichen Diphthonge lautet die Stimme während der Bewegung aus einer
	Vokalstellung in die andere und nur während dieser Bewegung, so dass eine
	wirkliche Mischung (κρᾶσις) ist wie zwischen Wasser und Wein; bei uneigentlichen
	Diphthongen dagegen bestehen die Laute neben einander, wie in einer Verflechtung
	(συμπλοκή).
	S. Rumpelt, \emph{das natürliche System der Sprachlaute}, S.\ 47.
}
