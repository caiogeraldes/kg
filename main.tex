\documentclass{memoir}
% Fontes, línguas, unicode
\RequirePackage{csquotes}
\RequirePackage{fontspec}
\RequirePackage{emoji}
\RequirePackage{mathpazo}
\RequirePackage{hyphenat}
\RequirePackage[main=german, bidi=basic]{babel}
\newfontfamily\cartafont{EBGaramond08-Regular.ttf}[
	Extension=.otf,
	ItalicFont=EBGaramond08-Italic,
	BoldFont=EBGaramondSC08-Regular,
]
\newfontfamily\fleurfont{EBGaramond12-AllSC.otf}[
	Extension=.otf,
	ItalicFont=EBGaramond08-Italic,
	BoldFont=EBGaramondSC08-Regular,
]
\newcommand{\fleurl}{{\fleurfont{}}}
\newcommand{\fleurr}{{\fleurfont{}}}
\defaultfontfeatures{Renderer=Harfbuzz}

\babelfont[german]{rm}[Script=Latin]{Gentium Book Plus}
\babelfont[german]{sf}{Noto Sans}
\babelfont[german]{tt}[Scale=0.8]{Mononoki Nerd Font}
\babeltags{de=german}
%
\babelfont[english]{rm}[Script=Latin]{Gentium Book Plus}
\babelfont[english]{sf}{Noto Sans}
\babelfont[english]{tt}[Scale=0.8]{Mononoki Nerd Font}

\babelprovide[import, onchar=ids fonts letters]{ancientgreek}
\babelfont[ancientgreek]{rm}[BoldFont=Gentium Book Plus Bold]{Brill}
\babeltags{grc = ancientgreek}

\babelprovide[import, onchar=ids fonts letters]{russian}
\babelfont[russian]{rm}{Brill}

\babelprovide[import]{hebrew}
\babelfont[hebrew]{rm}[Scale=0.8]{Ezra SIL}

\babelprovide[]{ipa}
\babelfont[ipa]{rm}{Gentium Plus}
\babeltags{ipa = ipa}

\babelprovide[]{syntax}
\babelfont[syntax]{rm}[Scale=0.85]{Noto Sans}
\babeltags{syntax = syntax}

\babelprovide{hittite}
\babelfont[hittite]{rm}{UllikummiA}

\babelprovide[import,onchar=ids fonts]{sanskrit}
\babelfont[sanskrit]{rm}[Scale=1]{AdishilaLetterpressHeavy}
\babelfont[sanskrit]{sf}[Scale=1]{AdishilaLetterpressHeavy}

\babelprovide[import, onchar=ids fonts letters]{armenian}
\babelfont[armenian]{rm}[Scale=0.8]{Noto Serif Armenian}

\babelprovide[]{linearb}
\babelfont[linearb]{rm}[]{Noto Sans Linear B}

\RequirePackage{microtype}
\usepackage{longtable}
\usepackage{lipsum}

\let\printindex\undefined
\usepackage{bibliocaiera}
\RequirePackage{xcolor}
\definecolor{green}{RGB}{16, 87, 87} % rgb(16,87,87)

\RequirePackage[pdfusetitle]{hyperref}
\hypersetup{%
	colorlinks=true, % false: boxed links; true: colored links
	linkcolor=green,  % color of internal links
	citecolor=green,  % color of links to bibliography
	filecolor=green,  % color of file links
	urlcolor=green,
}
\renewcommand{\passagem}[1]{\ifglshasparent{\glsentryparent{idx.#1}}{\glsxtrusefield{\glsentryparent{\glsentryparent{idx.#1}}}{useri}, }{}\glsxtrusefield{\glsentryparent{idx.#1}}{useri}, \gls{idx.#1}\ifglshasfield{useri}{idx.#1}{ =~\hyperref[corp:#1]{C\glsentryuseri{idx.#1}}}{}}
\newcommand{\passagemb}[1]{\ifglshasparent{\glsentryparent{idx.#1}}{\glsxtrusefield{\glsentryparent{\glsentryparent{idx.#1}}}{userii},
  }{}\glsxtrusefield{\glsentryparent{idx.#1}}{userii}, \gls{idx.#1}\ifglshasfield{useri}{idx.#1}{ =~\hyperref[corp:#1]{C\glsentryuseri{idx.#1}}}{}}
\newcommand{\authwork}[1]{\glsxtrusefield{\glsentryparent{#1}}{useri}, \glsxtrshort{#1}}
\newcommand{\passagemc}[1]{{\glsxtrusefield{\glsentryparent{\glsentryparent{idx.#1}}}{userii}
(\glsxtrusefield{\glsentryparent{idx.#1}}{userii}, \gls{idx.#1})}}

\begin{document}

\glssetwidest[0]{DyonisThraxkkkk}
\printunsrtglossary[
	type=authwork,
	title={Abkürzung},
	nonumberlist=true,
	style=mcoltree
]

\part{Elementarlehre. Erster Abschnitt: Laut- und Buchstabenlehre.}

Die Grammatik löst in ihrem ersten Teile das aus der zusammenhängenden Rede herausgehobene Wort in seine Bestandteile oder Elemente auf und schreitet nach Erörterung derselben zur Betrachtung des Wortes selbst fort.

\chapter{Von den Sprachlauten und den Buchstaben.}
% TeX root=../main.tex


\part{Laut- und Buchstabenlehre}
Die Grammatik löst in ihrem ersten Teile das aus der zusammenhängenden Rede herausgehobene Wort in seine Bestandteile oder Elemente auf und schreitet nach Erörterung derselben zur Betrachtung des Wortes selbst fort.

\chapter{Von den Sprachlauten und den Buchstaben}

\section{Alphabet (ἡ γραμματική oder τὰ γράμματα, τὰ στοιχεῖα)}\label{1}\par

Die griechische Sprache hat zur Bezeichnung ihrer Laute 24 Buchstaben [στοιχεῖα als Lautzeichen, γράμματα als Schriftzeichen\footnote{
			Bekker, Anecd. II, p.\ 774: στοιχεῖον μέν ἐστιν ἡ ἐκφώνησις, γράμματα δὲ αἱ
			εἰκόνες καἰ οἱ χαρακτῆρες.
			Das Wort στοιχεῖα erklärt \passagem{DionysThrax76} p.\ 630: διὰ τὸ ἔχειν
			στοῖχον τινα καὶ τάξιν.
			Wohl richtiger werden die Buchstaben στοιχεῖα genannt, als die Elemente,
			Grundbestandteile der Sprache; vergl. Anecd.\ p.\ 790;
			\authwork{DeCompVerb} p.\ 71 R.\ (ὅτι πᾶσα
			φωνὴ τὴν γένεσιν ἐκ τούτων λαμβάνει πρώτων κτἑ.).
		}], nämlich 7 Vokale und 17 Konsonanten:

	\begin{longtable}{llp{5cm}p{2.5cm}l}
		\toprule
		\multicolumn{2}{p{1.5cm}}{Gestalt (σχῆμα)}                                                                           & Aussprache (ἐκφώνησις)                             &
		\multicolumn{2}{l}{Name (ὄνομα)\footnote{Über die Benennungen der
				griechischen Buchstaben s.\ K.~E.~A.~Schmidt in Ztchr.~f.~d.~Gymnasialwesen
				v.\ Mützell 1851, Juni, S.\ 471--440; ders., Beiträge z.~Gesch.\
        d.~Grammatik\ des
		Gr.~u.~Lat., Halle 1859, S.\ 48ff.}}                                                                                                                                                                         \\
		\midrule
		\endfirsthead%

		\toprule
		\multicolumn{2}{p{1.5cm}}{Gestalt (σχῆμα)}                                                                           & Aussprache (ἐκφώνησις)                             &
		\multicolumn{2}{p{2cm}}{Name (ὄνομα)}                                                                                                                                                                        \\
		\midrule
		\endhead%

		\bottomrule
		\endfoot

		\bottomrule
		\endlastfoot%


		Α                                                                                                                    & α                                                  & a kurz oder lang & Ἄλφα  & Alpha \\
		Β                                                                                                                    & β                                                  & b
		                                                                                                                     & Βῆτα                                               &
		Bēta                                                                                                                                                                                                         \\
		Γ                                                                                                                    & γ                                                  & g
		                                                                                                                     &
                                                                                                                         Γάμμα\footnote{\label{foot:gemma}Bei
			\glsxtrlong{Demokrit}
		γέμμα, s. \glsxtrshort{Eustathius} z. Il. Γ Afg.}                                                                    & Gamma                                                                                 \\
		Δ                                                                                                                    & δ                                                  & d                & Δέλτα & Delta \\
		Ε                                                                                                                    & ε                                                  &
		e kurz und geschlossen                                                                                               & Εἶ,
		sp. ἔ, missbr. ἒ~ψιλόν\footnote{\label{not:psilon}Die Alten
			nannten das ε εἶ, das ο οὖ, das lange offene o ὦ und das Ypsilon ὖ.
			S.~\passagem{Cratylus393d} u.a. Das ε und ο nannten dann spätere Grammatiker ἔ
		und ὄ. Über den Zusatz ψιλόν s. \ref{1.anmerk2}.}                                                                    & Ei (Ĕ) [Epsīlon]                                                                      \\
		Ζ                                                                                                                    & ζ                                                  & sd, genauer zd
		(m.\ \glsxtrshort{franz} z), später frz.\ z = weichem s                                                                          & Ζῆτα                                               & Zēta                             \\
		Η                                                                                                                    & η
		                                                                                                                     & e lang und offen                                   &
		Ἦτα
                                                                                                                                                                                                                                                               &
                                                                                                                                                                                                                                                               Eta\footnote{\label{foot:heta}Eine merkwürdige Notiz findet
			sich bei \glsxtrshort{Theodosius} p.\ 7 extr.: τὸ Ἦτα δέ, τὸ ὄνομα τοῦ στοιχεῖον, δασύνεται
			(also ἧτα), ὅτι παρὰ ἀρχαίοις ὁ τύπος τοῦ Η έν τύπῳ δασείας ἔκειτο, ὥσπερ καὶ
			νῦν τοῖς παλαιοῖς Ῥωμαίοις (nämlich H). Unzweifelhaft haben diejenigen Griechen,
			welche das Zeichen Η in ihren lokalen Alphabeten für den Hauch gebrauchten (§
			2,2), und so auch die Attiker bis zur Reform der Ortographie, Heta gesagt; die
			Ioner indes, die den Hauch nicht hatten und das Zeichen für den Vokal anwandten,
			sagten natürlich Eta, und diese Benennung muss mit dieser Geltung massgebend
		sein.}                                                                                                                                                                                                       \\
		Θ                                                                                                                    & θ                                                  & th
		(d.i.~t~+~h)                                                                                                         & Θῆτα                                               & Thēta                            \\
		Ι                                                                                                                    & ι                                                  & i kurz
		oder lang                                                                                                            & Ἰῶτα                                               & Iōta                             \\
		Κ                                                                                                                    & κ                                                  & k
		                                                                                                                     & Κάππα                                              & Kappa                            \\
		Λ                                                                                                                    & λ                                                  & l
		                                                                                                                     & Λά{(μ)}βδα\footnote{Die
			besser bezeugte Namensform ist λάβδα, vgl. \passagem{Cratylus402e},
			\gls{idx.Cratylus405d}, \gls{idx.Cratylus427b}, \gls{idx.Cratylus434cd} nach
			dem cod.~Oxoniensis; Schmidt, Zeitschr.~f.~Gymn.-W., a.~a.~O. 423; Btr. S.\
			55f., Philodem. Fl. Jahrb. Suppl. XVII, 241, 258. Auch bei
			\glsxtrlong{Photius} v. λάμβδα (und \glsxtrlong{Eupolis} das.) ist λάβδα nach
		der Buchstabenfolge offenbar herzustellen (L.~Dindorf, Cobet).}                                                      & La{(m)}bda                                                                            \\
		Μ                                                                                                                    & μ                                                  & m
		                                                                                                                     &
                                                                                                                         Μῦ\footnote{\label{foot:mo}Bei
			\glsxtrlong{Demokrit}
			μῶ,
			\glsxtrshort{Eustathius}
			zu Il. Γ Afg.,
			\glsxtrshort{Photius}
			unter μῶ; dieser Form
			wird νῶ für νῦ
		entsprochen haben.}                                                                                                  & My                                                                                    \\
		Ν                                                                                                                    & ν                                                  & n
		                                                                                                                     & Νῦ                                                 & Ny                               \\
		Ξ                                                                                                                    & ξ                                                  & x
		                                                                                                                     & Ξῦ, Ξεῖ,
		sp.~Ξῖ\footnote{\label{foot:ksi}Die Schreibungen ξῖ, πῖ, φῖ, χῖ, ψῖ für ξεῖ, πεῖ u.~s.~f.\
			stammen aus der Zeit, wo ει mit ι gleichlautend geworden war; doch steht πεῖ,
			φεῖ, χεῖ im Cod. A des \passagemb{AthenaeusX453d}; πεῖ, χεῖ finden sich auf attischen
			Inschriften (Meisterhans, Gramm.~d.~att~.Inschr., 2. Aufl., Berl. 1888, S.\
			5); Helladios b.\ \authwork{PhotiusBibl}, p.\ 530 Bk.\ bezeugt φεῖ, χεῖ, ψεῖ;
			ξεῖ steht Philodem. Fl.\ Jahrb.\ a.~a.~O.\ 239. Man findet aber für ξεῖ auch den
			Namen ξῦ (Kallias b.\ \passagem{AthenaeusX453d}, \passagemb{DikeSymph9} u.~s.,
			Schmidt a.~a.~O., Btr.\ 56), der sich an μῦ, νῦ anschliesst; ξεῖ scheint nach
			Analogie der anderen neuen Buchstabennamen gebildet. Für ξῖ s.\ Priscan I,
		§9f.; das Excerpt aus Helladios nennt als Namen auf υ nur ὗ, μῦ, νῦ.}                                                & Xy, Xei (Xi)                                                                          \\
		Ο                                                                                                                    & ο
		                                                                                                                     & o
		kurz und geschlossen                                                                                                 & Οὖ, später ὄ, spät ὂ
		μικρόν\footnote{\label{foot:omicron}S. Anm. 2, S.~\pageref{1.anmerk2}, u. N.~\ref{not:psilon}.} & Ou, Ŏ (Omīkron)                                                                       \\
		Π                                                                                                                    & π                                                  & p
		                                                                                                                     & Πεῖ,
		missbr. Πῖ\textsuperscript{\ref{foot:ksi}}                                                                           & Pei (Pi)                                                                              \\
		Ρ                                                                                                                    & ρ                                                  & r
		                                                                                                                     & Ῥῶ                                                 & Rho                              \\
		Σ Ϲ                                                                                                                  & ς ϲ                                                & s  scharf
		                                                                                                                     &
                                                                                                                         Σῖγμα\footnote{\label{foot:san}Ein anderer Name war σάν, besonders bei den Doriern üblich,
			s.\ \passagemb{Hdt1139}: Δωριέες μὲν σὰν καλέουσι, Ἴωνες δὲ σῖγμα. Pindar,
			Frg. 47 (57 A, Bergk): τὸ σὰν κίβδαλον. Der Chalkedonier Thrasymachos (Epigram
			bei \passagemb{AthenaeusX454f}) buchstabiert seinen Namen im Hexameter θῆτα ῥῶ
			ἄλφα σὰν ὖ μῦ ἄλφα χεῖ οὖ σάν. Indes Müssen auch Andere als Dorier so gesagt
			haben: Achaeus von Eretria, der für die attische Bühne dichtete, gebraucht σάν
			beim Buchstabieren von Διονύσου, \passagem{AthenaeusX466f}. Schmidt a.~a.~O.\
			424, Brt.\ 57. -- Dass nicht σᾶν zu accentuieren, erweist Thrasymachos' Vers;
			unklar ist die Sache bei σίγμα, doch hat σῖγμα mehr Gewähr (Schmidt, 425,
		Btr.\ 58).}                                                                                                          & Sigma                                                                                 \\
		Τ                                                                                                                    & τ                                                  & t                & Ταῦ   & Tau   \\
		Υ                                                                                                                    & υ                                                  & ü kurz
		oder lang,                                                                                                           &
		ὖψιλόν\textsuperscript{\ref{foot:omicron}}                                                                           & Y [Ypsīlon]                                                                           \\
		Φ                                                                                                                    & φ                                                  & ph
		(d.~i.~p~+~h)                                                                                                        & Φεῖ, missbr. Φῖ\textsuperscript{\ref{foot:ksi}}    & Phei (Phi)                       \\
		Χ                                                                                                                    & χ                                                  & ch
		(d.~i.~k~+~h)                                                                                                        & Χεῖ, missbr. Χῖ\textsuperscript{\ref{foot:ksi}}    & Chei (Chi)                       \\
		Ψ                                                                                                                    & ψ                                                  & ps               & Ψεῖ,
		missbr.\ Ψῖ\textsuperscript{\ref{foot:ksi}}                                                                          & Psei (Psi)                                                                            \\
		Ω                                                                                                                    & ω                                                  & o
		lang und offen                                                                                                       & Ὦ, spät ὦ μέγα\textsuperscript{\ref{foot:omicron}} & Ō (Omĕga)                        \\
	\end{longtable}



\noindent\subparagraph{\label{1.anmerk1}} In der
Kursivschrift nimmt ς am Ende des Wortes die Gestalt ς an, als: σεισμός. Nach
dem Vorgange von H. Stephanus gebraucht man oft das ς auch in der Mitte
zusammengesetzter Wörter, als: προσφέρω, δυσγενής, vgl. Wolf,
Litter.~Analekt.~I, S. 460 ff., doch ist dies insofern eigentlich falsch, als
das Zeichen ς seine Gestalt nur dem Absetzen beim Wortschluss verdankt.

\noindent\subparagraph{\label{1.anmerk2}} Die nur allzu fest eingebürgerten Namen Epsilon und Ypsilon kommen,
wie Schmidt (Zeitschr.~f.~Gymn.-W., 1851, 433 ff., Beiträge\ z.~Gesch.\ d.~Gramm.,
S.~64 ff.) nachgewiesen hat, aus einem reinen Missverständnis. Byzantinische
Grammatiker, wenn sie Regeln über die mit αι oder ε, οι oder υ (welche
Schreibungen dazumal unter sich gleichlautend waren) zu schreibenden Wörter
geben, pflegen z.B.\ zu sagen: τὸ παῖδες κατὰ τὴν παραλήγουσαν διὰ τῆς αι
διφθόγγου (γράφεται), τὸ δὲ πέδαι διὰ τοῦ ε ψιλοῦ, d.~i.~mit einem blossen ε,
ohne damit im Geringsten dem Buchstaben einen vermehrten Namen geben zu wollen.
Als Namen der Buchstaben finden sich ἒ ψ.\ und ὖ ψ.\ nur bei dem Grammatiker
hinter dem Etymolog.~Gudianum und bei Chrysoloras. Die Bezeichnungen ὂ μικρόν
und ὦ μέγα sind eher als Namen zu fassen, doch erst als byzantinische, aus der
Zeit des Gleichlauts der beiden Zeichen; man kann das bekannte “harte und weiche
T (D)” damit zusammenstellen. Will man die alten Bezeichnungen εἶ und οὖ, über
deren Entstehung wir unten (S.\ 44) handeln, als missverständlich
nicht zulassen, so ist doch durch die Namen ἔ, ὄ (ὖ, ὦ), d.~i.~ĕ, ŏ (ü, ō)
allem Missverständnis vorgebeugt.

\section[Kurze Geschichte des griechischen Alphabets und der alten
	Schreibweise]{Kurze Geschichte des griechischen Alphabets und der alten
	Schreibweise\protect\footnote{Vgl.\ das klassische Buch von A.~Kirchhoff, Studien
		z.~Gesch.~d.~griechischen Alphabets, in 4. Aufl., Gütersloh 1887.}}


\paragraph{} Das Alphabet ist nach der Aussage der Alten, die sich überall bestätigt, den
Griechen von den Phöniciern zugebracht worden; die Sage knüpft die Einführung an
den Einwanderer Kadmos an \passagem{Hdt558}. Bei den Ioniern hiessen darum auch die
Buchstaben φοινικήια (Her.\ das., Ephoros in Bk.\ Anecd.\ 782, Inschrift von Teos
Σ.~I.~Gr.~3044 ὃς ἂν φοινικήια ἐκκόψει, d.~i.~γράμματα). Und zwar sind von
Anfang an sämtliche 22 phönikische Buchstaben von den Griechen übernommen
worden, unter leichter Umwandlung der Namen: Aleph = Alpha, Beth = Beta, Gimel =
Gamma (Gemma, s.~oben N.~\ref{foot:gemma}), Daleth = Delta, He = Ei, Vau = Ϝαῦ (Βαῦ,
Digamma), Sain = Zeta, Cheth = Eta (Heta, oben
N.~\ref{foot:heta}), Teth = Theta, Jod =
Jota, Kaph = Kappa, Lamed = La{(m)}bda, Mem = My (Mo, oben N.~\ref{foot:mo}) Nun = Ny,
Samech = Sigma (vgl.\ unten~\ref{par:san}; der Name wenigstens daher, wiewohl nicht
die Form), Ain = Ou, Phe = Pei, Zade griech. Μ (der Name \glsxtrshort{grc} nicht nachweisbar),
Koph = Koppa (\epigraphic{Ϙ}, \glsxtrshort{latin} Q), Resch = Rho, Schin = San (vergl.\
unten~\ref{par:san} und oben
N.~\ref{foot:san}), Thav = Tau.

\paragraph{} Aber die phönicischen Hauchzeichen wurden in dem griechischen
Alphabete zu Vokalzeichen, und damit die Konsonantenschrift des Semitischen zur
Lautschrift, was das hohe Verdienst der Griechen bleibt. Man nahm Aleph für a,
He für e, Cheth für Eta (d.~h.~die Ionier Asiens, während die anderen Stämme
dies Zeichen als Hauchzeichen beibehielten, in welcher Geltung es auch die
Lateiner bekamen und bewahrten), Jod für i, Ain für o.

\paragraph{\label{par:san}} Von den Konsonantenzeichen der Phönicier waren indes auch so noch manche nicht ohne weiteres verwendbar.
S-Laute giebt es im Semitischen vier: das weiche s (Sajin), das gewöhnliche scharfe (Samech), ein emphatisch gesprochenes scharfes (Zade) und den dicken Zischlaut, den wir sch schreiben (Schin).
Das Sajin nun ist im allgemeinen in seiner Geltung geblieben, wenn auch der weiche Zischlaut im griechischen mit d versetzt war; der Name Zeta scheint nach Eta Theta umgewandelt.
Samech hat bei den asiatischen Ioniern seinen Namen an den vorletzten phönicischen Buchstaben abgegeben, seinen Platz und seine Gestalt aber bewahrt, mit dem neuen Werte als ks, und dem neuen Namen ξῦ (nach νῦ) oder ξεῖ (nach πεῖ gebildet).
Die anderen Griechen haben meistens auch das Zeichen nicht angewandt, ehe sie das ionische Alphabet annahmen.
Für den scharfen S-Laut aber finden wir bei den verschiedenen Stämmen zwei
Zeichen verwendet, nicht nebeneinander, sondern eins oder das andere: Σ
(\epigraphic{}) und Μ, von denen jenes auf Schin (\glsxtrshort{grc} San), dieses auf Zade zurückzugehen scheint; ersteres ist schliesslich das allgemeine geworden.
-- Emphatische, im \Glsxtrlong{grc}en fehlende Laute waren im Phönicischen ferner Teth und Koph; die Griechen haben das Zeichen Teth für den aspirierten Laut (t + h) verwendet, das Koph aber lange Zeit neben Kaph ohne Unterschied des Lautes, wie es scheint, und mit der Massgabe gebraucht, dass sie vor o (und u υ) dem Namen entsprechend Koppa, im übrigen aber Kappa schrieben.
Die Römer, welche ausserdem auch dem dritten Zeichen des Alphabets den Wert der gutturalen Tenuis gegeben hatten, liessen dies, das Σ, das allgemeine Zeichen sein, während sie das Ka = Kappa vor a, das Ku = Koppa vor u (mit folgendem Vokale) gebrauchten. -- Die ausser Kurs gesetzten Zeichen wurden übrigens von den Griechen in den Alphabeten fortgeführt, und konnten als ἐπίσημα (Abzeichen, Kennzeichen, notae) noch weitere Verwendung finden.
Insbesondere als Zahlzeichen ist sowohl Koppa (im Werte von 90) als auch San
(für 900) geblieben, letzteres mit dem vermehrten Namen σανπῖ, der aus der
Gestalt \epigraphic{} mit ihrer scheinbaren Vereinigung von Σ (ς) und II hergeleitet
ist.

\paragraph{} In dieser Anpassung der phönicischen Zeichen ist zugleich auch die
Richtung schon fest bestimmt, in welcher das Alphabet auf griechischen Boden
vervollständigt wurde.
Zunächst musste für den fünften Vokal υ (u oder ü) ein Zeichen gebildet werden,
welches man, wie es scheint, aus einer Nebenform des Vau gewann und hinten an
den Schluss des Alphabets hängte.
Kein griechisches Alphabet ist ohne dieses Zeichen, während es allerdings
Alphabete giebt (auf den Inseln Kreta, Thera, Melos), in denen dies das einzig
nicht phönicische ist.
In diesen Alphabeten werden die gutturale und die labiale Aspirata entweder
durch die Tenuis mitausgedrückt (Kreta), oder durch Zusammensetzung mit dem
Hauchzeichen \epigraphic{Κ}, \epigraphic{Γ} (Thera, Melos), gemäss
der Aussprache und analog der späteren Schreibweise der Römer CH, PH\@.
An den meisten Orten indes zog das Vorhandensein eines Zeichens für die dentale
Aspirata frühzeitig die Erfindung von solchen für die beiden andern nach sich,
so zwar, dass für ph allgemein Φ verwandt wurde, für ch aber teils X, nämlich
bei den asiatischen Ioniern, den Athenern, Korinthiern, Argivern u.~a., teils
\epigraphic{}, unter Verwendung des Zeichens X für ξ, nämlich auf Euböa,
in Nord- und Mittelgriechenland ausser Attika, im grössten Teil des Peloponnes,
endlich in den meisten westlichen Kolonien, durch welche, nämlich durch die
chalkidischen Kumäer, auch die Römer das X im Werte von x erhielten. 
Die neuen Zeichen Φ X bezw.~X (ks) Φ \epigraphic{} (ch) wurden wieder an
den Schluss des Alphabets gehängt.
Endlich hat, namentlich bei den Ioniern Asiens, das Vorhandensein eines Zeichens
für den Doppellaut ks auch ein solches für den Doppellaut ps hervorgerufen, bei
den Ioniern in der Form, die bei den westlichen Griechen das ch bedeutete (Ψ),
und die Verwendung des Hauchzeichens für das offene e (mit welchem, nach dem
dialektischen Verluste des Hauches, der Name nun anfing) die Erfindung einer
Doppelbezeichnung auch für den Vokal o, der gleichfalls offen und geschlossen in
merklicher Verschiedenheit existierte.
Die Ionier Asiens haben dazu den Kreis des O unten geöffnet und die Linie nach
beiden Seiten auseinandergebogen; das neue Zeichen, Ω, entsprach dem H und
drückte den offenen Laut aus, während O für den geschlossenen blieb.
Mit Ψ und Ω ist das Alphabet abgeschlossen worden, und zwar, bei den Ioniern,
noch im 7. Jahrhundert v. Chr.

\paragraph{\label{par:ionicalpha}}Das ionische Alphabet (τὰ Ἰωνικὰ γράμματα) nun ist schliesslich, unter Verdrängung der übrigen lokalen und nationalen Alphabete, das allgemein griechische geworden.
Es umfasst 24 Buchstaben, nämlich 19 phönicische (nach Ausscheidung von Vau, Zade und Koppa) und fünf neue: Γ Φ Χ Ψ Ω.
Die Stämme indes, welche den Laut des Digamma nicht verloren hatten, behielten
auch nach Annahme des ionischen Alphabets das Zeichen Ϝ bei, hatten also 25
Buchstaben, wie die Böoter, oder, indem sie das halbierte Hauchzeichen
\epigraphic{} für den Hauch eingeführt hatten, sogar 26, wie die Tarentiner und Herakleoten in Italien.
In Athen wurde das einheimische Alphabet (τὰ Ἀττικὰ γράμματα) durch den
Staatsmann Archinos im J. 403/2, unter dem Archon Eukleides, auch für den
offiziellen Gebrauch abgeschafft; es hatte aus 20 Buchstaben bestanden: Α Β
\epigraphic{} (γ) Δ Ε (ε, η) {\sffamily\textsc{i}} Η (h) Θ Ι Κ \epigraphic{}
(λ) Μ Ν Ο (ο, ω) Γ Ρ Σ Τ Φ Χ (ch), und die Doppelbuchstaben ξ ψ waren durch ΧΣ
ΦΣ umschrieben worden.

\paragraph{\label{par:heta}}Die Zeichen Η (in der neuen Geltung) und Ω haben im
allgemeinen nur lange Laute ausgedrückt, indem das offene e (=
\glsxtrshort{franz} è ê) und das offene o (\glsxtrshort{franz} o in
\emph{alors}) im \Glsxtrlong{grc}en nur als Längen vorkamen.
Hingegen waren geschlossenes e (\glsxtrshort{franz}\ é) und o
(\glsxtrshort{franz} \emph{dos}, \emph{anneau}) sowohl kurz als lang
vorhanden, und darum haben Ε und Ο bei Ioniern und Attikern, auch nachdem diese
das ionische Alphabet angenommen hatten, kurze und lange Laute bezeichnet.
Die langen Laute dienten auch als Namen der betreffenden Buchstaben.
Das lange é indes hatte sehr frühzeitig einen Beiklang von i, und entsprechend
das lange geschlossene o einen solchen von u; darum kommen schon in sehr alter
Zeit in Ionien, Athen, namentlich auch in Korinth und dessen Kolonien für dies
ē̄ ō̄ die diphthongischen Schreibungen ΕΙ, ΟΥ vor, die im Laufe des
4.~Jahrh.~v.~Chr.~in Athen und anderwärts die allgemein angewandten geworden sind und das Ε Ο
auf die Bezeichnung des kurzen ĕ́, ŏ́ beschränkt haben.\footnote{Vgl.\ den
trefflichen Aufsatz von A.~Dietrich, Zum Vokalismus d.~gr.~Spr., Kuhns
Ztschr.~XIV, S.\ 48ff.}
Somit waren auch die Buchstabennamen nunmehr εἶ, οὖ, und es ist der quantitative
Unterschied von Ε und Η, Ο und Ω bereits für die alexandrinischen
Grammatiker\footnote{S.\ auch \passagem{Poetik1458a} (τὰ ἀεί
μακρά, d.i.~η ω, τὰ ἐπεκτεινόμενα d.i.~α ι υ, τὰ βραχέα, d.i.~ε ο).} der
einzige, während ursprünglich der qualitative es ausschliesslich war, der die
Verschiedenheit der Bezeichnung hervorrief. -- Die diphthongische Schreibung ΕΙ,
ΟΥ verwischt den Unterschied von echt diphthongischem ΕΙ = ε + ι und gedehntem
ε, von echt diphthongischem ΟΥ = ο + υ und gedehntem ο; die älteren Inschriften
geben diese Scheidung im allgemeinen wieder, und natürlich muss damals, im 5.
Jahrhundert, noch ein lautlicher Unterschied von ε + ι ei, ο + υ ou und
e\textsuperscript{i} =
ε̄, o\textsuperscript{u} = ο̄ bestanden haben, der nachher verschwand.
Schliesslich sind, wie wir im folgenden Paragraph sehen werden, beide ει zu ī,
beide ου zu ū geworden.
Ursprünglichen Diphthong ει haben z.B.\ λείπω (\glsxtrshort{altatt} \epigraphic{ΕΙΠΟ}),
ἔχει (\epigraphic{ΕΧΕΙ}), πρυτανεία (\epigraphic{ΓΡΥΤΑΝΕΙΑ}); zahlreicher aber
sind die Fälle, wo kein echter Diphthong, sondern Dehnung des ε, oder
Kontraktion aus εε vorliegt: ἐπεστάτει (\epigraphic{ΕΠΕΣΤΑΤΕ}), ὀφειλέτω
(\epigraphic{ΟΦΕΕΤΟ}), ἀβλαβεῖς (\epigraphic{ΑΒΑΒΕΣ}), εἴργασται
(\mbox{\epigraphic{ΕΡΛΑΣΤΑΙ}}) Κλειγένης (\epigraphic{ΚΕΛΕΝΕΣ}), ἐπιθεῖναι
(\epigraphic{ΕΙΠΘΕΝΑΙ}), ἔχειν (\epigraphic{ΕΧΕΝ}).
Vollends überwiegen die entsprechenden Fälle bei ου; für den echten Diphthongen
ου sind anzuführen:\footnote{Meisterhans Gramm.~d.~att.~Inschr., S.\
49\textsuperscript{2}.} οὐ \epigraphic{ΟΥ}, οὗτος
\epigraphic{ΟΥΤΟΣ} und darnach τοιοῦτος u.~s.~w., σπουδή Σπουδίας
\epigraphic{ΣΠΟΥΔΙΑΣ} vgl. σπεύδω, ἀκόλουθος \epigraphic{ΑΚΟΟΥΘΟΣ} vgl.
κέλευθος, ferner κρούω Προκρούστης, βοῦς (βούτης) Βουτάδης, δοῦλος, Σούνιον,
ξουθός, στροῦθος, ἄρουρα, θοῦρος Θούριοι, βροῦκος. Natürlich aber ist durchaus
nicht für alle Wörter, bei denen man über die Art des ου zweifelhaft sein kann,
ein inschriftliches Zeugnis vorhanden, und bei \epigraphic{ΦΡΟΥΡΟΣ ΦΡΟΡΟΣ}, was beides vorkommt (aus προϝοράω) ist schwer zu sagen, was das Richtige sei.

\paragraph{} Die Griechen schrieben ursprünglich, wie die Morgenländer, von der
Rechten zur Linken; nur wenige mehrzeilige Inschriften mit dieser Schreibweise
sind uns übrig geblieben; darauf nach Art der ackernden Stiere (βουστροφηδόν,
\passagem{PausanV176}), so dass die erste Zeile von der Rechten zur Linken, die
zweite von der Linken zur Rechten geht u.~s.~w., oder auch, doch seltener, so,
dass der Anfang von der Linken nach der Rechten, die nächste Fortführung von
dieser zu jener u.~s.~w.~geschieht.
Diese Schreibart, die sich auf zahlreichen Inschriften findet, herrschte
allgemein bis ins 6. Jahrh., und war z.B.\ auf den ἄξονες und κύρβεις des Solon
angewandt.
Doch kommt die rechtsläufige Schrift schon auf den Söldnerinschriften von
Abu-Simbel {(Ende 7.\ Jahrh.)} vor, und zu Herodots (\passagem{Hdt236}) Zeit
schrieb man schon nur nach der Rechten, ausser etwa auf Kreta, dessen Schrift
lange stabil blieb. Bei der Schrift von rechts nach links hatten die Buchstaben
die Richtung nach links, bei der aber von links nach rechts nahmen sie die
entgegengesetzte Richtung an, als: \epigraphic{} u.\ \epigraphic{Γ} (Gamma),
\epigraphic{} u.\ \epigraphic{Κ} (Kappa), \epigraphic{} u.\ \epigraphic{} (My),
\epigraphic{Π} u.\ \epigraphic{} (Pei) u.~s.~w.~Beide Schreibarten finden sich
auf den βουστροφηδὸν geschriebenen Inschriften.

\paragraph{} Die alten Griechen bedienten sich ursprünglich der sogenannten
Kapital- oder Unzial-{(Majuskel-)}Schrift, d.~i.~unserer grossen Buchstaben,
welche Schrift sich auf den Inschriften und Münzen und bis zum achten
Jahrhunderte n.~Chr.~in den Handschriften findet.
Neben dieser kam frühzeitig für den Privatgebrauch eine Kursivschrift auf, die
sich zu freieren Formen entwickelte; aus dieser ist die Minuskelschrift
hervorgegangen, die seit dem 9. Jahrh.~n.~Chr.~auch in den Handschriften der
Schriftsteller herrschend wird.
Die alte Majuskel ging aber damit nicht unter, und aus ihrem Gebrauche zu
Initialen und in Überschriften hat sich unser gegenwärtiger Schreibgebrauch, in
welchem die grossen und die kleinen Buchstaben ihre Stelle finden, entwickelt.


% TODO: Check abreviação ächtgriech.
\section[Von der Aussprache der Buchstaben]{Von der Aussprache der
Buchstaben\protect\footnote{Literatur aus unseren Jahrhundert (abgesehen von den
Grammatiken des Griechischen): G.~Seyfarth, de sonis litterarum Graecarum,
Leips.\ 1824; K.~F.~S.~Liskovius, über d.\ Aussprache des Griechischen, Leipz.\
1825; S.~N.~J.~Bloch, Revision der Lehre von der Aussprache des Altgriechischen,
Altona u.\ Leipz.\ 1826, dazu Nachträge in Seebode's Archiv 1827 u.\ 1829;
``Zweite Beleuchtung der Mattiäschen Kritik, die Aussprache des Altgriechischen
betreffend'', Altona 1832; R.~J.~F.~Henrichsen, über die neugriechischen
Aussprache der hellenischen Sprache, aus dem Dänischen übersetzt von
P.~Friedrichsen, Parchim und Ludwigslust 1839.
Bloch vertheidigt die \glsxtrshort{neugrc} Aussprache als die ächtgriech., wird
aber von Henrichsen gründlich widerlegt.
G.~Curtius, über die Ausspr.\ der griech.~Vokale u. Diphthonge,
Zeitschr.~f.~österr.~Gymn.\ 1852, S.\ 1ff.; ders.\ in den Erläuterungen zu s.\
Schulgrammatik, S.\ 16ff., u.\ in Curtius' Studien I, 2, 277 ff.
Für die \glsxtrshort{neugrc} Aussprache trat dann wieder ein: Ellissen,
Verhandl.\ d.\ XIII.\ Ver.\ deutscher Philologen, Göttingen 1853, S.\ 106ff; eine
gemischte Aussprache befürwortete Bursian, Verh.\ d.\ XX.\ Vers., Leipz. 1863,
S.\ 183ff. S.\ ferner Rangabė, d.\ Ausspr.\ d.\ Griech., 2. Aufl., Leipz. 1882,
der als Grieche seine Aussprache vertritt; Blass, Über die Ausspr.\ des
Griechischen, in 3.\ Aufl.\ Berlin 1888; K.~Zacher, d.\ Ausspr.\ d.\ Gr., Leipz.
1888.
}}\label{sec:buchstabenausprache}

\paragraph{} Die Aussprache der Buchstaben einer toten Sprache genau zu bestimmen ist sehr
schwierig, ja grossenteils ganz unmöglich, da selbst bei einer lebenden Sprache
eine durchaus richtige Aussprache nur aus dem Munde des sie redenden Volkes
erlernt werden kann. Allerdings lebt die griechische Sprache noch in dem Munde
der \Glsxtrlong{neugrc}en; aber sowie in jeder Sprache sich im Laufe der Zeiten die
Aussprache ändert, so ist dies gewiss in so langer Zeit in der griechischen
eingetreten, während die Orthographie infolge des durch das Mittelalter und
ebenso noch zu unserer Zeit ungebrochen herrschenden Klassicismus sich nicht
entsprechend ändern konnte. Schon hiernach darf man mit vollem Rechte
schliessen, dass die Neugriechen die Aussprache der Altgriechen nicht rein und
unverdorben bewahrt haben.

\paragraph{} Gegen Ende des XIV\@. und im XV\@. Jahrh.~n.~Chr.~wurde durch
Übersiedelung vieler griechischen Gelehrten nach Italien die Kenntnis der
griechischen Sprache und Literatur und mit ihr zugleich auch die damals in
Griechenland herrschende Aussprache der Buchstaben in dieses Land verpflanzt und
von hier aus über die übrigen Länder Europas verbreitet. In Deutschland wurde
die griechische Sprache, natürlich mit neugriechischer Aussprache, namentlich
von dem berühmten Joh. Reuchlin (geb. 1455, gest. 1522) gelehrt,  weshalb diese
Aussprache auch die Reuchlinische genannt wird.
Nach derselben wird η, υ, ει, οι und υι wie i, αι wie ä, αυ, ευ, ηυ ωυ vor einem
Vokale und vor den Konsonanten β, γ, δ, ζ, λ, μ, ν, ρ wie aw, ew, iw, ow, vor π,
κ, τ, φ, χ, θ, ξ, ψ, ς wie af, ef, if, of, ου wie u gesprochen. Von den
Konsonanten lautet κ vor e, i palatal, wie kj (tj, dialektisch auch tsch),
ausserdem κ, π, τ nach Nasal wie g (gj) b, d; φ wie f, χ wie ch in ach, jedoch
vor (nicht nach) e, i wie ch in “ich”; θ hat den scharfen englischen Laut wie in
\emph{think}, dazu δ den gelinden wie in \emph{this}; b ist v, g der gelinde
Laut zu χ, also vor e, i gleich j. Σ hat den scharfen, ζ den gelinden S - Laut.
Erasmus von Rotterdam (geb. 1467, gest. 1536) war einer der Ersten, die die
Richtigkeit dieser Aussprache bezweifelten.
Erasmus trug seine Bedenken in einem scherzhaften Zwiegespräche (\emph{Dialogus
de recta Latini Graecique sermonis pronuntiatione}, Basileae 1528) zwischen
einem Löwen und einem Bären vor und stellte eine andere Aussprache dagegen auf,
deren er selbst sich nicht bedient haben soll; gleichwohl geht aus der Art und
Weise, wie er seine Ansichten vorträgt, deutlich hervor, dass er nicht einen
blossen Scherz getrieben, sondern die Sache ernstlich gemeint hat.
Nach der Erasmischen Aussprache lautet η zwischen a u.~e, d.~i.~wie ein offenes
e, υ wie \glsxtrshort{franz} u, αι wie ai in Kaiser, αυ wie au, ει wie das
\Glsxtrlong{holländ}e ei, d.~h.~wie e mit i, ευ wie das \glsxtrshort{latin} eu
in \emph{euge}, d.~i.~wie e + u, οι wie das \glsxtrshort{altfranz} oi in
\emph{foi}, \emph{loi}, \emph{toi}, d.~h.~wie o + i, ου wie das holländische ou
in \emph{oudt}, \emph{kout}, \emph{gout} (alt, kalt, Gold), d.~h.~wie o mit u.
Diese Aussprache fand als die wissenschaftlich und rationell besser begründete
bald viele Anhänger und verdrängte zuletzt überall die neugriechische oder
Reuchlinische; jedoch ist sie mit der Zeit in den verschiedenen Ländern mehr
oder weniger ausgeartet, indem man der Bequemlichkeit und den Gewohnheiten der
eigenen Sprache folgte.
So wird jetzt in Deutschland ει wie unser ei, d.~i.~wie ai und gleich αι, ευ wie
unser eu und gleich οι, ου wie u gesprochen.
Insofern in der neugriechischen oder Reuchlinischen Aussprache der I-Laut
vorherrscht, und insbesondere das häufige und wichtige Eta diesen Laut hat, wird
sie auch Itacismus, die entgegengesetzte Aussprache Etacismus genannt.

\paragraph{} Die neugriechische Aussprache ist etwas innerhalb der griechischen Sprache
Gewordenes, nicht etwa, wie man wohl gemeint hat, eine von aussen
hineingekommene Barbarei; es lässt sich auch zeigen, wie die Tendenz, durch die
η zu ι wurde, schon von Urzeiten her vorhanden war und das ionische η für α
hervorrief. Denn dieselbe Bewegung zu einem helleren Laute hin lässt e aus a
werden und i aus e.\footnote{S.~A.~Dietrich, der Itacismus in d.\ altgr.\ Spr.,
N.\ Jahrb.\ f.\ Philolo.\ 105 (1872) S.\ 11ff.}  Aber dem Glauben der Neugriechen und ihrer Anhänger,
dass das gegenwärtige Stadium dieser Bewegung bereits im frühen Altertum
erreicht gewesen sei, stehen gewichtige Gründe entgegen. Wenn η, υ, ει, οι und
υι wirklich alle wie i gelautet hätten, so lässt sich kein vernünftiger Grund
einsehen, weshalb die alten Griechen sich so vieler Zeichen bedient hätten.
Keine andere Sprache bietet eine analoge Erscheinung. Wohl aber zeigt uns die
Geschichte der Sprachen, dass ursprüngliche Diphthonge im Laufe der Zeiten
allmählich in Einzellaute übergehen. Die Diphthonge αι, ει, οι, υι, αυ, ευ, ηυ,
ωυ, die doch von den alten Nationalgrammatikern als Diphthonge anerkannt werden,
hören auf Diphthonge zu sein; denn auch Lautverbindungen wie aw, ew, ow, af, ef,
of, verdienen sie wohl den Namen von Diphthongen? Der Gleichlaut von η, ι, υ,
ει, οι, υι und die Aussprache von αυ, ευ, ηυ, ωυ wie aw, ew, iw, ow oder wie af,
ef, if, of erzeugt einen Übellaut, der mit dem gerühmten Wohllaute der
griechischen Sprache in offenbarem Widerstreite steht. So lauten z. B. die
Worte: Πείθοιʼ ἄν, εἰ πείθοιʼ· ἀπειθοίης δἴσως (Aesch. Ag. 1049), pithi' an, i
pithi', apithiis d' isos. Εἴ μοι ξυνείη (S. OR. 863) i mi xinii.\ Σὺ δ εἰπέ μοι
μὴ μῆκος (S. Ant. 446) si d' ipe mi mi mikos, εὐχή wie efchi, βεβούλευνται
vevúlewnte, πέπαυνται wie pépawnte, ἐκελεύσθην wie ekelefsthin, γραῦς wie grafs,
ναῦς wie nafs, ναῦν wie nawn. Hätten αυ und ευ wie aw und ew gelautet, so würden
die Griechen das \Glsxtrshort{latin}\ av und ev nicht durch αου (αβ), εου ηου (εβ ηβ), sondern
durch αυ und ευ ausgedrückt haben, als: Ἀουεντῖνος Aventinus, Σεουῆρος Severus,
ἠουοκᾶτοι, evocati, Βατάουοι, Batavi, sowie auch die Römer nicht Orpheus, Peleus
nach der 2. Deklination abgewandelt hätten: Orphei, Orpheo, Orpheum.
Verbindungen von Lauten wie fs, wn widerstreiten durchaus den Lautgesetzen der
griechischen Sprache, ebenso wenn Ζεῦ wie Sew gesprochen wird; denn ein w als
Auslaut eines Wortes war dem griechischen Ohre unerträglich. Auch mit der
Prosodie verträgt sich diese Aussprache in unzähligen Fällen nicht, als: ἅρμα δέ
οἱ χρυσῷ τε καὶ ἀργύρῳ εὖ ἤσκηται Il. κ, 438 (ĕw); ebenso bei αυ vor einem
Vokale: ăw. Auch die Lateiner unterscheiden prosodisch sehr streng zwischen
lĕvis, ăvus und Euander (¯ ¯ ˘), Agaue (˘ ¯ ¯); in der Schrift hielten sie ja u
und v nicht auseinander. Viele Erscheinungen des Wandels der Vokale und
Diphthonge in der Flexion, in der Ableitung und in den Mundarten lassen sich
nach dieser Aussprache nicht erklären. Wie konnte aus τείχεα τείχη (tichi), aus
φάμα φήμη (fimi) entstehen? wie würden sich die Formen ἀϝῦδός (Böot.), ποῶ
ποητής (auch attisch ganz gewöhnlich) u.~s.~w.~zu ἀοιδός (aïdos), ποιῶ (pio),
ποιητής (piitis) u.~s.~w.~verhalten? Die Zeichen des Spiritus asper und lenis
werden von den Neugriechen zwar noch  geschrieben, aber jener nicht mehr
gesprochen. Auch die Verteilung von Längen und Kürzen hat bei ihnen einen
anderen Charakter angenommen, der von der altgriechischen in hohem Grade
abweicht und mehr mit der unserer Sprache übereinstimmt; nämlich der betonte
Vokal wird im allgemeinen gedehnt, der unbetonte stets verkürzt, während im
Altgriechischen wohl die Betonung einigermassen von der Quantität, aber durchaus
nicht diese von jener abhängt. So bilden nach neugriechischer Aussprache ξένους
(xḕnŭs), ὅρα (ṑră) Trochäen; γένοιτο (jḕnĭtŏ), πρόσωπον (prṑsŏpŏ[n], mit
scharfem s; das ν in der nicht affektierten Aussprache stumm), ἄνθρωπος ā́θrŏpŏs
(das θ wie \glsxtrshort{engl} th, ν vor θ in der nicht affektierten Aussprache stumm) bilden
Daktylen. Insbesondere ist zwischen ο und ω schlechterdings kein Unterschied,
weder der Qualität noch der Quantität: betontes ο wird gedehnt, unbetontes ω
wird verkürzt.

\paragraph{} Nach diesen allgemeinen Bemerkungen wollen wir zu den einzelnen Buchstaben
übergehen und versuchen, wie sich die Aussprache derselben wenigstens
annäherungsweise ermitteln lasse. Bezüglich des H haben wir oben gesehen, dass
die Verwendung dieses Hauchzeichens als Vokal erfolgte, um offenes (η) und
geschlossenes (ε) e zu scheiden; von Haus aus war es è, und ehe es i wurde, ist
es é gewesen. Nun wird η noch von griechischen und lateinischen Schriftstellern
des 2., 3., 4. Jahrhunderts n.~Chr.~als Länge des ε hingestellt und umgekehrt ε
als Kürze des η,\footnote{Sext.\ Empir.\ adv.\ mathem.\ p.\ 625 Bk: συσταλὲν μὲν τὸ ε
γίνεται η, ἐκταθὲν δὲ τὸ ε γίνεται η (es folgt Ensprechendes über ο, ω). --
Terentian.\ Maur.\ (Ende des 3.\ Jarhh.\ n.~Chr.) V\@.~450ff.:\ litteram namque ε
videmus esse ad ἦτα proximam, sicut ο et ω videntur sibi; temporum momenta
distant, non soni nativitas. S.\ ferner Marius Victorinus (4.\ Jahrh.) Ars
gramm.\ p.\ 39 Keil, Ausonius p. 202 ed. Bip., Martianus Capella III, §235
u.~s.~w.} so dass zwar der specielle qualitative Unterschied
verschwunden scheint, die allgemeine Qualität aber als e durchaus noch
feststeht. Sodann ist für die ältere Zeit, und zwar für die Aussprache als è,
das ein ganz unbezwinglicher Beweis, dass die Attiker (so die Komiker
\glsxtrlong{Kratinus}
und \glsxtrlong{Aristophanes}) den Naturlaut der Schafe durch βῆ βῆ
wiedergeben:\footnote{\passagemb{KratinusFr43} Kock: ὁ
δ' ἠλίθιος ὥσπερ πρόβατον βῆ βῆ λέγων βαδίζει.\ \passagem{AristophFr645} K.} niemals
haben die Schafe vi vi geblökt. So sagen auch die alten Griechen von den
Ziegen μηκᾶσθαι, von den Rindern μυκᾶσθαι, machen also einen Unterschied zwischen
ē und ü, während neugriechisch beides in mikasthe zusammenfliesst. Wenn ferner
\passagemc{Cratylus418b} angibt, dass im \glsxtrlong{altatt}en vielfach ι und ε statt
η gebraucht sei, so in ἱμέρα, ἑμέρα statt ἡμέρα, so ist klar nur das Eine, dass
er einen Unterschied der Aussprache setzt, indem er ἡμέρα als die
grandiosere bezeichnet;  im übrigen aber steht das, was \glsxtrlong{Platon} um seiner
Etymologien willen -- ἡμέρα von ἱμείρω -- hier und anderwärts vom Attischen
aussagt, derartig mit dem, was uns Denkmäler und Sprachwissenschaft lehren, in
Widerstreit, dass wir es ruhig gleich den Etymologien selbst als Scherz und Spiel
nehmen dürfen. Wenn das η wie ι gesprochen worden wäre, so sieht man nicht ein,
wie η mit zugeschriebenem ι (ῃ) entstanden sei, wie εα in η zusammengezogen
(τείχεα = τείχη), wie von τιμάω τιμήσω, von φιλέω φιλήσω gebildet werden, wie ε
und α in der Augmentation des Verbs in η, das alte ᾱ in η übergehen (φā́μā φήμη)
konnte,da im \Glsxtrlong{grc}en nie ι aus α hervorgeht. Endlich entspricht im
\Glsxtrlong{latin}en dem η in unzähligen Wörtern ē (Crates, Delus u.~s.~w.), und
umgekehrt dem lateinischen ē griech. η, als ῥήγι (Dat.\ rēgi) Plut.\ Qu.\ Rom.\ 63,
καρῆρε (carere) Plut.\ Rom.\ 21, σαπίηνς (sapiens), Ῥῆνος
(Rhenus).\footnote{Vgl.\ A.~Sickinger, de linguae latinae a.\ Plutarchum et
reliquiis et vestigiis, Freigb.\ i.\ Br.\ 1883.} -- Die
Aussprache des η wie ι findet sich dialektisch schon in vorchristlicher Zeit bei
den Böotern, bei denen es zunächst zu ει geworden war; in der Gemeinsprache
zeigen sich die ersten Spuren des Itacismus im 2. Jahrh.~n.~Chr., doch kann er
auch im 4.\ noch nicht herrschend gewesen sein.

\paragraph{} Nach der Lehre der Grammatiker ist ι stets Vokal, nie Konsonant, und daher stets
wie i, nie wie j auszusprechen. Allerdings hat die griechische Sprache, wie wir
weiter unten sehen werden, die grösste Abneigung gegen den Jod-Laut; allein in
dem Falle, wo ι (ε) mit folgendem Vokale mittelst der Synizese einsilbig
auszusprechen ist, ergibt sich notwendig eine dem Jod ähnliche
Aussprache.\footnote{Hermann, em.\ rat.\ gr.\ gr., p. 33 sqq.\ u\. 40sq.;
G.~Meyer, gr.\ Gramm.\textsuperscript{2}, §146ff.}
Auch wenn die Diphthongen αι ει οι vor Vokal verkürzt werden, oder wenn, was auf
Inschriften häufig, dem ε (α,ο) vor Vokal ein ι missbräuchlich zugesetzt wird
(ἐννεία), ist für die Aussprache ein schwaches halbvokalisches ι anzunehmen. --
Das \glsxtrshort{latin} j drücken die Griechen durch ι aus, als: Ἰούλιος.

\paragraph{} Dass υ nicht wie ι ausgesprochen worden sei,geht deutlich aus
\authwork{DeCompVerb} 14 p. 164 Schäf. (77 R.) hervor, wo er lehrt, bei υ würden die
Lippen stark zusammengezogen, der Laut gepresst und dünn herausgestossen, bei ι
geschehe der Luftstoss durch die Zähne,indem der Mund wenig geöffnet werde, und
die Lippen nicht mitwirkten, um den Laut hell und kräftig zu machen (καὶ οὐκ
ἐπιλαμπρυνόντων τῶν χειλέων τὸν ἦχον).Also lautete υ nicht wie ι, sondern wie
unser ü oder das französische u. Auch der Scholiast ad \passagem{Nubes31}
unterscheidet zwischen Ἀμυνίας und Ἀμινίας. Wäre υ wie ι gesprochen worden, so
würde es auch den  Diphthongen υι nichtgeben. Die Römer drückten υ in älterer
Zeit durch u, später, als das zu wenig genau erschien, durch das griechische
Zeichen y aus. Ursprünglich wurde υ ohne Zweifel wie das \glsxtrshort{latin} u und das deutsche
u gesprochen, s. § 5, 3, erst später wie unser ü; Quintil. XII.\ 10, 27 nennt υ
und φ die lieblichsten Laute der griechischen Sprache. (S. Nr. 14.) Der Übergang
zu ι ist allgemeiner erst mitten in byzantinischer Zeit, nicht vor dem 9.\ u.\ 10.
Jahrh., erfolgt; noch in Suidas Lexikon (10. Jahrh.) werden ι η ει einerseits
und υ οι andererseits in der Buchstabenfolge als verschiedene Laute behandelt,
indem ει η ι zusammen hinter ζ und vor θ, οι und υ für sich an den Platz des
letzteren gestellt sind.\footnote{Den Byzantinern sind ει η ἀντίστοιχα von ι, αι
  von ε, οι von υ, d.~h.\ gleichwertige und in der Lehre von der Ortographie
Künstlich geschiedene Bezeichnungen. So in den ortograph.\ κανόνες des
Theognostos (Ende des 9\. u~Afg.\ des 10.\ Jahrh.), s. Egenolff, d.\ ortograph.\
Stücke d.\ byzant.\ Litteratur, Prog.\ Heidelberg 1888, S.\ 21ff.}

\paragraph{\label{par:diphthongen}} Von den Diphthongen besprechen wir zunächst das αι, bei welchem die
neugriechische Aussprache wie ä von namhaften Gelehrten unserer Zeit in Schutz
genommen und geübt worden ist, hauptsächlich wegen seiner Beziehungen zum
\glsxtrshort{latin} ae.
Die Römer nämlich drücken αι durch ae aus, als: σκαιός scaevus, Φαῖδρος
Phaedrus,und die Griechen das \glsxtrshort{latin} ae durch αι, als: Καικίλιος
Caecilius, Πραινεστῖνοι Praenestini.
Indes das römische ae ist eine Abschwächung des
ursprünglichen ai,welches sich in der älteren Latinität in zahlreichen
inschriftlichen Belägen findet, als: Ailius, Gnaivod (= Gnaevo), aidilis,
quaistor, quairatis, aiquom,Aimilius.\footnote{S.~K.~L.~Schneider, Ausf.\ Gr.\
d.\ lat.\ Spr.\ I, 1, S.\ 50ff.} Man darf aber auch für ae mit guten
Gründen annehmen, dass die Römer es nicht als einen Einzellaut, sondern als
Diphthongen ae gesprochen haben.\footnote{Seelmann, Ausspr.\ d.\ Latein., S.\
222ff.} Auch aus der lateinischen Verwandlung des
griechischen αι mit folgendem Vokale in āj, als: Αἴας Ajax, Μαῖα Maja, lässt sich
schliessen, dass die Griechen αι wie ai sprachen. Die Behauptung, die der
Skeptiker \glsxtrshort{SextusEmpiricus} (um 200 n. Chr.) aus “gewissen Philosophen” anführt
(adv.\ mathem.\ p.\ 625 Bk.), dass αι ει ου einfache, von Anfang bis zu Ende des
Ertönens sich gleichbleibende Laute seien, kann natürlich unter allen Umständen
nur für die Zeit des Autors und seiner Gewährsmänner beweisen, wird aber
dadurch unverwendbar, dass es sich hier ausdrücklich um neue, im Alphabete noch
nicht vorkommende Laute handelt, was αι ä (e) kaum und ει i schlechterdings nicht
ist. Unzweideutig aber legt der Musiker Aristides Quintilianus (3. Jahrh.
oderspäter) dem αι die Geltung eines gedehnten ε bei,\footnote{Aristides π.\
μουσικῆς, p.\ 56 Jahn (93 Meibom); s.\ Blass, Ausspr.\textsuperscript{3}, S.\ 67
n.\ 240\textsuperscript{a}.}  gleichwie entsprechend
lateinische Grammatiker der gleichen Zeit ae als Dehnung des ĕ (d.i.\ des offenen
kurzen e) bezeichnen. Die griechischen Grammatiker dagegen (wie Choeroboskus p.\
1214 in Bekkeri Anecd., \glsxtrshort{Theodosius} p. 34 Göttl.,
Schol.~\glsxtrshort{DionysThrax} p.\ 804 in Bekkeri Anecd., \glsxtrshort{Moschopulos} p. 24 sq. Titze) unterscheiden die Diphthonge
von den στοιχείοις\footnote{Henrichsen a.~a.~O., S.\ 95ff.} und lehren, dass
zwar ει, ῃ, ῳ, ᾳ δίφθογγοι κατὰ ἐπικράτειαν seien, d.~h.~solche, in welchen der Laut des einen Vokales so das
Übergewicht hat, dass er allein gehört wird; αι aber nennen sie ἡ ᾶῖ δίφθογγος ἡ
ἐκφωνοῦσα τὸ ι, woraus die diphthongische Natur deutlich hervorgeht.
Choeroboskus
stellt den Diphthongen αι ausdrücklich dem ᾳ entgegen, welches τὸ ι ἀνεκφώνητον
habe. Demnach müssen wir αι sowohl als οι auch für die alexandrinische und die
nächstfolgende Zeit, wo diese grammatische Theorie sich bildete, nicht als
Einzellaute (ä oder e und oe), sondern als wirkliche Diphthonge ansehen. Wenn wir
οι als Diphthong gelten lassen, so müssen wir auch αι als solchen ansehen; denn
beide haben manche Erscheinungen mit einander gemein. Beide werden in der Flexion
(mit Ausnahme des Optativs), wenn ihnen kein Konsonant beigefügt ist, in
Beziehung auf die Betonung als kurz betrachtet;beide entstehen häufig aus αϊ und
οϊ, als: πάϊς (\glsxtrshort{Homerus}) u. παῖς, ὄϊς u. οἶς, ὀΐομαι u. οἴομαι
u.~s.~w.; im Dat.\ Pl.\ und im Optative stehen sich αις u. οις, αι u. οι
gegenüber; ebenso die äolischen Formen παῖσα (aus πάντ-ια) st. πᾶσα u.\ μένοισα
(aus μένοντ-ια) st.\ μένουσα.
Einen sehr starken Beweis liefert die Krasis: aus καὶ ἔστι wird κἄστι, mit Bewahrung
des α, welches also auch in καί erhalten gewesen sein muss. Entsprechend ist μοι
ἐστί μοὐστί. Dass aber in der böotischen Mundart statt αι η (λεγόμενη st.
λεγόμεναι, τύπτομη st. τύπτομαι, Θειβῆος st.\ Θηβαῖος u.~s.~w.) und in der
äolischen αι zuweilen st. ῃ, η (θναίσκω, μιμναίσκω, μαχαίτας st. μαχητής, αἴμισυς
st. ἥμισυς) gebraucht wurde,beweist bei richtiger Betrachtung nicht die
Gleichheit der Aussprache von η und αι, sondern vielmehr die Verschiedenheit.
Übrigens müssen die Griechen αι und οι da, wo sie in Beziehung auf die Betonung
als kurz behandelt wurden, kürzer und flüchtiger ausgesprochen haben als da, wo
sie als lang angesehen wurden; vgl.\ βούλευσαι, βουλεῦσαι, βουλεύσαι, οἶκοι,
Häuser, οἴκοι, zu Hause. In diesen Verbalendungen mit Ausnahme des Optativs muss
auch schon in alexandrinischer Zeit das αι, nach den häufigen Verwechselungen mit
ε auf Papyrus zu schliessen,sich wenig oder gar nicht von ε unterschieden haben.
Aber weiter als auf diesen Fall erstrecken sich diese Verwechselungen nicht in
einem Beispiel,  sodass für καί, ἡμέραι u.~s.~f. die diphthongische Aussprache
auch für diese Zeit eben hieraus unzweifelhaft ist.

\paragraph{} Auf den Diphthongen αι lassen wir den Diphthongen οι folgen, weil sie sich, wie
wir \pararef{sec:buchstabenausprache}{par:diphthongen} 
gesehen haben, einander mehrfach entsprechen. Die Römer gebrauchten in
älteren Zeiten oi, später oe,als: foideratei, foederati, Coilius, Coelius, und
drückten οι in den älteren Zeiten durch oi, später durch oe aus, als: Φοῖβος
Phoebus, Κροῖσος Croesus; wie Ajax aus Αἴας ist Troja aus Τροία. Aber auch oe
bildete ohne Zweifel nicht einen Einzellaut wie das deutsche ö, sondern war ein
Diphthong. Die neugriechische Aussprache des οι wie i ist offenbar eine durchaus
verderbte und junge, indem es noch zu Suidas' Zeit (vgl. oben 6) wenigstens noch
wie ü lautete. Dass nach der Lehre der alten Grammatiker οι kein Einzellaut,
sondern ein wirklicher Diphthong sei, dass οι häufig aus οϊ entstehe, dass im
lesbischen Aeolismus die Endung οισα aus οντια (μένοισα) hervorgehe, dass in der
Krasis von οι mit ε das ο erhalten bleibe, haben wir \pararef{sec:buchstabenausprache}{par:diphthongen} gesehen. Hierzu kommt,
dass οι vor Vokal in der attischen und anderen Mundarten mit ο wechselt, als:
att. χρόα st. χροιά, πόα st. des ion. ποίη, des dor. ποία; dass in der Ableitung
ει in οι, sowie ε in ο,übergeht, als: λείπω λέλοιπα, μένω μέμονα; dass in der
Augmentation des Verbs οι in ῳ übergeht, als: οἴομαι ᾠόμην; endlich bei
\passagem{Op243} die Verbindung von λοιμὸν ὁμοῦ καὶ λιμόν, welche beide Wörter nach der
neugriechischen Aussprache nicht zu unterscheiden gewesen wären.\footnote{Ganz
verkehrt führen die Reuchlianer für ihre Aussprache die Weissagung bei
\passagem{Thuc254} an: ἥξει Δωριακὸς πόλεμος καὶ λοιμὸς ἅμ' αὐτῷ. Es entstand
ein Streit unter den Athenern, ob in der Weissagung λοιμός oder vielmehr λιμός
gesagt sei. Aber gerade aus dem Streite geht hervor, dass beide Wörter
verschieden gelautet haben müssen.} Hiernach
wurde das οι, und zwar bis weit in die Kaiserzeit hinein, der Schreibung
entsprechend wie ein geschlossenes ο mit i ausgesprochen, welcher Laut übrigens
mit dem unseres eu keineswegs gleich, und von dem eines ü nicht weit abliegend
ist. Es ist darum auch nicht nur im Böotischen statt οι vielfach υ geschrieben
worden (ϝυκία st. οἰκία, καλύ st. καλοί), sondern auch anderweitig zeigen sich
zwischen οι und υ auffällige Berührungen: λοιγός -- λυγρός, κοίρανος -- κύριος
(Curtius Etymol.\textsuperscript{5} 658 f.), in Eigennamen -- οίτης u. -- ύτης
(Ἀνδροίτας, Μενοίτας, Κλεοίτης, Ἀνδρύτας, Ἀρχύτας, Φιλύτης).
So lässt sich erklären, wie die Aussprache von οι erst zu υ und von da zu ι
überging.

\paragraph{} In betreff des Diphthongen ει haben wir oben (\pararef{sec:buchstabenausprache}{par:diphthongen}) gesehen, dass ihn die alten
Grammatiker zu den Diphthongen κατὰ ἐπικράτειαν rechneten, also ει als einen
Einzellaut (entweder als langes e oder  als langes i) ansahen. Hierin liegt aber
kein Beweis dafür, dass schon die älteren Griechen ει wie ῑ gesprochen hätten,
sondern nur dafür, dass zur Zeit der Grammatiker, d.~h.~in der alexandrinischen
und römischen, der Diphthong als ein Einzellaut ausgesprochen worden sei. Hierzu
kommt noch, dass Choeroboskus zu den Diphthongen κατὰ ἐπικράτειαν nur ῃ, ῳ und ᾳ
rechnet, aber ει weglässt. Die Römer drücken ει vor Vokalen gewöhnlich durch ē,
vor Konsonanten gewöhnlich durch ī aus, als: Aenēas, Galatea, Medea, Sigeum;
Nīlus, Pisistratus, Phidias.\footnote{S.~K.~L.~Schneider, Ausf.\ Gr.\ d.\ lat.\
Spr. I, S.\ 69ff.}\textsuperscript{,}\footnote{Wenn die Reuchlinianer für ihre
Aussprache des ει als ι und des αι als ε als Beweis anführen, bei Callim.\ Ath.\
Pal.\ 12,28 antworte das Echo ἔχει (echi) auf ναίχι (naechi); so begehen sie eine
argen Fehler.
Der Dichter ruft aus: Λυσανίη, σὺ δὲ ναίχι καλός καλός· ἀλλὰ πρὶν εἰπεῖν | τοῦτο
σαφῶς, ἠχώ φησί τις ἄλλος ἔχει.
Das Echo kann doch auf ναίχι καλός nicht rückwärts antworten ἄλλος ἔχει, sondern
entweder es liegt bloss in dem Worte ἄλλος in Beziehung auf καλός (Henrichsen
a.~a.~O., S.\ 135), oder ``Echo'' bedeutet hier nur die sicher folgenden
Erwiderung (v.\ Wilamowitz, Homer.\ Untersuchungen S.\ 353), oder die Worte sind
zu emendieren: -- -- τοῦτο σαφῶς Ἠχώ (näml.\ καλός, welches wiederholt ist), φησί
τις ἄλλος ἔχειν (E. Petersen, Prog.\ Dorpat 1878; man kann auch einfach das
Komma ver legen: Ἠχώ, φησί τις ``ἄλλος ἔχει'').}
Hieraus und aus zahllosen Verwechselungen auf
Inschriften und Papyrus folgt mit voller Evidenz, dass bereits im 1. (2.)
Jahrh.\ v.\ Chr.\ das ει, dessen Entstehung und ursprünglichen Lautwert wir oben
(\ref{par:heta})
betrachtet haben, zu einem langen i vereinfacht war, ausser vor Vokalen, wo es
damals noch im ganzen den E-Laut hatte und in griechischen Denkmälern mit (ε
oder) η verwechselt wird. 
Zu beachten sind auch die Worte Priscians (I. 9, 50): \emph{I quoque apud antiquos post  e ponebatur et ei diphthongum faciebat, quam pro omni i
  longa scribebant more antiquo Graecorum}.
Die alten Römer hatten nämlich wie die
Griechen den Diphthongen ei und gebrauchten zumal in der Schrift ihn noch lange
da, wo die jüngeren das lange i anwendeten. Aber auch der hier hervorgehobene
griechische,in vielen Denkmälern nachweisbare Gebrauch, das ει allgemein zur
Bezeichnung des langen ι zu verwenden, war zu Priscians Zeit veraltet, indem
inzwischen (durch \glsxtrlong{Herodian}) die grammatische Regelung zwischen ει und ι auf
Grund der ursprünglichen Schreibung erfolgt war. Dass übrigens ει nicht von
Anfang an ī war, erhellt aus zahlreichen Umständen. Sowie häufig αι und οι aus αϊ
und οϊ entstehen, ebenso auch ει aus εϊ, als: ὄρεϊ ὄρει, Ἀτρεΐδης Ἀτρείδης. Auch
die Zusammenziehung von εε in ει, als: φίλεε = φίλει; der Name εἶ für den
Buchstabe nε; die Stelle bei \passagemb{Cratylus402e}: τὸν οὖν ἄρχοντα τῆς
δυνάμεως ταύτης θεὸν ὠνόμασε Ποσειδῶνα, ὡς ποσίδεσμον ὄντα, τὸ δὲ ε ἔγκειται
ἴσως εὐπρεπείας ἕνεκα(der  Zierlichkeit wegen, zur Verschönerung); das ionische
(dorische u.~s.~w.) ηι st. ει, als: στρατηίη st. στρατεία; die Angabe der Grammatiker, dass die Böotier langes ι st. ει gebrauchten, als: λέγις, ἠΐ, ἶμι st. λέγεις, αἰεί, εἶμι:alles dies spricht gegen die neugriechische Aussprache des ει wie ι.

\paragraph{} Was gegen die neugriechische Aussprache der Diphthonge αυ und ευ zu erinnern
ist, habenwir schon Nr. 3 gesehen. Αυ und ευ werden von den alten Grammatikern
(s. \pararef{sec:buchstabenausprache}{par:diphthongen}) als δίφθογγοι κατἀ κρᾶσιν angeführt, d.~h.~als solche, bei welchen
συγκιρνῶσινἑαυτὰ τὰ δύο φωνήεντα καὶ ἀποτελοῦσιν μίαν φωνὴν ἁρμόζουσαν τοῖς δύο
φωνήεσιν.Die Diphthonge αυ und ευ erleiden zuweilen die Diäresis, als ἄϋσαν (ῡ)
b. \glsxtrshort{Homerus} von αὔω, ἐΰ (b. \glsxtrshort{Homerus}) st. εὖ; so wird im \Glsxtrshort{latin} zuweilen das griechische ευ
in einzweisilbiges e-u aufgelöst, als: Orpheus als Daktylus.\footnote{} Hieraus erhellt,
dassin beiden Diphthongen die beiden Laute vernommen worden sind. Da υ, wie wir
§ 5 sehen werden, ursprünglich wie u lautete, so ist anzunehmen, dass αυ wie
unserau und entsprechend ευ als e + u gesprochen worden sind; denn die mit
υgebildeten Diphthonge waren ohne Zweifel eher vorhanden, als υ den
getrübten Laut ü angenommen hatte; weshalb im \Glsxtrlong{neugrc}en auch das υ dieser
Diphthongesich in w und f verhärten konnte.

\paragraph{} Ου war ursprünglich, wenigstens in einer Anzahl von Wörtern, ein diphthongischer
Laut, ähnlich dem altdeutschen ou z. B.in troum, noch mehr dem alt\glsxtrshort{latin} ou z. B.
in ioudico, s. § 2, 6 S. 45, wurde aberspäter ein Einzellaut wie das
französische ou, gleich unserem und dem \glsxtrshort{latin} langen u. Die Römer drücken ου durch
das einfache u aus, als: Mūsa Μοῦσα,eunūchus εὐνοῦχος, sowie die Griechen das
\glsxtrshort{latin} ū durch ου, als: ΒροῦτοςBrūtus, nachmals auch ŭ, als: Νουμᾶς Nŭma,
Ῥήγουλος Regŭlus, in älterer Zeitdies jedoch durch ο, als: Φονδάνιος Fundanius,
Λέντολος (Λέντλος) Lentulus (in einzelnen Fällen ū ŭ durch υ: Σύλλας Sulla,
Ῥωμύλος Romulus, Καπύη Capŭa).\footnote{}
Die Neugriechen sprechen es nicht, nach
Analogie von αυ, ευ, ωυ, wie ow oder of,sondern gleichfalls wie u aus. Wie wir
oben (§ 2, 6) gesehen haben, ist das ουin den meisten Fällen ein verlängertes ο,
demnach eigentlich wie langesgeschlossenes o lautend; doch mischte sich
frühzeitig ein U-Laut hinzu, und zurrömischen Zeit war der Endpunkt der
Entwickelung, die ἐπικράτεια dieses u, schonlange erreicht. Vgl. Nigidius
Figulus b. Gell. 19, 14: Graecos non tantaeinscitiae arcesso, qui ου ex O et Y
scripserunt, quantae, qui ει ex E  et I; illud enim inopia fecerunt, hoc nulla re
subacti, d.~h.~ich beschuldige die Griechen nicht deshalb so sehr des
Unverstandes, weil sie den Laut des langen udurch ου ausgedrückt haben; denn
dazu sind sie durch die Not gezwungen worden,weil sie kein einfaches Zeichen
dafür hatten, wohl aber deshalb, weil sie ganzunnötiger Weise statt ῖ ει
schreiben [falls sich in der nicht unversehrterhaltenen Stelle dies letzte nicht
		vielmehr ursprünglich auf die Römer und ihrei bezog]. Auch die griechische
Bezeichnung des lateinischen v durch ου, als: Οὐάῤῥων Varro, Οὐενουσία Venusia,
Σκαιουόλας Scaevola zeigt deutlich ου als Einzellaut.

\paragraph{} Die Diphthonge ηυ, ωυ und υι werden von den § 3, 7 angeführten alten Grammatikern
δίφθογγοι κατὰ διέξοδον genannt, d.~h.~solche, in welchen der Laut jedes der zwei
verbundenen Vokale getrennt (χωρίς) gehört wird: also sprachen die Grammatiker
e-ü, o-ü, ü-i. Für ηυ und ωυ indes, welche Diphthonge damals inder wirklichen
Sprache nicht mehr existierten, kann diese Aussprache nicht wohl angenommen
werden; denn wie ηυ aus αυ (ηὔχουν von αὐχῶ) oder ευ (ηὐχόμην vonεὔχομαι)
hervorgeht, so der fast nur ionische (dorische) Diphthong ωυ aus ο + αυ(ευ):
ωὐτός ion. aus ὀ αὐτός, ἐμεωυτοῦ aus ἐμέο αὐτοῦ; es muss somit das υ in ηυωυ so
gut wie in αυ ευ den Wert von u gehabt haben. Dagegen das υι, welches beiden
Attikern im 4.~Jahrh.~v.~Chr. völlig in ῡ aufgegangen war, im Hellenistischen
indes erhalten blieb, lautete wohl in der That wie üi(einsilbig), also wie das
französische ui z. B. in lui, pluie, als: μυῖα müia.

\paragraph{} Die Diphthonge ᾳ, ῃ, ῳ werden von den alten Grammatikern (s.~\pararef{sec:buchstabenausprache}{par:diphthongen})
als δίφθογγοι κατὰ ἐπικράτειαν bezeichnet, also als solche, in welchen das ι
ἀνεκφώνητον ist.\footnote{}
Vor Einführung des η und ω schrieb man ΕΙ st. ΗΙ und ΟΙ st.ΩΙ, und im ganzen Altertum das Ι dieser drei Diphthonge in einer Reihe mit den übrigen Buchstaben; dass es von Haus aus nicht ein unnützes Zeichen war, ist schon hiernach selbstverständlich. Vgl. ferner γρᾴδιον aus γραΐδιον, λῃστής aus ληϊστής, πατρῷος aus πατρώιος, ᾕρουν von αἱρῶ, ᾤκουν von οἰκῶ. Es lautete das ι auch noch in der Zeit, wo die Römer die Wörter comoedia, tragoedia, Thraex aufnahmen; denn hier ist ῳ, ᾳ gerade so behandelt wie sonst οι, αι. Dagegen istseit dem 2.~Jahrh.~v.~Chr. das ι verstummt, und wurde zu Strabos Zeit (unter Augustus und Tiberius) von Vielen als unnütz und in dem wirklichen Laute nichtbegründet weggelassen (Str. XIV, p. 648: πολλοὶ γὰρ χωρὶς τοῦ ι  γράφουσι τὰς δοτικάς, καὶ ἐκβάλλουσι δὲ τὸ ἔθος φυσικὴν αἰτίαν οὐκ ἔχον). Darum wird es auch von den Römern in den später aufgenommenen Wörtern nicht berücksichtigt:odeum ᾠδεῖον, rhapsodus ῥαψῳδός, Thracia.

\paragraph{} Über die Aussprache der Konsonanten ist nur Weniges zu bemerken. B lautet im
\Glsxtrlong{neugrc}en ganz wie unser w (\glsxtrshort{franz} v); nur nach Nasal ist in der lebendigen
Volksaussprache der alte B-Lauter halten. Dass die Alten b sprachen, folgt schon
daraus, dass sie den Konsonanten zu den Mutae zählten, was w schlechterdings
nicht ist; gleiches gilt von γ (\glsxtrshort{neugrc} vor e und i j) und δ
(\glsxtrshort{neugrc} wie weiches
\glsxtrshort{engl} th). Γ hatte vorden Kehllauten γ, κ, χ, ξ den Nasenlaut wie ng in Engel,
Angst; im Lateinischen steht dafür n, als: Γάγγης Ganges, συγκοπή syncope,
Ἀγχίσης Anchises, λάρυγξ larynx. Das \Glsxtrlong{sanskrit} hat für diesen Nasallaut einen
besonderen Buchstaben; Nigidius Figulus b. Gell. N. A. 19, 14, 7 nennt dieses n n
adulterinum. Varro
überliefert für γ vor κ u.~s.~w.~den Namen agma (ἄγγμα?), eine
Umdrehung von γάμμα.\footnote{} Manche nehmen eine solche Aussprache des γ auch vor μ ν
an (πρᾶγμα,γίγνομαι), indes wollen sich dem die Thatsachen, wie die stete
Syllabierung πρᾶ-γμα, γί-γνομαι, durchaus nicht fügen. Viel weniger noch kann das
nasale γ indem Homerischen κὰγ γόνυ (Il. υ, 458) vorliegen, da hier doch eine
völlige Angleichung des τ mit dem folgenden γ stattfindet; ebenso in ἔγγονος d.
i.ἔκγονος eggonos. -- Das ς wurde scharf gesprochen, ausser vor Media oder
Liquida,wo es auch im \Glsxtrlong{neugrc}en gelinde lautet, und von den Alten oft mit
ζvertauscht wurde: Ζμύρνα, ζβεννύναι. -- Z ist nicht wie unser z (= ts)
zusprechen, auch nicht, da es als Doppelkonsonant Position bewirkt, wie
das neugriechische ζ, das wie ein weiches s oder wie das französische z
gesprochenwird, sondern nach dem einhelligen Zeugnisse der griechischen
Grammatiker wie sdoder genauer zd (mit \glsxtrshort{franz} Werte des
z).\footnote{} Es wächst daher
vielfach in der Wortbildung und Komposition ein ζ aus σδ zusammen: Ἀθήναζε aus
Ἀθήνας-δε, βύζηνaus βύς-δην (vgl. βέβυσμαι und πλέγδην), Θεόζοτος aus
Θεός-δοτος. Fernerverliert σύν vor ζ = σδ das ν so gut wie vor στ, σπ u. s. w.:
συζῆν συζυγία --σύστημα συσπᾶν. Den persischen Gottesnamen Auramazda schreibt
\glsxtrlong{Platon} Ὠρομάζης,die Stadt Aschdod in Palästina heisst bei
\glsxtrlong{Herodotus} u.~a.~Ἄζωτος.
Indes  istseit der hellenistischen Zeit das zd zu z (\glsxtrshort{franz}) vereinfacht worden,
weshalbin der Septuaginta Ἀσδώδ, auf einer Inschrift des 1. Jahrh. v. Chr.
Ὠρομάσδηςgeschrieben wird. -- Θ ist nicht wie das lispelnde neugriechische θ oder
dasenglische th zu sprechen, sondern, da es aus τ und ', wie φ aus π und ', χ
aus κund ', entstanden ist, wie ein τ mit Hauch dahinter, also τ, z. B.
ἀνθέλκω,entstanden aus ἀντ (ὶ) und ἕλκω. Entsprechend ist X nicht unser ch noch
das neugriechische χ, sondern ein κ mit Hauch darnach (κ): οὐκ ὅτι (ouk hoti)
wirdgeschrieben οὐχ ὅτι (ΟΥΧΟΤΙ) oukhoti. Der Beweis wird hierfür auch durch
die Geltung von φ χ θ als Mutae geliefert; denn englisches th, unser ch, f
sindnicht Mutae, sondern Spiranten und gehören zu den ἡμίφωνα (§ 7, 2). Über φ
s.unten besonders. -- Die Liquidae Λ, Μ, Ρ hatten anlautend einen volleren
Klangund werden in alten Inschriften im Anlaut auch wohl mit Hauch geschrieben
(ΛΗ,ΡΗ, besonders ΜΗ); die Grammatiker haben nur das P als im Anlaut und in
der Verdoppelung aspiriert gehört und bezeichnet. Vgl. § 8, 1. Übrigens wurde das
Pnach Dionysios' Beschreibung (\glsxtrshort{DeCompVerb} p. 79 R.) mit der
Zungenspitzegesprochen, war also wie im \Glsxtrlong{neugrc}en dental, nicht guttural.
-- Über Ξ s.weiter unten unter Ψ. -- Σχ bildete nicht wie das deutsche sch einen
Laut,sondern wurde wie das lateinische sch getrennt gesprochen, und zwar sk',
als:σχολή (d.~i.~sk' olē), \glsxtrshort{latin} schola, woraus sich das Schwanken zwischen σχ
und σκz. B. in σχινδάλαμος σκινδάλαμος erklärt, s. Fritzsche ad Aristoph. Thesm.
p.611. -- Τι lautete wie ti ohne Zischlaut, als: Κριτίας. -- Φ wurde nicht wie
das\glsxtrshort{latin} f, sondern wie ein aspiriertes π, also π gesprochen. Wenn daher die
Römerdas φ in ihrer Sprache ausdrücken wollen, so gebrauchen sie dem Laute
gemäss ph,als: Phaedrus Φαῖδρος, und nur in urverwandten Wörtern, wie fuga,
fama,bedienten sie sich ihres f. Dagegen bezeichnen die Griechen das \glsxtrshort{latin} f (aus
Not)stets durch φ, als: Fabius Φάβιος, φερῖρε ferire. Quintilian 12, 10, 27
nenntdie beiden griechischen Laute φ und υ die lieblichsten Laute der Griechen.
“Wennwir”, fährt er fort, “(im “Sprechen) dieselben gebrauchen, nescio quo
modohilarior protinus “renidet oratio, ut in Zephyris et zophoris (?).
Werdendie“selben durch unsere Buchstaben (f und u) ausgedrückt, surdum “quiddam
etbarbarum efficient, et velut in locum earum succedent “tristes et
horridae,quibus Graecia caret. Denn das f wird paene “non humana voce inter
discriminadentium herausgestossen.” -- Die beiden Doppellaute ξ und ψ sind wie ks
und ps zusprechen (vgl. Dionys. Thrax Bk. Anecd. p. 632, \authwork{DeCompVerb}
p. 82R., Sext. Emp. adv. gramm. § 103, p. 622 Bk.), auch wenn sie aus γς, χς,
βς, φςentstanden sind, da γ, χ, β, φ vor ς in die tenues ü  bergehen müssen.Vgl.
scrib-o, scrip-si. Also: κόραξ, G. κόρακ-ος, λέξω v. λέγ-ω, ὄνυξ, G.ὄνυχ-ος,
βλέψω v. βλέπ-ω, χάλυψ, G. κάλυβ-ος, κατῆλιψ, G. κατήλιφ-ος. Wenn aufalten
Inschriften, die der Zeichen für ξ ψ entbehren, dieselben nicht sowohldurch ΚΣ
und ΠΣ, als durch ΧΣ und ΦΣ umschrieben werden, so kommt dies daher,weil ς als
γράμμα πνευματῶδες (\passagemb{Cratylus427a}) der Tenuis einen Hauchmitzuteilen
schien. -- Was endlich das Vau ϝ betrifft, so werden diejenigen Rechthaben, die
in demselben den Halbvokal w (\glsxtrshort{engl}), nicht den weichen Spiranten
v (\glsxtrshort{engl};
deutsch w) erblicken. Jenes war auch der Laut des lateinischen v,40)
undentsprechend beschreibt \glsxtrlong{DionysHalic} (Antiq. Rom. 1, 20)
dasaltgriechische Digamma als τὴν ου συλλαβὴν ἑνὶ στοιχείῳ γραφομένην. Wäre das
Vauunser w gewesen, so hätte dieser sehr konsistente Laut durchaus nicht so
leichtverschwinden können.



\clearpage
\glsfindwidesttoplevelname%
\printunsrtglossary[
	type=passagens,
	style=bookindex,
	title={Index Locorum},
]

\end{document}
