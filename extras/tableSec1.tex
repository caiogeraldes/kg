\begin{longtable}{llp{4.5cm}p{2.5cm}l}
	\toprule
	\multicolumn{2}{p{1.5cm}}{Gestalt (σχῆμα)}                                                      & Aussprache (ἐκφώνησις)                             &
	\multicolumn{2}{l}{Name (ὄνομα)\footnote{Über die Benennungen der
			\langlongdecl{grc}{en} Buchstaben s.\ K.~E.~A.~Schmidt in \emph{Ztchr.~f.~d.~Gymnasialwesen
				v.}\ Mützell 1851, Juni, S.\ 471--440; ders., \emph{Beiträge z.~Gesch.\
				d.~Grammatik\ des
	Gr.~u.~Lat.}, Halle 1859, S.\ 48ff.}}                                                                                                                                                                                                                                                                   \\
	\midrule
	\endfirsthead%

	\toprule
	\multicolumn{2}{p{1.5cm}}{Gestalt (σχῆμα)}                                                      & Aussprache (ἐκφώνησις)                             &
	\multicolumn{2}{p{2cm}}{Name (ὄνομα)}                                                                                                                                                                                                                                                                   \\
	\midrule
	\endhead%

	\bottomrule
	\endfoot%

	\bottomrule
	\endlastfoot%


	Α                                                                                               & α                                                  & \ipa{a} kurz oder lang & Ἄλφα                                                                                                            & Alpha \\
	Β                                                                                               & β                                                  & \ipa{b}                & Βῆτα                                                                                                            & Bēta  \\
	Γ                                                                                               & γ                                                  & \ipa{g}                & Γάμμα\footnote{\label{foot:gemma}Bei \authorlong{Demokrit} γέμμα, s.\ \glsxtrshort{Eustathius} z.\ Il.\ Γ Afg.} & Gamma \\
	Δ & δ                                                  & \ipa{d}                      & Δέλτα                                                                                                           & Delta \\
	Ε & ε                                                  & \ipa{e} kurz und geschlossen & Εἶ, sp. ἔ,\newline missbr. ἒ~ψιλόν\footnote{\label{not:psilon}Die Alten nannten das ε εἶ, das ο οὖ, das lange offene o ὦ und das Ypsilon ὖ. S.~\passagem{Cratylus393d} u.~a. Das ε und ο nannten dann spätere Grammatiker ἔ und ὄ. Über den Zusatz ψιλόν s.\ \ref{1.anmerk2}.}                                              & Ei (Ĕ) [Epsīlon]                                                                                                                                                                                      \\
	Ζ
	                                                                                                &
	ζ                                                                                               & {\ipa{sd},~genauer
	\ipa{zd}\newline später weichem \ipa{s}}                                                        & Ζῆτα                                               & Zēta                                                                                                                                             \\
	Η
                                                                                                    &
  η & \ipa{e} lang und offen                                   & Ἦτα &
  Eta\footnote{\label{foot:heta}Eine merkwürdige Notiz findet sich bei
  \passagem{Theodosios7} extr.: τὸ Ἦτα δέ, τὸ ὄνομα τοῦ στοιχεῖον, δασύνεται (also ἧτα), ὅτι παρὰ ἀρχαίοις ὁ τύπος τοῦ Η έν τύπῳ δασείας ἔκειτο, ὥσπερ καὶ νῦν τοῖς παλαιοῖς Ῥωμαίοις (nämlich H). Unzweifelhaft haben diejenigen Griechen, welche das Zeichen Η in ihren lokalen Alphabeten für den Hauch gebrauchten (\pararef{sec:kurzgeschischte}{par:phenichauc}), und so auch die Attiker bis zur Reform der Ortographie, Heta gesagt; die Ioner indes, die den Hauch nicht hatten und das Zeichen für den Vokal anwandten, sagten natürlich Eta, und diese Benennung muss mit dieser Geltung massgebend sein.}                                                                                                                                                                                                                                                                                                  \\
	Θ & θ                                                  & \ipa{th} (d.i.~\ipa{t~+~h})                                                                                    & Θῆτα                                               & Thēta                                                                                                                                            \\
	Ι & ι                                                  & \ipa{i} kurz oder lang                                                                                       & Ἰῶτα                                               & Iōta                                                                                                                                             \\
  Κ & κ                                                  & \ipa{k} & Κάππα                                              & Kappa                                                                                                                                            \\
	Λ & λ                                                  & \ipa{l} &
  Λά{(μ)}βδα\footnote{Die besser bezeugte Namensform ist λάβδα, vgl.\
\passagem{Cratylus402e}, \gls{idx.Cratylus405d}, \gls{idx.Cratylus427b},
\gls{idx.Cratylus434cd} nach dem cod.~Oxoniensis; Schmidt,
\emph{Zeitschr.~f.~Gymn.-W.}, a.~a.~O. 423; Btr. S.\ 55f.,
\glsxtrshort{Philodemos}, Jahrb. Suppl. XVII, 241, 258. Auch bei \authorlong{Photius} v. λάμβδα (und \authorlong{Eupolis} das.) ist λάβδα nach der Buchstabenfolge offenbar herzustellen (L.~Dindorf, Cobet).}                                 & La{(m)}bda                                                                                                                                                                                            \\
	Μ & μ                                                  & \ipa{m} & Μῦ\footnote{\label{foot:mo}Bei \authorlong{Demokrit} μῶ, \glsxtrshort{Eustathius} zu Il. Γ Afg., \glsxtrshort{Photius} unter μῶ; dieser Form wird νῶ für νῦ entsprochen haben.}                                                                             & My                                                                                                                                                                                                    \\
	Ν & ν                                                  & \ipa{n} & Νῦ                                                 & Ny                                                                                                                                               \\
	Ξ & ξ                                                  & \ipa{x} & Ξῦ, Ξεῖ,
  sp.~Ξῖ\footnote{\label{foot:ksi}Die Schreibungen ξῖ, πῖ, φῖ, χῖ, ψῖ für ξεῖ,
πεῖ u.~s.~f.\ stammen aus der Zeit, wo ει mit ι gleichlautend geworden war; doch
steht πεῖ, φεῖ, χεῖ im Cod. A des \passagemb{AthenaeusX453d}; πεῖ, χεῖ finden
sich auf attischen Inschriften (Meisterhans, \emph{Gramm.~d.~att~.Inschr.}, 2.
Aufl., Berl. 1888, S.\ 5); Helladios b.\ \authwork{PhotiusBibl}, p.\ 530 Bk.\
bezeugt φεῖ, χεῖ, ψεῖ; ξεῖ steht \glsxtrshort{Philodemos}, Jahrb.\ a.~a.~O.\ 239. Man findet aber für ξεῖ auch den Namen ξῦ (Kallias b.\ \passagem{AthenaeusX453d}, \passagemb{DikeSymph9} u.~s., Schmidt a.~a.~O., Btr.\ 56), der sich an μῦ, νῦ anschliesst; ξεῖ scheint nach Analogie der anderen neuen Buchstabennamen gebildet. Für ξῖ s.\ \passagem{Priscian19f}; das Excerpt aus Helladios nennt als Namen auf υ nur ὗ, μῦ, νῦ.}         & Xy, Xei (Xi)                                                                                                                                                                                          \\
	Ο & ο & \ipa{o} kurz und geschlossen & Οὖ, später ὄ,\newline spät ὂ μικρόν\footnote{\label{foot:omicron}S. Anm. 2, S.~\pageref{1.anmerk2}, u. N.~\ref{not:psilon}.} & Ou, Ŏ (Omīkron)                                                                                                                                                                                       \\
	Π & π                                                  & \ipa{p} & Πεῖ, missbr. Πῖ\textsuperscript{\ref{foot:ksi}}                                                      & Pei (Pi)                                                                                                                                                                                              \\
	Ρ & ρ                                                  & \ipa{r} & Ῥῶ                                                 & Rho                                                                                                                                              \\
	Σ Ϲ & ς ϲ                                                & \ipa{s} scharf &
  Σῖγμα\footnote{\label{foot:san}Ein anderer Name war σάν, besonders bei den
Doriern üblich, s.\ \passagemb{Hdt1139}: Δωριέες μὲν σὰν καλέουσι, Ἴωνες δὲ
σῖγμα.\ \passagemb{PindarFr47}: τὸ σὰν κίβδαλον. Der Chalkedonier Thrasymachos
(Epigram bei \passagemb{AthenaeusX454f}) buchstabiert seinen Namen im Hexameter
θῆτα ῥῶ ἄλφα σὰν ὖ μῦ ἄλφα χεῖ οὖ σάν. Indes Müssen auch Andere als Dorier so
gesagt haben: Achaeus von Eretria, der für die attische Bühne dichtete,
gebraucht σάν beim Buchstabieren von Διονύσου, \passagem{AthenaeusX466f}.
Schmidt a.~a.~O.\ 424, Brt.\ 57.\ -- Dass nicht σᾶν zu accentuieren, erweist Thrasymachos' Vers; unklar ist die Sache bei σίγμα, doch hat σῖγμα mehr Gewähr (Schmidt, 425, Btr.\ 58).}                                                                                     & Sigma                                                                                                                                                                                                 \\
	Τ & τ                                                  & \ipa{t}                      & Ταῦ                                                                                                             & Tau   \\
	Υ & υ                                                  & \ipa{ü} kurz oder lang                                                                                      & ὖψιλόν\textsuperscript{\ref{foot:omicron}}                                                      & Y [Ypsīlon]                                                                                                                                                                                           \\
	Φ & φ                                                  & \ipa{ph} (d.~i.~\ipa{p~+~h})                                                                                   & Φεῖ, missbr. Φῖ\textsuperscript{\ref{foot:ksi}}    & Phei (Phi)                                                                                                                                       \\
	Χ
    & χ                                                  & \ipa{ch} (d.~i.~\ipa{k~+~h})                                                                                   & Χεῖ, missbr. Χῖ\textsuperscript{\ref{foot:ksi}}    & Chei (Chi)                                                                                                                                       \\
	Ψ & ψ                                                  & \ipa{ps}                     & Ψεῖ, missbr.\ Ψῖ\textsuperscript{\ref{foot:ksi}}                                                     & Psei (Psi)                                                                                                                                                                                            \\
	Ω & ω                                                  & \ipa{o} lang und offen                                                                                  & Ὦ, spät ὦ μέγα\textsuperscript{\ref{foot:omicron}} & Ō (Omĕga)                                                                                                                                        \\
\end{longtable}
